% !TeX program = lualatex
% !TeX encoding = utf8
% !TeX spellcheck = uk_UA
% !BIB program = biber

\documentclass[10pt]{beamer}
\usetheme{QuantumChemistry}
\usepackage{QuantumChemistry}

\def\vect#1{\mathbf{#1}}
\title[Лекції з квантової хімії]{\huge\bfseries Вступ до квантової хімії}
\subtitle{Лекції з квантової хімії}
\author{Пономаренко С. М.}
%\addbibresource{../Textbook/QChem.bib}
\graphicspath{{pictures/}}
\date{}
\begin{document}
%=======================================================================================================
\begin{frame}
	\titlepage
\end{frame}
%=======================================================================================================

\begin{frame}
	\frametitle{Що таке квантова хімія?}

\begin{block}{}\justifying
    	<<\emph{КВАНТОВА ХІМІЯ, розділ теоретич. хімії, в якому \alert{будова} і \alert{властивості хімічних сполук}, їх взаємодія і перетворення в хімічних реакціях розглядаються на основі уявлень і за допомогою методів квантової механіки.}>>

    \bigskip

	{\small <<Химическая энциклопедия>>, в 5-ти томах, под ред. Кнунянц И.Л. М.: Советская энциклопедия, 1988 -- 1999.}
\end{block}

\begin{alertblock}{}\centering
    Як людство прийшло то поняття атомно-молекулярної будови речовини і при чому тут кванти?
\end{alertblock}

\end{frame}


%\begin{frame}
%	\frametitle{Приблизна структура курсу}
%	\begin{enumerate}
%		\item Історія квантової хімії
%		\item Квантова механіка водню та воднеподібних атомів
%		\item Атом гелію
%		\item Багатоелектронні атоми. Періодична система елементів
%		\item Молекулярний іон водню \ce{H_2^+} та природа хімічного зв'язку
%		\item Квантово-хімічні моделі
%		\item Опис двоатомних молекул
%		\item Багатоатомні молекули
%		\item Хімічний зв'язок в конденсованих системах
%		\item Комп'ютерні програми для квантово-хімічних розрахунків
%		\item Екзамен
%	\end{enumerate}
%\end{frame}

%\begin{frame}[allowframebreaks]
%	\frametitle{Література}
%    \printbibliography
%	\nocite{*}
%\end{frame}


% ============================== Слайд ## ===================================
\begin{frame}{Історія розвитку уявлень про будову матерії}{Найважливіша гіпотеза}
\begin{columns}
	\begin{column}{0.2\linewidth}\centering
         \includegraphics[width=\linewidth]{Faynman}
	\end{column}
	\begin{column}{0.8\linewidth}
         \begin{block}{}\justifying
            Якщо внаслідок якоїсь катастрофи все наукове знання було б знищене і тільки одне речення могло б бути передане наступним поколінням людей, який вислів дав би найбільшу інформацію в найменшій кількості слів? Я вірю, що це атомна гіпотеза... що \alert{всі речі складаються з атомів --- маленьких частинок, які перебувають у безперервному русі, притягуються одна до одної на малих відстанях і відштовхуються під час стиснення}.
        \end{block}
	\end{column}
\end{columns}
\begin{center}
    \includegraphics[width=0.3\linewidth]{FLF}
\end{center}
\end{frame}
% ===========================================================================


%============================================================================
\begin{frame}[t]{Історія розвитку уявлень про будову матерії}{Чи існують атоми?}
	\begin{center}
		\includegraphics[width=0.25\linewidth]{Lavuazie} \includegraphics[width=0.25\linewidth]{Prust} \includegraphics[width=0.25\linewidth]{Dalton}
	\end{center}
	\begin{itemize}\small
		\item \emph{Закон збереження маси}, (1789 році \emph{Антуаном Лораном Лавуазьє}) --- маса речовини при хімічній реакції не змінюється;
		\item \emph{Закон сталості складу} (1799 році\emph{ Жозефом Луї Прустом}) --- будь-яка хімічна сполука, не залежно від способу її отримання, складається з одних і тих же хімічних елементів («елементи» --- це прості речовини);
		\item \emph{Закон кратних відношень} (1803, \emph{Джон Дальтон}) --- якщо два елементи утворюють кілька сполук, маси одного елемента відносяться як прості числа.   Приклад: CO (C:O = 12:16) і CO$_2$ (C:O = 12:32) → відношення 1:2.

	\end{itemize}
\end{frame}
%============================================================================




\begin{frame}[t]{Історія розвитку уявлень про будову матерії}{Як атоми взаємодіють?}
	Єнс Якоб Берцеліус --- 1818, электрохімічна теорія хімічного зв'язку.

	\begin{columns}[c]
		\begin{column}{.1\linewidth}
			\begin{center}
				\includegraphics[width=1.7\linewidth]{Berzelius}
			\end{center}
		\end{column}
		\begin{column}{.8\linewidth}
			\begin{quote}
				<<кожна хімічна сполука залежить від двох протилежних сил, додатної та від'єної електрики, так як ніякої третьої сили не існує>>
			\end{quote}
			\begin{center}
				\includegraphics[width=0.75\linewidth]{DBAS.jpg}
			\end{center}
		\end{column}
	\end{columns}
\end{frame}

% ============================== Слайд ## ===================================
\begin{frame}{Історія розвитку уявлень про будову матерії}{І тут Ейнштейн!}



       	\begin{columns}[c]\centering
		\begin{column}{.2\linewidth}
			\begin{center}
				\includegraphics[width=\linewidth]{Einstein}
			\end{center}
		\end{column}
		\begin{column}{.8\linewidth}
        \begin{block}{}\justifying
            Альберт Ейнштейн у 1905 році дав кількісний опис броунівського руху --- хаотичного руху мікрочастинок у рідині чи газі. Він показав, що ця випадкова поведінка є наслідком теплового руху молекул середовища. Його \alert{теорія підтвердила атомістичну будову матерії} та дала спосіб обчислювати розміри молекул і число Авогадро.
        \end{block}
		\end{column}
	\end{columns}

\begin{columns}
	\begin{column}{0.7\linewidth}
\begin{block}{}\justifying
    Вільгельм Освальд --- авторитет у фізичній хімії, прихильник «енергетизму» --- відкидав атоми як «гіпотетичні сутності». Коли молодий Ейнштейн опублікував роботу про броунівський рух (\alert{1905}), Освальд як редактор журналу спершу відхилив її як «незначну». Лише наполягання колег змусило його прийняти статтю. Згодом, після експериментів Перрена, які підтвердили передбачення Ейнштейна, Освальд визнав реальність атомів і фактично капітулював у цій суперечці.
\end{block}
	\end{column}
	\begin{column}{0.3\linewidth}\centering
    \includegraphics[width=\linewidth]{Brownian}
	\end{column}
\end{columns}


\end{frame}
% ===========================================================================

%============================================================================
\begin{frame}[t]{Історія розвитку уявлень про будову матерії}{А-том вже не атом}
	Дж. Дж. Томсон --- 1906, Нобелівська премія за відкриття електрона.
	\begin{columns}[c]
		\begin{column}{.1\linewidth}
			\begin{center}
				\includegraphics[width=1.7\linewidth]{Thomson}
			\end{center}
		\end{column}
		\begin{column}{.8\linewidth}
			\begin{center}
				\includegraphics[width=0.7\linewidth]{ThomsonAtom}
			\end{center}
		\end{column}
	\end{columns}
\end{frame}
%============================================================================
%============================================================================
\begin{frame}[t]{Історія розвитку уявлень про будову матерії}{Атом подібний до сонячної системи?}
	1911, планетарна модель атома Резерфорда\\
	\begin{columns}[c]
		\begin{column}{.1\linewidth}
			\includegraphics[width=1.7\linewidth]{rutherford}
		\end{column}
		\begin{column}{.8\linewidth}
			\begin{center}
				\includegraphics[width=0.5\linewidth]{planet_model}
			\end{center}
		\end{column}
	\end{columns}
\end{frame}
%============================================================================
%============================================================================
\begin{frame}[t]{Історія розвитку уявлень про будову матерії}{Менделеєв і його пріодичний закон}
	1913, зв'язок між зарядом ядра, атомним номером і положенням атома в періодичній таблиці Д.~І.~Менделеєва
	\begin{center}
		\includegraphics[width=0.35\linewidth]{atom}
		\includegraphics[width=0.35\linewidth]{mendeleev}
	\end{center}
\end{frame}
%============================================================================
%============================================================================
\begin{frame}[t]{Історія розвитку уявлень про будову матерії}{Квантовий атом}
	Нільс Бор --- 1913, перша квантова теорія воднеподібного атома.
	\begin{equation*}
		\tcbhighmath[drop fuzzy shadow]{E_1, E_2, \ldots, E_n, \ldots, \quad mv_nr_n = n \hbar, \quad \hbar\omega_{} = E_m - E_n}
	\end{equation*}
	\begin{columns}[c]
		\begin{column}{.1\linewidth}
			\includegraphics[width=1.7\linewidth]{NielsBohr}
		\end{column}
		\begin{column}{.8\linewidth}
			\begin{center}
				\includegraphics[width=0.8\linewidth]{BohrTheory}
			\end{center}
		\end{column}
	\end{columns}
\end{frame}
%============================================================================
%============================================================================
\begin{frame}[t]{Історія розвитку уявлень про будову матерії}{Частинка --- хвиля?}
	1923, хвильова теорія речовини Л. де Бройля
	\begin{equation*}
		\tcbhighmath[drop fuzzy shadow]{p = \frac{2\pi\hbar}{\lambda} = \hbar k}
	\end{equation*}
	\begin{columns}[c]
		\begin{column}{.1\linewidth}
			\includegraphics[width=1.7\linewidth]{Broglie}
		\end{column}
		\begin{column}{.8\linewidth}
			\begin{center}
				\includegraphics[width=0.7\linewidth]{debroil}
			\end{center}
		\end{column}
	\end{columns}
\end{frame}
%============================================================================
%============================================================================
\begin{frame}[t]{Історія розвитку уявлень про будову матерії}{Частинка --- де вона?}
	1923, принцип невизначеності Гейзенберга
	\begin{equation*}
		\tcbhighmath[drop fuzzy shadow]{\Delta x \cdot \Delta p \ge \frac{\hbar}{2}}
	\end{equation*}
	\begin{columns}[c]
		\begin{column}{.3\linewidth}
			\includegraphics[width=1\linewidth]{HeisenbergOld}
		\end{column}
		\begin{column}{.7\linewidth}
			\begin{center}
				\includegraphics[width=0.6\linewidth]{HeisUncert}
			\end{center}
		\end{column}
	\end{columns}
\end{frame}
%============================================================================
%============================================================================
\begin{frame}[t]{Історія розвитку уявлень про будову матерії}{У нас є механіка!}
	1926, матрична механіка В. Гайзенберга, нерелятивістське хвильове рівняння Е. Шредінґера
	\begin{columns}[c]
		\begin{column}{.5\linewidth}
			\begin{center}
				\includegraphics[width=0.5\linewidth]{Heisenberg}\\
				\includegraphics[width=1.1\linewidth]{HeisMatrix}
			\end{center}
		\end{column}
		\begin{column}{.5\linewidth}
			\begin{center}
				\includegraphics[width=1\linewidth]{Sroedinger} \\
				\includegraphics[width=0.7\linewidth]{ShEq}
			\end{center}
		\end{column}
	\end{columns}
\end{frame}
%============================================================================
%============================================================================
\begin{frame}[t]{Історія розвитку уявлень про будову матерії}{Erwin with his psi \ldots}
	\begin{center}
		\begin{equation*}
			\tcbhighmath[drop fuzzy shadow]{\hat{H} \Psi = i\hbar \frac{\partial \Psi}{\partial t}}
		\end{equation*}
	\end{center}
	Свого часу серед фізиків ходила епіграма на Шредінгера, складена на англійській та німецькій мовах і приписувана Е. Хюккелю:

	\begin{center}\itshape
		Erwin with his psi can do \\
		Calculations quite a few \\
		But one thing has not been seen \\
		Just what does psi really mean?
	\end{center}

	\begin{center}\huge
		$\Psi$ - ?
	\end{center}
\end{frame}
%============================================================================
%============================================================================
\begin{frame}{Історія розвитку уявлень про будову матерії}{Де електрон? Ймовірно він \ldots}
	Макс Борн --- 1926, ймовірнісна інтерпретація хвильової функції.
	\begin{overprint}
		\onslide<1>
		\begin{equation*}
			\tcbhighmath[drop fuzzy shadow]{W = \int |\Psi|^2 dV}
		\end{equation*}
		\onslide<2>
		\begin{equation*}
			\tcbhighmath[drop fuzzy shadow]{\vect{p} =  \int e|\Psi|^2  \vect{r} dV = \int \rho  \vect{r} dV }
		\end{equation*}
	\end{overprint}

	\bigskip
	\begin{minipage}[t][][c]{.2\textwidth}
		\includegraphics[width=1\linewidth]{MaxBorn.jpg}
	\end{minipage}
	\begin{minipage}[t][][c]{.7\textwidth}
		\begin{overprint}
			\onslide<1>
			\begin{quote}
				\rmfamily <<є  ймовірність того, що електрон буде знайдено саме в елементі об'єма $dV$>>
			\end{quote}
			\onslide<2>
			\begin{quote}\rmfamily\small
				В классической теории излучение определяется \ldots скоростью изменения $\vect{p}$ во времени. В волновой механике дипольный момент $\vect{p}$ легко вычисляется \ldots

				\ldots для всех стационарных состояний атома этот интеграл обращается в нуль, так что производная дипольного момента, а вместе с ней и излучение равны нулю; таким образом, в стационарных состояниях излучение отсутствует. Это объясняет \ldots что вращающийся вокруг ядра электрон может двигаться по своей орбите, не излучая \ldots .
			\end{quote}
		\end{overprint}
	\end{minipage}
\end{frame}
%============================================================================
%============================================================================
\begin{frame}[t]{Історія розвитку уявлень про будову матерії}{Електрон --- юла і магнітик}
	Отто Штерн і Вальтер Герлах --- 1922, дослід з розщепленням пучка іонів срібла, який показав, що проекція магнітного моменту електрона квантується, і може набувати лише двох значень.
	\begin{center}
		\includegraphics[width=0.45\linewidth]{Shtern-Gellax}
		\includegraphics[width=0.45\linewidth]{electronSpin}
	\end{center}
\end{frame}
%============================================================================
%============================================================================
\begin{frame}{Історія розвитку уявлень про будову матерії}{Електрони --- індивідуалісти}
	В. Паулі --- 1925, принцип виключення Паулі
	\begin{multline*}
		\psi_A(\vec{r}_1, \sigma_1; \vec{r}_2, \sigma_2; \ldots; \vec{r}_i, \sigma_i; \ldots; \vec{r}_j, \sigma_j; \vec{r}_N, \sigma_N) = \\ = - \psi_A(\vec{r}_1, \sigma_1; \vec{r}_2, \sigma_2; \ldots; \vec{r}_j, \sigma_j; \ldots; \vec{r}_i, \sigma_i; \vec{r}_N, \sigma_N)
	\end{multline*}
	\begin{minipage}[t][][b]{.2\textwidth}
		\begin{center}
			\includegraphics[width=1\linewidth]{Pauli}
		\end{center}
	\end{minipage}
	\begin{minipage}[t][][b]{.7\textwidth}
		\begin{quote}\small
			Система частинок з напівцілими спінами має описуватися хвильовою функцією, яка змінює знак при перестановці координат і спінових змінних будь-якої пари електронів.
		\end{quote}
		\begin{center}
			\includegraphics[width=0.55\linewidth]{pauli-exclusion-principle}
		\end{center}
	\end{minipage}
\end{frame}
%============================================================================
%============================================================================
\begin{frame}[t]{Історія розвитку уявлень про будову матерії}{Таємниця валентного штриха!}
	Після відкриття електрона стало можливим подальший розвиток теорії зв'язку. З'являються  Косселя (1915) і електронна теорія валентності Льюїса (1916), яка є найбільш загальною і охоплює основні типи хімічного зв'язку --- ковалентну і іонну.
	\begin{itemize}
		\item Коссель (1915) --- іонна теорія хімічного зв'язку.
		\item Льюїса (1916) --- електронна теорія валентності: хімічний зв'язок утворюється парою електронів.
	\end{itemize}

	Основним досягненням електронних теорій Косселя і Льюїса слід вважати те, що вони, аби відродити правильний електрохімічний підхід Берцеліуса до розуміння природи хімічного зв'язку, використовували все цінне, що було досягнуто завдяки <<структурній>> теорією валентності.
\end{frame}
%============================================================================

%============================================================================
\begin{frame}[t]{Історія розвитку уявлень про будову матерії}{Молекула... Ні, давайте спочатку йон!}
	1927, О.~Барооу розв'зує рівняння Шредінґера для  \href{https://en.wikipedia.org/wiki/Dihydrogen_cation}{молекулярного іона водню \ce{H_2^+}}
	\begin{center}
		\includegraphics[width=0.55\linewidth]{dihydrogen-kation}
	\end{center}
\end{frame}
%============================================================================
%============================================================================
\begin{frame}{Історія розвитку уявлень про будову матерії}{Так ось яка ти --- молекула}
	В. Гайтлер, Ф. Лондон --- 1927, квантово-механічна теорія ковалентного зв'язку (теорія молекули \ce{H2})
	\begin{equation*}
		\tcbhighmath[drop fuzzy shadow]{\Phi_\text{AB} (\overrightarrow{r_1},\overrightarrow{r_2}) = \frac{1}{\sqrt{2+2S_\text{AB}^2}} \left[ \chi_\text{A} (\overrightarrow{r_1}) \cdot \chi_\text{B} (\overrightarrow{r_2}) + \chi_\text{B} (\overrightarrow{r_1}) \cdot \chi_\text{A} (\overrightarrow{r_2}) \right]}
	\end{equation*}
	\begin{minipage}[t][][c]{.3\textwidth}
		\includegraphics[width=1\linewidth]{Heitler-London}
	\end{minipage}
	\begin{minipage}[t][][c]{.65\textwidth}
		\begin{center}
			\includegraphics[width=0.55\linewidth]{Valence-Bond}
		\end{center}
	\end{minipage}
\end{frame}
%============================================================================
%============================================================================
\begin{frame}{Історія розвитку уявлень про будову матерії}{Тепер у нас є все що необхідно! Точно?}
	Поль Дірак --- 1928, релятивістське рівняння руху вільного електрона (позитрона)
	\begin{equation*}
		\tcbhighmath[drop fuzzy shadow]{i\hbar \gamma ^{\mu }\partial _{\mu }\psi -mc\psi =0}
	\end{equation*}
	\begin{columns}[c]
		\begin{column}{.1\linewidth}
			\includegraphics[width=1.7\linewidth]{Dirac}
		\end{column}
		\begin{column}{.8\linewidth}
			\begin{quote}
				\rmfamily <<Нарешті, основні фізичні закони  необхідні для побудови математичної теорії здебільшого фізики і всієї хімії, повністю відомі, і єдина складність полягає в тому, що в результаті застосування цих законів ми приходимо до дуже складних для розв'язання рівнянь>>

				{\tiny\rmfamily Dirac P.A.M. Quantum Mechanics of Many Electron Systems // Proceedings of the Royal Society A123 (1929): 713}.
			\end{quote}
		\end{column}
	\end{columns}
\end{frame}
%============================================================================


%============================================================================
\begin{frame}[t]
	\frametitle{Історія квантової хімії}
	\framesubtitle{Що дає розв'язок рівняння квантової механіки}

	Рівняння Шредінґера (Дірака) можна записати для системи, що складається з багатьох ядер і електронів (тобто для атомів, молекул, іонів, кристалів), і його розв'язок у вигляді хвильової функції.
	\only<1>{
		\bigskip

		{\footnotesize     Знаючи \emph{хвильову функцію}, \textbf{в принципі}, можна визначити розподіл електричного заряду, розрахувати моменти молекули, обчислити її спектроскопічні і резонансні характеристики, описати її реакційну здатність, розрахувати зонну структуру кристала і тощо. Для простих систем хвильові функції можна досить точно розрахувати чисельно; для систем більш складних і представляють практичний інтерес для хімії це неможливо.

			\begin{equation*}\label{eq:Psi}
				\Psi = \Psi(\vect{r}_1, \sigma_1, \vect{r}_2, \sigma_2, \ldots, \vect{r}_N, \sigma_N; \vect{R}_1, \vect{R_2}, \ldots, \vect{R}_K),
			\end{equation*}
			де $\vect{r}_i$ та $\sigma_i$~--- просторові та спінові координати електронів, відповідно, $\vect{R}_\alpha$~--- координати ядер, $N$~--- число електронів, $K$~--- число ядер.}

		\bigskip
		Основна перешкода полягає в тому, що навіть за наявності всього двох електронів це рівняння аналітично не розв'язується, а при збільшенні їх числа складнощі зростають. Тому при розрахунках доводиться вводити різні наближення.
	}
	\only<2>{

		\bigskip
		Але є одне але \ldots \alert{А чи треба знати точну хвильову функцію?}

		\bigskip
		{\footnotesize Істотним обставиною при цьому виявляється принципова нерозрізненність всіх електронів молекули і як наслідок уявлення про хімічний зв'язок між атомами, про геометрію молекули, її симетрії і топології і багато інших <<уявлень>> про молекулу втратять сенс.

			\begin{equation*}\label{eq:Psi}
				\Psi = \Psi(\vect{r}_1, \sigma_1, \vect{r}_2, \sigma_2, \ldots, \vect{r}_N, \sigma_N; \vect{R}_1, \vect{R_2}, \ldots, \vect{R}_K).
			\end{equation*}}

		Всі уявлення про молекулу мають сенс тільки в рамках \emph{певних наближень}, які \emph{взагалі} не випливають з основних принципів (\emph{ab inito}) квантової механіки.
	}
\end{frame}
%============================================================================
%============================================================================
\begin{frame}
	\frametitle{Історія квантової хімії}
	\framesubtitle{Основні наближення квантової хімії}
	\begin{enumerate}
		\item Наближення \emph{Борна-Оппенгеймера}, засноване на ідеї окремого розгляду хвильових функцій, що описують стан електронів і ядер.

		      \medskip
		      {\footnotesize  Більш важкі ядра рухаються набагато повільніше електронів і при описі багатьох електронних процесів можуть вважатися нерухомими. В результаті математична задача визначення електронних хвильових функцій значно спрощується. Теорія хімічного зв'язку, наприклад, побудована головним чином в цьому наближенні.}

		      \bigskip
		\item Наближення\emph{ незалежних частинок} (\emph{одноелектронне наближення}), в якому замість взаємодії заданого електрона з іншими електронами і ядрами розглядають його взаємодію з електричним полем молекули або кристала, усередненим за положеннями інших частинок.

		      \medskip
		      {\footnotesize Завдяки цьому наближенню проблема розрахунку хвильових функцій для складних систем зводиться до визначення одноелектронних хвильових функцій кожного електрона в середньому полі інших частинок. Сучасні прості і потужні методи квантової хімії --- методи Хартрі-Фока і Кона-Шема --- дозволяють розрахувати саме такі функції --- атомні, молекулярні або кристалічні орбіталі.}
	\end{enumerate}

\end{frame}
%============================================================================


\begin{frame}
	\frametitle{Історія квантової хімії}
	\framesubtitle{Основні віхи в історії розвитку квантової хімії}
	\begin{itemize}
		\item 1929, Д.~Хартрі, метод самоузгодженого поля для знаходження розв'язків
		      рівняння Шредінгера для багатьох частинок.
		\item 1929,  \href{https://en.wikipedia.org/wiki/John_Lennard-Jones}{Д.~Е.~Леннард-Джонс}, ідея про поділ всіх електронів молекули на внутрішні і валентні. Роль валентних електронів --- визначають в основному хімічні властивості молекули
		\item 1929 -- 1930, В.~О.~Фок --- вдосконалення методу Хартрі (врахував симетрію хвильової функції).
		\item Д.~К.~Слейтер вводить функції для опису радіальної залежності атомних орбіталей, які отримали назву слейтерівскіх орбіталей.

	\end{itemize}
\end{frame}

\begin{frame}
	\frametitle{Історія квантової хімії}
	\framesubtitle{Основні віхи в історії розвитку квантової хімії}
	\begin{itemize}
		\item 1931, Лайнус Полінг узагальнює результати розрахунків молекули водню Гайтлера і Лондона на багатоатомні молекули, формулюючи спінову теорію валентності.
		\item 1930, Е.~Хюккель розробляє простий метод молекулярних орбіталей (МО) --- передбачення властивостей молекул, складніших за \ce{H2}
		\item 1932, \href{https://en.wikipedia.org/wiki/Robert_S._Mulliken}{Роберт Маллікен} остаточно вводить поняття молекулярної орбіталі як одноелектронної хвильової функції для опису електронів в молекулах.
		\item 1935, Берта Свірлс, застосування рівняння Дірака в квантовій хімії (метод Дірака-Фока-Брейта).
		\item 1935, Г.~Г.~Гельман вводить назву <<Квантова хімія>>~\cite{Gelman}.
		\item 1951, \href{https://en.wikipedia.org/wiki/Clemens_C._J._Roothaan}{К.~Рутаан} сформулював метод Хартрі-Фока для молекулярних систем з замкнутими оболонками
	\end{itemize}
\end{frame}

\begin{frame}
	\frametitle{Історія квантової хімії}
	\framesubtitle{Вдосконалення методів}
	\begin{itemize}
		\item  1965, Вальтер Кон (Нобелівська премія з хімії 1998) і Лу Джей Шем  --- практичний метод знаходження електронної густини і енергії системи через електронну густину, що став обчислювальною основою методу функціонала густини.
		\item 1964, П'єр Хоенберг, Вальтер Кон (Нобелівська премія з хімії 1998) ---  теорія функціонала електронної густини.
		\item Джон Попл (Нобелівська премія з хімії 1998)--- напівемпіричні квантово-хімічних методів, заснованих на наближенні нульового диференціального перекривання.
		\item 1970-ті,  перші версії програмного комплексу \texttt{GAUSSIAN}.\\ {\scriptsize Офіційний web-сайт програми \url{http://gaussian.com/}~--- програмного пакету для розрахунку структури і властивостей молекулярних систем, що включає велику різноманітність методів обчислювальної хімії, квантової хімії, молекулярного моделювання.}
	\end{itemize}
\end{frame}

\begin{frame}
	\frametitle{Історія квантової хімії}
	\framesubtitle{Можливості квантової хімії}
	\begin{quote}
		З появою високопродуктивних комп'ютерів, квантова хімія починає ставати обчислювальною.
	\end{quote}

	На сьогоднішній день квантова хімія дозволяє з високою точністю обчислювати:
	\begin{itemize}
		\item рівноважні міжядерні відстані;
		\item валентні кути;
		\item бар'єри внутрішнього обертання;
		\item енергії утворення і енергії дисоціації;
		\item частоти і ймовірності переходів;
		\item енергії активації реакцій;
		\item перерізи і константи швидкості найпростіших хімічних реакцій.
	\end{itemize}

\end{frame}

% ============================== Слайд ## ===================================
\begin{frame}{Ядра, атоми, молекули та спектри}{}

	\begin{center}
		\includegraphics[width=0.6\linewidth]{mol_types_of_motion}
	\end{center}

	\begin{block}{}\small\justifying
		Діапазони енергій і часів, які відповідають електронним відповідають \emph{електронним}, \emph{коливальним}, \emph{обертальним} і
		\emph{поступальним} рухам молекули. Кожен із цих рухів є квантовим, тобто стаціонарні стани для цих рухів є розв'язанням якогось якогось
		рівняння Шредингера.
	\end{block}



	\begin{onlyenv}<1>
		\begin{block}{}\small\justifying
			Щоб ініціювати перехід, фотон має потрапити в резонанс із потрібним рухом молекули, тому частота/довжина хвилі фотона теж характеризує
			сам рух.
		\end{block}
	\end{onlyenv}

	\begin{onlyenv}<2>
		\begin{block}{}\small\justifying
			Але не тільки ЕМ хвилі можуть викликати зміну станів: якщо до молекули підходять інші молекули, то вони можуть механічно
			віддавати/приймати енергію відповідних рухі (а для багатоатомних молекул роль інших частинок можуть виконувати інші шматки/рухи цієї ж
			самої молекули).
		\end{block}
	\end{onlyenv}

\end{frame}
% ===========================================================================

% ============================== Слайд ## ===================================
\begin{frame}{$\gamma$-Випромінювання}{}
	\begin{equation*}
		\nu \ge 10^{19}\  \text{Гц}
	\end{equation*}
	\begin{block}{}\small\justifying
		У цій області відбуваються здебільшого всілякі мегаенергетичні процеси, типу ядерних реакцій, випромінювання всяких космічних об'єктів і
		випромінювань всяких космічних об'єктів \ldots . І це один із видів радіації, тому що у всіх хімічних сполук під час взаємодії з фотонами
		такої енергії відбувається вибивання електронів з різних енергетичних рівнів (іонізація).
	\end{block}

	\begin{center}
		\includegraphics[width=0.5\linewidth]{gamma_rays}
	\end{center}
\end{frame}
% ===========================================================================

% ============================== Слайд ## ===================================
\begin{frame}{Рентгенівське випромінювання}{}
	\begin{equation*}
		10^{19} \ge \nu \ge 10^{16}\ \text{Гц}.
	\end{equation*}

	\begin{block}{}\small\justifying
		Ця область спектра належить до переходів остовних електронів. Так, наприклад, рентгенівські трубки працюють за рахунок "звалювання"
		електронів з якого-небудь 2p-рівня в атомі на 1s. Таким чином, ця область відповідає енергіям і частотам руху остовних електронів в
		атомах/молекулах. Тут електрони мають характеристичні періоди обертання навколо ядер у районі долей ангстрема. Запам'ятати довжини хвиль
		цього діапазону можна з таких міркувань: рентгенівські промені можуть дифрагувати на кристалічних решітках, отже, довжина хвилі повинна
		відповідати але довжина хвилі має бути порівнянна з довжиною міжатомних відстаней. атомних відстаней. Характеристична довжина хімічного
		зв'язку --- це $1$ \AA, що і є хорошою оцінкою для довжини хвилі рентгенівських променів, що лежать у діапазоні $10$ нм --  $10$ пм.
	\end{block}

	\begin{center}
		\includegraphics[width=0.3\linewidth]{x-rays}
	\end{center}
\end{frame}
% ===========================================================================

% ============================== Слайд ## ===================================
\begin{frame}{Ультрафіолетове та видиме випромінювання}{}
	\begin{columns}
		\begin{column}{0.5\linewidth}
			\begin{equation*}
				10^{16} \ge \nu \ge 0.77 \cdot 10^{15}\ \text{Гц}
			\end{equation*}

			\begin{equation*}
				0.77 \cdot 10^{15} \ge \nu \ge 0.43 \cdot 10^{15}\ \text{Гц}
			\end{equation*}
		\end{column}
		\begin{column}{0.5\linewidth}
			\includegraphics[width=0.75\linewidth]{xps_ups_prinzip}
		\end{column}
	\end{columns}


	\begin{block}{}\small\justifying
		Тут уже відбуваються електронні переходи для валентних електронів.
		Зокрема UV/Vis спектрометри є незамінним атрибутом будь-якої органічної лабораторії. Вони дозволяють <<бачити>>, наприклад, переходи в
		$\pi$-системах ароматичних молекул. І ми самі здатні бачити тільки в цьому вузькому шматочку всього неосяжного ЕМ-спектра, тому що процеси
		ізомеризації ретиналю в наших колбочках в очах ініціюються за рахунок електронного переходу, який розриває подвійний зв'язок \ce{C=C} в
		молекулі, а це якраз валентні електрони.
	\end{block}

	\begin{block}{}\small\justifying
		Температура фотонів цього діапазону має порядок $10^4$ К, що можна порівняти з температурою на поверхні Сонця. За наших, земних, умов, з
		температурами порядку $10^2$ К, складно збудити електронні стани термічно,
		тобто якщо молекулу спеціально не злити і не тикати високочастотними фотонами або іншими високоенергетичними фотонами, або іншими
		високоенергетичними частинками, то вона перебуватиме в основному електронному стані.
	\end{block}
\end{frame}
% ===========================================================================

% ============================== Слайд ## ===================================
\begin{frame}{Інфрачервоне випромінювання}{}
	\begin{equation*}
		0.43 \cdot 10^{15} \ge \nu \ge 0.3 \cdot 10^{12}\ \text{Гц}
	\end{equation*}
	\begin{block}{}
		Тут, в основному, вотчина всяких коливальних переходів у молекулах. У дальній же області (за низьких частот) уже можна спостерігати обертання
		малих і дуже легких молекул, типу водню.
	\end{block}
	\begin{center}
		\includegraphics[width=1\linewidth]{mol_vibrations}
	\end{center}
\end{frame}
% ===========================================================================

% ============================== Слайд ## ===================================
\begin{frame}{Мікрохвильовий діапазон}{}
	\begin{equation*}
		0.3\cdot 10^{12} \ge \nu \ge 0.3 \cdot 10^{9}\ \text{Гц}
	\end{equation*}

	\begin{block}{}
		Тут відбуваються тільки обертальні переходи вільних молекул. Чим більша, розгалуженіша, важча молекула, тим нижча частота переходу. Часи
		обертальних рухів мають порядок від пікосекунд до долей мікросекунд.
	\end{block}



	\begin{columns}
		\begin{column}{0.45\linewidth}
			\includegraphics[width=\linewidth]{microwaveoven}
		\end{column}
		\begin{column}{0.45\linewidth}
			\includegraphics[width=\linewidth]{water_rotation}
		\end{column}
	\end{columns}

\end{frame}
% ===========================================================================

% ============================== Слайд ## ===================================
\begin{frame}{Шкала електромагнітних хвиль}{}
	\begin{center}
		\includegraphics[width=\linewidth]{em_waves}
	\end{center}
\end{frame}
% ===========================================================================


% ============================== Слайд ## ===================================
\begin{frame}{Спектроскопія}{}
%\UseTblrLibrary{booktabs}
       \begin{tblr}{|Q[l, m]|X[j, m]|}
           \hline
           Тип спектроскопії                              & Переходи в молекулах.                          \\ \hline
           Інфрачервона (ІЧ)                              & Коливальні та обертальні переходи в молекулах \\
           Раманівська (Раман)                            & Обертальні та коливальні переходи в молекулах. \\
           УФ/видима (UV/Vis)                             & Електронні переходи в атомах або молекулах.    \\
           Електронний парамагнітний резонанс (EPR)       & Електронні переходи зі зміною спіна.           \\
           Ядерно-магнітний резонанс (ЯМР)                & Переорієнтація ядерних магнітних моментів.     \\
           \hline
       \end{tblr}
\end{frame}
% ===========================================================================$

%http://hyperphysics.phy-astr.gsu.edu/hbase/mod3.html
\end{document}
