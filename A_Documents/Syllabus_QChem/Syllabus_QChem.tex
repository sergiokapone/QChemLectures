% !TeX program = xelatex
% !TeX encoding = utf8
% !TeX spellcheck = uk_UA
% !BIB program = biber
\documentclass{Syllabus}
\addbibresource{../../Bibliography/QuantumChemistry.bib}

% ================================================
%                Дані  дисципліни
% ================================================
\logokaf{logokaf.png}
\creator{доцент, к.ф.-м.н., доцент Пономаренко Сергій Миколайович} % Розробник програми
\lecturer[s.ponomarenko@kpi.ua]{доцент, к.ф.-м.н., доцент Пономаренко Сергій Миколайович}
\disciplinename{Квантова хімія і квантово-механічні методи обчислення} % Назва дисципліни
\kursurl{https://do.ipo.kpi.ua/course/view.php?id=1263}
\kursurl{https://classroom.google.com/c/MjI0NzY1NTA5OTMy?cjc=ozzjpz7}
\shedule{http://ipt.kpi.ua/navchalnij-protses}
\kafname{прикладної фізики}
% -------------- Номера протоколів ---------------
\kafday{11} % Число ухвалення кафедрою
\kafmonth{червня} % місяць ухвалення кафедрою
\kafyear{2025}
\kafnum{{8}}% Номер протокола ухвалення кафедрою
%
\methodcomday{30} % Число ухвалення метод момісією
\methodcommonth{червня} % місяць ухвалення метод момісією
\methodcomyear{2025}
\methodcomnum{6} % Номер протокола ухвалення метод момісією
% ---------------- Syllabus head -----------------
\disciplinecode{ПО1}
\cycle{цикл професійної підготовки}
\level{Другий (магістерський)} % освітньо-кваліфікаційний рівень
\status{Обов'язкова}
\efield{E Природничі науки}
\speciality{E6 Прикладна фізика та наноматеріали}
\teachingprogramm{Прикладна фізика}
%-------- Семестр 2 --------------------
\course{1}
\semestr{весняний}
\controltype{екзамен}
\AudDistribHours%
{5}%
{30}%
{30}%
{}%
[]%
\MKR{1}
\DKR{1}
\RGR{}
\Referat{}
% ================================================


\def\zsrs{\textit{Завдання на СРС:}}
\def\lit{\textit{Література для опрацювання:\ }}


\begin{document}
\printhead


\syllabuschapter{Реквізити навчальної дисципліни}
\bluetableprint

\syllabuschapter{Програма навчальної дисципліни}

\section{Опис навчальної дисципліни, її мета, предмет вивчання та результати навчання}

%Викладач обґрунтовує необхідність вивчення навчальної дисципліни, відповідаючи на питання «Чому майбутньому фахівцю варто вчити саме цю дисципліну?», визначає мету, предмет дисципліни та програмні результати 3 навчання (компетентності, знання, уміння, навички, досвід, послідовність дій в стандартних виробничих ситуаціях тощо), які студент/аспірант набуде після вивчення дисципліни з розподілом на окремі освітні компоненти (якщо дисципліна вивчається декілька семестрів).

%Дисципліна <<\discipline>> грунтується на квантовій хімії, яка в свою чергу є міждисциплінарною галуззю науки і являється  фундаментом теоретичних уявлень сучасної хімії. Квантова хімія використовує засади квантової механіки для інтерпретації всіх явищ, що протікають в атомах, молекулах та твердих тілах. Методи, розроблені в квантовій хімії є універсальним і застосовується для опису будови речовин та пояснення їх властивостей і в наш час використовуються не лише в хімії, а і в прикладній науці в цілому.
%
%Стрімкий розвиток сучасних технологій швидко змінює структуру промислового виробництва і ставить перед людством задачу підготовки сучасних фахівців в галузі прикладної науки, які здатні адаптуватися і орієнтуватися в швидких змінах наукоємних високих технологіях. Вивчення дисципліни <<\discipline>> має на меті донесення знань про принципи і методи квантової хімії до фахівця в галузі прикладної фізики, що дасть йому змогу застосовувати їх при створенні речовин і матеріалів з наперед заданими властивостями та у інших наукоємних технологіях.

%Після засвоєння навчальної дисципліни студенти мають продемонструвати такі результати навчання:
%
%\begin{enumerate}
%    \item[\bfseries знання:] методів квантово-механічних обчислення структури і властивостей молекул та кристалів; наближень що використовуються при розробці цих методів; фізичної природи хімічного зв'язку.
%    \item[\bfseries уміння:] проводити базові квантово-механічні обчислення різними методами (напівемпіричні і неемпіричні методи); визначати необхідну інформацію для розрахунку електронної структури молекул і аналізувати дані розрахунків; орієнтуватися у великому обсязі літератури, що стосується квантово-механічних обчислень; застосовувати отримані знання на практиці для вирішення теоретичних і прикладних задач фізики та матеріалознавства.
%    \item[\bfseries досвід:] користування сучасними програмами квантово-механічних обчислень на персональному ком\-п'ю\-те\-рі; інтерпретації розрахунків для визначення характеристики молекул.
%\end{enumerate}

%Згідно з вимогами освітньо-наукової програми
%\footnote{\href{https://docs.google.com/document/d/1IOGn2p1Boq6-4fbsm9HKYVe9Bki8GPg-/edit}{ОСВІТНЬО-НАУКОВА ПРОГРАМА
%<<Прикладна фізика>> спеціальності 105 Прикладна фізика та наноматеріали другого (магістерського) рівня вищої освіти}}
%студенти після засвоєння навчальної дисципліни <<\discipline>> мають продемонструвати такі результати навчання:

Дисципліна «\discipline» є фундаментальною професійною складовою підготовки сучасного фахівця в галузі прикладної фізики та наноматеріалів. Вона базується на квантовій механіці та квантовій хімії, інтегруючи їх із методами комп'ютерного моделювання, що забезпечує кількісний опис електронної структури, молекулярної організації та властивостей речовини. Курс поєднує теоретичні засади побудови моделей з практичними алгоритмами чисельного розв’язання, які необхідні для аналізу та прогнозування поведінки складних фізичних і хімічних систем. Особливий акцент зроблено на обчислювальному експерименті, який у сучасній науці поряд із теорією та лабораторними дослідженнями виступає рівноправним інструментом пізнання та створення нових матеріалів і технологій.

Завдяки цьому дисципліна є підґрунтям, що дає можливість вивчати широкий спектр об’єктів --- від наноструктур і функціональних поверхонь до біомолекул та енергетичних матеріалів. Вона формує не лише глибоке розуміння природи мікроскопічних процесів, але й практичні навички застосування квантово-механічних методів для розв’язання актуальних завдань науки й технологій.

\subsection*{Загальні компетентності ОНП}

\begin{enumerate}
    \item [ЗК 1:] Здатність до абстрактного та аналітичного мислення, розуміння основних концепцій, парадигми та ідей прикладної фізики.
    \item [ЗК 2:] Здатність до навчання та самонавчання  шляхом пошуку, аналізу та конструктивного синтезу інформації з різних джерел.
    \item [ЗК 7:] Здатність ініціативно застосовувати знання в області прикладної фізики при вирішенні робочих питань, організації командної роботи, оцінці та забезпеченні якості виконуваних робіт, реалізації проектів
    \item [ЗК 8:] Здатність до кваліфікованого проведення досліджень на відповідному рівні під керівництвом фахівців, включаючи аналіз проблем, постановку цілей і завдань, вибір методів дослідження та аналіз отриманих результатів.
    \item [ЗК 9:] Здатність адаптуватися та діяти в нових ситуаціях під тиском обставин, зокрема, здатність до самостійного освоєння нових методів дослідження, зміни наукового й виробничого профілю своєї діяльності.
    \item [ЗК 10:] Здатність генерувати нові ідеї й нестандартні підходи до їх реалізації (креативність).
\end{enumerate}

\subsection*{Фахові компетентності ОНП}

\begin{enumerate}
\item [ФК 3:] Здатність застосовувати теоретичні знання для аналізу фізичних систем, явищ і процесів в галузі прикладної фізики та наноматеріалів.
\item [ФК 5:] Здатність аналізувати та обробляти результати експерименту із використанням сучасного прикладного програмного забезпечення.
\item [ФК 8:] Здатність використовувати методи і засоби математичного моделювання для опису фізичних об'єктів та процесів.
\end{enumerate}

\subsection*{Програмні результати навчання}

\begin{enumerate}
    \item [ПРН 5:] Знання основ професійно-орієнтованих дисциплін спеціальності, зокрема,високих фізичних технологій, сучасного матеріалознавства, біофізики та фізики енергетичних систем (залежно від освітньої траєкторії) на рівні, необхідному для успішної роботи в наукових колективах, що працюють в галузі прикладної фізики.
    \item [ПРН 9:] Вміння застосовувати фізичні, математичні та комп'ютерні моделі для дослідження фізичних явищ, розробки приладів, нових матеріалів і наукоємних технологій в області біофізики, енергетичних та інформаційних систем (залежно від освітньої траєкторії).
    \item [ПРН 11:] Вміння знаходити науково-технічну інформацію з різних джерел з використанням сучасних інформаційних технологій.
    \item [ПРН 12:] Вміння класифікувати, аналізувати та інтерпретувати науково-технічну, патентну, популярну інформацію в галузі прикладної фізики.
    \item [ПРН 13:] Вміння використовувати сучасні методи і технології наукової комунікації українською та іноземною мовами, вміння читати та розуміти фахові англомовні джерела.
\end{enumerate}


\section{Пререквізити та постреквізити дисципліни (місце в структурно-логічній схемі\\ навчання за відповідною освітньою програмою)}

%Зазначається перелік дисциплін, або знань та умінь, володіння якими необхідні студенту (вимоги до рівня підготовки) для успішного засвоєння дисципліни (наприклад, «базовий рівень володіння англійською мовою не нижче А2»). Вказується перелік дисциплін які базуються на результатах навчання з даної дисципліни.

Для засвоєння матеріалу курсу <<\discipline>> студенти повинні засвоїти термінологію та поняття курсів:
\begin{enumerate}
    \item Програмування;
    \item Хімія;
    \item Обчислювальні методи;
    \item Атомна фізика;
    \item Квантова механіка;
    \item Статистична фізика;
    \item Фізика твердого тіла.
\end{enumerate}

Також повинні вміти програмувати, використовувати математичний апарат: операції з матрицями,  диференціювати, інтегрувати, розв'язувати диференціальні рівняння.

Отримані практичні навички та засвоєні теоретичні знання під час вивчення навчальної дисципліни <<\discipline>> можна використовувати в подальшому в навчальних дисциплінах, пов’язаних з теоретичними та практичними аспектами прикладної фізики.

\section{Зміст навчальної дисципліни}\label{sec:zmist}

%Надається перелік розділів і тем всієї дисципліни.

\begin{Rozdil}
\itemR{Предмет, методи, гіпотези та моделі квантової хімії.} \label{R:predmet}
    \begin{Rozdil}
        \item Рівняння Шредінґера. Квантовомеханічний опис структури атома водню та його спектрів.
        \item Спін електрона. Рівняння Паулі.
        \item Історія виникнення квантової хімії та квантово-механічних розрахунків. Сучасні досягнення та перспективи розвитку галузі.
    \end{Rozdil}
\itemR{Багатоелектронні атоми.}\label{R:bagatoelekronni}
    \begin{Rozdil}
        \item Одноелектронна модель та методи її реалізації для розрахунку багатоелектронних атомів.
        \item Методи теорії збурень.
        \item Варіаційний принцип.
        \item Рівняння Хартрі та Хартрі-Фока.
        \item Програми для квантово-механічних обчислень електронної структури та властивостей багатоелектронних атомів.
    \end{Rozdil}
\itemR {Молекулярна структура.} \label{R:molstructura}
    \begin{Rozdil}
       \item Поняття молекулярної структури та наближення Борна-Оппенгеймера.
       \item Методи молекулярної динаміки.
    \end{Rozdil}
\itemR {Кавнтово-механічні методи розрахунку структури.} \label{R:kchmetody}
    \begin{Rozdil}
        \item Природа хімічного зв'язку. Метод МО ЛКАО та метод валентних схем.
        \item Одноелектронне наближення. Рівняння Хартрі-Фока.
        \item Неемпіричні (\emph{ab-initio}) методи квантово-механічних обчислень та їх точність. Рівняння Хартрі-Фока-Рутаана. Базисні набори.
        \item Врахування електронної кореляції. Пост-хартріфоківські методи.
        \item Поняття про метод функціонала електронної густини (\texttt{DFT}-method).
%        \item Напівемпіричні методи обчислень.
        \item Застосування прикладних програм для квантово-хіміччних розрахунків структури молекул.
    \end{Rozdil}
\itemR {Властивості молекулярних систем.}  \label{R:vlastyv}
    \begin{Rozdil}
        \item Електричні та магнітні властивості молекул.
        \item Спектральні характеристики молекул.
        \item Обчислення спектрів та властивостей у квантово-хімічних програмах.
    \end{Rozdil}

\end{Rozdil}

%Зазначається: базова (підручники, навчальні посібники) та додаткова (монографії, статті, документи, електронні ресурси) література, яку потрібно прочитати або використовувати для опанування дисципліни.
%
%Можна надати рекомендації та роз’яснення:
%\begin{itemize}
%\item де можна знайти зазначені матеріали (бібліотека, методичний кабінет, інтернет тощо);
%\item що з цього є обов’язковим для прочитання, а що факультативним;
%\item як саме студент/аспірант має використовувати ці матеріали (читати повністю, ознайомитись тощо);
%\end{itemize}
% зв’язок цих ресурсів з конкретними темами дисципліни.
%Бажано зазначати не більше п’яти базових джерел, які є вільно доступними, та не більше 20
%додаткових.

\section{Навчальні матеріали та ресурси}

Нижче наводиться перелік навчальних матеріалів та ресурсів для засвоєння матеріалу, розглядуваного на лекційних заняттях та для додаткового вивчення.
%Крім того, наводиться перелік online-ресурсів для підготовки та виконання розрахунково-графічної роботи.

\nocite{
YatsimirsijQChem,
Sleta,
%Yatsimirsij,
%Cirelson,
Szabo,
%Zulicke,
Levine,
BlinderHouse,
%Gelman,
AtkinsQM,
Sauer,
%Dmitriev1,
PySCF,
Gaussian,
ORCA,
ORCAInput,
Multiwfn,
ChemCraft,
CCCBDB,
BSE,
GB,
sun2018pyscf,
sun2020pyscf,
Neese,
Cao2019,
Dral2020,
McArdle2020,
Liu2010,
Haag2013,
EGMI1,
EGMI2,
EGMI3,
}

\printbibliography[category=Main, title={Основні}, heading=subbibliography]
\addtocategory{Main}{YatsimirsijQChem, Sleta}
\printbibliography[category=Additional, title={Додаткові}, heading=subbibliography]
\addtocategory{Additional}{
%Yatsimirsij,
%Cirelson,
%Szabo,
Zulicke, Levine, BlinderHouse,
%Gelman,
AtkinsQM, Sauer,
%Dmitriev1
}
\printbibliography[category=Applications, title={Програмні продукти}, heading=subbibliography]
\addtocategory{Applications}{Gaussian, ORCA, PySCF, ORCAInput, Multiwfn, ChemCraft}
\printbibliography[category=OnlineResources, title={Online-ресурси}, heading=subbibliography]
\addtocategory{OnlineResources}{CCCBDB, BSE, GB}
\printbibliography[category=SciArticles, title={Оглядові наукові статті}, heading=subbibliography]
\addtocategory{SciArticles}{sun2018pyscf, sun2020pyscf, Neese, Cao2019, Dral2020, McArdle2020, Liu2010, Haag2013, EGMI1, EGMI2, EGMI3}

%\clearpage
\syllabuschapter{Навчальний контент}

\section{Методика опанування навчальної дисципліни (освітнього компонента)}

Навчання здійснюється на основі студентоцентрованого підходу та стратегії взаємодії викладача та студента для засвоєння студентами матеріалу та розвитку у них практичних навичок. Для проведення занять застосовується практичний метод. Для лекційних занять використовуються пояснювально-ілюстративний метод та метод проблемного виконання, для проведення лабораторних робіт використовується частково-пошуковий та дослідницький методи навчання, при яких викладач ставить перед студентами проблему, і ті вирішують її самостійно або під керівництвом викладача, висуваючи ідеї, перевіряючи їх, підбираючи для цього необхідні джерела інформації, методи, підходи тощо.

%Надається інформація (за розділами, темами) про всі навчальні заняття (лекції, практичні,
%семінарські, лабораторні) та надаються рекомендації щодо їх засвоєння (наприклад, у формі
%календарного плану чи деталізованого опису кожного заняття та запланованої роботи).

\subsection*{Програмне забезпечення}

\noindent%
\begin{tblr}{
  colspec={|Q[l, m, 3cm]|X[j, h]|},
  row{1} = {c, font=\bfseries, bg=rowfill, fg=main},
  cell{2}{1} = {valign=t},
  cell{2}{2} = {valign=t},
  hlines = {1-Z}{infotablecolor},
  vlines = {1-Z}{infotablecolor},
}
Назва ПЗ & Характеристика та призначення \\
\href{https://orcaforum.kofo.mpg.de/app.php/portal}{ORCA 6.x} & Спеціалізоване програмне забезпечення для квантово-хімічних обчислень, яке використовується в курсі для моделювання молекулярних систем.\\
\href{https://www.python.org/}{Python 3.x} & Вільно розповсюджуване середовище програмування. \href{https://numpy.org/}{NumPy}-бібліотека використовується для розрахунків. \href{https://matplotlib.org/}{Matplotlib}-бібліотека --- для побудови графіків. \href{https://simpy.readthedocs.io/en/latest/}{SymPy}-бібліотека --- для символьної математики, використовується для аналітичного розв’язання рівнянь електростатики та магнетизму, перевірки формул.\\
\href{https://pyscf.org/}{PySCF} & \href{https://pyscf.org/}{PySCF} --- це безкоштовна бібліотека для Python із відкритим вихідним кодом для квантово-хімічних обчислень, яка випускається під ліценією Apache-2.0. \\
\href{http://sobereva.com/multiwfn/}{Multiwfn} & Програма для проведення аналізу електронних хвильових функцій. Вона безкоштовна, з відкритим вихідним кодом, високоефективна, дуже зручна і гнучка, вона підтримує майже всі найважливіші методи аналізу хвильових функцій.\\
\href{https://colab.research.google.com/}{Google Colab} & Хмарна платформа для запуску Python-коду без локального встановлення. Дає змогу студентам виконувати обчислення, будувати графіки, подавати звіти в інтерактивному вигляді \\
\href{https://www.latex-project.org/}{\LaTeX} & Система верстки для підготовки наукових звітів. Використовується студентами для оформлення звітів, розрахунково-графічних робіт з формулами та графіками.
\end{tblr}


\subsection*{Лекційні заняття}

\DefTblrTemplate{contfoot-text}{default}{}
\DefTblrTemplate{conthead-text}{default}{}
\DefTblrTemplate{caption}{default}{}
\DefTblrTemplate{conthead}{default}{}
\DefTblrTemplate{capcont}{default}{}
\begin{longtblr}[]{
	colspec = {Q[l, m, fg=main]X[j,m,fg=main]},
    rowhead=1,
	row{1} = {bg=rowfill, c},
	hlines = {1-Z}{infotablecolor},
	vlines = {1-Z}{infotablecolor},
	}
	№                                                                           & Назва теми лекції та перелік основних питань                                                                                                                                                                                                                                                                                                                                                                                                                                                                                                                                                   \\
%======================================================
\SetCell[c=2]{h, c}{Розділ \nref{R:predmet}}
 \\
%======================================================
\rownumber.
& \textbf{Рівняння Шредінґера. Квантовомеханічний опис структури атома водню}. Стани електрона в атомі водню та квантові числа. Пояснення спектрів.  Спін електрона. Поняття мультиплетності. Історія виникнення квантової хімії та квантово-механічних розрахунків. Обє'єкт, предмет дослідження квантової хімії. Сучасні досягнення та перспективи розвитку галузі.
\newline
\lit{}\cite[Розділ 1]{YatsimirsijQChem}, \cite[Глава I, II, IV]{Yatsimirsij}, \cite[Глава 3, \S\  3.1]{Cirelson}
\\
%======================================================
\SetCell[c=2]{h, c}{Розділ \nref{R:bagatoelekronni}}
\\*
%======================================================
\rownumber.
& \textbf{Варіаційний принцип та розв'язки рівняння Шредінгера}. Атом гелію. Ортогелій, парагелій. Застосування теорії збурень та варіаційного принципу для розрахунку енергії основного стану атома гелію. Проблема точного опису двоелектронного атому. Атомні електронні конфігурації і терми.
\newline
\lit{}\cite[Розділ 2. \S 2.2]{YatsimirsijQChem}, \cite[Глава V]{Yatsimirsij}, \cite[Глава 2]{Cirelson}
\\
\rownumber.
& \textbf{Одноелектронна модель та методи її реалізації для розрахунку багатоелектронних атомів}. Метод самоузгодженого поля (метод Хартрі) та метод Хартрі-Фока для розрахунку електронної структури багатоелектронного атому. Детермінант Слейтера.  Атомні орбіталі. Канонічні орбіталі. Орбіталі слейтерівьского типу (STO). Оболонкова модель атома та метод Кона-Шема. Трактування розв'язків одноелектронної моделі з точки зору хімії. Теорема Купманса.
\newline
\lit{}\cite[Глава VI]{Yatsimirsij},  \cite[Глава 2, \S\  2.1, 2.2, 2.3, 2.6]{Cirelson}
\\
\rownumber.
& \textbf{Програми для квантово-механічних обчислень електронної структури та властивостей багатоелектронних атомів.} Алгоритми розрахунку властивостей багатоелектронних атомів методом Хартрі-Фока.  Порівняння результатів хартрі-фоковських розрахунків атомів з експериментом.  Обмінна та кулонівська кореляції. Застосування прикладних програми для розрахунків електронної густини та характеристик багатоелектронних атомів.
\newline
\lit{}\cite[Глава 2, \S\   2.6, 2.7]{Cirelson}, \cite{Gaussian}, \cite{Multiwfn}
\newline
\zsrs{} Розрахуйте характеристики атомів 1-го  та 2-го періоду
\\
%======================================================
\SetCell[c=2]{h, c}{Розділ \nref{R:molstructura}}
\\
%======================================================
\rownumber.
&  \textbf{Силовий та енергетичний аспекти опису хімічного зв'язку}. Проблема означення поняття <<хімічний зв'язок>>. Метод валентних зв'язків як розвиток теорії Гайтлера-Лондона. Резонансні структури. Теорема Гельмана-Файнмана.  Теорема віріалу.
\newline
\lit{}\cite[Глава 11]{Sleta}, \cite[Глава 4, \S\   4.1]{Cirelson}
\\
\rownumber.
& \textbf{Пояснення природи хімічного зв'язку в рамках одноелектронної моделі}. Молекулярні орбіталі та їх класифікація. Заселеність атомних орбіталей. Геометрична будова багатоатомних молекул. Локалізація і гібридизація орбіталей. Просторовий розподіл електронної густини. Індекси вільної валентності. Дипольні та квадрупольні моменти молекул. Спорідненість до електрона, електронегативність.
\newline
\lit{}\cite[Глава 3, \S\   3.7]{Cirelson}, \cite[Глава 9]{Sleta}
\\
\rownumber.
& \textbf{Поняття молекулярної структури та наближення Борна-Оппенгеймера}. Наближення Борна-Оппенгеймера. Відокремлення електронної задачі від ядерної. Теорія Гайтлера-Лондона для молекули \ce{H2}. Порівняння результатів розрахунків \ce{H2} методом Гайтлера-Лондона з експериментом.
\newline
\lit{}\cite[Глава VIII]{Yatsimirsij}, \cite[Глава 3, \S\   3.1]{Cirelson}
\\
\rownumber.
& \textbf{Застосування одноелектронної моделі для розрахунку молекул}. Метод Хартрі-Фока. Наближення МО ЛКАО. Рівняння Рутана. Способи врахування електронної кореляції в одноелектронній моделі.  Молекулярний іон водню \ce{H_2^+}.
\newline
\lit{}\cite[Глава 3, \S\S\   3.2, 3.3]{Cirelson}, \cite[Глава 14, \S\  14.2]{Sleta}
\\
\rownumber.
& \textbf{Неемпіричні (\emph{ab-inito}) методи квантово-механічних розрахунків та їх точність}. Базисні функції. Поняття про мінімальний та розширений базиси. Орбіталь гаусового типу (GTO).  Точність неемпіричних методів розрахунку. рахування електронної кореляції. Ієрархія методів розрахунку (діаграма Поппла).
\newline
\lit{}\cite[Глава 3, \S\S\   3.5, 3.6]{Cirelson}, \zsrs{} Розрахуйте характеристики двоатомних молекул
\\
\rownumber.
& \textbf{Поняття про метод функціонала електронної густини (DFT-method)}. Застосування DFT для розрахунку молекул.
\newline
\lit{}\cite[Глава 3, \S\   3.4]{Cirelson}
\\
\rownumber.
& \textbf{Напівемпіричні методи розрахунку молекул}. Ідея нульового диференціального перекриття. Метод Хюккеля.
\newline
\lit{}\cite[Глава XI]{Yatsimirsij}, \cite[Глава 3, \S\   3.7]{Cirelson}, \cite[Глава 13, \S\S\   13.2, 13.3]{Sleta}
\\
\rownumber.
& \textbf{Застосування прикладних програм для квантово-механічних розрахунків молекул.} Застосування комплексу прикладних програм \href{https://orcaforum.kofo.mpg.de/index.php}{ORCA} та пакета \href{https://pyscf.org/quickstart.html}{PySCF}. Візуалізатори \href{https://avogadro.cc}{Avogadro}. Вибір розрахункового методу для вирішення різних типів задач. Способи задавання вихідних координат атомів та розрахунок їх характеристик. Точкові розрахунки (Single Point). Оптимізація геометрії (Geometry Optimization) та перерізи поверхонь потенціальної енергії.
\newline
\lit{}\cite{ORCA, ORCAInput, Multiwfn, ChemCraft, PySCF}
\\\pagebreak
%======================================================
\SetCell[c=2]{h, c}{Розділ \nref{R:kchmetody}}
\\
 %======================================================
\rownumber.
&  \textbf{Силовий та енергетичний аспекти опису хімічного зв'язку}. Проблема означення поняття <<хімічний зв'язок>>. Метод валентних зв'язків як розвиток теорії Гайтлера-Лондона. Резонансні структури. Теорема Гельмана-Файнмана.  Теорема віріалу.
\newline
\lit{}\cite[Глава 11]{Sleta}, \cite[Глава 4, \S\   4.1]{Cirelson}
\\
\rownumber.
& \textbf{Пояснення природи хімічного зв'язку в рамках одноелектронної моделі}. Молекулярні орбіталі та їх класифікація. Заселеність атомних орбіталей. Геометрична будова багатоатомних молекул. Локалізація і гібридизація орбіталей. Просторовий розподіл електронної густини. Індекси вільної валентності. Дипольні та квадрупольні моменти молекул. Спорідненість до електрона, електронегативність.
\newline
\lit{}\cite[Глава 3, \S\   3.7]{Cirelson}, \cite[Глава 9]{Sleta}
\\
%======================================================
\SetCell[c=2]{h, c}{Розділ \nref{R:vlastyv}} \\
%======================================================
% ------------------------------------------------------------
\rownumber.
& \textbf{Електричні та магнітні властивості.} Дипольний момент, квадрупольний момент та вищі мультипольні моменти: характеризують розподіл заряду у молекулі.  Поляризовність і гіперполяризовність: відповідають за нелінійно-оптичні ефекти та індуковані моменти.   Магнітна сприйнятливість і магнітні моменти: важливі для опису діамагнітних, парамагнітних та ферромагнітних систем.  Спінові густини та розподіл електронів: ключові для інтерпретації магнітно-резонансних експериментів.
\newline
\lit{}\cite[4.1 -- 4.4, 5.1 -- 5.4]{Sauer}, \cite[12]{AtkinsQM}
\\
% ------------------------------------------------------------
\rownumber.
& \textbf{Спектральні характеристики молекул.} Електронні спектри (поглинання, люмінесценція): пов’язані з переходами між електронними станами.  Коливальні та обертальні спектри: дають інформацію про геометрію та динаміку молекул.  Ядерний магнітний резонанс (ЯМР), електронний парамагнітний резонанс (ЕПР): дозволяють дослідити локальні електронні й магнітні середовища атомів. Інфрачервона (ІЧ) та раманівська спектроскопія: чутливі до симетрії та типу хімічних зв’язків.
\newline
\lit{}\cite[4.5 -- 4.6, 5.5 -- 5.10, 10]{Sauer}, \cite[13]{AtkinsQM}
\\
% ------------------------------------------------------------
\rownumber.
& \textbf{Обчислення спектрів та властивостей у квантово-хімічних програмах.} Розрахунки спектрів: CI, CC та TDDFT для електронних збуджень, аналіз нормальних мод для ІЧ і Рамана, квантово-хімічні параметри для ЯМР та ЕПР.
\newline
\lit{}\cite[11]{Sauer}
\\
% ------------------------------------------------------------
%%======================================================
%\SetCell[c=2]{h, c}{Розділ \nref{R:kondensovani}} \\
%%======================================================
%\rownumber.
%& \textbf{Невалентні взаємодії в молекулярних системах}. Міжмолекулярні взаємодії. Донорно-акцепторні молекулярні комплекси. Водневий зв'язок
%\newline
%\lit{}\cite[Розділ 6.б \S 6.2]{YatsimirsijQChem}, \cite[Глава 5, \S\S\   5.1  -- 5.3]{Cirelson}
%%\\
%%\rownumber.
%%& \textbf{Класифікація конденсованих систем за типами хімічного зв’язку}. Хімічний зв'язок в іонних сполуках. Ковалентний зв'язок в твердих тілах. \newline
%%\lit{}\cite[Розділ 6., \S\S 6.1, 6.2]{YatsimirsijQChem}, \cite[Глава XIV, \S\S\  1,4]{Yatsimirsij}
%%\\
%%\rownumber.
%%& \textbf{Застосування одноелектронного методу для опису фізичних властивостей твердих тіл}. Модель вільних електронів. Зонна теорія твердих тіл. Методи розрахунку хвильових функцій в кристалах. \newline
%%\lit{}\cite[Глава XIV, \S\  6]{Yatsimirsij}
\end{longtblr}

\subsection*{Практичні заняття}

\begin{center}\setcounter{magicrownumbers}{0}
\DefTblrTemplate{contfoot-text}{default}{}
\DefTblrTemplate{conthead-text}{default}{}
\DefTblrTemplate{caption}{default}{}
\DefTblrTemplate{conthead}{default}{}
\DefTblrTemplate{capcont}{default}{}
\def\ORCA{\href{https://orcaforum.kofo.mpg.de/index.php}{ORCA}}
\begin{longtblr}[]{
	colspec = {Q[l, m, fg=main]X[j,m,fg=main]},
    rowhead=1,
	row{1} = {bg=rowfill, c},
	hlines = {1-Z}{infotablecolor},
	vlines = {1-Z}{infotablecolor},
	}
        № & Назва теми заняття та перелік розглядуваних питань
        \\
        \rownumber. & Основи роботи \ORCA{} та в \texttt{Python/PySCF}. Встановлення та налаштування.
        \\
        \rownumber. & Розрахунок атома водню в \ORCA{} та в \texttt{Python/PySCF}.
        \\
        \rownumber. & Розрахунок атома гелію в \ORCA{} в \texttt{Python/PySCF}. Побудова молекулярних орбіталей на основі базису.
        \\
        \rownumber. & Розрахунок атома літію та інших багатоелектронних атомів в \ORCA{} та в в \texttt{Python/PySCF}. Побудова молекулярних орбіталей та радіального розподілу густини за допомогою  \href{http://sobereva.com/multiwfn/}{Multiwfn}.
        \\
        \rownumber. & Побудова молекулярних систем в програмі \href{https://avogadro.cc}{Avogadro}. Оптимізація структури за допомогою вбудованих методів молекулярної механіки.
        \\
        \rownumber. & Розрахунок енергії основного стану для двоатомних молекул. Оптимізація геометрії квантово-хімічними методами.
        \\
        \rownumber. & Пост-Хартрі-Фоківські методи. Розрахунки методами CI, CC та DFT.
        \\
        \rownumber. & Розрахунок спектрів та властивостей багатоатомних молекул.
        \\
        \rownumber. & Розрахунок властивостей кристалів.
    \end{longtblr}
\end{center}

\section{Самостійна робота студента}

%Зазначаються види самостійної роботи (підготовка до аудиторних занять, проведення
%розрахунків за первинними даними, отриманими на лабораторних заняттях, розв’язок задач,
%написання реферату, виконання розрахункової роботи, виконання домашньої контрольної
%роботи тощо) та терміни часу, які на це відводяться.

Самостійна робота студентів має на меті	розвиток творчих здібностей та активізація розумової діяльності студентів, а також формування у студентів потреби безперервного самостійного поповнення знань. Завдяки самостійній роботі студенти повинні навчитись самостійно працювати з літературою, творчо сприймати навчальний матеріал і осмислювати його, сформувати навички щоденної самостійної роботи з метою одержання та узагальнення знань, умінь і навичок.

Самостійна робота над засвоєнням навчального матеріалу може виконуватися у бібліотеці, ком\-п'ю\-тер\-них класах, а також у домашніх умовах. При використанні студентами програмних продуктів передбачаються можливості отримання необхідної консультації або допомоги з боку викладача.

На самостійну роботу виділено \SRSHours~години і відводяться наступні види завдань:

\begin{enumerate}[label=$\bullet$]
\item обробка і осмислення інформації, отриманої безпосередньо на лекціях, робота з відповідними підручниками та особистим конспектом лекцій (30 годин).
%\item самостійне вивчення окремих тем або питань із розробкою конспекту;
%\item робота з відповідною літературою;
\item виконання підготовчої роботи написання МКР (10 годин);
\item робота з відповідними програмними продуктами та виконання ДКР (20 годин);
\item підготовка до складання семестрового контролю (30 годин).
\end{enumerate}




\syllabuschapter{Політика та контроль}

\section{Політика навчальної дисципліни (освітнього компонента)}

%Зазначається система вимог, які викладач ставить перед студентом/аспірантом:
%\begin{itemize}\setlength\itemsep{0ex}
%\item правила відвідування занять (як лекцій, так і практичних/лабораторних);
%\item правила поведінки на заняттях (активність, підготовка коротких доповідей чи текстів,
%відключення телефонів, використання засобів зв’язку для пошуку інформації на гугл-
%диску викладача чи в інтернеті тощо);
%\item правила захисту лабораторних робіт;
%\item правила захисту індивідуальних завдань;
%\item правила призначення заохочувальних та штрафних балів;
%\item  політика дедлайнів та перескладань;
%\item політика щодо академічної доброчесності;
%\item  інші вимоги, що не суперечать законодавству України та нормативним документам
%Університету.
%\end{itemize}

\subsection*{Відвідування занять}
Відвідування лекцій, а також відсутність на них, не оцінюється. Однак, студентам рекомендується відвідувати заняття, оскільки на них викладається теоретичний матеріал та розвиваються навички, необхідні для успішного написання МКР. В разі великої кількості пропусків студент може бути недопущений до \control у.

\subsection*{Пропущені контрольні заходи}

Результат модульної контрольної роботи для студента, який не з’явився на контрольний захід, є нульовим. У такому разі, студент має можливість написати модульну контрольну роботу, але максимальний бал за неї буде дорівнювати 50~\% від загальної кількості балів. Повторне написання модульної контрольної роботи не допускається.

\subsection*{Календарний контроль}

Календарний контроль: проводиться двічі на семестр як моніторинг поточного стану виконання вимог силабусу,  базується на поточній рейтинговій оцінці. Умовою позитивної атестації є значення поточного рейтингу студента не менше 50\% від максимально можливого на час атестації. Бал, необхідний для отримання позитивного календарного контролю доноситься до студентів викладачем не пізніше ніж за 2 тижні до початку календарного контролю.

%\begin{center}
%\begin{tabular}{|l|c|c|}
%\hline
%Термін атестації                            & \thead{Перша атестація\\ 8-й тиждень}     & \thead{Друга атестація\\ 14-й тиждень}     \\\hline
%\thead{Критерій: поточний контроль}         & $\ge 10$~ балів                           & $\ge 20$~ балів \\\hline
%\end{tabular}
%\end{center}

\subsection*{Академічна доброчесність}

Політика та принципи академічної доброчесності визначені у розділі 3 Кодексу честі Національного технічного університету України «Київський політехнічний інститут імені Ігоря Сікорського». Детальніше: \url{https://kpi.ua/code}.

\subsection*{Норми етичної поведінки}

Норми етичної поведінки студентів і працівників визначені у розділі 2 Кодексу честі Національного технічного університету України «Київський політехнічний інститут імені Ігоря Сікорського». Детальніше: \url{https://kpi.ua/code}.

\subsection*{Процедура оскарження результатів контрольних заходів}

Студенти мають можливість підняти будь-яке питання, яке стосується процедури контрольних заходів та очікувати, що воно буде розглянуто згідно із наперед визначеними процедурами (згідно <<Положення про систему забезпечення якості вищої освіти у Національному технічному університеті України «Київський політехнічний інститут імені Ігоря Сікорського>>, <<Положення про організацію навчального процесу>>).

\section{Види контролю та рейтингова система оцінювання результатів навчання (РСО)}

%Вказуються всі види контролю та бали за кожен елемент контролю, наприклад:
%Поточний контроль: експрес-опитування, опитування за темою заняття, МКР, тест тощо
%Календарний контроль: провадиться двічі на семестр як моніторинг поточного стану
%виконання вимог силабусу.
%
%Семестровий контроль: екзамен / залік / захист курсового проекту (роботи)
%Умови допуску до семестрового контролю: мінімально позитивна оцінка за індивідуальне
%завдання / зарахування усіх лабораторних робіт / семестровий рейтинг більше ХХ балів.

Видами контролю успішності засвоєння матеріалу дисципліни є оцінка на лекціях під час бліц-опитувань,  модульна контрольна робота (МКР), домашня контрольна робота (ДКР) та семестровий контроль.

\subsection*{Бліц-опитування на лекційних заняттях}

\pgfmathsetmacro{\lecBal}{2}
На початку заняття проводиться бліц-опитування, за відповідь на запитання якого, студент може отримати максимум \lecBal~бали.

\subsection*{Модульна контрольна робота}

\pgfmathsetmacro{\mkrBal}{10}
МКР проводиться після завершення третього розділу курсу <<\discipline>> і проводиться протягом 1-ї академічної години. МКР являє собою тестування знань термінології, формулювань, основних положень та теоретичних підходів. Всі студенти отримують завдання з 10-ти тестових питань, повна відповідь на кожне з яких вимагає не більше 4-х хвилин.

Оцінюється за чіткими критеріями з позначенням коректної або некоректної відповіді, а також з коментарями, зауваженнями тощо.

Критерії оцінювання модульної контрольної роботи (максимум \mkrBal~балів):
\begin{enumerate}[label=$\bullet$]
    \item максимальна кількість балів за кожне теоретичне питання --- 1: повна правильна відповідь, 95\% інформації, якщо треба наведено рисунок, \item 0.5 бали --- не всі умови попереднього пункту виконано,
    \item 0 балів --- не надано правильної відповіді, розв’язок неправильний.
\end{enumerate}

\subsection*{Домашня контрольні роботи}

\pgfmathsetmacro{\rgrBal}{20}
Домашня контрольна робота виконується студентами поступово. Після кожного практичного заняття задається завдання з ДКР, яке задаються студентами до початку наступного практичного заняття. Остаточна оцінка за ДКР виставляється на останньому практичному  занятті.

Критерії оцінювання ДКР (максимум \rgrBal~балів):
\begin{enumerate}[label=$\bullet$]
    \item максимальна кількість балів ставиться у випадку, якщо наведено 95\% інформації, там де треба наведено рисунки, позначення, є письмовий коментар щодо базових понять, методів розрахунку, які використовуються під час виконання роботи,
    \item 75\% балів --- виконання правильне, не всі умови попереднього пункту виконано,
    \item 60\% балів --- наведено основні розрахунки, неправильні методи розрахунку.
    \item Не зараховуються --- студент не виконав роботу, або не може її пояснити.
\end{enumerate}

\subsection*{Умови допуску до \control у}

В таблиці наведені умови допуску до семестрового контролю.

\begin{center}\setcounter{magicrownumbers}{0}
\begin{tabular}{|c|l|l|}
	\hline
	 \thead{№}   &  \thead{Обов’язкова умова допуску до \control у}  & \thead{Критерій} \\\hline
	 \rownumber      &  Потчний рейтинговий бал                          & $\ge 40$         \\ \hline
	 \rownumber      &  МКР                                              & виконана         \\ \hline
	 \rownumber      &  ДКР                                          & здана            \\ \hline
\end{tabular}%
\end{center}

Додаткові умови допуску до \control у, які заохочуються:
\begin{enumerate}[label=$\bullet$]
    \item Залучення при виконанні домашньої контрольної роботи нових програмних засобів та застосунків для візуалізації результатів обрахунків, оптимізації обрахунків, використання оригінальних методик (додаються заохочувальні бали).
    \item Активна самостійна робота над теоретичним матеріалом: пошук та використання інформаційних ресурсів, ілюстрацій, відео, медіа ресурсів, що доповнюють поточний курс (додаються заохочувальні бали).
    \item Позитивний результат першої та другої атестації.
\end{enumerate}


\subsection*{Семестровий контроль (\control)}

\pgfmathsetmacro{\kontrolBal}{40}
Питання, що виносяться на \control{} складаються із 2-х теоретичних питань, за кожне з яких дається максимум \pgfmathparse{int(round(\kontrolBal/2))}\pgfmathresult~балів.

Критерії оцінювання:
\begin{enumerate}[label=$\bullet$]
    \item максимальна кількість балів – 95\% інформації, повна правильна відповідь, там де треба наведено рисунки, позначення, є письмовий коментар щодо базових понять та наведені основні формули, що повністю розкривають зміст питання.
    \item 75\% балів --- питання розкрито з незначними неточностями, не всі умови попереднього пункту виконано,
    \item 60\% балів --- питання розкрито з суттєвими неточностями.
    \item списані відповіді, незнання обов'язкових формул та співвідношень що розкривають зміст питання.
\end{enumerate}

Остаточна оцінка $R$ є сумою рейтингових балів отриманих за поточний контроль та балів отриманих на \control\ і після співбесіди зі студентом.

\begin{center}\setcounter{magicrownumbers}{0}
    \begin{spreadtab}{{tabular}{|c|l|c|c||c|}}
    \hline
    @ \thead{№}   & @ \thead{Контрольний захід}                & @ \thead{Бал}  & @ \thead{Кількість}  & @ \thead{Всього}   \\ \hline
    @ \rownumber               & @ Активність на лекційних заняттях   & \lecBal        & 15                   & c2*d2           \\ \hline
    @ \rownumber               & @ Модульна контрольна робота         & \mkrBal        & 1                    & c3*d3           \\ \hline
    @ \rownumber               & @ Реферат                            & \rgrBal        & 1                    & c4*d4           \\ \hline
    @ \rownumber               & @ \Control                  & \kontrolBal    & 1                    & c5*d5           \\ \hline\hline
    @                          & @ \multicolumn{3}{l||}{Всього, $R$ }                      & sum(e1:[0,-1])  \\  \hline
    \end{spreadtab}
\end{center}


%\pgfmathsetmacro{\lectRSO}{10}
%\pgfmathsetmacro{\mkrRSO}{20}
%\pgfmathsetmacro{\refRSO}{30}
%\pgfmathsetmacro{\ekzRSO}{40}
%\begin{enumerate}
%	\item Робота протягом семестру:
%	      \begin{enumerate}[label=\alph*)]
%	      	\item Відвідування всіх лекцій, активність на лекціях. Максимальна кількість балів~--- \lectRSO.
%	      	\item МКР. Максимальна кількість балів~--- \mkrRSO.
%	      	\item Індивідуальне завдання. Максимальна кількість балів~--- \refRSO.
%	      \end{enumerate}
%	\item Екзамен:
%	      максимальна кількість балів~--- \ekzRSO.
%	      \begin{enumerate}[label=\alph*)]
%	      	\item Вичерпна відповідь $35-40$ балів;
%	      	\item Відповідь з незначними неточностями $25-35$ балів;
%	      	\item Правильні формулювання з відсутністю доведень		$10-25$ балів;
%	      	\item Грубі помилки при формулюванні з відсутністю доведень $5-10$ балів;
%	      	\item Незнання обов'язкових формул та співвідношень чи постановки основних задач $0$ балів.
%	      \end{enumerate}
%	\item Максимальний сумарний рейтинг складає:\pgfmathsetmacro{\RDRSO}{\lectRSO + \mkrRSO + \refRSO + \ekzRSO}
%	      \[\mathbf{RD} = \lectRSO + \mkrRSO + \refRSO + \ekzRSO  = \RDRSO \text{ балів}.\]
%\end{enumerate}
%
%Календарний контроль: проводиться двічі на семестр як моніторинг поточного стану виконання вимог силабусу.


Таблиця відповідності рейтингових балів оцінкам за університетською шкалою.
%\footnote{Оцінювання результатів навчання здійснюється за рейтинговою системою оцінювання відповідно до рекомендацій Методичної ради КПІ ім. Ігоря Сікорського, ухвалених протоколом №7 від 29.03.2018 року.}.

\begin{center}
\begin{tblr}{
colspec={|c|c|c|},
hlines,
vlines
}
    \hline
           Значення рейтингу      & Оцінка ECTS  \\
            $95 \le R \le 100$         &      відмінно        \\
             $85 \le R < 95$           &     дуже добре       \\
             $75 \le R < 85$           &        добре         \\
             $65 \le R < 75$           &     задовільно       \\
             $60 \le R < 65$           &    достатньо        \\
        	 $ R < 60$                 &    незадовільно     \\
            Не здані ДКР               &     не допущено     \\
\end{tblr}%
\end{center}

%\section{Додаткова інформація з дисципліни (освітнього компонента)}

%\begin{itemize}
%\item перелік питань, які виносяться на семестровий контроль (наприклад, як додаток до силабусу);
%\item  можливість зарахування сертифікатів проходження дистанційних чи онлайн курсів за відповідною тематикою;
%\item  інша інформація для студентів/аспірантів щодо особливостей опанування навчальної дисципліни.
%\end{itemize}

%\subsection*{Приблизний перелік тем рефератів}
%
%Виконання реферату, передбаченого навчальним планом, і мають на меті набуття студентами уміння та навичок опанувати та застосовувати застосовувати програмні пакети для здійснення квантово-механічних обчислень.
%
%\begin{enumerate}
%    \item Сучасний стан методів самоузгодженого поля зі скінченним базисними наборами.
%    \item Розрахунок збуджених станів легких атомів.
%	\item Розрахунок структури та властивості молекул \ce{H2O}, \ce{CH4}, \ce{CO2}.
%    \item Сучасні досягнення квантово-механічних методів розрахунку кристалів.
%    \item Досягнення в релятивістській квантовій хімії.
%    \item Сучасні програми для квантово-хімічних розрахунків: порівняльний аналіз можливостей.
%    \item Застосування квантово-хімічних методів до розрахунку міжмолекулярних взаємодій.
%    \item Квантово-хімічне моделювання поведінки молекул речовини під впливом високих тисків.
%\end{enumerate}

%\begin{enumerate}
%	\item Розрахуйте та опишіть структуру та властивості молекул \ce{H2O}, \ce{CH4}, \ce{CO2}.
%    \item Дослідіть залежність повної енергії молекули гідразину \ce{N2H4} від кута внутрішнього обертання і вкажіть для неї найбільш стійку конфігурацію.
%    \item Обчисліть величину переносу заряду від атома азоту до молекули \ce{BF3}. Яка частина заряду локалізована на атомах бору і фтору?
%    \item Обчисліть силову константу та енергію нульових коливань для молекули \ce{HI}.
%    \item Побудуйте графік потенціальної функції $U(\varphi)$ у молекулі \href{https://en.wikipedia.org/wiki/1,2-Dibromoethane}{1,2-диброметану}:
%    \begin{center}
%        \chemfig{H-C(-[:90]H)(-[:-90]Br)-C(-[:90]Br)(-[:-90]H)-H}
%    \end{center}
%\end{enumerate}

%\subsection*{Перелік питань, які виносяться на \control}

%\loaditemsfromfile[enumerate]{Qchem.dat}
%
%\thispagestyle{empty}

\vfill
\printrequisites
\end{document}



