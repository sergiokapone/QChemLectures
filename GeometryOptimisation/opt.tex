%%============================ Compiler Directives =======================%%
%%                                                                        %%
% !TeX program = pdflatex							    	
% !TeX encoding = utf8
% !TeX spellcheck = uk_UA
%%                                                                        %%
%%============================== Клас документа ==========================%%
%%                                                                        %%
\documentclass[]{article}
\usepackage[%
paperheight=15in,
paperwidth=15in,
margin=1in,
heightrounded,
%showframe
]{geometry}   

\usepackage{tikz}
\makeatletter
  \def\xwhilenum #1\do{\@whilesw{\ifnum #1}\fi }
\makeatother
\newcounter{it}

\usepackage{multicol}


\usepackage[%colorlinks=true,
	%urlcolor = blue, %Colour for external hyperlinks
	%linkcolor  = malina, %Colour of internal links
	%citecolor  = green, %Colour of citations
	bookmarks = false,
	bookmarksnumbered=true,
	unicode,
	linktoc = all,
	hypertexnames=false,
	pdftoolbar=false,
	pdfpagelayout=SinglePage,
	pdfauthor={Ponomarenko S.M. aka sergiokapone},
	pdfdisplaydoctitle=true,
	pdfencoding=auto
	]%
	{hyperref}
		\makeatletter
	\AtBeginDocument{
	\hypersetup{
		pdfinfo={
		Title={\@title},
		}
	}
	}
	\makeatother


%\ExplSyntaxOn
%\NewDocumentCommand\looping{ m m }{
%\int_set:Nn \l_tmpa_int {1}
% \int_do_while:nn {\l_tmpa_int < #1}
%{
%    #2
%    \int_incr:N \l_tmpa_int
%}
%}
%\ExplSyntaxOff

\begin{document}
\pagestyle{empty}

%\begin{tabular}[t]{c|c}
%R & E \\ \hline

%\end{tabular}

\foreach \i[count = \ci] in {0.3,0.4,...,3.2} {
\clearpage
\begin{center}\Huge\bfseries
Optimization chemical bond length
\end{center}

%---------------------------------------------------------
\begin{tikzpicture}[remember picture, overlay]
\node[anchor=north west] at ([xshift=1in, yshift=-1.5in]current page.north west) {%
\begin{tabular}{l|l}
  $R$ & $E$ \\\hline
  \setcounter{it}{0}%
  \xwhilenum{\ci>\value{it}}\do {%
    \stepcounter{it}%
     $R_{\theit}$ & $E_{\theit}$ \\
  }%
\hline
\end{tabular}    
};
\end{tikzpicture}%
%---------------------------------------------------------

\begin{center}
    \begin{tikzpicture}[scale=3]
    \draw[-latex] (-0.3,0) -- ++(3.8,0) node[below] {$R$};
    \draw[-latex] (0,-2.5) -- ++(0,4) node[left] {$E$};
    \draw[domain=0.3:3.2, smooth, variable=\x, red, samples = 100]  plot ({\x}, {-3/\x + 1/(\x*\x)});
    \node[circle, fill=red, inner sep= 2pt] (A) at ({\i}, {-3/\i + 1/(\i*\i)}) {};
    \draw[dashed] (A) -| node[left]  {$E_{\ci}$} (0,0) ;
    \draw[dashed] (A) |- node[above] {$R_{\ci}$} (0,0) ;
    \end{tikzpicture}%    
\end{center}
%---------------------------------------------------------




\vfill

\begin{center}
\begin{tikzpicture}[scale=3]
\path[inner color=red, outer color=white, opacity=1] (-\i/2,0) circle(3);
\path[inner color=blue, outer color=white, opacity=0.5] (\i/2,0) circle(3);
\draw[ball color=red] (-\i/2,0) circle (0.1) node [color=white] {H};
\draw[ball color=blue] (\i/2,0) circle (0.1) node [color=white] {H};
\draw (-\i/2,-0.2) --  (-\i/2,-0.4) (\i/2,-0.2) --  (\i/2,-0.4) ;
\draw [latex-latex] (-\i/2,-0.3) -- node [below] {$R$}(\i/2,-0.3);
\end{tikzpicture}
\end{center}
}
\end{document}
