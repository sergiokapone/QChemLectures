% !TeX program = lualatex
% !TeX encoding = utf8
% !TeX spellcheck = uk_UA
% !BIB program = biber

\documentclass[]{beamer}
\usetheme{QuantumChemistry}
\usepackage{QuantumChemistry}
\usepackage{multirow}
\usepackage{makecell}
\usepackage{tabularray}


\graphicspath{{pictures/}}

\addbibresource{d:/Projects/LaTeX/QChem/Textbook2/QChemBook.bib}
\AtEveryCitekey{\clearfield{url}\clearfield{doi}}
\usetikzlibrary{chains, positioning}
\usepackage{ragged2e}
\usepackage{ifthen}

\newcommand\ddfrac[2]{\frac{\displaystyle #1}{\displaystyle #2}}



\newcommand\vertarrowbox[3][6ex]{%
  \begin{array}[t]{@{}c@{}} #2
  \left\uparrow\vcenter{\hrule height #1}\right.\kern-\nulldelimiterspace\\
  \makebox[0pt]{\scriptsize#3}
  \end{array}%
}
\usepackage{pdfpages}
\title[Лекції з квантової хімії]{\huge\bfseries Наближені методи квантової механіки}
\subtitle{Лекції з квантової хімії}
\author{Пономаренко С. М.}
\date{}
\let\vphi\varphi
\def\vxi{\vec{\xi}}
\begin{document}
%============================================================================
\usebackgroundtemplate{
	\tikz\node[opacity=0.15]{\includegraphics[width=\paperwidth,height=\paperheight]{background}};%
}
\begin{frame}
	\titlepage
\end{frame}
%============================================================================

% ============================== Слайд ## ===================================
\begin{frame}{Навіщо наближені методи?}{}
	\begin{alertblock}{}\justifying
		У квантовій механіці часто неможливо знайти точний розв'язок рівняння Шредінгера для складних систем. Тому використовують наближені методи. Серед основних методів --- стаціонарна теорія збурень та варіаційні методи.
	\end{alertblock}
\end{frame}
% ===========================================================================

\section{Стаціонарна теорія збурень}

% ============================== Слайд ## ===================================
\begin{frame}[t]{Стаціонарна теорія збурень}{}
	\begin{equation*}
		\tcbhighmath{\hat{H}\psi_n = E_n\psi_n} \qquad \hat{H} = \tikzmarknode[minimum height=1em, blue]{H0}{\hat{H}^{(0)}} + \tikzmarknode[minimum height=1.2em, red]{V}{\hat{V}} \qquad  \tcbhighmath{\hat{H}^0\psi_n^{(0)} = E^{(0)}_n\psi_n^{(0)}}
		\tikz[remember picture, overlay, >=latex]{
			\draw[blue, ->] (H0.south) -- ++(-0.3,-.3) node[below, font=\scriptsize, text width= 1.5cm, align=center, text=blue] {нульове\\ наближення};
			\draw[red, ->] (V.south) -- ++(+.3,-.3) node[below, font=\scriptsize, text width= 1.5cm, align=center, text=red] {оператор збурення};
		}
	\end{equation*}

	\vspace*{2em}

	\begin{overprint}\small
		\onslide<1>
		\begin{block}{}\justifying\small

			Оператором збурення може бути частина оператора Гамільтона, яка не враховувалася в ідеалізованій задачі, або потенціальна енергія зовнішнього впливу (поля).

		\end{block}

		\begin{alertblock}{}\justifying\small

			Завданням теорії збурень є відшукання формул, що визначають енергію і хвильові функції стаціонарних станів через відомі значення енергій $E^{(0)}_n$ і хвильових функцій $\psi_n^{(0)}$ <<незбуреної>> системи, що описується гамільтоніаном $\hat{H}^{(0)}$.

		\end{alertblock}
		\onslide<2>
		\begin{block}{}
			\textcolor{blue}{Перше} наближення теорії збурень:
			\begin{equation*}
				E_n = E_n^{(0)} + \bracket<\psi_n^{(0)}|\hat{V}|\psi_n^{(0)}>, \quad
				\psi_n = \psi_n^{(0)} + \sum\limits_{\substack{m\\m \neq n}} \frac{\bracket<\psi_n^{(0)}|\hat{V}|\psi_m^{(0)}>}{E_n^{(0)} - E_m^{(0)}} \psi_m^{(0)}.
			\end{equation*}

			\textcolor{blue}{Друге} наближення теорії збурень:
			\(
			E_n = E_n^{(0)} + V_{nn}  + \sum\limits_{\stackrel{m}{m \neq n}}\frac{|V_{nm}|^2}{E^{(0)}_n - E^{(0)}_m} \quad
			\)

		\end{block}
		\begin{block}{}\justifying\scriptsize
			Умову застосовності теорії збурень:
			\(
			\bracket<\psi_n^{(0)}|\hat{V}|\psi_n^{(0)}> = V_{nn} \ll |E_n^{(0)} - E_m^{(0)}|,\ \text{для}\ n \neq m.
			\)
		\end{block}
	\end{overprint}
\end{frame}
% ===========================================================================

\section{Варіаційний метод}

% ============================== Слайд ## ===================================
\begin{frame}{Варіаційний принцип}{}
	\begin{block}{}\justifying
		Суть методу полягає у тому, щоб \alert{підібрати хвильову функцію} так, щоб енергія системи була мінімальною. Чим краще підібрана функція, тим ближчий результат до істинної енергії основного стану:

    	\begin{equation*}
			E[\psi]  = \bracket<{\psi}|{\hat{H}}|{\psi}> = \int \psi^* \hat{H} \psi d\xi, \quad \int \psi^*  \psi d\xi = 1.
		\end{equation*}

        \bigskip

		Метод заснований на принципі, що \alert{енергія, обчислена з будь-якою пробною хвильовою функцією, завжди буде не нижчою за істинну енергію}:

        \begin{equation*}
            E_{\min}[\psi_\text{trial}] \geqslant E_\text{true}
        \end{equation*}

        \bigskip

		Це дає можливість тестувати різні функції, змінюючи їхні параметри, щоб знайти найкраще наближення.
	\end{block}
\end{frame}
% ===========================================================================


% ============================== Слайд ## ===================================
\begin{frame}[t]{Варіаційний принцип і рівняння Шредінгера}{}
        Припустимо, що ми знаємо \alert{точну хвильову функцію} \(\psi\)
	\begin{block}{}\justifying


		\textbf{Як врахувати нормування:} вводимо множник Лагранжа $\lambda$:
		\[
			F[\psi] = \bracket<\psi|\hat{H}|\psi> - \lambda (\bracket<\psi|\psi>-1)
		\]

		\bigskip
		\textbf{Мінімізуємо функціонал:}
		\[
			\delta F[\psi] = 0 \quad \Rightarrow \quad (\hat{H} - \lambda)\psi = 0
		\]
	\end{block}

\begin{alertblock}{}
    	Якщо ми знаємо точну функцію $\psi$ $\rightarrow$ отримуємо рівняння Шредінґера.

        \medskip

		\begin{center}
			\alert{Зазвичай ми не знаємо точну функцію!}
		\end{center}
	\end{alertblock}
\end{frame}
% ===========================================================================

\subsection{Метод Рітца}

%============================================================================
\begin{frame}[t]{Варіаційний метод Рітца}
	Точна хвильова функція $\phi$ в загальному випадку \emphs{апроксимується} у вигляді \emphs{обмеженої лінійної комбінації} лінійно незалежних функцій (\emphs{які можуть бути і не ортонормованими}): $\chi_1$, $\chi_2$, $\ldots$ , $\chi_n$:
	\begin{equation}\label{eq:basis}
		\phi = \sum_{j = 1}^n c_j \chi_j,
	\end{equation}
	\begin{overprint}
		\onslide<1>
		коефіцієнти $c_j$ є параметрами, які визначаються шляхом мінімізації варіаційного інтеграла:
		\begin{equation}\label{key}
			E = \ddfrac{\int \phi^* \hat{H} \phi\, d\tau}{\int \phi^* \phi\, d\tau}.
		\end{equation}
		\onslide<2>
		\begin{multline*}\label{}
			\int \phi^* \hat{H} \phi\, d\tau = \int \sum_{j = 1}^n c_j^* \chi_j^* \hat{H} \sum_{k = 1}^n c_k \chi_k = \\ = \sum_{j = 1}^n c_j^* c_k \sum_{k = 1}^n \int \chi_j^* \hat{H} \chi_k\, d\tau = \sum_{j = 1}^n \sum_{k = 1}^n c_j^* c_k H_{jk}.
		\end{multline*}
		\onslide<3>
		\begin{multline*}\label{}
			\int \phi^*  \phi\, d\tau = \int \sum_{j = 1}^n c_j^* \chi_j^*  \sum_{k = 1}^n c_k \chi_k = \\ = \sum_{j = 1}^n c_j^* c_k \sum_{k = 1}^n \int \chi_j^*  \chi_k\, d\tau = \sum_{j = 1}^n \sum_{k = 1}^n c_j^* c_k S_{jk}.
		\end{multline*}
		де $S_{jk} = \int \chi_j^*  \chi_k\, d\tau $ --- інтеграли перекривання.
		\onslide<4>
		Інтеграл $W$
		\begin{equation}\label{}
			E = \ddfrac{\sum_{j = 1}^n \sum_{k = 1}^n c_j^* c_k H_{jk}}{\sum_{j = 1}^n \sum_{k = 1}^n c_j^* c_k S_{jk}}.
		\end{equation}
		є функцією від $n$ незалежних змінних $c_1$, $c_2$, $\ldots$, $c_n$:
		\begin{equation*}\label{}
			E = E(c_1, c_2, \ldots, c_n).
		\end{equation*}
		\onslide<5-7>
		\begin{onlyenv}<5>
			Умовою мінімуму у функції $E = E(c_1, c_2, \ldots, c_n)$ є:
			\begin{equation*}\label{}
				\frac{\partial E}{\partial c_i^*} = 0, \quad i = 1,2, \ldots, n,
			\end{equation*}

		\end{onlyenv}
		\begin{onlyenv}<5-6>
			Яка приводить до рівнянь, розв'язками яких є коефіцієнти $c_i$:
			\begin{equation}\label{eq:eq_c}
				\sum_{k = 1}^n (H_{ik} - S_{ik} E) c_k = 0, \quad i = 1,2, \ldots, n.
			\end{equation}
		\end{onlyenv}
		\noindent\only<6>{Отримана система лінійних однорідних рівнянь має \emphs{нетривіальні розв'язки} тільки тоді, коли її \emphs{детермінант дорівнює нулю}:}
		\begin{onlyenv}<6-7>
			\begin{equation}\label{}
				\mathrm{det}(H_{ij} - S_{ij}W) = 0.
			\end{equation}
		\end{onlyenv}
		\begin{onlyenv}<7>
			Розв'язком цього характеристичного рівняння знаходять $n$ коренів $W_1$, $W_2$, $\ldots$, $W_n$:
			\begin{equation*}
				E_1 \le E_2 \le \ldots \le E_n.
			\end{equation*}
			Найменше значення $W_1$ є оцінкою зверзу енергії основного стану, решта коренів в рамках варіаційного методу Рітца є оцінками зверху для енергії відповідних \emphs{збуджених станів}.

			%			Для знаходження хвильової функції основного стану необхідно найменший корінь рівняння підставити в систему рівнянь~\eqref{eq:eq_c} і знайти коефіцієнти $c_i$. Аналогічно знаходять хвильові функції збуджених станів.
		\end{onlyenv}
	\end{overprint}
\end{frame}
%============================================================================

\section{Метод Самоузгодженого поля}

\end{document}
