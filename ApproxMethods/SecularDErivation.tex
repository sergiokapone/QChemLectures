\documentclass{article}

\usepackage[fontsize=14pt]{fontsize}
\usepackage{fontsetup}
\usepackage[english, russian, ukrainian]{babel}

\usepackage[%
	a4paper,%
	footskip=1cm,%
	headsep=0.3cm,%
	top=2cm, %поле сверху
	bottom=2cm, %поле снизу
	left=2cm, %поле ліворуч
	right=2cm, %поле праворуч
    ]{geometry}

\usepackage{amsmath}


\usepackage[%colorlinks=true,
	%urlcolor = blue, %Colour for external hyperlinks
	%linkcolor  = malina, %Colour of internal links
	%citecolor  = green, %Colour of citations
	bookmarks = true,
	bookmarksnumbered=true,
	unicode,
	linktoc = all,
	hypertexnames=false,
	pdftoolbar=false,
	pdfpagelayout=TwoPageRight,
	pdfauthor={Ponomarenko S.M. aka sergiokapone},
	pdfdisplaydoctitle=true,
	pdfencoding=auto
	]%
	{hyperref}

\title{Виведення секулярного рівняння}
\date{}
\begin{document}
\maketitle

Точна хвильова функція $\phi$ апроксимується лінійною комбінацією базисних функцій:
\begin{equation}
\phi = \sum_{j=1}^n c_j \chi_j,
\end{equation}
де коефіцієнти $c_j$ є параметрами, які визначаються мінімізацією варіаційного інтегралу
\begin{equation}
E[c_1,\dots,c_n] = \frac{\langle \phi | \hat H | \phi \rangle}{\langle \phi | \phi \rangle}.
\end{equation}

Підставляючи $\phi = \sum_j c_j \chi_j$, отримуємо
\begin{align}
\langle \phi | \hat H | \phi \rangle &= \sum_{j=1}^n \sum_{k=1}^n c_j^* c_k H_{jk},
\quad H_{jk} = \int \chi_j^* \hat H \chi_k \, d\tau, \\
\langle \phi | \phi \rangle &= \sum_{j=1}^n \sum_{k=1}^n c_j^* c_k S_{jk},
\quad S_{jk} = \int \chi_j^* \chi_k \, d\tau.
\end{align}

Тоді енергія як функція коефіцієнтів:
\begin{equation}
E(c_1,\dots,c_n) = \frac{\sum_{j,k=1}^n c_j^* c_k H_{jk}}{\sum_{j,k=1}^n c_j^* c_k S_{jk}}.
\end{equation}

Мінімізація по $c_i^*$:
\begin{equation}
\frac{\partial E}{\partial c_i^*} = 0, \quad i=1,\dots,n,
\end{equation}

Позначимо
\[
N = \sum_{jk} c_j^* c_k H_{jk}, \quad D = \sum_{jk} c_j^* c_k S_{jk}, \quad E = \frac{N}{D}.
\]

Беремо похідну по $c_i^*$:
\begin{equation}
\frac{\partial E}{\partial c_i^*} = \frac{\partial}{\partial c_i^*}\frac{N}{D}
= \frac{\frac{\partial N}{\partial c_i^*} D - N \frac{\partial D}{\partial c_i^*}}{D^2}.
\end{equation}

Оскільки
\[
\frac{\partial N}{\partial c_i^*} = \sum_k H_{ik} c_k, \quad
\frac{\partial D}{\partial c_i^*} = \sum_k S_{ik} c_k,
\]
умова мінімуму $\frac{\partial E}{\partial c_i^*} = 0$ дає
\begin{equation}
\left(\sum_k H_{ik} c_k \right) D - N \left(\sum_k S_{ik} c_k \right) = 0.
\end{equation}

Поділивши на $D \neq 0$ та підставивши $E = N/D$, отримуємо
\begin{equation}
\sum_{k=1}^n (H_{ik} - E S_{ik}) c_k = 0, \quad i=1,\dots,n.
\end{equation}

Нетривіальний розв'язок для $\mathbf{c}=(c_1,\dots,c_n)$ існує лише якщо
\begin{equation}
\det(H - E S) = 0,
\end{equation}
що і є \emph{секулярним рівнянням}.

%дає систему лінійних рівнянь:
%\begin{equation}
%\sum_{k=1}^n (H_{ik} - E S_{ik}) c_k = 0, \quad i=1,\dots,n.
%\label{eq:secular}
%\end{equation}
%
%Це і є \emph{секулярне рівняння}. Воно визначає допустимі значення енергії $E$, для яких система \eqref{eq:secular} має нетривіальний розв'язок. Для знаходження $E$ треба розв'язати визначник:
%\begin{equation}
%\det(H - E S) = 0.
%\end{equation}
%
%Власні значення $E$ відповідають енергіям, а власні вектори $(c_1,\dots,c_n)$ — коефіцієнтам для хвильової функції у вибраному базисі.

\subsection*{Приклад: двофункціональний базис для гелію}

Нехай хвильова функція електронів гелію апроксимується лінійною комбінацією двох 1s-функцій з різними екрануючими параметрами $\zeta_1$ і $\zeta_2$:
\begin{equation}
\phi = c_1 \chi_1 + c_2 \chi_2, \quad
\chi_j = \left(\frac{\zeta_j^3}{\pi}\right)^{1/2} e^{-\zeta_j r}, \quad j=1,2.
\end{equation}

Тоді матриці Гамільтоніана та перекривання мають вигляд:
\begin{equation}
H =
\begin{pmatrix}
H_{11} & H_{12} \\
H_{21} & H_{22}
\end{pmatrix}, \quad
S =
\begin{pmatrix}
1 & S_{12} \\
S_{12} & 1
\end{pmatrix},
\end{equation}
де
\[
H_{jk} = \int \chi_j^* \hat H \chi_k \, d\tau, \quad
S_{12} = \int \chi_1^* \chi_2 \, d\tau.
\]

Секулярне рівняння:
\begin{equation}
\det(H - E S) = 0 \quad \Rightarrow \quad
\det
\begin{pmatrix}
H_{11}-E & H_{12}-E S_{12} \\
H_{12}-E S_{12} & H_{22}-E
\end{pmatrix} = 0.
\end{equation}

Розкриваємо визначник:
\begin{equation}
(H_{11}-E)(H_{22}-E) - (H_{12}-E S_{12})^2 = 0.
\end{equation}

Це квадратне рівняння для $E$, яке легко розв'язати:
\begin{equation}
E = \frac{1}{2} \left[ H_{11}+H_{22} \pm \sqrt{(H_{11}-H_{22})^2 + 4(H_{12}-E S_{12})^2} \right].
\end{equation}

%Після знаходження $E$ підставляємо у систему
%\begin{equation}
%(H - E S) \begin{pmatrix} c_1 \\ c_2 \end{pmatrix} = 0
%\end{equation}
%і знаходимо **коефіцієнти $c_1, c_2$** для власної функції хвильової комбінації.

%\textbf{Примітка:} В реальних обчисленнях для гелію $H_{jk}$ включає як одноелектронні енергії, так і кулонівські інтеграли між електронами, а $S_{12} = \langle \chi_1 | \chi_2 \rangle$ — це перекривання базисних орбіталей.

\subsection*{Числовий приклад: гелій з двома 1s-базисами}

Візьмемо два 1s-орбіталі з різними екрануючими параметрами:
\[
\zeta_1 = 1.7, \quad \zeta_2 = 2.0,
\quad \chi_j = \left(\frac{\zeta_j^3}{\pi}\right)^{1/2} e^{-\zeta_j r}.
\]

Припустимо, що інтеграли Гамільтоніана та перекривання (обчислені аналітично або чисельно) мають значення:
\[
H = \begin{pmatrix} -2.85 & -2.65 \\ -2.65 & -3.00 \end{pmatrix}, \quad
S = \begin{pmatrix} 1 & 0.85 \\ 0.85 & 1 \end{pmatrix}.
\]

Секулярне рівняння:
\[
\det(H - E S) = 0 \quad \Rightarrow \quad
\det \begin{pmatrix} -2.85-E & -2.65-0.85 E \\ -2.65-0.85 E & -3.00-E \end{pmatrix} = 0.
\]

Розкриваємо визначник:
\[
(-2.85-E)(-3.00-E) - (-2.65-0.85 E)^2 = 0.
\]

Після розкриття дужок отримаємо квадратне рівняння для $E$:
\[
E^2 + 5.85 E + 0.4875 = 0.
\]

Розв’язок:
\begin{align*}
E = \frac{-5.85 \pm \sqrt{5.85^2 - 4\cdot0.4875}}{2}
\approx -0.0835 \quad \text{(вищий)} , \\
E \approx -5.7665 \quad \text{(нижчий, фізично значущий)}.
\end{align*}

Вибираємо нижчий (основний стан):
\[
E_0 \approx -5.767 \text{ a.u.}
\]

Підставляємо $E_0$ у систему
\[
(H - E_0 S) \begin{pmatrix} c_1 \\ c_2 \end{pmatrix} = 0
\quad \Rightarrow \quad
\begin{pmatrix} -2.85 - (-5.767) & -2.65 - 0.85(-5.767) \\ -2.65 -0.85(-5.767) & -3.00 - (-5.767) \end{pmatrix} \begin{pmatrix} c_1 \\ c_2 \end{pmatrix} = 0.
\]

Обчислюємо числово:
\[
\begin{pmatrix} 2.917 & 2.402 \\ 2.402 & 2.767 \end{pmatrix} \begin{pmatrix} c_1 \\ c_2 \end{pmatrix} = 0.
\]

Відношення коефіцієнтів:
\[
c_2/c_1 = -2.917/2.402 \approx -1.215.
\]

Нормуємо:
\[
c_1^2 + c_2^2 + 2 S_{12} c_1 c_2 = 1 \quad \Rightarrow \quad c_1 \approx 0.664, \quad c_2 \approx -0.806.
\]

\textbf{Отже, хвильова функція основного стану гелію у двобазисному наближенні:}
\[
\phi_0 \approx 0.664 \, \chi_1 - 0.806 \, \chi_2.
\]


\end{document}