%%============================ Compiler Directives =======================%%
%%                                                                        %%
% !TeX program = lualatex
% !TeX encoding = utf8
% !TeX spellcheck = uk_UA
%%                                                                        %%
%%============================== Клас документа ==========================%%
%%                                                                        %%
\documentclass[14pt]{extarticle}

%%                                                                        %%
%%========================== Мови, шрифти та кодування ===================%%
%%                                                                        %%
\usepackage{fontspec}
\setsansfont{CMU Sans Serif}%{Arial}
\setmainfont{Bookman Old Style}%{Times New Roman}
\setmonofont{CMU Typewriter Text}%{Consolas}
\defaultfontfeatures{Ligatures={TeX}}
\usepackage[math-style=TeX]{unicode-math}
\usepackage[english, russian, ukrainian]{babel}
\usepackage[most]{tcolorbox}

%%                                                                        %%
%%============================= Геометрія сторінки =======================%%
%%                                                                        %%
\usepackage[%
	a4paper,%
	footskip=1cm,%
	headsep=0.3cm,%
	top=2cm, %поле сверху
	bottom=2cm, %поле снизу
	left=2cm, %поле ліворуч
	right=2cm, %поле праворуч
    ]{geometry}
%%                                                                        %%
%%============================== Інтерліньяж  ============================%%
%%                                                                        %%
\renewcommand{\baselinestretch}{1}
%-------------------------  Подавление висячих строк  --------------------%%
\clubpenalty =10000
\widowpenalty=10000
%---------------------------------Інтервали-------------------------------%%
\setlength{\parskip}{0.5ex}%
\setlength{\parindent}{2.5em}%
%%                                                                        %%
%%=========================== Математичні пакети і графіка ===============%%
%%                                                                        %%
\usepackage{amsmath}
\usepackage{graphicx}
\usepackage{floatflt}
%%                                                                        %%
%%========================== Гіперпосилення (href) =======================%%
%%                                                                        %%
\usepackage[colorlinks=true,
	%urlcolor = blue, %Colour for external hyperlinks
	%linkcolor  = malina, %Colour of internal links
	%citecolor  = green, %Colour of citations
	bookmarks = true,
	bookmarksnumbered=true,
	unicode,
	linktoc = all,
	hypertexnames=false,
	pdftoolbar=false,
	pdfpagelayout=TwoPageRight,
	pdfauthor={Ponomarenko S.M. aka sergiokapone},
	pdfdisplaydoctitle=true,
	pdfencoding=auto
	]%
	{hyperref}
		\makeatletter
	\AtBeginDocument{
	\hypersetup{
		pdfinfo={
		Title={\@title},
		}
	}
	}
	\makeatother
%%                                                                        %%
%%============================ Заголовок та автори =======================%%
%%                                                                        %%
\title{\bfseries Теорія малих коливань молекул}
\author{}
\date{}
%%                                                                        %%
%%========================================================================%%


\begin{document}
\maketitle

\section{Потенціальна енергія}
Розкладемо потенціальну енергію $U(\mathbf{q})$ в ряд Тейлора навколо положення рівноваги $\mathbf{q}
=
\mathbf{0}$:

\[
U(\mathbf{q}) = U(0) + \left. \frac{\partial U}{\partial q_i} \right|_0 q_i + \frac{1}{2} \left.
\frac{\partial^2 U}{\partial q_i \partial q_j} \right|_0 q_i q_j + \cdots
\]

Де:
\begin{itemize}
\item Перший член $U(0) = U_0$ --- потенціальна енергія в положенні рівноваги (константа)
\item Перша похідна $\left. \frac{\partial U}{\partial q_i} \right|_0 = 0$, оскільки в точці рівноваги
сила дорівнює нулю ($F_i = -\frac{\partial U}{\partial q_i} = 0$)
\item Друга похідна $H_{ij} = \left. \frac{\partial^2 U}{\partial q_i \partial q_j} \right|_0$ утворює
матрицю Гессе
\end{itemize}

Таким чином, в гармонічному наближенні:
\[
U(\mathbf{q}) \approx U_0 + \frac{1}{2} H_{ij} q_i q_j
\]


\subsection{Кінетична енергія}
\[
T = \frac{1}{2} m_i \dot{q}_i \dot{q}_i
\]

\subsection{Рівняння Лагранжа}
\[
\frac{d}{dt}\left(\frac{\partial T}{\partial \dot{q}_i}\right) - \frac{\partial T}{\partial q_i} +
\frac{\partial U}{\partial q_i} = 0 \implies m_i \ddot{q}_i + H_{ij} q_j = 0
\]

\section{Силова константа \( k \) для двохатомної молекули}


Для двоатомної молекули силова константа визначається як:

\[
k = \left. \frac{d^2 U}{dr^2} \right|_{r_e},
\]
де \( r_e \) --- рівноважна відстань.

\subsection{Зв'язок із частотою коливань}

Частота коливань пов'язана з \( k \) через приведену масу \( \mu \):

\[
\omega = \sqrt{\frac{k}{\mu}}, \quad \mu = \frac{m_1 m_2}{m_1 + m_2}.
\]

Звідси:

\[
k = \mu \omega^2.
\]


\section{Нормальні моди коливань молекул}

\subsection{Фізичний зміст нормальних мод}
Нормальні моди — це \textbf{спеціальні типи коливань}, при яких всі атоми молекули рухаються
\textbf{синхронно} з однаковою частотою $\omega_i$.


\subsection{Математичний опис}
Для кожної нормальної моди $k$:
\[
q_i(t) =  A_i e^{(\omega_i t + \phi_k)}
\]
де:
\begin{itemize}
\item $A_k$ — амплітуда коливань
\item $\phi_k$ — фаза коливань
\item $\omega_k$ — власна частота
\end{itemize}


\subsection{Фізична інтерпретація}
Кожна нормальна мода відповідає:
\begin{itemize}
\item Для двоатомних молекул --- простому розтягу/стиску зв'язку
\item Для багатоатомних молекул --- складним колективним рухам (наприклад:
\begin{itemize}
\item Валентні коливання (розтяг зв'язків)
\item Деформаційні коливання (зміна кутів)
\end{itemize}
\end{itemize}


\subsection{Розв'язок у вигляді нормальних мод}
Підстановка \(q_i(t) =  A_i e^{(\omega_i t + \phi_k)}  \) дає:
\[
\sum_{j=1}^{3N} H_{ij}A_j = \omega^2 m_i A_i
\]
або в матричній формі:
\[
\mathbf{H}\mathbf{A} = \omega^2 \mathbf{M}\mathbf{A}
\]

\begin{enumerate}
\item \textbf{Нормування за масою}:
Вводимо нові змінні:
\begin{equation}
B_i = \sqrt{m_i} A_i
\end{equation}

\item \textbf{Переписуємо систему}:
\begin{equation}
\sum_{j=1}^{3N} \frac{H_{ij}}{\sqrt{m_i m_j}} B_j = \omega^2 B_i
\end{equation}

\item \textbf{Нормальні координати}:
Визначаємо:
\begin{equation}
Q_k = \sum_{i=1}^{3N} \frac{B_i^{(k)}}{\sqrt{m_i}} q_i
\end{equation}
\end{enumerate}

\subsection{Фінальний вигляд}

Отримуємо незалежні рівняння руху:
\begin{equation}
\ddot{Q}_k + {H}^{(Q)}_{kk} Q_k = 0
\end{equation}\
або
\begin{equation}
\ddot{Q}_k + \omega_k^2 Q_k = 0
\end{equation}


До базису нормальних координат $\mathbf{H}^{(Q)} $ матриця Гессе перетворюється за формулою
\begin{equation}
    \mathbf{H}^{(Q)} = \mathbf{Q}^T \mathbf{H}  \mathbf{Q}.
\end{equation}
і має діагональну форму:

\[
\mathbf{H}^{(Q)} =
\begin{pmatrix}
\omega_1^2 & 0 & \cdots & 0 \\
0 & \omega_2^2 & \cdots & 0 \\
\vdots & \vdots & \ddots & \vdots \\
0 & 0 & \cdots & \omega_{3N}^2
\end{pmatrix}
\]












\subsection{Властивості}

\begin{itemize}
\item Діагональні елементи --- квадрати власних частот:
\[
H_{kk}^{(Q)} = \omega_k^2
\]

\item Недіагональні елементи дорівнюють нулю:
\[
H_{kl}^{(Q)} = 0 \quad (k \neq l)
\]

\item Для фізичних коливань ($k > 6$ для нелінійних молекул):
\[
\omega_k^2 > 0
\]

\item Для поступальних/обертальних мод ($k \leq 6$):
\[
\omega_k^2 = 0
\]
\end{itemize}

\subsection{Розподіл ступенів вільності}

Для молекули з $N$ атомами:

\[
\text{Загальна кількість мод} = 3N =
\underbrace{3}_{\text{поступальні}} +
\underbrace{3}_{\text{обертальні}} +
\underbrace{3N-6}_{\text{коливальні}}
\]

\subsection{Матричний вигляд}

У нормальних координатах матриця Гессе має вигляд:

\[
\mathbf{H} =
\begin{pmatrix}
0 & & & & \\
& \ddots & & \text{\huge0} & \\
& & 0 & & \\
& \text{\huge0} & & \omega_7^2 & \\
& & & & \ddots \\
\end{pmatrix}
\]

\subsection{Доведення для поступальних рухів}

Для поступального руху ($\mathbf{q} = \mathbf{a}$ --- постійний вектор):

\[
\mathbf{Ha} = 0 \Rightarrow \omega^2 = 0
\]

\subsection{Доведення для обертальних рухів}

Для малих поворотів $\theta$:

\[
\mathbf{q} = \boldsymbol{\theta} \times \mathbf{r}_i \Rightarrow \mathbf{H}(\boldsymbol{\theta} \times
\mathbf{r}) = 0
\]

\subsection{Властивості}

\begin{itemize}
\item Перші 6 мод - нульові частоти
\item Коливальні моди ($k>6$) мають $\omega_k^2 > 0$
\item Для лінійних молекул: 5 нульових мод (3 поступальні + 2 обертальні)
\end{itemize}


\subsection{Фізична інтерпретація результатів}

\begin{itemize}
\item Для нелінійної молекули:
\begin{itemize}
\item 3 нульові частоти --- поступальні рухи
\item 3 нульові частоти --- обертальні рухи
\item $3N-6$ додатних частот --- коливальні моди
\end{itemize}

\item Для лінійної молекули:
\begin{itemize}
\item 3 нульові частоти --- поступальні рухи
\item 2 нульові частоти --- обертальні рухи
\item $3N-5$ додатних частот --- коливальні моди
\end{itemize}
\end{itemize}

\subsection{Приклад для молекули води (H$_2$O)}

\begin{itemize}
\item $N=3$ атоми $\Rightarrow$ $9\times9$ матриця Гессе
\item 3 поступальні моди ($\omega=0$)
\item 3 обертальні моди ($\omega=0$)
\item 3 коливальні моди:
\begin{itemize}
\item Симетричний розтяг ($\sim 3650$ см$^{-1}$)
\item Деформація ($\sim 1590$ см$^{-1}$)
\item Антисиметричний розтяг ($\sim 3750$ см$^{-1}$)
\end{itemize}
\end{itemize}

\end{document}


