%%============================ Compiler Directives =======================%%
%%                                                                        %%
% !TeX program = lualatex
% !TeX encoding = utf8
% !TeX spellcheck = uk_UA
%%                                                                        %%
%%============================== Клас документа ==========================%%
%%                                                                        %%
\documentclass[14pt]{extarticle}
%%                                                                        %%
%%========================== Мови, шрифти та кодування ===================%%
%%                                                                        %%
\usepackage{fontspec}
\setsansfont{CMU Sans Serif}%{Arial}
\setmainfont{CMU Serif}%{Times New Roman}
\setmonofont{CMU Typewriter Text}%{Consolas}
\defaultfontfeatures{Ligatures={TeX}}
\usepackage[math-style=TeX]{unicode-math}
\usepackage[english, russian, ukrainian]{babel}
\usepackage[most]{tcolorbox}
\usepackage{microtype}
\usepackage{lua-widow-control}
%%                                                                        %%
%%============================= Геометрія сторінки =======================%%
%%                                                                        %%
\usepackage[%
	a4paper,%
	footskip=1cm,%
	headsep=0.3cm,%
	top=2cm, %поле сверху
	bottom=2cm, %поле снизу
	left=2cm, %поле ліворуч
	right=2cm, %поле праворуч
    ]{geometry}
%%                                                                        %%
%%============================== Інтерліньяж  ============================%%
%%                                                                        %%
\renewcommand{\baselinestretch}{1}
%-------------------------  Подавление висячих строк  --------------------%%
\clubpenalty =10000
\widowpenalty=10000
%---------------------------------Інтервали-------------------------------%%
\setlength{\parskip}{0.5ex}%
\setlength{\parindent}{2.5em}%
%%                                                                        %%
%%=========================== Математичні пакети і графіка ===============%%
%%                                                                        %%
\usepackage{amsmath}
\usepackage{graphicx}
\usepackage{floatflt}
%%                                                                        %%
%%========================== Гіперпосилення (href) =======================%%
%%                                                                        %%
\usepackage[colorlinks=true,
	%urlcolor = blue, %Colour for external hyperlinks
	%linkcolor  = malina, %Colour of internal links
	%citecolor  = green, %Colour of citations
	bookmarks = true,
	bookmarksnumbered=true,
	unicode,
	linktoc = all,
	hypertexnames=false,
	pdftoolbar=false,
	pdfpagelayout=TwoPageRight,
	pdfauthor={Ponomarenko S.M. aka sergiokapone},
	pdfdisplaydoctitle=true,
	pdfencoding=auto
	]%
	{hyperref}
		\makeatletter
	\AtBeginDocument{
	\hypersetup{
		pdfinfo={
		Title={\@title},
		}
	}
	}
	\makeatother
%%                                                                        %%
%%============================ Заголовок та автори =======================%%
%%                                                                        %%
\title{Частоти коливань молекули}
\author{}
\date{}
%%                                                                        %%
%%========================================================================%%

\def\JK#1#2#3#4{ \left\langle \phi_{#1} \phi_{#2} | \phi_{#3} \phi_{#4} \right\rangle }

\begin{document}
\maketitle

%% --------------------------------------------------------
\section{Пошук локального мінімума на ППЕ}
%% --------------------------------------------------------

Багато стратегій, які намагаються знайти мінімуми на ландшафтах молекулярної потенціальної енергії, починаються з апроксимації потенціальної енергії
для геометрій (разом позначених через $3N$ декартових координат ${q_j}$ у розкладанні в ряд Тейлора навколо деякої <<початкової точки>> геометрії (тобто поточної молекулярної геометрії в ітераційному процесі або геометрії, яку ви вважаєте розумною для мінімуму або перехідного стану, який ви шукаєте):

\begin{equation}
	V (g_k) = V(0) + \sum_k \left(\dfrac{\partial V}{\partial q_k}\right) q_k + \dfrac{1}{2} \sum_{j,k} q_j H_{j,k} q_k \, + \, ... \label{3.1.1}
\end{equation}
де,
$V(0)$ --- енергія при поточній геометрії,
$\dfrac{\partial{V}}{\partial{q_k}} = g_k$ ---   градієнт енергії вздовж координати $q_k$,
$H_{j,k} = \dfrac{\partial^2{V}}{\partial{q_j}\partial{q_k}}$  ---  є другою похідною або матрицею Гессе, а
$g_k$ --- довжина <<кроку>>, який потрібно зробити вздовж цього декартового напрямку.

Якщо єдиним доступним знанням є $V(0)$ та компоненти градієнта (наприклад, обчислення других похідних зазвичай є набагато більш трудомістким, ніж оцінка градієнта, тому часто доводиться працювати без знання елементів матриці Гессіана), то лінійна апроксимація:
\begin{equation*}
    V(q_k) = V(0) + \sum_k g_K \,q_k
\end{equation*}
підказує, що слід вибирати <<кроки>> $q_k$ протилежні за знаком до відповідних елементів градієнта $g_k$ якщо ми хочемо рухатися <<з гори>> до мінімуму. Величина кроків тримається невеликою, щоб залишатися в межах, в якому лінійна апроксимація до $V$ є вірним з деякою наперед визначеною бажаною точністю (тобто, потрібно бути впевненим, що $\sum_k g_K q_k$ не є надто великою).

За наявності даних про другу похідну існують різні підходи до прогнозування того, який крок $g_k$ треба зробити для пошуку мінімуму, і саме в рамках таких стратегій, заснованих на Гессіані, виникає концепція кроків по $3N-6$ незалежних змінних. Спочатку запишемо квадратичний розклад Тейлора:
\begin{equation}
    V (g_k) = V(0) + \sum_k g_K g_k + \dfrac{1}{2} \sum_{j,k} q_j H_{j,k} g_k\label{3.1.3}
\end{equation}
у матрично-векторній нотації:
\begin{equation}
    V(\mathbf{q}) = V(0) + ​\mathbf{q}^{\mathbf{T}} \cdot \mathbf{g} + \dfrac{1}{2} \mathbf{q}^{\mathbf{T}} \cdot \mathbf{H} \cdot \mathbf{q} \label{3.1.4}
\end{equation}

Ввівши унітарну матрицю $\mathbf{U}$, яка діагоналізує симетричну матрицю $\mathbf{H}$, наведене вище рівняння набуває вигляду:
\begin{equation}
    V(\mathbf{q}) = V(0) + \mathbf{q}^{\mathbf{T}} \mathbf{U} \, \mathbf{U}^{\mathbf{T}} \mathbf{q} + \dfrac{1}{2}​ \mathbf{q}^{\mathbf{T}} \mathbf{U} \, \mathbf{U}^{\mathbf{T}} \mathbf{H} \mathbf{U}\, \mathbf{U}^{\mathbf{T}} \mathbf{q}. \label{3.1.5}
\end{equation}

Оскільки $\mathbf{U}^{\mathbf{T}}\mathbf{H}\mathbf{U}$ є діагональним, ми маємо:
\begin{equation}
    (\mathbf{U}^{\mathbf{T}}\mathbf{H}\mathbf{U})_{k,l} = \delta_{k,l} \lambda_k \label{3.1.6}
\end{equation}
де $\lambda_k$ --- власні значення матриці Гессе. Для нелінійних молекул $3N-6$ з цих власних значень будуть відмінними від нуля; для лінійних молекул $3N-5$ будуть відмінними від нуля. $5$ або $6$ нульових власних значень $\mathbf{H}$ мають власні вектори, які описують трансляцію і обертання всієї молекули; вони дорівнюють нулю, оскільки енергетична поверхня $V$ не змінюється, якщо молекула обертається або транслюється. Буває важко правильно визначити $5$ або $6$ власних значень трансляції та обертання гесіану, оскільки проблеми числової точності часто призводять до того, що вони виявляються дуже малими додатними або від'ємними власними значеннями. Якщо досліджувана молекула насправді має дуже малі внутрішні (тобто коливальні) власні значення (наприклад, крутильний рух метильної групи в етані має дуже малий енергетичний бар'єр, внаслідок чого енергія дуже слабко залежить від цієї координати), потрібно бути обережним, щоб правильно ідентифікувати трансляцію-обертання і внутрішні власні значення. Досліджуючи власні вектори, що відповідають усім низьким власним значенням Гессіана, можна ідентифікувати і, таким чином, відокремити перші від других. Далі я припускатиму, що обертання і трансляції були належним чином ідентифіковані, а стратегії, які я обговорюватиму, стосуватимуться використання решти 3N-5 або 3N-6 власних значень і власних векторів для виконання серії геометричних "кроків", призначених для знаходження енергетичних мінімумів і перехідних станів.

Власні вектори $\mathbf{H}$ утворюють стовпці масиву:
\begin{equation}
    \sum_{\lambda} H_{k,l} U_{l,m} = \lambda_m U_{k,m} \label{3.1.7}
\end{equation}
\end{document}


