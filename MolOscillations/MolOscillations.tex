% !TeX program = lualatex
% !TeX encoding = utf8
% !TeX spellcheck = uk_UA

\documentclass[]{beamer}
\usetheme{QuantumChemistry}
\usepackage{QuantumChemistry}
\usepackage{minted}
\graphicspath{{pictures/}}
\addbibresource{../Bibliography/QuantumChemistry.bib}
\title[Лекції з квантової хімії]{\bfseries\huge Спектри молекул}
\subtitle{\bfseries Лекції з квантової хімії}
\author{Пономаренко С. М.}
\date{}

\begin{document}

%============================================================================
\begin{frame}
	\thispagestyle{empty}
	\titlepage
\end{frame}
%============================================================================

% ============================== Слайд ## ===================================
\begin{frame}{Молекулярні спектри}{}
	\begin{block}{}\justifying
		Молекулярні спектри --- спектри поглинання, випромінювання або розсіювання, що виникають при
		квантових переходах молекул з одного енергетичного стану в інший.
	\end{block}
	\begin{columns}
		\begin{column}{0.6\linewidth}
			\begin{block}{}\justifying\scriptsize
				Спектри визначаються \alert{складом молекули}, її \alert{структурою}, \alert{характером хімічного
					зв'язку} і \alert{взаємодією із зовнішніми полями} (тобто, з навколишніми її атомами і
				молекулами).

				\bigskip

				Найбільш характерними виходять спектри розріджених молекулярних газів, коли відсутнє розширення
				спектральних ліній тиском: такий спектр складається з вузьких ліній з доплерівською шириною.

				\bigskip

				На рисунку показана схема рівнів енергії двоатомної молекули: a і б --- електронні рівні; $v'$ і $v''$
				--- коливальні
				квантові числа; $J'$ і $J''$ --- обертальні квантові числа.

			\end{block}
		\end{column}
		\begin{column}{0.4\linewidth}
			\begin{center}
				\includegraphics[height=6cm]{spectra}
			\end{center}
		\end{column}
	\end{columns}

\end{frame}
% ===========================================================================

% ============================== Слайд ## ===================================
\begin{frame}{Ядра, атоми, молекули та спектри}{}

	\begin{center}
		\includegraphics[width=0.6\linewidth]{mol_types_of_motion}
	\end{center}

	\begin{block}{}\justifying
		В молекулах можливі \alert{електронні збудження} та \alert{коливальні}, \alert{обертальні} і
		\alert{поступальні} рухи. Кожен із цих рухів є квантовим.
	\end{block}

	\begin{onlyenv}<1>
		\begin{block}{}\justifying
			Щоб ініціювати перехід, фотон має потрапити в резонанс із потрібним рухом молекули, тому частота/довжина хвилі фотона теж характеризує
			сам рух.
		\end{block}
	\end{onlyenv}

	\begin{onlyenv}<2>
		\begin{block}{}\justifying
			Однак, не тільки ЕМ хвилі можуть викликати зміну станів: якщо до молекули підходять інші молекули, то вони можуть механічно
			віддавати/приймати енергію відповідних рухів.
		\end{block}
	\end{onlyenv}

\end{frame}
% ===========================================================================

% ============================== Слайд ## ===================================
%\begin{frame}{$\gamma$-Випромінювання}{}
%	\begin{equation*}
%		\nu \ge 10^{19}\  \text{Гц}
%	\end{equation*}
%	\begin{block}{}\small\justifying
%		У цій області відбуваються здебільшого всілякі мегаенергетичні процеси, типу ядерних реакцій, випромінювання всяких космічних об'єктів і
%		випромінювань всяких космічних об'єктів \ldots . І це один із видів радіації, тому що у всіх хімічних сполук під час взаємодії з фотонами
%		такої енергії відбувається вибивання електронів з різних енергетичних рівнів (іонізація).
%	\end{block}
%
%	\begin{center}
%		\includegraphics[width=0.5\linewidth]{pictures/gamma_rays}
%	\end{center}
%\end{frame}
% ===========================================================================

% ============================== Слайд ## ===================================
%\begin{frame}{Рентгенівське випромінювання}{}
%	\begin{equation*}
%		10^{19} \ge \nu \ge 10^{16}\ \text{Гц}.
%	\end{equation*}
%
%	\begin{block}{}\small\justifying
%		Ця область спектра належить до переходів остовних електронів. Так, наприклад, рентгенівські трубки працюють за рахунок "звалювання"
%		електронів з якого-небудь 2p-рівня в атомі на 1s. Таким чином, ця область відповідає енергіям і частотам руху остовних електронів в
%		атомах/молекулах. Тут електрони мають характеристичні періоди обертання навколо ядер у районі долей ангстрема. Запам'ятати довжини хвиль
%		цього діапазону можна з таких міркувань: рентгенівські промені можуть дифрагувати на кристалічних решітках, отже, довжина хвилі повинна
%		відповідати але довжина хвилі має бути порівнянна з довжиною міжатомних відстаней. атомних відстаней. Характеристична довжина хімічного
%		зв'язку --- це $1$ \AA, що і є хорошою оцінкою для довжини хвилі рентгенівських променів, що лежать у діапазоні $10$ нм --  $10$ пм.
%	\end{block}
%
%	\begin{center}
%		\includegraphics[width=0.3\linewidth]{pictures/x-rays}
%	\end{center}
%\end{frame}
% ===========================================================================

% ============================== Слайд ## ===================================
\begin{frame}{Ультрафіолетове та видиме випромінювання}{}
	\begin{columns}
		\begin{column}{0.6\linewidth}
			Ультрафіолетове випромінювання:
			\begin{equation*}
				10^{16} \ge \nu \ge 0.77 \cdot 10^{15}\ \text{Гц}
			\end{equation*}
			Видиме випромінювання:
			\begin{equation*}
				0.77 \cdot 10^{15} \ge \nu \ge 0.43 \cdot 10^{15}\ \text{Гц}
			\end{equation*}
		\end{column}
		\begin{column}{0.4\linewidth}
			\includegraphics[width=0.75\linewidth]{pictures/xps_ups_prinzip}
		\end{column}
	\end{columns}

	\begin{block}{}\justifying
		Електронні переходи валентних електронів атомів/молекул є джерелом видимого та ультрафіолетового випромінювання.
		%		UV/Vis спектрометри є незамінним атрибутом органічної хімії. Вони дозволяють <<бачити>>, переходи в
		%		$\pi$-системах ароматичних молекул. І ми самі здатні бачити тільки завдяки процесам
		%		ізомеризації ретиналю в наших колбочках.
	\end{block}

	\begin{block}{}\scriptsize\justifying
		Температура фотонів цього діапазону має порядок $10^4$ К, що можна порівняти з температурою на поверхні Сонця. За наших, земних, умов, з
		температурами порядку $10^2$ К, складно збудити електронні стани термічно,
		тобто якщо молекулу спеціально не злити і не тикати високочастотними фотонами або іншими високоенергетичними фотонами, або іншими
		високоенергетичними частинками, то вона перебуватиме в основному електронному стані.
	\end{block}
\end{frame}
% ===========================================================================

% ============================== Слайд ## ===================================
\begin{frame}{Інфрачервоне випромінювання}{}
	\begin{equation*}
		0.43 \cdot 10^{15} \ge \nu \ge 0.3 \cdot 10^{12}\ \text{Гц}
	\end{equation*}
	\begin{block}{}\justifying
		Це випромінювання завдячує коливальним переходам у молекулах. У дальній ІЧ області (за низьких частот) уже можна спостерігати обертання
		малих і дуже легких молекул, типу \ce{H2}.
	\end{block}
	\begin{center}
		\includegraphics[width=1\linewidth]{pictures/mol_vibrations}
	\end{center}
\end{frame}
% ===========================================================================

% ============================== Слайд ## ===================================
\begin{frame}{Мікрохвильовий діапазон}{}
	\begin{equation*}
		0.3\cdot 10^{12} \ge \nu \ge 0.3 \cdot 10^{9}\ \text{Гц}
	\end{equation*}

	\begin{block}{}\justifying
		В цьому діапазоні відбуваються тільки обертальні переходи вільних молекул. Чим більша, розгалуженіша, важча молекула, тим нижча частота
		переходу. Часи 		обертальних рухів мають порядок від пікосекунд до долей мікросекунд.
	\end{block}

	\begin{columns}
		\begin{column}{0.45\linewidth}
			\includegraphics[width=\linewidth]{pictures/microwaveoven}
		\end{column}
		\begin{column}{0.45\linewidth}
			\includegraphics[width=\linewidth]{pictures/water_rotation}
		\end{column}
	\end{columns}

\end{frame}
% ===========================================================================

% ============================== Слайд ## ===================================
\begin{frame}{Шкала електромагнітних хвиль}{}
	\begin{center}
		\includegraphics[width=\linewidth]{pictures/em_waves}
	\end{center}
\end{frame}
% ===========================================================================

% ============================== Слайд ## ===================================
\begin{frame}{Спектроскопія}{}\small
	%\UseTblrLibrary{booktabs}
	\begin{tblr}{|Q[0.55\linewidth, l, m]|X[j, m]|}
		\hline
		Тип спектроскопії                        & Переходи в молекулах.                         \\ \hline
		Інфрачервона (ІЧ) та раманівська         & Коливальні та обертальні переходи в молекулах \\
		УФ/видима (UV/Vis)                       & Електронні переходи в атомах або молекулах.   \\
		Електронний парамагнітний резонанс (EPR) & Електронні переходи зі зміною спіну.          \\
		Ядерно-магнітний резонанс (ЯМР)          & Переорієнтація ядерних магнітних моментів.    \\
		\hline
	\end{tblr}
\end{frame}
% ===========================================================================

% ============================== Слайд ## ===================================
\begin{frame}[fragile]{Ультрафіолетові та видимі спектри молекул}{}
	\begin{onlyenv}<1>
		\begin{block}{}\justifying
			Для збудження електронних станів в молекулі необхідне випромінювання в ультрафіолетовій і видимій областях спектра. При падінні світла на
			молекулу відбуваються їх переходи в  збуджений стан --- так виникає молекулярний \alert{електронний спектр поглинання}. Конфігурація молекули
			при цьому змінюється.
		\end{block}%
	\end{onlyenv}
	\begin{onlyenv}<1-2>
		\begin{center}
			\includegraphics[width=\linewidth]{ElTransitions}
			\begin{block}{}\justifying\footnotesize
				Електронні переходи в двоатомних молекулах: а) з основного коливального стану в основний коливальний стан, б) з основного
				коливального
				стану в коливально-збуджений стан, в) фотодисоціація.
			\end{block}
		\end{center}
	\end{onlyenv}
	\begin{onlyenv}<2>
		\begin{block}{}\justifying
			Найінтенсивніші електронні переходи відбуваються без зміни положення ядер (\alert{принцип Франка-Кондона}).
			%Інтенсивність  переходів пропорційна до квадрата інтеграла перекривання між коливальними хвильовими функціями двох станів, які залучені в
			%переході.
		\end{block}
		\begin{block}{}\justifying
			Електронні переходи в молекулі можуть супроводжуватись змінами коливальних станів (випадок б).
		\end{block}
	\end{onlyenv}
	\framesubtitle<3-4>{Обчислення електронних спектрів в ORCA}
	\begin{onlyenv}<3>
		\begin{minted}[
        autogobble,
        fontsize=\scriptsize,
        ]
        {ruby}
! UHF SVP OPT

%casscf
    Nel 2
    Norb 2
    Nroots 2
    iroot 1
end

%geom scan
    B 0 1 = 0.5,2.25,50
    end
end

* int 0 1
   H     0     0     0        0.00000     0.00000        0.00000
   H     1     0     0        0.10000     0.00000        0.00000
*
        \end{minted}
	\end{onlyenv}
	\begin{onlyenv}<4>
		\begin{center}
			\pgfplotstableread[]{tikz/H2cas.trjscf.dat}\atable
\pgfplotstableread[]{tikz/H2cas.trjscf_nocas.dat}\btable

\begin{tikzpicture}[   oxygen/.style={circle, ball color=red, minimum size=6mm, inner sep=0},
		hydrogen/.style={circle, ball color=white, minimum size=2.5mm, inner sep=0},
		carbon/.style={circle, ball color=black!75, minimum size=7mm, inner sep=0}
	]
	\begin{axis}[%
            legend style={nodes={scale=0.5, transform shape},
%            at={(current axis.south east)},anchor=south east,
            },
			xlabel = {Довжина зв'язку, \r{A}},
			ylabel = {Енергія, Eh},
			tick label style={font=\tiny},
			xtick distance={0.5},
%			xticklabel style = {rotate=45},
            ytick distance={0.1},
			width=5cm,
			height=5cm,
			scale only axis,
			enlargelimits=false,
			line join=round,
            xmin=0,
            ymax=-0.9,
			ymin=-1.2,
			% === Налаштування сітки ===
			grid = both,
			grid style={line width=.1pt, draw=gray!10},
			major grid style={line width=.2pt,draw=gray!50},
			minor grid style = {line width=.1pt,draw=gray!10},
            minor x tick num=1,
            minor y tick num=4,
            width=0.75\linewidth,
		]
        \addplot [color=red, mark=none, smooth] table[y index = 1] {\atable};
        \addlegendentry{{def2-SVP, RHF метод}}
        \addplot [color=blue, mark=none, smooth] table[y index = 1] {\btable};
        \addlegendentry{{def2-SVP, CAS SCF метод}}

%        \addplot [color=green, mark=none, smooth] table {\ctable};
        \coordinate (MINCAS) at (axis cs:0.759142884817,  -1.163182705);
        \coordinate (MINHF) at (axis cs:0.74489796,  -1.10855245);
        \node[circle, fill=red, inner sep=0.5pt] at (MINCAS) {};
        \node[circle, fill=blue, inner sep=0.5pt] at (MINHF) {};
        \node[below, font=\tiny, anchor=north west, text=red] at (MINCAS) {
            \pgfplotspointgetcoordinates{(MINCAS)}
            $(
                \pgfmathprintnumber[fixed]
                {\pgfkeysvalueof{/data point/x}},
                \pgfmathprintnumber[fixed]
                {\pgfkeysvalueof{/data point/y}}
            )$
        };
         \node[below, font=\tiny, anchor=north west, text=blue] at (MINHF) {
            \pgfplotspointgetcoordinates{(MINHF)}
            $(
                \pgfmathprintnumber[fixed]
                {\pgfkeysvalueof{/data point/x}},
                \pgfmathprintnumber[fixed]
                {\pgfkeysvalueof{/data point/y}}
            )$
        };
	\end{axis}
\end{tikzpicture}%
		\end{center}
	\end{onlyenv}
\end{frame}
% ===========================================================================

% ============================== Слайд ## ===================================
\begin{frame}[fragile]{Розрахунок інфрачервоних спектрів в ORCA}{}
	\tikz[remember picture,overlay] \node[opacity=0.3,inner sep=0pt, anchor=north east] at ([yshift=-1cm]current page.north
	east){\includegraphics[width=2cm]{orca_logo}};

	\begin{onlyenv}<1>
		\begin{block}{}
			Щоб знайти ІЧ-спектри молекули, треба додати наступні рядки:

			\begin{minted}[
        autogobble,
        fontsize=\scriptsize,
        ]
        {ruby}

    ! Opt
    ! AnFreq # or NumFreq
    \end{minted}
		\end{block}

		\begin{block}{}\justifying
			Аналітичні (\texttt{AnFreq}) частоти вимагають менше часу на обчислення ніж чисельні (\texttt{NumFreq}).
		\end{block}

		\begin{minted}[
        autogobble,
        fontsize=\scriptsize,
        ]
        {ruby}
            ! RHF Opt AnFreq def2-SVP

            * int 0 1
               O     0     0     0        0.00000        0.00000        0.00000
               H     1     0     0        0.99025        0.00000        0.00000
               H     1     2     0        0.99025      104.51004        0.00000
            *
        \end{minted}
	\end{onlyenv}
	\begin{onlyenv}<2>
		\begin{block}{}\justifying
			Перші кілька частот завжди дорівнюють нулю, оскільки вони відповідають обертальним і поступальним модам. Їх має бути п'ять для лінійних молекул і
			шість для нелінійних, решта відповідають власне коливальним модам.
			\begin{minted}[
        autogobble,
        fontsize=\scriptsize,
        ]
        {ruby}
        -----------------------
        VIBRATIONAL FREQUENCIES
        -----------------------

        Scaling factor for frequencies =  1.000000000  (already applied!)

           0:         0.00 cm**-1
           1:         0.00 cm**-1
           2:         0.00 cm**-1
           3:         0.00 cm**-1
           4:         0.00 cm**-1
           5:         0.00 cm**-1
           6:      1750.33 cm**-1
           7:      4148.51 cm**-1
           8:      4244.72 cm**-1
        \end{minted}
		\end{block}
	\end{onlyenv}
	\begin{onlyenv}<3>
		\begin{block}{}
			Потім програма виводить нормальні моди коливань:

			\begin{minted}[
        autogobble,
        fontsize=\scriptsize,
        ]
        {ruby}
-----------
IR SPECTRUM
-----------

 Mode   freq       eps      Int      T**2         TX        TY        TZ
       cm**-1   L/(mol*cm) km/mol    a.u.
----------------------------------------------------------------------------
  6:   1750.33   0.016362   82.68  0.002917  (-0.033072 -0.042700 -0.000000)
  7:   4148.51   0.005311   26.84  0.000400  ( 0.012232  0.015807 -0.000000)
  8:   4244.72   0.014716   74.37  0.001082  ( 0.026011 -0.020132 -0.000000)
 \end{minted}
		\end{block}

		\begin{block}{}
			ІЧ-спектр можна побудувати за допомогою утиліти \texttt{orca\_mapspc}:
			\begin{minted}[
        autogobble,
        ]
        {bash}
> orca_mapspc H2O.out IR
 \end{minted}
		\end{block}
	\end{onlyenv}
	\begin{onlyenv}<4>
		Одиницею енергії переходу для ІЧ є см\textsuperscript{-1}:
		\begin{equation*}
			1\ \text{еВ}=8.0655\cdot 10^3\ \text{см\textsuperscript{-1}}
		\end{equation*}
		Програма розраховує \href{https://en.wikipedia.org/wiki/Beer%E2%80%93Lambert_law}{коефіцієнт молярної екстинції} $\varepsilon$ в
		$\frac{\text{літр}}{\text{моль}\cdot\text{см}} = \frac{1000\ \text{см}^2}{\text{моль}}$ та інтенсивність за формулою:
		\begin{equation*}
			I = \int \varepsilon(\nu)d\nu,
		\end{equation*}
		в одиницях $\frac{1000\ \text{см}^2}{\text{моль}}\cdot\text{см}^{-1} = \frac{0.01\ \text{km}}{\text{моль}}$.
	\end{onlyenv}
\end{frame}
% ============================== Слайд ## ===================================

% ============================== Слайд ## ===================================
\begin{frame}{Парниковий ефект}{}


\begin{block}{}\justifying\scriptsize

\begin{wrapstuff}[l, type=figure, width=0.4\linewidth, top=2]
\includegraphics[width=\linewidth]{Greenhouse-effect.pdf}
\end{wrapstuff}
    Атмосфера, що складається переважно з гомоядерних двоатомних молекул, як-от \ce{N2} і \ce{O2}, майже не поглинає інфрачервоне випромінювання, оскільки їхні коливальні переходи \emph{ІЧ-неактивні}.

\smallskip

Сонячне світло у видимому та ближньому ІЧ-діапазоні нагріває поверхню Землі, яка, своєю чергою, випромінює тепло у вигляді інфрачервоного випромінювання. Деякі атмосферні гази (наприклад, \ce{CO2}, \ce{CH4}, \ce{H2O}) поглинають цю енергію і повертають частину тепла назад до Землі, подібно до того, як скло в парнику утримує тепло

\smallskip

Середня температура на поверхні Землі визначається балансом між вхідним високочастотним (УФ і видимим) та вихідним низькочастотним (ІЧ) випромінюванням. Збільшення концентрацій парникових газів --- призводить до зменшення тепловтрат.
\end{block}

\end{frame}
% ===========================================================================

% ============================== Слайд ## ===================================
\begin{frame}{Чому вода блакитна?}{}

\begin{block}{}\justifying\scriptsize
    \begin{wrapstuff}[l, type=figure, width=0.4\linewidth, top=2]
\includegraphics[width=\linewidth]{blueWater}
\end{wrapstuff}
Коливальні переходи у молекулах води (\ce{H2O}) частково відповідають за блакитний колір води. Моря, річки та озера часто здаються блакитними не лише через те, що блакитне світло неба відбивається від поверхні, --- біле світло, проходячи крізь товщу чистої води, також набуває блакитного відтінку.

\smallskip

Це зумовлено сильним поглинанням водою інфрачервоного випромінювання, де розташовані фундаментальні коливальні переходи $n = 0 \to 1$ з частотою близько $3700$ см\textsuperscript{-1}. Висока густина води й довжина пробігу світла в ній посилюють цей ефект.

\smallskip

Цікаво, що завдяки великій дипольній похідній та значній ангармонічності потенціалу молекули води навіть вищі обертонові переходи, зокрема $n = 0 \to 4$, можуть помітно поглинати світло в діапазоні $600$–$800$ нм. Це область червоного кінця видимого спектра, тому такі переходи частково відсіюють червоне світло. У результаті синє світло з коротшою довжиною хвилі проходить крізь воду майже без втрат.

\smallskip

Цей ефект, хоч і слабкий через низьку інтенсивність обертонових переходів, є унікальним прикладом того, як саме коливальні (а не електронні) переходи можуть визначати сприйманий колір речовини.
\end{block}

\end{frame}
% ===========================================================================

\begin{frame}{Термохімічні функції}{}
	\begin{block}{}\justifying
		Якщо успішно виконано розрахунок коливальних частот, то ORCA автоматично дає результати розрахунків термодинамічних функцій, таких як
		внутрішня
		енергія $U$, ентальпія $H$, ентропія $S$ та вільна енергія Гіббса $G$ на основі формул статистичної фізики:

		\begin{enumerate}
			\item Внутрішня енергія: \( U(T) = E_0 + \sum_i \left( \frac{h\nu_i}{2} + \frac{h\nu_i}{\exp(h\nu_i/kT) - 1} \right) \).
			\item Ентальпія: \( H = U + kT \).
			\item Ентропія: \( S(T) = S_0 - R\ln\left(\frac{V}{N} \frac{(2\pi m k T)^{3N/2}}{h^{3N}}\right) + R\sum_i
			      \left(\frac{\nu_i}{T}\frac{\exp(-h\nu_i/kT)}{1-\exp(-h\nu_i/kT)}\right) \).
			\item Енергія Гіббса: \(G(T) = H(T) - TS(T) \).
		\end{enumerate}

		%		\begin{minted}[
		%        autogobble,
		%        fontsize=\scriptsize,
		%        ]
		%        {ruby}
		%--------------------------
		%THERMOCHEMISTRY AT 298.15K
		%--------------------------
		%
		%Temperature         ... 298.15 K
		%Pressure            ... 1.00 atm
		%Total Mass          ... 18.02 AMU
		%...
		%Total thermal energy                     -75.93539552 Eh
		%Total Enthalpy                    ...    -75.93445131 Eh
		%Final entropy term                ...      0.02137025 Eh     13.41 kcal/mol
		%Final Gibbs free energy           ...    -75.95582156 Eh
		% \end{minted}
	\end{block}
\end{frame}
% ===========================================================================

% ============================== Слайд ## ===================================
\begin{frame}{Аналітичні розрахунки градієнтів та Гесіана}{}
	\fullcite[p. 103, \S 11.9, p. 370]{Jensen}
\end{frame}
% ===========================================================================

\end{document}