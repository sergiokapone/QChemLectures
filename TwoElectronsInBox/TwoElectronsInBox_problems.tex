% !TeX program = lualatex
% !TeX encoding = utf8
% !TeX spellcheck = uk_UA

\documentclass[10pt]{beamer}
\usetheme{QuantumChemistry}
\usepackage{QuantumChemistry}


\title[Лекції з квантової хімії]{{\bfseries\huge Два електрони в одновимірній потенціальній ямі} \\ {Прообраз багатоелектронної системи}}
\subtitle{Задачі}
\date{}
\begin{document}
\begin{frame}{Задачі}
    \begin{enumerate}
    \item Запишіть детермінант Слейтера для основного стану системи трьох частинок (прикладом може бути основний стан атома \ce{Li}). Яка мультиплетність основного стану?
    \item Знайдіть вираз електронної густини для цього випадку через орбіталі використовуючи формулу:
    	\begin{equation*}
    		\rho(x,y,z) = 3  \int\limits_{V_2}  \int\limits_{V_3} \int\limits_{\sigma_1} \int\limits_{\sigma_2} \int\limits_{\sigma_3}  |\Phi (\vxi_1, \vxi_2, \vxi_3)|^2  dV_2dV_3 d\sigma_1d\sigma_2d\sigma_3.
    	\end{equation*}
    \item Зробіть висновки з попереднього розв'язку: як має виглядати електронна густина для системи $N_e$ електронів (виведення для загальному випадку робити не треба)?
    \end{enumerate}
\end{frame}
\end{document}