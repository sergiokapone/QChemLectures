% !TeX program = lualatex
% !TeX encoding = utf8
% !TeX spellcheck = russian-aot
% !BIB program = biber

\documentclass{article}

\usepackage[%
	a4paper,%
	footskip=1cm,%
	headsep=0.3cm,%
	top=2cm, %поле сверху
	bottom=2cm, %поле снизу
	left=2cm, %поле ліворуч
	right=2cm, %поле праворуч
    ]{geometry}

\usepackage[fontsize=14pt]{fontsize}
\usepackage{fontsetup}
\usepackage[english, russian, ukrainian]{babel}

\usepackage{amsmath}
\usepackage[most, many]{tcolorbox}


% Defining custom commands for clarity
\newcommand{\pd}[2]{\frac{\partial #1}{\partial #2}}
\newcommand{\pdd}[2]{\frac{d^2 #1}{d #2^2}}

% Starting the document
\begin{document}

\section{Решение уравнения Шрёдингера для задачи Келлера}

Уравнение Шрёдингера для задачи Келлера имеет вид

\begin{equation}
\Delta \psi + \frac{8\pi m^2}{\hbar^2} \left( E + \frac{e^2Z}{r} \right) \psi = 0.
\end{equation}

Вводя сферические координаты \( r, \theta, \varphi \), приведем его к виду

\begin{equation}
\left\{ \frac{\partial^2}{\partial r^2} + \frac{2}{r} \frac{\partial}{\partial r} + \frac{1}{r^2} \frac{\partial^2}{\partial \theta^2} + \frac{\cot \theta}{r^2} \frac{\partial}{\partial \theta} + \frac{1}{r^2 \sin^2 \theta} \frac{\partial^2}{\partial \varphi^2} \right\} \psi +
\frac{8\pi m^2}{\hbar^2} \left( E + \frac{e^2Z}{r} \right) \psi = 0.
\end{equation}

В этом уравнении переменные опять разделяются; если положить

\begin{equation}
\psi = R(r) \Theta(\theta) \Phi(\varphi),
\end{equation}
%
то уравнение можно расщепить на три уравнения:

\begin{equation}
\left\{ \frac{d^2}{dr^2} + \frac{2}{r} \frac{d}{dr} + \frac{8\pi m^2}{\hbar^2} \left( E + \frac{e^2Z}{r} \right) - \frac{\lambda}{r^2} \right\} R = 0,
\end{equation}

\begin{equation}
\left\{ \frac{d^2}{d\theta^2} + \cot \theta \frac{d}{d\theta} + \lambda - \frac{m^2}{\sin^2 \theta} \right\} \Theta = 0,
\end{equation}

\begin{equation}
\frac{d^2 \Phi}{d\varphi^2} + m^2 \Phi = 0,
\end{equation}
%
где \( m \) (как и раньше) и \( \lambda \) — параметры разделения. Решение третьего уравнения уже известно:

\begin{equation}
\Phi = \begin{bmatrix}
\cos(m\varphi)\\
\sin(m\varphi)
\end{bmatrix}
, \text{или} \quad \Phi = e^{im\varphi},
\end{equation}
%
где \( m \) должно быть целым, иначе решение \( \Phi \) окажется неоднозначным.

Второе уравнение — это уравнение, определяющее шаровые функции $P_l^m(\cos \theta)$, где $l$ принимает значения $l(l+1)$ а $|m| \leq l$. Для других значений уравнение не имеет конечных однозначных решений. Докажем это в общем виде, для чего объединим зависимость от $\theta$ и от $\phi$ в обобщенную шаровую функцию $Y_l(\theta, \phi)$. Если для краткости ввести обозначение

\begin{equation}
\Delta = \frac{1}{\sin \theta} \frac{\partial}{\partial \theta} \sin \theta \frac{\partial}{\partial \theta} + \frac{1}{\sin^2 \theta} \frac{\partial^2}{\partial \phi^2},
\end{equation}
%
так что

\begin{equation}
\Delta = \frac{\partial^2}{\partial r^2} + \frac{2}{r} \frac{\partial}{\partial r} + \frac{\Lambda}{r^2},
\end{equation}
%
то будет ясно, что функция $Y_l(\theta, \phi)$ должна удовлетворять дифференциальному уравнению

\begin{equation}
\Lambda Y_l + \lambda Y_l = 0.
\end{equation}

Общее решение этого уравнения подходит следующим образом. Рассмотрим однородные полиномы $U_l$ степени $l$ по $x$, $y$, $z$, удовлетворяющие уравнению Лапласа

\begin{equation}
\Delta U_l = 0.
\end{equation}

Определим теперь функцию $Y_l$ отношений $x/r$, $y/r$, $z/r$ (т. е. функцию, зависящую только от углов $\theta$ и $\phi$) при помощи уравнения

\begin{equation}
U_l = r^l Y_l.
\end{equation}

Затем подставим эту функцию в уравнение Лапласа $\Delta U_l = 0$. Выполнив дифференцирование по $r$, мы приходим к уравнению

\begin{equation}
\left( \frac{\partial^2}{\partial r^2} + \frac{2}{r} \frac{\partial}{\partial r} \right) r^l Y_l + \frac{\Lambda}{r^2} r^l Y_l = r^{l-2} \left\{ \Lambda Y_l + l(l+1) Y_l \right\} = 0.
\end{equation}

Таким образом, введенные выше функции являются решениями дифференциального уравнения $\Delta Y_l + \lambda Y_l = 0$, только если

\begin{equation}
\lambda = l(l+1).
\end{equation}

\begin{tcolorbox}[breakable]
\section*{Вывод собственных значений оператора $\Lambda$}

Рассмотрим задачу
\begin{equation}
\Lambda Y_l + \lambda Y_l = 0,
\end{equation}
где $\Lambda$~--- угловая часть лапласиана.
Лапласиан в сферических координатах имеет вид
\begin{equation}
\Delta = \frac{\partial^2}{\partial r^2}
+ \frac{2}{r}\frac{\partial}{\partial r}
+ \frac{1}{r^2}\Lambda.
\end{equation}

\subsection*{Однородные гармонические многочлены}
Пусть $U_l(x,y,z)$~--- однородный многочлен степени $l$, удовлетворяющий уравнению Лапласа
\begin{equation}
\Delta U_l = 0.
\end{equation}
Так как $U_l$ однороден, его можно записать в виде
\begin{equation}
U_l(x,y,z) = r^l Y_l\!\left(\tfrac{x}{r},\tfrac{y}{r},\tfrac{z}{r}\right),
\end{equation}
где функция $Y_l$ зависит только от углов $\theta,\phi$.

\subsection*{Подстановка в лапласиан}
Подставим $U_l = r^l Y_l$ в лапласиан:
\begin{equation}
\Delta U_l =
\left( \frac{\partial^2}{\partial r^2}
+ \frac{2}{r}\frac{\partial}{\partial r} \right) r^l Y_l
+ \frac{1}{r^2}\Lambda(r^l Y_l).
\end{equation}

Вычислим радиальные производные:
\begin{align}
\frac{\partial}{\partial r}(r^l Y_l) &= l r^{l-1} Y_l, \\
\frac{\partial^2}{\partial r^2}(r^l Y_l) &= l(l-1) r^{l-2} Y_l,
\end{align}
откуда
\begin{equation}
\left( \frac{\partial^2}{\partial r^2}
+ \frac{2}{r}\frac{\partial}{\partial r} \right) r^l Y_l
= l(l+1) r^{l-2} Y_l.
\end{equation}

Угловая часть даёт
\begin{equation}
\frac{1}{r^2}\Lambda(r^l Y_l) = r^{l-2}\Lambda Y_l.
\end{equation}

Таким образом,
\begin{equation}
\Delta U_l = r^{l-2}\big( l(l+1) Y_l + \Lambda Y_l \big).
\end{equation}

\subsection*{Условие гармоничности}
Так как $U_l$ гармонический многочлен, то
\begin{equation}
\Delta U_l = 0,
\end{equation}
следовательно
\begin{equation}
\Lambda Y_l + l(l+1) Y_l = 0.
\end{equation}

\subsection*{Собственные значения}
Мы получили уравнение
\begin{equation}
\Lambda Y_l = -\,l(l+1) Y_l,
\end{equation}
т.~е. собственные значения оператора $\Lambda$ равны
\begin{equation}
\lambda = l(l+1).
\end{equation}
Таким образом, функции $Y_l(\theta,\phi)$, возникающие из однородных гармонических многочленов, являются собственными функциями углового лапласиана, то есть сферическими гармониками.

\end{tcolorbox}

Можно доказать, что никаких другие значения $\lambda$ не дают конечных, неприводимых к однообразным решений этого уравнения. В соответствии с этим соотношением
\begin{equation*}
    \Lambda Y_l + \lambda Y_l = 0
\end{equation*}
равны $l(l+1)$. Кроме того, число произвольных параметров в общем решении равно степени $l$ равно определено. Например, полином степени $l$ по $x$, $y$, $z$ содержит $\frac{1}{2}(l+1)(l+2)$ произвольных коэффициентов (он имеет $n$ членов с $x^n$, два члена с $x^{l-1}$, три с $x^{l-2}$ и т. д. Наконец, $(l+1)$ членов, больше не содержит $x$). Очевидно
условие $\Delta U_l = 0$ приводит к определенной зависимости между постоянными; это условие эквивалентно $\frac{1}{2} l(l-1)$ уравнениям для определений коэффициентов, так как $\Delta U_l$ есть однородная функция степени $(l-2)$, тождественно равная нулю. Поэтому $U_l$ содержит

\begin{equation}
\frac{1}{2} \{ (l+1)(l+2) - l(l-1) \} = 2l + 1
\end{equation}
%
независимых коэффициентов. В соответствии с этим имеется $2l+1$ линейно независимых шаровых функций степени $l$. Если записать их в обычной форме

\begin{equation}
Y_l^{(m)} = P_l^m e^{im\phi},
\end{equation}
%
то они соответствуют $2l+1$ возможному значению третьего (магнитного) квантового числа $m$.

Теперь перейдем к дифференциальному уравнению для радиальной функции $R$:

\begin{equation}
\left\{ \frac{d^2}{dr^2} + \frac{2}{r} \frac{d}{dr} + \frac{8\pi^2 m^2}{h^2} \left( E + \frac{e^2 Z}{r} \right) - \frac{l(l+1)}{r^2} \right\} R = 0.
\end{equation}

Его решения должны быть конечными и непрерывными при всех значениях $r$ от нуля до бесконечности. Здесь нас главным образом уравнение имеет решение, удовлетворяющее заданным условиям при которых это уравнение имеет решение $E$, при котором это решение не имеет конечных решений. В частности, займемся случаем $E<0$. В теории Бора это соответствует эллиптическим орбитам, когда для уравнения электрона на бесконечное расстояние от ядра необходимо придать ему дополнительную энергию. Случай $E>0$ соответствует гиперболическим орбитам.

Для простоты введем относительные единицы. За единицу радиуса примем боровский радиус $h^2/4\pi^2 me^2 Z$, а за единицу энергии --- энергию основного состояния атома Бора, $-2\pi^2 me^4 Z^2/h^2$. Другими словами, положим (гл. V, § 1)

\begin{equation}
    r = \rho \frac{h^2}{4\pi^2 me^2 Z}, \quad E = \varepsilon \left( \frac{-2\pi^2 me^4 Z^2}{h^2} \right).
\end{equation}

Тогда волновое уравнение принимает более простой вид

\begin{equation}
\left\{ \frac{d^2}{d\rho^2} + \frac{2}{\rho} \frac{d}{d\rho} - \varepsilon + \frac{2}{\rho} - \frac{l(l+1)}{\rho^2} \right\} R = 0.
\end{equation}

Начнем с определения свойств функции $R$ при очень больших значениях $\rho$. Соответственно отгорим в дифференциальном уравнении члены с $1/\rho$ и $1/\rho^2$, так что поведение на бесконечности будет определяться уравнением

\begin{equation}
\left\{ \frac{d^2}{d\rho^2} - \varepsilon \right\} R_\infty = 0.
\end{equation}

Оно имеет решение

\begin{equation}
R_\infty = e^{\pm \rho \sqrt{\varepsilon}}.
\end{equation}

Однако решение со знаком «плюс» надо отбросить, так как волновая функция в этом случае неограниченно возрастает бы экспоненциальным образом при возрастании $r$ (или $\rho$) и поэтому не могла бы быть собственной функцией.

Другая особенность находится у начала координат. Прежде всего уравнение при очень малых значениях $r$ стремится к бесконечности медленнее, чем $1/\rho^2$, при условии приближенное уравнение

\begin{equation}
\left\{ \frac{d^2}{d\rho^2} + \frac{2}{\rho} \frac{d}{d\rho} - \frac{l(l+1)}{\rho^2} \right\} R_0 = 0.
\end{equation}

Решением его будут функции $R_0 = \rho^l$ и $R_0 = \rho^{-l-1}$. Вторая из них неприемлема, так как обращается в бесконечность в нуле.

Итак, мы знаем поведение искомой функции вблизи двух особых точек, $\rho = 0$ и $\rho = \infty$. Разумно предположить, что вся функция $R$ имеет вид

\begin{equation}
R = e^{-\rho \sqrt{\varepsilon}} \rho^l f(\rho),
\end{equation}
%
где $f$ --- функция от $\rho$, которая, конечно, должна быть регулярной в обеих точках (т. е. на бесконечности не должна возрастать быстрее $e^{+\rho \sqrt{\varepsilon}}$ и которая определяет поведение интевалы в которых доминирует либо степенная $\rho^l$, либо экспоненциальная зависимость. Подставляем в дифференциальное уравнение для $R$ и приравниваем коэффициенты, так как $\Lambda Y_l + \lambda Y_l = 0$, только если

\begin{equation}
\frac{d^2 f}{d\rho^2} + \frac{2(l+1)}{\rho} \frac{df}{d\rho} - 2 \sqrt{\varepsilon} \frac{df}{d\rho} + \frac{2}{\rho} (1 - \sqrt{\varepsilon} (l+1)) f = 0.
\end{equation}

Попытаемся решить его, разложив $f$ в ряд по степеням $\rho$ (или, лучше, по степеням $2\rho \sqrt{\varepsilon}$); запишем соответственно

\begin{equation}
f = \sum_{v=0}^{\infty} a_v (2\rho \sqrt{\varepsilon})^v.
\end{equation}

Подставляя это выражение в дифференциальное уравнение и несколько иначе располагая члены, получаем

\begin{equation}
\sum_{v=0}^{\infty} a_v (2\rho \sqrt{\varepsilon})^{v-2v} (v + 2l + 1) - \sum_{v=0}^{\infty} a_v (2\rho \sqrt{\varepsilon})^{v-1} \left( v + l + 1 - \frac{1}{\sqrt{\varepsilon}} \right) = 0.
\end{equation}

Этот ряд должен тождественно обращаться в нуль; таким образом, мы приходим к рекуррентному соотношению для коэффициентов

\begin{equation}
a_{v+1} (v + 1)(v + 2l + 2) = a_v \left( v + l + 1 - \frac{1}{\sqrt{\varepsilon}} \right).
\end{equation}

Разумеется, в начале координат функция $f$ конечна и равна первому члену, как показывает подгоня анализ, быстрее, чем $e^{+\rho \sqrt{\varepsilon}}$ во всех случаях, кроме того, когда ряд для $f$ обрывается на каком-то члене, превращаясь в конечный полином. В этом последнем случае $f$ и обращается в бесконечность, но $R$ на бесконечности исчезает благодаря экспоненциальному множителю $e^{-\rho \sqrt{\varepsilon}}$. Условие, при котором происходит обрыв ряда, получается из рекуррентного соотношения. Ряд обрывается на $n_r$-м члене, если

\begin{equation}
n_r + l + 1 = \frac{1}{\sqrt{\varepsilon}}.
\end{equation}

Таким образом, $1/\sqrt{\varepsilon}$ должно быть положительным целым числом или

\begin{equation}
\varepsilon = \frac{1}{n^2},
\end{equation}
%
где $n = n_r + l + 1$; число $n$ есть главное квантовое число, а $n_r$ — радиальное квантовое число.

Итак, мы видим, что решения дифференциального уравнения, существуют только для некоторых дискретных значений параметров $\varepsilon$, именно для значений $\varepsilon = 1/n^2$. Значит, возможны лишь некоторые определенные энергетические уровни, именно

\begin{equation}
E_n = - \frac{2\pi^2 me^4 Z^2}{h^2 n^2} = - \frac{h R_\infty Z^2}{n^2},
\end{equation}
%
которые и дает теория Бора.

Для $n=1$ имеем $l=0$, $n_r=0$ и $f$ сводится к постоянной. Волновая функция $\psi$ не зависит от $\theta$ и $\phi$, и если принять нормировку

\begin{equation}
\int \psi^2 dr = 1,
\end{equation}
%
получим просто

\begin{equation}
\psi = \frac{1}{\sqrt{\pi}} \left( \frac{Z}{a_1} \right)^{3/2} e^{-Z r / a_1}, \quad a_1 = \frac{h^2}{4 \pi^2 m e^2}.
\end{equation}

Можно видеть, что $\psi$ быстро уменьшается при $r > a_1/Z$. Величина $a_1/Z$ есть радиус первой боровской орбиты для атома с зарядом $Ze$.
%
Он равен среднему значению (см. приложение 25) $\overline{r} = \int \psi^2 r dr$ для электронного облака в состоянии $n=1$. Такие средние величины можно вычислить для всех высших состояний с $l=0$, и, как оказалось, они совпадают с главными полусами эллипсов теории Бора.

Можно добавить, что полиномы $f$ --- это известные полиномы Лагерра. Однако мы не будем вдаваться в подробности; заметим только, что нули их определяют положение узловых поверхностей в нуле. Функция $R$ имеет $n_r$ узлов, не считая нулеи при $r=0$ (в случае $l>0$) и при $r=\infty$.

\end{document}