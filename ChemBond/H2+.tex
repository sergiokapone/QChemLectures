% !TeX program = lualatex
% !TeX encoding = utf8
% !TeX spellcheck = uk_UA

\documentclass{article}
\usepackage[fontsize=14pt]{fontsize}

\usepackage{fontsetup}
\usepackage[english, russian]{babel}
\usepackage{microtype}

\usepackage{amsmath}
%\usepackage{amsfonts}
%\usepackage{amssymb}
\usepackage{graphicx}
\usepackage{geometry}
\usepackage{fancyhdr}
\usepackage{xcolor}
\usepackage{tcolorbox}
\usepackage{tikz}
\usepackage{pgfplots}
\usepackage{booktabs}
\usepackage{multirow}
\usepackage{hyperref}
\usepackage{listings}

\geometry{margin=2cm}
\pgfplotsset{compat=1.15}

% Настройка заголовков
\pagestyle{fancy}
\fancyhf{}
\fancyhead[L]{\textbf{МО-ЛКАО метод}}
\fancyhead[R]{\textbf{Страница \thepage}}
\renewcommand{\headrulewidth}{0.4pt}

% Цветовые схемы для блоков
\definecolor{theoremcolor}{RGB}{230,230,250}
\definecolor{examplecolor}{RGB}{240,248,255}
\definecolor{warningcolor}{RGB}{255,245,238}
\definecolor{importantcolor}{RGB}{255,240,245}

\newtcolorbox{theorembox}[1]{
    colback=theoremcolor,
    colframe=blue!75!black,
    title={\textbf{#1}},
    fonttitle=\bfseries,
    rounded corners
}

\newtcolorbox{examplebox}[1]{
    colback=examplecolor,
    colframe=cyan!75!black,
    title={\textbf{#1}},
    fonttitle=\bfseries,
    rounded corners
}

\newtcolorbox{warningbox}[1]{
    colback=warningcolor,
    colframe=orange!75!black,
    title={\textbf{⚠️ #1}},
    fonttitle=\bfseries,
    rounded corners
}

\newtcolorbox{importantbox}[1]{
    colback=importantcolor,
    colframe=red!75!black,
    title={\textbf{🎯 #1}},
    fonttitle=\bfseries,
    rounded corners
}

\title{\textbf{\Huge Метод МО-ЛКАО}\\
\Large Полное методическое пособие для студентов}
\author{\textbf{Квантовая химия и вычислительные методы}}
\date{\today}

\begin{document}

\maketitle

\tableofcontents
\newpage

\section{Введение}

\begin{importantbox}{Цели изучения}
После изучения данного материала студент должен:
\begin{itemize}
\item Понимать физические основы МО-ЛКАО метода
\item Уметь применять вариационный принцип для определения коэффициентов
\item Осознавать ограничения приближения и пути их преодоления
\item Понимать необходимость расширения базисных наборов
\end{itemize}
\end{importantbox}

Метод молекулярных орбиталей в приближении линейной комбинации атомных орбиталей (МО-ЛКАО) является одним из фундаментальных подходов в квантовой химии. Этот метод позволяет описать электронную структуру молекул, используя атомные орбитали как строительные блоки.

\section{Теоретические основы}

\subsection{Постановка задачи}

Рассмотрим простейшую молекулярную систему -- ион водорода H$_2^+$, состоящий из двух протонов и одного электрона.

\begin{theorembox}{Основная идея МО-ЛКАО}
Молекулярная орбиталь $\psi$ представляется в виде линейной комбинации атомных орбиталей:
\begin{equation}
\psi = c_1\phi_1 + c_2\phi_2 + \ldots + c_N\phi_N = \sum_{i=1}^{N} c_i\phi_i
\end{equation}
где $\phi_i$ -- атомные орбитали, $c_i$ -- коэффициенты разложения.
\end{theorembox}

Для иона H$_2^+$ в минимальном базисе:
\begin{equation}
\psi = c_1\phi_1 + c_2\phi_2
\end{equation}
где $\phi_1$ и $\phi_2$ -- 1s орбитали водорода на ядрах A и B соответственно.

\subsection{Вариационный принцип}

\begin{theorembox}{Вариационный принцип Рэлея-Ритца}
Для любой нормированной пробной функции $\psi_{\text{пробн}}$ выполняется:
\begin{equation}
E_{\text{пробн}} = \frac{\langle\psi_{\text{пробн}}|\hat{H}|\psi_{\text{пробн}}\rangle}{\langle\psi_{\text{пробн}}|\psi_{\text{пробн}}\rangle} \geq E_{\text{точн}}
\end{equation}
где $E_{\text{точн}}$ -- точная энергия основного состояния.
\end{theorembox}

Это означает, что варьируя параметры пробной функции и минимизируя энергию, мы получаем наилучшее приближение в рамках выбранного функционального пространства.

\section{Вывод секулярного уравнения}

\subsection{Функционал энергии}

Для ЛКАО функции $\psi = c_1\phi_1 + c_2\phi_2$ энергия записывается как:
\begin{equation}
E = \frac{\langle\psi|\hat{H}|\psi\rangle}{\langle\psi|\psi\rangle} = \frac{c_1^2H_{11} + c_2^2H_{22} + 2c_1c_2H_{12}}{c_1^2S_{11} + c_2^2S_{22} + 2c_1c_2S_{12}}
\end{equation}

где введены обозначения:
\begin{align}
H_{ij} &= \langle\phi_i|\hat{H}|\phi_j\rangle \quad \text{(матричные элементы гамильтониана)} \\
S_{ij} &= \langle\phi_i|\phi_j\rangle \quad \text{(интегралы перекрывания)}
\end{align}

\subsection{Условия минимума}

Применяя условия экстремума:
\begin{equation}
\frac{\partial E}{\partial c_1} = 0, \quad \frac{\partial E}{\partial c_2} = 0
\end{equation}

После алгебраических преобразований получаем систему линейных уравнений:
\begin{align}
c_1(H_{11} - ES_{11}) + c_2(H_{12} - ES_{12}) &= 0 \\
c_1(H_{21} - ES_{21}) + c_2(H_{22} - ES_{22}) &= 0
\end{align}

\begin{theorembox}{Секулярное уравнение}
Система имеет нетривиальное решение только при условии:
\begin{equation}
\begin{vmatrix}
H_{11} - ES_{11} & H_{12} - ES_{12} \\
H_{21} - ES_{21} & H_{22} - ES_{22}
\end{vmatrix} = 0
\end{equation}
\end{theorembox}

\section{Решение для H$_2^+$}

\subsection{Матричные элементы}

Для симметричной системы H$_2^+$:
\begin{align}
H_{11} = H_{22} &= \alpha \quad \text{(кулоновские интегралы)} \\
H_{12} = H_{21} &= \beta \quad \text{(резонансные интегралы)} \\
S_{11} = S_{22} &= 1 \quad \text{(нормировка)} \\
S_{12} = S_{21} &= S \quad \text{(интеграл перекрывания)}
\end{align}

\subsection{Собственные значения энергии}

Секулярное уравнение принимает вид:
\begin{equation}
\begin{vmatrix}
\alpha - E & \beta - ES \\
\beta - ES & \alpha - E
\end{vmatrix} = 0
\end{equation}

Раскрывая определитель:
\begin{equation}
(\alpha - E)^2 - (\beta - ES)^2 = 0
\end{equation}

\begin{examplebox}{Решение квадратного уравнения}
\begin{align}
\alpha - E &= \pm(\beta - ES) \\
E_{1,2} &= \frac{\alpha \pm \beta}{1 \pm S}
\end{align}

\textbf{Два решения:}
\begin{itemize}
\item \textbf{Связывающая МО:} $E_1 = \frac{\alpha + \beta}{1 + S}$ (нижняя энергия)
\item \textbf{Антисвязывающая МО:} $E_2 = \frac{\alpha - \beta}{1 - S}$ (верхняя энергия)
\end{itemize}
\end{examplebox}

\subsection{Собственные функции}

\subsubsection{Связывающая МО}

Для $E_1 = \frac{\alpha + \beta}{1 + S}$ из первого уравнения системы:
\begin{equation}
c_1(\alpha - E_1) + c_2(\beta - E_1S) = 0
\end{equation}

Поскольку $\alpha - E_1 = \beta - E_1S$, получаем $c_1 = c_2$.

Из условия нормировки:
\begin{equation}
c_1^2 + c_2^2 + 2c_1c_2S = 1 \Rightarrow 2c_1^2(1 + S) = 1
\end{equation}

\begin{equation}
c_1 = c_2 = \frac{1}{\sqrt{2(1 + S)}}
\end{equation}

\textbf{Связывающая МО:}
\begin{equation}
\boxed{\psi_1 = \frac{1}{\sqrt{2(1 + S)}}(\phi_1 + \phi_2)}
\end{equation}

\subsubsection{Антисвязывающая МО}

Аналогично для $E_2 = \frac{\alpha - \beta}{1 - S}$ получаем $c_1 = -c_2$:

\textbf{Антисвязывающая МО:}
\begin{equation}
\boxed{\psi_2 = \frac{1}{\sqrt{2(1 - S)}}(\phi_1 - \phi_2)}
\end{equation}

\section{Критический анализ приближения}

\subsection{Подстановка в уравнение Шрёдингера}

\begin{warningbox}{Ключевой вопрос}
Является ли МО-ЛКАО функция точным решением уравнения Шрёдингера?
\end{warningbox}

Проверим: $\hat{H}\psi_{\text{ЛКАО}} \stackrel{?}{=} E_{\text{ЛКАО}}\psi_{\text{ЛКАО}}$

\subsubsection{Молекулярный гамильтониан}

Для H$_2^+$:
\begin{equation}
\hat{H} = -\frac{1}{2}\nabla^2 - \frac{1}{r_A} - \frac{1}{r_B} + \frac{1}{R_{AB}}
\end{equation}

\subsubsection{Действие на атомную орбиталь}

\begin{align}
\hat{H}\phi_1 &= \left(-\frac{1}{2}\nabla^2 - \frac{1}{r_A} - \frac{1}{r_B} + \frac{1}{R_{AB}}\right)\phi_1 \\
&= \underbrace{-\frac{1}{2}\nabla^2\phi_1 - \frac{1}{r_A}\phi_1}_{E_{1s}\phi_1} + \underbrace{-\frac{1}{r_B}\phi_1 + \frac{1}{R_{AB}}\phi_1}_{\text{возмущение}}
\end{align}

где $E_{1s} = -\frac{1}{2}$ а.е. -- энергия 1s орбитали водорода.

\subsubsection{Полный результат}

\begin{align}
\hat{H}\psi_{\text{ЛКАО}} &= \frac{1}{\sqrt{2(1+S)}}\bigl[E_{1s}(\phi_1 + \phi_2) \\
&\quad + \left(-\frac{1}{r_B} + \frac{1}{R_{AB}}\right)\phi_1 + \left(-\frac{1}{r_A} + \frac{1}{R_{AB}}\right)\phi_2\bigr]
\end{align}

\begin{importantbox}{Критическое наблюдение}
В левой части уравнения присутствуют \textbf{дополнительные слагаемые}:
\begin{equation}
\left(-\frac{1}{r_B} + \frac{1}{R_{AB}}\right)\phi_1 + \left(-\frac{1}{r_A} + \frac{1}{R_{AB}}\right)\phi_2
\end{equation}
которых \textbf{НЕТ} в правой части!
\end{importantbox}

\subsection{Анализ "малости" дополнительных слагаемых}

\begin{warningbox}{Распространённое заблуждение}
Часто утверждается, что слагаемые $\left(-\frac{1}{r_B} + \frac{1}{R_{AB}}\right)\phi_1$ малы и ими можно пренебречь. Это \textbf{НЕ всегда верно}!
\end{warningbox}

\subsubsection{Численный анализ}

\begin{table}[h]
\centering
\begin{tabular}{@{}lcccc@{}}
\toprule
\textbf{Расстояние} $R_{AB}$ & $\frac{1}{R_{AB}}$ & $\langle\phi_1|-\frac{1}{r_B}|\phi_1\rangle$ & \textbf{Разность} & \textbf{Качество} \\
\midrule
5.0 Å & 0.2 & -0.05 & 0.15 & Хорошее ✅ \\
2.5 Å & 0.4 & -0.25 & 0.15 & Умеренное ⚠️ \\
1.0 Å & 1.0 & -0.8 & 0.2 & Плохое ❌ \\
0.5 Å & 2.0 & -1.5 & 0.5 & Очень плохое ❌❌ \\
\bottomrule
\end{tabular}
\caption{Качество МО-ЛКАО приближения в зависимости от межъядерного расстояния}
\end{table}

\section{Ограничения метода и пути улучшения}

\subsection{Основные проблемы МО-ЛКАО}

\begin{enumerate}
\item \textbf{Жёсткость атомных орбиталей}
\begin{itemize}
\item Атомные орбитали не деформируются при образовании связи
\item Игнорируется поляризация под влиянием соседних ядер
\item Неучёт релаксации при изменении окружения
\end{itemize}

\item \textbf{Неполнота минимального базиса}
\begin{itemize}
\item Всего 2 функции для описания бесконечномерного пространства
\item Отсутствие гибкости для описания возбуждённых состояний
\item Невозможность описания поляризации
\end{itemize}

\item \textbf{Одноэлектронное приближение}
\begin{itemize}
\item Игнорирование электронной корреляции
\item Неучёт многочастичных эффектов
\item Приближение среднего поля
\end{itemize}
\end{enumerate}

\subsection{Стратегии улучшения}

\subsubsection{Расширение базиса}

\begin{theorembox}{Иерархия базисных наборов}
\begin{align}
\text{Минимальный:} \quad &\psi = c_1\phi_{1s} + c_2\phi_{2s} \\
\text{Удвоенный дзета:} \quad &\psi = c_1\phi_{1s} + c_2\phi_{2s} + c_3\phi_{1s}' + c_4\phi_{2s}' \\
\text{Поляризационный:} \quad &\psi = c_1\phi_{1s} + c_2\phi_{2s} + c_3\phi_{2p_x} + c_4\phi_{2p_y} + c_5\phi_{2p_z}
\end{align}
\end{theorembox}

\section{Численный пример для H$_2^+$}

\subsection{Исходные данные}

При равновесном расстоянии $R_{AB} = 2.5$ Å:
\begin{align}
S &= \langle\phi_1|\phi_2\rangle = 0.586 \\
\alpha &= \langle\phi_1|\hat{H}|\phi_1\rangle = -13.61 \text{ эВ} \\
\beta &= \langle\phi_1|\hat{H}|\phi_2\rangle = -1.76 \text{ эВ}
\end{align}

\subsection{Результаты расчёта}

\begin{examplebox}{Энергии молекулярных орбиталей}
\begin{align}
E_1 &= \frac{\alpha + \beta}{1 + S} = \frac{-13.61 + (-1.76)}{1 + 0.586} = -9.69 \text{ эВ} \\
E_2 &= \frac{\alpha - \beta}{1 - S} = \frac{-13.61 - (-1.76)}{1 - 0.586} = -28.62 \text{ эВ}
\end{align}
\end{examplebox}

\begin{examplebox}{Коэффициенты МО}
\textbf{Связывающая МО:}
\begin{equation}
c_1 = c_2 = \frac{1}{\sqrt{2(1 + S)}} = \frac{1}{\sqrt{2(1.586)}} = 0.563
\end{equation}

\textbf{Антисвязывающая МО:}
\begin{equation}
c_1 = -c_2 = \pm\frac{1}{\sqrt{2(1 - S)}} = \pm\frac{1}{\sqrt{2(0.414)}} = \pm 0.778
\end{equation}
\end{examplebox}

\section{Задачи для самостоятельного решения}

\begin{examplebox}{Задача 1: HeH$^+$ -- ион гелий-гидрида}
Рассмотрите ион HeH$^+$ в МО-ЛКАО приближении:
\begin{enumerate}
\item Запишите МО как $\psi = c_1\phi_{\text{He}} + c_2\phi_{\text{H}}$
\item Обсудите, почему $H_{11} \neq H_{22}$ (несимметричная система)
\item Как изменится секулярное уравнение?
\item Какая орбиталь (He или H) внесёт больший вклад в связывающую МО?
\end{enumerate}
\end{examplebox}

\section{Заключение}

\begin{importantbox}{Ключевые выводы}
\begin{enumerate}
\item МО-ЛКАО метод даёт \textbf{качественно правильную} картину химической связи
\item Количественная точность требует \textbf{расширения базиса} и учёта корреляций
\item Метод является \textbf{отправной точкой} для более сложных подходов
\item Понимание ограничений критически важно для правильной интерпретации результатов
\end{enumerate}
\end{importantbox}

МО-ЛКАО метод остаётся фундаментальным инструментом квантовой химии благодаря:
\begin{itemize}
\item Ясной физической интерпретации
\item Связи с химической интуицией
\item Вычислительной эффективности
\item Возможности систематического улучшения
\end{itemize}

\section{Рекомендуемая литература}

\begin{enumerate}
\item Левин А.А. \textit{Квантовая химия}. -- М.: Высшая школа, 2019.
\item Цирельсон В.Г. \textit{Квантовая химия. Молекулы, молекулярные системы и твёрдые тела}. -- М.: БИНОМ, 2017.
\item Szabo A., Ostlund N.S. \textit{Modern Quantum Chemistry}. -- Dover, 1996.
\end{enumerate}

\end{document}