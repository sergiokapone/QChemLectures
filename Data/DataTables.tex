% !TeX program = lualatex
% !TeX encoding = utf8
% !TeX spellcheck = uk_UA

\documentclass[]{article}
\usepackage[fontsize=14pt]{fontsize}

\usepackage{fontspec}
\setsansfont{CMU Sans Serif}%{Arial}
\setmainfont{CMU Serif}%{Times New Roman}
\setmonofont{CMU Typewriter Text}%{Consolas}

\defaultfontfeatures{Ligatures={TeX}}
\usepackage[math-style=TeX]{unicode-math}
\usepackage[russian, english, ukrainian]{babel}

\usepackage{biblatex}

\usepackage{microtype}
\usepackage{xurl}
\usepackage{indentfirst}

\usepackage[most]{tcolorbox}
\tcbset{highlight math style={enhanced,
  colframe=red,colback=white,arc=0pt,boxrule=1pt}}
\tcbset{
    mybox/.style={
        enhanced,
        colback=gray!5,
        fonttitle=\bfseries,
        sharp corners,
        boxrule=0pt,
        frame empty
    }
}
\usepackage{subcaption}
\captionsetup[subfigure]{justification=centering}
\usepackage{hyperref}

\usepackage{titlesec}
\titlelabel{\thetitle.\space}
\usepackage{minted}
\usepackage{tikz, calc}
\usetikzlibrary{backgrounds}

\usepackage{enumitem}
\usepackage{tabularray}


\addbibresource{../Bibliography/QuantumChemistry.bib}

%%                                                                        %%
%%============================= Геометрія сторінки =======================%%
%%                                                                        %%
\usepackage[%
	a4paper,%
	footskip=1cm,%
	headsep=0.3cm,%
	top=2cm, %поле сверху
	bottom=2cm, %поле снизу
	left=2cm, %поле ліворуч
	right=2cm, %поле праворуч
    ]{geometry}
%%                                                                        %%
%%============================== Інтерліньяж  ============================%%

\usepackage{tikz}
\usepackage{pgfplots}
\usepgfplotslibrary{units}
\pgfplotsset{compat=newest}
\usepackage{pgfplotstable}
\usepgfplotslibrary{groupplots}

\usepackage{mhchem}[version=4.10]



\def\C#1{{\color{teal}#1}}
\def\a#1{{\color{magenta}#1}}
\def\sgto#1{\left(\frac{2\cdot\a{#1}}{\pi}\right)^{3/4} e^{-\a{#1}\cdot r^2}}
\NewDocumentCommand{\He}{}{
\tikz[baseline=-5pt]{
    \coordinate (dv) at (0,0);
    \coordinate (base) at (50pt, 0pt);
    \coordinate (height) at (0pt,50pt);
    \coordinate (diag) at ($(base)+(height)$);
    \path[draw=teal!80!black, rounded corners=2pt, fill=teal, thick] ($(dv)-.5*(diag)$) rectangle +(diag);
    \node[white] at (dv) {\sffamily\huge He};
    \node[white, inner sep=2pt] (dvtext) at ($(dv)-.5*(height)$) [anchor=south] {\sffamily\tiny Helium};
    \node[white, inner sep=2pt] (dvnum) at ($(dv)+.5*(height)-.5*(base)$) [anchor=north west] {\sffamily\tiny 2};
    \node[white, inner sep=2pt] (elestruct) at ($(dv)+.5*(height)+.1*(base)$) [anchor=north west] {\sffamily\tiny $1s^2$};
}
}

\title{\bfseries Таблиці даних}
\date{}
%===============================================================================
\begin{document}
\maketitle

\begin{table}\centering
\caption{Точні значення енергії основного стану для деяких атомів}
\begin{tblr}{
colspec={QQX[l, m]X[c, m]},
row{1} = {c, bg=gray!50},
hlines,
vlines
}
    $Z$ & Атом & Конфігурація                     & Точне значення енергії, Eh  \\
    2   & He   & \ce{1s^2}               \ce{^1S} & −2.903385       \\
    3   & Li   & \ce{1s^2 2s}            \ce{^2S} & −7.477976       \\
    4   & Be   & \ce{1s^2 2s^2}          \ce{^1S} & −14.668449      \\
    5   & B    & \ce{1s^2 2s^2 2p}       \ce{^2P} & −24.658211      \\
    6   & C    & \ce{1s^2 2s^2 2p^2}     \ce{^3P} & −37.855668      \\
    7   & N    & \ce{1s^2 2s^2 2p^3}     \ce{^4S} & −54.611893      \\
    8   & O    & \ce{1s^2 2s^2 2p^4}     \ce{^3P} & −75.109991      \\
    9   & F    & \ce{1s^2 2s^2 2p^5}     \ce{^2P} & −99.803888      \\
    10  & Ne   & \ce{1s^2 2s^2 2p^6}     \ce{^1S} & −128.830462     \\
    11  & Na   & \ce{[Ne] 3s}            \ce{^2S} & −162.428221     \\
    12  & Mg   & \ce{[Ne] 3s^2}          \ce{^1S} & −200.309935     \\
    13  & Al   & \ce{[Ne] 3s^2 3p}       \ce{^2P} & −242.712031     \\
    14  & Si   & \ce{[Ne] 3s^2 3p2}      \ce{^3P} & −289.868255     \\
    15  & P    & \ce{[Ne] 3s^2 3p^3}     \ce{^4S} & −341.946219     \\
    16  & S    & \ce{[Ne] 3s^2 3p^4}     \ce{^3P} & −399.034923     \\
    17  & Cl   & \ce{[Ne] 3s^2 3p^5}     \ce{^2P} & −461.381223     \\
    18  & Ar   & \ce{[Ne] 3s^2 3p^6}     \ce{^1S} & −529.112009     \\
    19  & K    & \ce{[Ar] 4s}            \ce{^2S} & −601.967492     \\
    20  & Ca   & \ce{[Ar] 4s^2}          \ce{^1S} & −680.101971
\end{tblr}
\end{table}


\begin{table}\centering
\caption{Молекулярні коливальні константи основного стану для вибраних двоатомних молекул}
\begin{tblr}{
colspec={QQ[c, m]Q[c, m]Q[c, m]},
row{1} = {c, bg=gray!50},
hlines,
vlines
}
    Molecule & $\mu$ (а. о.) & $E$, Eh & $R$, \AA & $k$, Н/м & $\omega$,  см$^{-1}$   \\
    \ce{H2}  & 0.50          &         & 0.742    & 570      & 4401.21                \\
    \ce{HF}  & 0.96          &         & 0.917    & 970      & 4138.32                \\
    \ce{HCl} & 0.98          &         & 1.275    & 520      & 2990.95                \\
    \ce{HBr} & 1.00          &         & 1.414    & 410      & 2649.67                \\
    \ce{HI}  & 1.00          &         & 1.609    & 310      & 2309.60                \\
    \ce{CO}  & 6.86          &         & 1.128    & 1900     & 2169.82                \\
    \ce{N2}  & 7.00          &         & 1.098    & 2290     & 2358.07                \\
    \ce{NO}  & 7.47          &         & 1.151    & 1600     & 1904.41                \\
    \ce{O2}  & 8.00          &         & 1.207    & 1180     & 1580.36                \\
    \ce{F2}  & 9.50          &         & 1.418    & 440      & 891.20                 \\
    \ce{Cl2} & 17.48         &         & 1.988    & 320      & 560.50                 \\
    \ce{Br2} & 39.46         &         & 2.670    & 250      & 325.29                 \\
    \ce{IBr} & 48.66         &         & 2.470    & 210      & 268.71                 \\
    \ce{I2}  & 63.45         &         & 2.664    & 170      & 214.52
\end{tblr}
\end{table}

\end{document}
