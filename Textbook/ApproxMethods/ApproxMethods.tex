% !TeX program = lualatex
% !TeX encoding = utf8
% !TeX spellcheck = uk_UA
% !BIB program = biber
% !TeX root =../QChem.tex


\chapter{Наближені методи квантової механіки}

Лише невелика кількість задач квантової механіки можуть бути розв'язані математично точно. До таких задач відноситься, наприклад, задача про атом водню. Для систем, які складаються з багатьох частинок (атомів та молекул) точний розрахунок хвильової функції вже неможливий. Якщо відволіктись від проблем, які пов'язані з релятивістськими ефектами, то основна причина, яка не дає точно розв'язати рівняння Шредінґера полягає у взаємодії між електронами, саме наявність оператора міжелектронної взаємодії не дає можливості розділити змінні і отримати точний розв'язок.

Внаслідок цього практична придатність квантової механіки в значній мірі залежить від того, наскільки добре вдасться розробити наближені методи розрахунку хвильової функцій і фізичних величин. Однак, розробка наближених методів ні в якому разі не  обмежується лише завданням чисельно визначити фізичні або фізико-хімічні величини. Ці методи, крім цього, ще й повинні сприяти систематизації та інтерпретації емпіричних даних, розуміння основних закономірностей, формування нових модельних уявлень і розробці на їх основі загальних прогнозів.

Це стосується і прикладанню квантової механіки до проблем хімічного зв'язку і реакційної здатності --- тобто, до квантової хімії. Від результатів квантової хімії слід вимагати, щоб вони відповідали якісним правилам і уявленням про структуру і властивості молекул, які отримані з великого числа експериментальних даних. 



