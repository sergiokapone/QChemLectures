% !TeX program = lualatex
% !TeX encoding = utf8
% !TeX spellcheck = uk_UA
% !TeX root =../EMProblems.tex

%========================================================================================================
%
%									      Палітурка
%
%========================================================================================================

\newcommand{\CoverPage}{
	\begin{tikzpicture}
	[
	remember picture,
	overlay,
	font=\sffamily\bfseries\huge,
	inner sep=0pt,
	outer sep=0pt,
	]
%	\node at (current page.center) {
%	\includegraphics[height=\paperheight]{Title}
%	};
	\foreach \i in {0,1,...,10}
	{
	\path (current page.west) to coordinate [pos=\i/10] (c\i) (current page.east);
	%\filldraw [red] (c\i) circle (1pt);
	}
	%\draw[fill=white,opacity=0.4] (current page.west |- 0,0) rectangle (current page.east |- 0,1);
	%\node [scale=0.7, anchor=west] at (c1 |- 0,0.4) {Кафедра фізики енергетичних систем};
	\draw[fill=olive,opacity=0.4] (current page.west |- 0,-10) rectangle (current page.east |- 0,-15);
	\node [text width=1\linewidth, anchor=west] at (c1 |- 0,-20) {\realtitle};
	\node [anchor=west] at (c1 |- 0,-22) {\LARGE \authors};
	\end{tikzpicture} 
}

%========================================================================================================
%
%									      Титульна сторінка
%
%========================================================================================================

\renewcommand\maketitle{
	\begin{alwayssingle}
		\begin{center}
			\begin{flushright}\bfseries\sffamily
				\MakeUppercase{Міністерство освіти і науки України}\\
				КПІ ім. Ігоря Сікорського\\
			\end{flushright}
			\begin{tcolorbox}[titlepagestyle,
					toprule=0.10cm,
					bottomrule=0.10cm,
					overlay={%
						\node (picture) at ([xshift=4cm]frame.west) {\includegraphics{logo_PTI}};
					}
			]%
			\begin{flushright}
				\large\bfseries\color{white}Фізико-технічний інститут
			\end{flushright}
			\end{tcolorbox}
%			\vspace*{-2em}
%		 	\begin{Large}\color{themecolordark!90!black}
%			\begin{equation}
%				\vect{\nabla}&\times&&\Efield &&= -\dfrac{1}{c}\dfrac{\partial\Bfield}{\partial t} \\
%				\vect{\nabla}&\,\cdot&&\Bfield &&= 0 \\
%				\vect{\nabla}&\,\cdot&&\Dfield &&= 4\pi\rho \\ 
%				\vect{\nabla}&\times&&\Hfield &&= \dfrac{4\pi}{c} \vect{j}+\dfrac{1}{c}\dfrac{\partial\Dfield}{\partial t} 
%			\end{equation}
%			\end{Large}
			\vspace*{10em}
			\begin{tcolorbox}[
				titlepagestyle,
				toprule=0.15cm,
				bottomrule=0.15cm,
				top=1.3cm,
				bottom=0.7cm,
				overlay={%
				\node[%
							fill=white,
							rounded corners = 15pt,	
							draw=themecolorlight,
							line width=0.15cm,
							inner sep=0pt,
							text width=8cm,
							minimum height=2cm,
							align=center,
							%anchor=east,
							font=\sffamily\bfseries\Large
						] (title) at (frame.north) {\authors};
				}
			]
			\centering
			\Huge\sffamily\bfseries\textcolor{white}{\realtitle}
			\end{tcolorbox}	
			\vfill 
			{\large\sffamily\bfseries{Навчальний посібник}\par}
			\vfill
			\ifelectronic \noindent Електронне видання \fi
			%\par {Версія від~\href{http://www.istpravda.com.ua/dates}{\today}} \par\else \par  \fi
			\vfill
			\begin{tcolorbox}[titlepagestyle,
					toprule=0.10cm,
					bottomrule=0.10cm]
				\begin{center}
					\color{white}\bfseries\normalsize\MakeUppercase{Київ~\the\year}
				\end{center}			       	
			\end{tcolorbox}
		\end{center}
		\clearpage
	\end{alwayssingle}	
}


%========================================================================================================
%
%									      Друга сторінка
%
%========================================================================================================
\newcommand\makeinfopage{
	\begin{alwayssingle}
		\noindent%	
		\begin{minipage}[t]{0.5\textwidth}
				\begin{flushleft}
					УДК  537\\
					ББК  22.3\\
					П 563
				\end{flushleft}
		\end{minipage}
		\hfill
		\begin{minipage}[t]{0.5\textwidth}
				\begin{flushleft}
					Рекомендовано Вченою радою ФТІ\\
					Протокол № 5/\the\year~від  25.04.\the\year~р.
				\end{flushleft}
		\end{minipage}
		
		\bigskip\noindent%
		\begin{flushleft}
			\begin{tabular}{p{4cm}p{0.65\textwidth}}
				\makecell[l]{Автор\\та укладач:}                   & \makecell[l]{\href{http://phes.ipt.kpi.ua/ponomarenko-sergij-mikolajovich}{\authors}, к.ф.-м.н., доцент \\кафедри фізики енергетичних систем ФТІ} \\
				                         &                                                                             \\
				\makecell[l]{Рецензент:}               & \makecell[l]{В.~О.~Кондаков, к.ф.-м.н., доцент кафедри прикладної\\фізики ФТІ}                 \\
				                         &                                                                             \\
				\makecell[l]{Відповідальний\\редактор:} & \makecell[l]{Т.~В.~Литвинова, к.т.н., доцент, в.~о. директора\\ФТІ}
			\end{tabular}
		\end{flushleft}
		
		\noindent%
		\begin{flushleft}
			\begin{tabular}{lp{0.9\textwidth}}
				     & \textbf{\href{http://phes.ipt.kpi.ua/ponomarenko-sergij-mikolajovich}{\authors}}                                                                                                                                                                        \\
				П 563 & \hspace*{3ex} \inlinetitle : навчальний посібник / \href{http://phes.ipt.kpi.ua/ponomarenko-sergij-mikolajovich}{\authors} --- К.:~КПІ ім. Ігоря Сікорського, \the\year. --~\pageref{LastPage}~с. -- Бібліогр.: с.~\pageref{BibPage}. \ifelectronic\relax\else-- 80~прим.\fi
			\end{tabular}
		\end{flushleft} 
		\vfill		 

		Для студентів фізико-технічного інституту КПІ ім. Ігоря Сікорського, які навчаються за спеціальністю 105~<<Прикладна фізика та наноматеріали>>.
		
		\vfill
				
		\begin{flushleft}\small
			Ілюстративний матеріал підручника підготовлений за допомогою пакету \href{http://pgf.sourceforge.net}{TikZ/Pgf}. Верстка тексту проведена в видавничій системі \LaTeXe{} (компілятор Lua\LaTeX) на базі системи комп'ютерної верстки \TeX{} (Збірка  \href{https://www.tug.org/texlive/}{\TeX Live~\the\year}) з використанням оболонки \href{https://www.texstudio.org}{\TeX Studio}.
		\end{flushleft}	
		\vfill
		
		\begin{flushleft}
			\begin{tabular}{p{\textwidth - 45ex}l}
				& \small\textcopyright\quad \authors, \the\year~р. \\
				& \small \textcopyright\quad КПІ ім. Ігоря Сікорського (ФТІ), \the\year~р.           
			\end{tabular}
		\end{flushleft}
		\newpage%
	\end{alwayssingle}
}




