%%============================ Compiler Directives =======================%%
%%                                                                        %%
% !TeX program = lualatex
% !TeX encoding = utf8
% !TeX spellcheck = uk_UA
%%                                                                        %%
%%============================== Клас документа ==========================%%
%%                                                                        %%
\documentclass[14pt]{extarticle}
\IfFileExists{ukrcorr.sty}{\usepackage{ukrcorr}}{}
%%                                                                        %%
%%========================== Мови, шрифти та кодування ===================%%
%%                                                                        %%
\usepackage{fontspec}
\setsansfont{CMU Sans Serif}%{Arial}
\setmainfont{CMU Serif}%{Times New Roman}
\setmonofont{CMU Typewriter Text}%{Consolas}
\defaultfontfeatures{Ligatures={TeX}}
\usepackage[math-style=TeX]{unicode-math}
\usepackage[english, russian, ukrainian]{babel}
\usepackage[most]{tcolorbox}

%%                                                                        %%
%%============================= Геометрія сторінки =======================%%
%%                                                                        %%
\usepackage[%
	a4paper,%
	footskip=1cm,%
	headsep=0.3cm,%
	top=2cm, %поле сверху
	bottom=2cm, %поле снизу
	left=2cm, %поле ліворуч
	right=2cm, %поле праворуч
    ]{geometry}
%%                                                                        %%
%%============================== Інтерліньяж  ============================%%
%%                                                                        %%
\renewcommand{\baselinestretch}{1}
%-------------------------  Подавление висячих строк  --------------------%%
\clubpenalty =10000
\widowpenalty=10000
%---------------------------------Інтервали-------------------------------%%
\setlength{\parskip}{0.5ex}%
\setlength{\parindent}{2.5em}%
%%                                                                        %%
%%=========================== Математичні пакети і графіка ===============%%
%%                                                                        %%
\usepackage{amsmath}
\usepackage{graphicx}
\usepackage{QuantumChemistry}
\usepackage{mhchem}
%%                                                                        %%
%%========================== Гіперпосилення (href) =======================%%
%%                                                                        %%
\usepackage[colorlinks=true,
	%urlcolor = blue, %Colour for external hyperlinks
	%linkcolor  = malina, %Colour of internal links
	%citecolor  = green, %Colour of citations
	bookmarks = true,
	bookmarksnumbered=true,
	unicode,
	linktoc = all,
	hypertexnames=false,
	pdftoolbar=false,
	pdfpagelayout=TwoPageRight,
	pdfauthor={Ponomarenko S.M. aka sergiokapone},
	pdfdisplaydoctitle=true,
	pdfencoding=auto
	]%
	{hyperref}
		\makeatletter
	\AtBeginDocument{
	\hypersetup{
		pdfinfo={
		Title={\@title},
		}
	}
	}
	\makeatother
%%                                                                        %%
%%============================ Заголовок та автори =======================%%
%%                                                                        %%
\title{Пост Хартрі-Фоківські методи для \ce{H2}}
\author{}
\date{}
%%                                                                        %%
%%========================================================================%%


\begin{document}
\maketitle


%% --------------------------------------------------------
\section{Методи CI}
%% --------------------------------------------------------

Найбільш послідовний підхід до урахування кореляційних ефектів заснований на понятті електронно-збуджених конфігурацій, кожній з яких ставиться у
відповідність певний слейтерівський детермінант. Електронно-збуджені конфігурації припускають інший розподіл електронів за спін-орбіталями у
порівнянні з тим, як це реалізується в методі Хартрі-Фока. У зв’язку з різними можливими способами розподілу електронів розрізняють
електронно-збуджені конфігурації (детермінанти) різної кратності збудження. Так, однократно-збуджені (single) конфігурації одержують, коли один з
електронів переміщають із зайнятої спін-орбіталі на вакантну.

%%---------------------------------------------------------
%\begin{figure}[h!]\centering
%    \begin{tikzpicture}[scale=0.95]

				\def\xshiftnode{1.15}
				\def\distance{0.75}
				\def\lofl{0.75}
				\foreach \i[count = \c from 0] in {1,...,3,5,6,8}{
						\foreach \j in {1,...,6}{
								\draw[ultra thick]  (\i*\xshiftnode,\j*\distance) coordinate (LD\i\j)
								\ifnum\i=1 node[left=5pt] (phi\j) {$\phi_{\j}$}\fi -- coordinate (O\i\j) ++(\lofl,0) coordinate (RU\i\j);
								\node (Ph\c) at ({\i*\xshiftnode+0.5*\lofl}, 0) {$\Phi_{\c}$};
							}
					}
				\path[dotted] (O33.east) -- node {$\ldots$} (O53.west);
				\path[dotted] (O63.east) -- node {$\ldots$} (O83.west);
				\path [
					updown=O11, updown=O12, updown=O13,
					updown=O21, updown=O22, up    =O23, down=O24,
					updown=O31, updown=O32, up    =O33, down=O35,
					updown=O51, updown=O52, up    =O54, down=O55,
					updown=O61, updown=O62, up    =O64, down=O66,
					updown=O81, up    =O82, down  =O83, up  =O84, down=O85, updown=O86,
				];
				\draw [curlybrace]   (Ph0.west) -- (Ph0.east) node[midway, below=15pt, font=\scriptsize]{Unexcited};
				\draw [curlybrace]   (Ph1.west) -- (Ph2.east) node[midway, below=15pt, font=\scriptsize]{Single Excited};
				\draw [curlybrace]   (Ph3.west) -- (Ph4.east) node[midway, below=15pt, font=\scriptsize]{Double Excited};
				\draw [curlybrace]   (Ph5.west) -- (Ph5.east) node[midway, below=15pt, font=\scriptsize]{Triple Excited};
				\draw [curlybracel]   (phi1.west) -- (phi3.west) node[midway, font=\scriptsize, sloped, above=5pt]{occupied};
				\draw [curlybracel]   (phi4.west) -- (phi6.west) node[midway, font=\scriptsize, sloped, above=5pt]{virtual};
				% ----------------------- Обводка ------------------------
				\fill[opacity=0.1, red!50]
				[stroke={LD11}{RU13}]
				[stroke={LD21}{RU24}]
				[stroke={LD31}{RU35}]
				[stroke={LD51}{RU55}]
				[stroke={LD61}{RU66}]
				[stroke={LD81}{RU86}]
				;

			\end{tikzpicture}
%\caption{}
%\label{}
%\end{figure}
%%---------------------------------------------------------


\end{document}


