% !TeX program = lualatex
% !TeX encoding = utf8
% !TeX spellcheck = uk_UA
% !BIB program = biber

\documentclass[]{beamer}
\usetheme{QuantumChemistry}
\usepackage{QuantumChemistry}

\graphicspath{{pictures/}}
\addbibresource{../Bibliography/QuantumChemistry.bib}
\AtEveryCitekey{\clearfield{url}\clearfield{doi}}
\usetikzlibrary{chains, positioning}
%\usepackage{ragged2e}


\newcommand\vertarrowbox[3][6ex]{%
  \begin{array}[t]{@{}c@{}} #2 \\
  \left\uparrow\vcenter{\hrule height #1}\right.\kern-\nulldelimiterspace\\
  \makebox[0pt]{\scriptsize#3}
  \end{array}%
}
\def\connect#1#2#3{%
    % #1: starting node
    % #2: ending node
    % #3: attributes for the shape connecting nodes
    \path let
      \p1 = ($(#2)-(#1)$),
      \n1 = {veclen(\p1)},
      \n2 = {atan2(\y1,\x1)} % <- Update
    in
      (#1) -- (#2) node[#3,
      midway,
      rotate = \n2,
      shading angle = \n2+90,
      minimum
      height=\n1,
      minimum
      width=5pt,
      inner sep=1pt] {};
            }
\title[Лекції з квантової хімії]{\huge\bfseries Пост Хартрі-Фоківські методи}
\subtitle{Лекції з квантової хімії}
\author{Пономаренко С. М.}
\date{}
\let\vphi\varphi
\def\vxi{\vec{\xi}}
\begin{document}
%=============================================================================

\begin{frame}
	\titlepage
	%	\epigraph{\itshape Существенный скрытый дефект метода Хартри-Фока состоит в пренебрежении электронной корреляцией в движении электронов с антипараллельными спинами.}{\fullcite{Popl:2002}}
\end{frame}
%=============================================================================


%%============================================================================
%\begin{frame}{Зміст}{}
%	\tableofcontents
%\end{frame}
%%============================================================================




%============================================================================
\begin{frame}{Сер Джон Ентоні Попл}{Лауреат Нобелівської премії по хімії (1998)}
	\begin{columns}
		\begin{column}{0.35\linewidth}
			\begin{center}
				\includegraphics[width=\linewidth]{pople_john}\\
				\href{https://en.wikipedia.org/wiki/John_Pople}{\scriptsize Попл Дж.~Е.  (1925 -- 2004)}
			\end{center}
		\end{column}
		\begin{column}{0.65\linewidth}

			\begin{block}{\small\fullcite{Popl:2002}}\itshape\justifying
				Істотний прихований дефект методу Хартрі-Фока полягає у нехтуванні електронною кореляцією в русі електронів з антипаралельними спінами ($\alpha\beta$-кореляція).

				\medskip

				При використанні однодетермінантних хвильових функцій неявно мається на увазі нехтування електронною $ \alpha\beta $-кореляцією; робота з уточненими хвильовими функціями неминуче означає використання кількох детермінантів.
			\end{block}

		\end{column}
	\end{columns}
\end{frame}
%============================================================================





%=============================================================================
%\begin{frame}[t]{Гамільтоніан системи}
%
%	\begin{multline*}
%		\hat{H} = \hat{T}_{n} + \hat{T}_{e} + \hat{V}_{en} + \hat{V}_{ee} + \ldots = \\ = -\frac12 \sum\limits_{i}^{N_e} \vec{\nabla}^2_i - \sum\limits_\alpha^{N_n}\sum\limits_{i}^{N_e} \frac{Z_\alpha}{r_i} + \frac12 \sum\limits_{i}^{N_e}\sum\limits_{\substack{j,\, j \neq i}}^{N_e}\frac{1}{r_{ij}}+\\ + \sum\limits_{\alpha}\sum\limits_{\beta, \alpha \neq \beta}\frac{Z_\alpha Z_\beta}{R_{\alpha\beta}} + \\ + \hat\varepsilon_{Orb-Spin} + \hat\varepsilon_{Spin-Spin} +\hat\varepsilon_{Orb-Orb},
%	\end{multline*}
%\end{frame}
%============================================================================






%============================================================================
\begin{frame}{Чому не точний метод Хартрі-Фока?}

	\begin{onlyenv}<1>
		Що точно і що не точно в методі Хартрі-Фока:
		\begin{enumerate}
			\item  Відносно добре описує \alert{стан молекулярних систем поблизу їх стійких (рівноважних) конфігурацій}, що відповідають точкам мінімуму на поверхнях потенціальної енергії основного електронного стану;
			\item дає \alert{некоректні оцінки} не тільки \alert{енергій дисоціації}, а й \alert{енергетичних бар'єрів}, що відповідають перетворенню однієї стійкої молекулярної форми на іншу;
			\item не дозволяє розглядати \alert{збуджені електронні стани} молекул.
		\end{enumerate}
	\end{onlyenv}
	\begin{onlyenv}<2>
		Основні причини неточності:
		\begin{enumerate}
			\item Метод використовує \alert{наближення незалежних частинок}, а електронну взаємодію враховує як суму взаємодій кожного електрона із середньою електронною густиною інших електронів.

			      {\scriptsize\color{blue} Насправді, між усіма електронами існує миттєве кулонівське відштовхування, тобто, їхній рух корельований.}

			\item неповнота базисного набору функцій;
			\item нехтування релятивістськими ефектами;

			      {\scriptsize\color{blue} Важливо при розгляді важких атомів}

			\item відхилення від наближення Борна–Оппенгеймера.

			      {\scriptsize\color{blue} Відхилення від наближення Борна-Оппенгеймера зазвичай незначні для основного стану молекули.}
		\end{enumerate}
	\end{onlyenv}
\end{frame}
\begin{frame}{Енергія електронної кореляції}
	\begin{center}
		\begin{tikzpicture}[scale=0.6]
			\draw[-latex] (0,3.5) -- (0,11) node[left] {$E$};
			\draw (0,10) -- +(1,0) node[right, text=red!90!blue] (Ehf) {Енергія Хартрі-Фока};
			\draw (0,8) -- +(1,0) node[right, text=red!70!blue] (HFlim) {Хартрі-Фоківський ліміт};
			\draw[->] (Ehf.south) -- node[right, font=\scriptsize] {Покращення базисного набору} (Ehf.south|-HFlim.north);
			\draw (0,6) node (A) {} -- +(1,0) node[right, text=red!50!blue] (PHF) {Пост Хартрі-Фоківські методи};
			\draw[->] (Ehf.south|-HFlim.south) coordinate (H) -- node[right, font=\scriptsize] {Врахування електронної кореляції} (H|-PHF.north);
			\draw (0,5) node (B) {} -- +(1,0) node[right, text width=7cm, text=red!30!blue] (Sh) {Точний розв'язок рівняння Шредінґера};
			\draw (0,4) -- +(1,0) node[right, text=red!10!blue] {Релятивістська енергія};

			\draw[<->] (A.west) -- ++(-0.5,0) -- coordinate (m)  ++(0,-1) -- (B.west);
			\node[text=red!80, left] at (m) {$E_\text{кор}$};
		\end{tikzpicture}
	\end{center}
	Результати розрахунків, виконаних за методом Хартрі-Фока, відрізняються від результатів, отриманих з точного розв'язку рівняння Шредінґера:
	\begin{equation*}
		E_\text{кор} = E_\text{точн} - E_\text{ХФ}
	\end{equation*}
	%Ці відмінності пов'язують з кореляційними ефектами, маючи на увазі під цим кулонівську кореляцію.
	{\scriptsize Створення точних і ефективних методів для визначення внеску кореляції досі є актуальним предметом досліджень в квантової хімії.}
\end{frame}
%============================================================================





%============================================================================
\begin{frame}{Пост Хартрі-Фоківські методи}{}
	\begin{enumerate}
		\item \href{https://en.wikipedia.org/wiki/Configuration_interaction}{\color{blue}Конфігураційна взаємодія} (англ. Configuration interaction, CI).
		\item \href{https://en.wikipedia.org/wiki/Multi-configurational_self-consistent_field}{\color{blue}Багатоконфігураційне самоузгоджене поле} (англ. Multiconfigurational Self-Consistent Field, MCSCF).
		\item \href{https://en.wikipedia.org/wiki/M\%C3\%B8ller\%E2\%80\%93Plesset_perturbation_theory}{\color{blue}Теорія збурень Меллера-Плессета} (англ. Møller–Plesset perturbation theory: MP2, MP3, MP4 ...).
		\item \href{https://en.wikipedia.org/wiki/Density_functional_theory}{\color{blue}Теорія функціоналу електронної густини} (англ. Density-functional theory, DFT).
	\end{enumerate}
\end{frame}
%============================================================================





%============================================================================
\begin{frame}[fragile]{Метод конфігураційної взаємодії}{}
	\framesubtitle<1-2>{Ідея методу}
	\begin{onlyenv}<1>
		\begin{itemize}\small
			\item  Розв'язок рівнянь Хартрі-Фока для основного стану атома гелію дає двічі зайняту сферично-симетричну орбіталь ($s$-орбіталь), на якій розміщуються два $\alpha$ та $\beta$ електрони. \alert{Кореляція не врахована}.

			\item Для врахування кулонівської кореляції необхідно включити в хвильову функцію з певним коефіцієнтом орбіталі із вищими орбітальними числами (наприклад, орбіталі $p_x$, $p_y$ та $p_z$), на яких можуть перебувати електрони.
		\end{itemize}
		\begin{center}
			\begin{tikzpicture}
				\coordinate (s) at (0,0);
				\orbital[pos = {(s)},  scale = 2, pcolor = red,]{s}
				\path [updown=s];
				\node[font=\Huge] at (2,0) {+};
				\coordinate (p) at (5,0);
				\orbital[pos = {(p)},  scale = 2, pcolor = red,]{pz}
				\pic at ([xshift=1cm]p) {upspinarrow};
				\orbital[pos = {(p)},  scale = 2, pcolor = blue,]{py}
				\pic at ([yshift=1cm]p) {downspinarrow};
			\end{tikzpicture}
		\end{center}
	\end{onlyenv}
	\begin{onlyenv}<2>
		\begin{enumerate}\small
			\item Розв'язок рівнянь Хартрі-Фока дає набір $M$ спін-орбіталей, а для побудови детермінанта Слейтера використовують лише $N$, які \alert{відповідають мінімальним орбітальним енергіям} ({\scriptsize\color{blue} Aufbau principle}).

			\item Частину з $M-N$ функцій, що залишилися (\alert{які відповідають віртуальним орбіталям}) використовують для \alert{побудови додаткових детермінантів Слейтера}.

			      \medskip

			      {{
						      \scriptsize Ці детермінанти отримують заміною певної кількості спін-орбіталей вихідного визначника $\Phi_{0}$ на відповідне число віртуальних спін-орбіталей. Одержані детермінанти називають збудженими і позначають як $\Phi_k$.

						      Хвильова функція має вигляд:
						      \begin{equation*}\label{}
							      \Phi = C_0\Phi_{0} + \sum_{k = 1}^{n} C_k\Phi_k,
						      \end{equation*}%
						      $ n $ --- число детермінантів.
					      }}%
			\item Задача пошуку $\Phi$ зводиться до варіаційної задачі з мінімізації електронної енергії шляхом варіювання коефіцієнтів $C_k$.

			      %        \item При формуванні набору $\{\Phi_K\}_{1,\ldots, L}$ часто \alert{обмежуються одно- та двократно-збудженими визначниками} по відношенню до $\Phi^{0}$ ({\scriptsize правило Кондона-Слейтера}). У такому варіанті метод називається ({\scriptsize\color{blue} CISD, Cl with single and double excitations}).
		\end{enumerate}
	\end{onlyenv}
	\framesubtitle<3>{Ілюстрація методу}
	\begin{onlyenv}<3>
		\begin{center}
			\begin{tikzpicture}[scale=0.95]

				\def\xshiftnode{1.15}
				\def\distance{0.75}
				\def\lofl{0.75}
				\foreach \i[count = \c from 0] in {1,...,3,5,6,8}{
						\foreach \j in {1,...,6}{
								\draw[ultra thick]  (\i*\xshiftnode,\j*\distance) coordinate (LD\i\j)
								\ifnum\i=1 node[left=5pt] (phi\j) {$\phi_{\j}$}\fi -- coordinate (O\i\j) ++(\lofl,0) coordinate (RU\i\j);
								\node (Ph\c) at ({\i*\xshiftnode+0.5*\lofl}, 0) {$\Phi_{\c}$};
							}
					}
				\path[dotted] (O33.east) -- node {$\ldots$} (O53.west);
				\path[dotted] (O63.east) -- node {$\ldots$} (O83.west);
				\path [
					updown=O11, updown=O12, updown=O13,
					updown=O21, updown=O22, up    =O23, down=O24,
					updown=O31, updown=O32, up    =O33, down=O35,
					updown=O51, updown=O52, up    =O54, down=O55,
					updown=O61, updown=O62, up    =O64, down=O64,
					updown=O81, up    =O82, down  =O83, up  =O84, down=O85, updown=O86,
				];
				\draw [curlybrace]   (Ph0.west) -- (Ph0.east) node[midway, below=15pt, font=\scriptsize]{Unexcited};
				\draw [curlybrace]   (Ph1.west) -- (Ph2.east) node[midway, below=15pt, font=\scriptsize]{Single Excited};
				\draw [curlybrace]   (Ph3.west) -- (Ph4.east) node[midway, below=15pt, font=\scriptsize]{Double Excited};
				\draw [curlybrace]   (Ph5.west) -- (Ph5.east) node[midway, below=15pt, font=\scriptsize]{Triple Excited};
				\draw [curlybracel]   (phi1.west) -- (phi3.west) node[midway, font=\scriptsize, sloped, above=5pt]{occupied};
				\draw [curlybracel]   (phi4.west) -- (phi6.west) node[midway, font=\scriptsize, sloped, above=5pt]{virtual};
				% ----------------------- Обводка ------------------------
				\fill[opacity=0.1, red!50]
				[stroke={LD11}{RU13}]
				[stroke={LD21}{RU24}]
				[stroke={LD31}{RU35}]
				[stroke={LD51}{RU55}]
				[stroke={LD61}{RU64}]
				[stroke={LD81}{RU86}]
				;

			\end{tikzpicture}
		\end{center}
		\begin{center}
			Формування збуджених гамільтоніанів шляхом переміщення електронів з зайнятих орбіталей в детермінантах Слейтера на віртуальні (\alert{промотування}).
			% При формуванні набору $\{\Phi_k\}_{1,\ldots, n}$ для складних молекул часто \alert{обмежуються одно- та двократно-збудженими визначниками} по відношенню до $\Phi_{0}$ ({\scriptsize правило Кондона-Слейтера}). У такому варіанті метод називається ({\scriptsize\color{blue} CISD, Cl with single and double excitations}).
		\end{center}
	\end{onlyenv}
	\framesubtitle<4>{Енергія кореляції}
	\begin{onlyenv}<4>
		Середнє значення енергії, обчислене з хвильовою функцією
		\[
			\Phi = C_0 \Phi_{0} + \sum_{k = 1}^{n} C_k\Phi_k, \quad C_0 \approx 1,
		\]
		визначається співвідношенням
		\begin{multline*}\label{}
			E_{CI} = \opbracket{\Phi_{0} + \sum_{k = 1}^{n} C_k\Phi_k}{\hat{H}}{\Phi_{0} + \sum_{l = 1}^{n} C_P\Phi_l} = \\
			= \opbracket{\Phi_{0}}{\hat{H}}{\Phi_{0}} + \opbracket{\sum_{k = 1}^{n} C_k\Phi_k}{\hat{H}}{\sum_{l = 1}^{n} C_l\Phi_l} \approx E_{HF} + \Delta E.
		\end{multline*}
		$E_{HF} = \opbracket{\Phi_{0}}{\hat{H}}{\Phi_{0}}$. Добавка до Хартрі-Фоківської енергії $\Delta E$ і є енергією кореляції $E_\text{кор}$.
	\end{onlyenv}
	\framesubtitle<5>{Методи CIS та CISD}
	\framesubtitle<6>{Метод Full CI}
	\begin{onlyenv}<5-6>

		\[
			\Phi = C_0 \Phi_{0} + \sum_{k = 1}^{n} C_k\Phi_k, \quad C_0 \approx 1,
		\]
		Застосовуючи варіаційний принцип, отримуємо набір рівнянь:
		\[
			\sum_{k = 0}^n C_k(H_{ik} - E\delta_{ik}) = 0, \quad H_{ik} = \opbracket{\Phi_i}{\hat{H}}{\Phi_k}.
		\]
		\begin{overprint}\footnotesize
			\onslide<5>
			% -----------------------------------------------------------------
			\begin{itemize}
				\item Матричні елементи точного гамільтоніану $\opbracket{\Phi_0}{\hat{H}}{\Phi_1}$ між основним і одноразово збудженими електронними конфігураціями дорівнюють нулю (\textcolor{blue}{теорема Бріллюена}).

				\item Тільки двічі збуджені детермінанти дають відмінні від нуля матричні елементи \alert{CID} (\alert{Configuration Interaction with Doubles}).

				\item Одноразово збуджені детермінанти  дають відмінні від нуля матричні елементи з дворазово збудженими детермінантами, тому їх включають в розкладання \alert{CISD} (\alert{CI with Singles and Doubles}).
			\end{itemize}

			\onslide<6>
			% -----------------------------------------------------------------
			\begin{itemize}
				\item В методі повної конфігураційної взаємодії (\alert{Full CI}) враховуються всі можливі розселення всіх $N$ електронів системи за всіма наявними орбіталями системи. Для синглетного стану $S = 0$ число детермінантів:
				      \begin{equation*}\label{}
					      n = \frac{M!(M+1)!}{\left(\frac12 N\right)! \left(\frac12 N + 1\right)!\left(M - \frac12N\right)!\left(M - \frac12N + 1\right)}.
				      \end{equation*}%
				\item  Для будь-якої реальної системи це означає врахування величезної кількості детермінантів, тому метод Full CI застосовується, як правило, \alert{тільки для простих систем}.
			\end{itemize}
		\end{overprint}
	\end{onlyenv}
\end{frame}
%============================================================================





%============================================================================
\begin{frame}{Діаграма Попла}{}
	{\scriptsize \fullcite{Popl:2002}}
	\begin{center}
		%            \includegraphics[width=\linewidth]{Pople_diagram_reverse_final.pdf}
		\begin{tikzpicture}
			\pgfmathsetmacro{\x}{7}
			\pgfmathsetmacro{\y}{6}
			\draw[-{Latex[scale=2]}, thick, gray] (0,0) -- node[sloped, anchor=north, font=\scriptsize] {кількість базисних функцій}++(0,\y);
			\draw[-{Latex[scale=2]}, thick, gray] (0,0) -- node[above, font=\scriptsize] {врахування кореляції} ++(\x,0);
			\node[draw, dashed, text width=3cm, font=\scriptsize, align=center, blue] (S) at  (\x, \y) {Точний розв'язок\\ рівняння Шредінгера};
			%    \draw [thick, -{Latex[scale=2]}] (0,0) --
			%            node[sloped, above, font=\scriptsize, red] {зростання часу обчислення}
			%            node[sloped, below, font=\scriptsize,] {зростання точності}
			%            (S.south west);
			\connect{0,0}{S.south west}{single arrow, top color=red, bottom color=blue};
			\path (0,0) --
			node[sloped, above, font=\scriptsize, red] {зростання часу обчислення}
			node[sloped, below, font=\scriptsize, blue] {зростання точності}
			(S.south west);


			\node[left, font=\scriptsize] at (0,0.5) {minimal basis set};
			\node[left, font=\scriptsize] at (0,1) {6-31G(d)};
			\node[left, font=\scriptsize] at (0,1.5) {6-31+G(d,p)};
			\node[left, font=\scriptsize] at (0,2) {6-311++G(2df,pd)};
			\node[left, font=\scriptsize] at (0,2.5) {$\vdots$};
			\node[left, font=\scriptsize] at (0,3) {(aug)-cc-pVDZ};
			\node[left, font=\scriptsize] at (0,3.5) {(aug)-cc-pVTZ};
			\node[left, font=\scriptsize] at (0,4) {(aug)-cc-pVQZ};
			\node[left, font=\scriptsize] at (0,4.5) {$\vdots$};
			\node[left, font=\scriptsize, text width=50pt] at (0,{\y}) {Хартрі-фоківський\\ ліміт};
			\node[left, font=\scriptsize, rotate=90] at (0.5, 0) {HF};
			\node[left, font=\scriptsize, rotate=90] at (1, 0) {MP2};
			\node[left, font=\scriptsize, rotate=90] at (1.5, 0) {MP4};
			\node[left, font=\scriptsize, rotate=90] at (2, 0) {CISD};
			\node[left, font=\scriptsize, rotate=90] at (2.5, 0) {CCSD};
			\node[below, font=\scriptsize] at (3, 0) {$\cdots$};
			\node[left, font=\scriptsize, rotate=90] at (\x, 0) {Full CI};
		\end{tikzpicture}
	\end{center}
\end{frame}
%============================================================================





%============================================================================
%\begin{frame}{Рівняння Хартрі-Фока}{Алгоритм розв'язку для атома гелію}\footnotesize
%	Базисні функції
%	\begin{equation*}\label{}
%		\chi_1 = \frac1{\sqrt{4\pi}}2\zeta_1^{3/2}e^{-\zeta_1r}, \quad \chi_2 = \frac1{\sqrt{4\pi}}2\zeta_1^{3/2}e^{-\zeta_2r}, \zeta_1 = 1.45, \quad \zeta_2 = 2.91.
%	\end{equation*}
%	\begin{overprint}
%		\onslide<1>
%		Компоненти матриці перекривання
%		\begin{equation*}\label{}
%			S_{11} = S_{22} = 1, \quad S_{12} = S_{21} = \frac{8\zeta_1^{3/2} \zeta_2^{3/2}}{(\zeta_1 + \zeta_2)^3} = 0.8366
%		\end{equation*}
%
%		Компоненти $\opbracket{\chi_p}{\hat{h}}{\chi_q}$
%		\begin{align*}\label{}
%			\opbracket{\chi_1}{\hat{h}}{\chi_1} = -\frac12\zeta_1^{3/2} + (\zeta_1 - 2)\zeta_1 = -1.8488, \quad \opbracket{\chi_2}{\hat{h}}{\chi_2} = -1.5860 \\
%			\opbracket{\chi_1}{\hat{h}}{\chi_2} = \opbracket{\chi_2}{\hat{h}}{\chi_1} = \frac{\zeta_1^{3/2}\zeta_2^{3/2}(4\zeta_1\zeta_2 - 8(\zeta_1 + \zeta_2))}{(\zeta_1 + \zeta_2)^3} = -1.8826.
%		\end{align*}
%		\onslide<2>
%		Кулонівські та обмінні інтеграли
%		\begin{align*}\label{}
%			(11|11) & = \frac58\zeta_1 = 0.9062, \quad (22|22) = \frac58\zeta_2 = 1.8188,                                                           \\
%			(11|22) & = (22|11) = (\zeta_1^4\zeta_2 + 4\zeta_1^3\zeta_2^2 + \zeta_1\zeta_2^4 + 4\zeta_1^2\zeta_2^3)/(\zeta_1 + \zeta_2)^4 = 1.1826, \\
%			(12|12) & = (21|12) = (12|21) = (21|21) = 20\zeta_1^3\zeta_2^3/(\zeta_1 + \zeta_2)^5 = 0.9535,                                          \\
%			(11|12) & = (11|21) = (12|11) = (21|11) =                                                                                               \\ &= \frac{16\zeta_1^{9/2}\zeta_2^{3/2}}{(3\zeta_1 + \zeta_2)^4}\left[\frac{12\zeta_1 + 8\zeta_2}{(\zeta_1 + \zeta_2)^2} + \frac{9\zeta_1 + \zeta_2}{2\zeta_1^2}\right] = 0.9033,\\
%			(12|22) & = (22|12) = (21|22) = (22|21) = 1.2980.
%		\end{align*}
%		\onslide<3>
%		Для вибору початкових коефіцієнтів візьмемо співвідношення $c_{11}/c_{21} \approx 2 = k$ . \\~\\
%		Умова нормування $\int |\phi_1|^2 dv = \int (c_{11}\chi_1 + c_{21}\chi_2)^2 dv = 1$ дає співвідношення $c_{21} = \sqrt{1 + k^2 + 2kS_{12}} \approx  0.3461$, $c_{11} = 0.6922$.\\~\\
%		Елементи матриці густини:
%		\begin{align*}\label{}
%			P_{11} = 2c_{11}c_{11} \approx 0.9583, \quad P_{12} = 2c_{11}c_{12} \approx 0.4791, \\
%			P_{21} = 2c_{21}c_{11} \approx 0.4791, \quad P_{22} = 2c_{21}c_{21} \approx 0.2396.
%		\end{align*}
%		\onslide<4>
%		Елементи матриці Фока:
%		\begin{align*}\label{}
%			F_{11} & = h_{11} + \frac12P_{11}(11|11) +  P_{12}(11|12) +P_{22}\left[(11|22) - \frac12(12|21)\right],              \\
%			F_{12} & = h_{12}+ \frac12P_{11}(12|11) + P_{12}\left[\frac32(12|12) - \frac12(11|22)\right] + \frac12P_{22}(12|22), \\
%			F_{22} & = F_{12},                                                                                                   \\
%			F_{22} & = h_{22} + P_{11}\left[(22|11) - \frac12(21|12)\right] + P_{12}(22|12) + \frac12P_{22}(22|22).
%		\end{align*}
%		\onslide<5>
%		Елементи матриці Фока:
%		\begin{align*}\label{}
%			F_{11} & = -1.8448 + 0.4531P_{11} +0.9033P_{12} + 0.7058P_{22},          \\
%			F_{12} & = F_{12} = -1.8826 + 0.45165P_{11} + 0.8391P_12 + 0.6490P_{22}, \\
%			F_{22} & = -1.5860 + 0.7058P_{11} + 1.2980P_{12} + 0.9094P_{22}.
%		\end{align*}
%		\begin{equation*}\label{}
%			F_{11} \approx -0.813, \quad F_{12} = F_{12} \approx -0.892, \quad F_{22} \approx -0.070\\
%		\end{equation*}
%		\onslide<6>
%		Розв'яжемо секулярне рівняння $\mathrm{det}(F_{pq} - S_{pq}\varepsilon_i) = 0$
%		\begin{equation*}\label{}
%			\left|
%			\begin{matrix}
%				-0.813 - \varepsilon_i       & -0.892 - 0.8366\varepsilon_i \\
%				-0.892 - 0.8366\varepsilon_i & -0.070-\varepsilon_1
%			\end{matrix}
%			\right| \approx 0.
%		\end{equation*}
%		\begin{equation*}\label{}
%			0.3001\varepsilon_1^2 - 0.6095\varepsilon_i - 0.739 \approx 0
%		\end{equation*}
%		\begin{equation*}\label{}
%			\varepsilon_1 \approx -0.854, \quad \varepsilon_2 \approx 2.885
%		\end{equation*}
%		\onslide<7>
%		Вибираємо корінь з меншою енергією і підставляємо його в рівняння Хартрі-Фока-Рутаана з  $p = 2$.
%		\begin{equation*}\label{}
%			c_{11}(F_{21} - \varepsilon_1S_{21}) + c_{21}(F_{22} - \varepsilon_1S_{22}) \approx 0.
%		\end{equation*}
%		\begin{equation*}\label{}
%			-0.1775 c_{11}  + 0.784 c_{21} \approx 0.
%		\end{equation*}
%		\begin{equation*}\label{}
%			c_{11}/c_{21} \approx 4.42.
%		\end{equation*}
%		Отримуємо поправлені коефіцієнти $c_{12} = 0.189$, $c_{11} = 0.836$.
%	\end{overprint}
%\end{frame}
%============================================================================

\end{document}
