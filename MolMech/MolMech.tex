% !TeX program = lualatex
% !TeX encoding = utf8
% !TeX spellcheck = uk_UA

\documentclass[]{beamer}
\usetheme{QuantumChemistry}
\usepackage{QuantumChemistry}
\addbibresource{\jobname.bib}
\usepackage{unicode-math-luatex}
\usepackage{makecell, booktabs, pgf}
\usepackage{minted, xcolor}

\graphicspath{{pictures/}}

\newcommand{\lenitem}[2][.5\linewidth]{\parbox[t]{#1}{\strut #2\strut}}

\newcommand{\mol}{
\tikz[baseline]{
    %\node[trapezium, draw, minimum width=3cm, trapezium left angle=135, trapezium right angle=45, minimum height=1.65cm, fill=cyan!10] at (0.25,0) {};

	\node[atom] (A) at (-1,0) {};
	\node[atom] (B) at (1,0) {};
	\node[atomH] (H1) at ($ (A) + (135:1) $) {};
	\node[atomH] (H2) at ($ (A) + (180:1) $) {};
	\node[atomH] (H3) at ($ (A) + (225:1) $) {};

	\node[atomH] (H4) at ($ (B) + (65:1) $) {};
	\node[atomH] (H5) at ($ (B) + (0:1) $) {};
	\node[atomH] (H6) at ($ (B) + (-45:1) $) {};

	\draw[bound] (A) -- (B);
	\draw[boundH] (A) -- (H1); \draw[boundH] (A) -- (H2); \draw[boundH] (A) -- (H3);
	\draw[boundH] (B) -- (H4); \draw[boundH] (B) -- (H5); \draw[boundH] (B) -- (H6);
	\draw[gray!50] (A) -- ++(-90:0.5) coordinate[pos=0.5] (A1);
	\draw[gray!50] (B) -- ++(-90:0.5) coordinate[pos=0.5] (B1);
	\draw[<->, gray] (A1) -- node[below, sloped, font=\scriptsize] {bond length} (B1);
	\draw[<->, gray] (A) ++(125:0.5) arc(125:30:0.5) node[pos=0.5, anchor=south] {$\alpha$};
}
}

\title[Лекції з квантової хімії]{{\bfseries\huge Молекулярна механіка} \\ {Оптимізація структури молекули}}

\subtitle{\bfseries Лекції з квантової хімії}
\author{Пономаренко С. М.}
\date{}
\begin{document}
%============================================================================


\begin{frame}
	\titlepage
\end{frame}

%============================================================================




%============================================================================
\begin{frame}{Будова молекули}{}
	\begin{itemize}
		\item Просторовою \emphs{будовою молекули} називають рівноважне розташування ядер атомів, що її утворюють. Енергія взаємодії атомів залежить від розташування ядер та стану електронної підсистеми.
		\item Рівноважні відстані у молекулах та розташування атомних ядер (геометрію молекули) визначаються методами \emphs{спектроскопії}, \emphs{рентгенівського структурного аналізу}, \emphs{електронографії} та \emphs{нейтронографії}. Ці методи також дозволяють отримати інформацію про розподіл електронів (електронну густину) в молекулі.
		\item Теоретично, геометрію молекули можна визначити методами \emphs[red]{молекулярної механіки} та \emphs[red]{квантової хімії}.
	\end{itemize}
\end{frame}
%============================================================================





%============================================================================
\begin{frame}{Моделювання структури та властивостей молекул}{}
	\begin{block}{}\justifying
		\emphs[red]{Молекулярна механіка} --- метод розрахунку геометрії та енергетичних характеристик молекулярних частинок з використанням емпіричних потенціальних функцій, вид яких взято з класичної механіки.
	\end{block}
	\hfill\raisebox{-2\height}[0pt][0pt]{\mol}
	\vspace{-1.5em}

	Припущення:
	\begin{itemize}\small
		\item кожен атом симулюється як одна окрема частинка;
		\item \lenitem{кожній частинці присвоюється радіус (зазвичай радіус Ван дер Ваальса\footnote{\scriptsize Характеристика атома, радіус уявної твердої сфери, якою можна було б замінити атом для опису властивостей газів та рідин із цих атомів за допомогою рівняння ван дер Ваальса.}), поляризовність та електричний заряд (виведений з квантово-хімічних розрахунків та/або експерименту);}
		\item взаємодії розглядаються як <<пружини>> із рівноважною довжиною рівною експериментальній чи розрахованій довжині хімічного зв'язку.
	\end{itemize}




\end{frame}
%============================================================================


\begin{frame}{Способи опису структури хімічної системи}{}
	Для опису атомно-молекулярних систем існує кілька способів задавання координат, кожен з яких має свої переваги і
	недоліки:
	\begin{enumerate}
		\item декартові координати;
		\item внутрішні координати.
		      %\item координати $Z$-матриці;
		      %\item дробні координати.
	\end{enumerate}
\end{frame}
%============================================================================





%============================================================================
\begin{frame}[fragile, t]{Декартові координати}{}
	\begin{columns}
		\begin{column}{0.3\linewidth}\scriptsize
			\begin{center}
				\includegraphics[width=\linewidth]{cart}
			\end{center}
		\end{column}
		\begin{column}{0.7\linewidth}\scriptsize
			Одиниці виміру координат $X$, $Y$, $Z$ можуть бути а ангстремах (\AA) або в атомних одиниці довжини (а.о., бори) (a.u., Bohr).
			\\~\\
			Приклад задавання декартових координат атомів етану \ce{C2H6}
			\begin{minted}[mathescape,
        gobble=8,
        %        breaklines,
        fontsize=\scriptsize,
        ]
        {ruby}
        %coords
        CTyp xyz   # the type of coordinates = xyz or internal
        Charge 0   # the total charge of the molecule
        Mult 1     # the multiplicity = 2S+1 S = 0
        Units Angs # the unit of length = angs or bohrs
        coords
           C       -4.61712        0.67819        0.00000
           C       -3.27135        1.38303        0.00000
           H       -5.41470        1.38930        0.30192
           H       -4.59746       -0.17109        0.71505
           H       -4.84008        0.29216       -1.01697
           H       -3.04839        1.76906        1.01697
           H       -3.29101        2.23231       -0.71505
           H       -2.47377        0.67192       -0.30192
        end
        end
        \end{minted}
		\end{column}
	\end{columns}
\end{frame}

%============================================================================
\begin{frame}[fragile, t]{Внутрішні координати}{}
	\begin{alertblock}{}
		Число внутрішніх координат дорівнює числу ступенів свободи молекули
		\begin{equation*}
			3N - 6
		\end{equation*}
	\end{alertblock}
	\begin{overprint}
		\onslide<1>
		\begin{columns}
			\begin{column}{0.3\linewidth}
				\begin{center}
					\begin{center}
						\begin{tikzpicture}[scale=0.8]
							\draw[fill=cyan, opacity=0.3] (-1.5, 0) rectangle (1.5, 1.5);
							\draw[fill=cyan, opacity=0.3] (-1.5, 0) -- ++(-135:1.5) -- ++ (0:3) -- ++(45:1.5) -- cycle;
							\node[atom] (A) at (-1, 0) {}; \node[below=5pt] at (A) {$A$};
							\node[atom] (B) at (1, 0) {}; \node[above=5pt] at (B) {$B$};
							\node[atom] (C) at (-1, 1) {}; \node[right=5pt] at (C) {$C$};
							\node[atom] (D) at (0.3, -0.7) {}; \node[left=5pt] at (D) {$D$};
							\draw[double] (C) -- (A) -- (B) -- (D);
							\draw[<->] (A) ++(0, 0.5) arc(90:0:0.5) node[pos=0.5, anchor=south west] {$\alpha$};
							\draw[<->] (-1.5,1) arc(90:210:0.5 and 1) node[pos=0.5, left] {$\phi$};
						\end{tikzpicture}
					\end{center}
				\end{center}
				\[
					N = 4
				\]
				\[
					3 \cdot 4 - 6 = 3 + 2 + 1
				\]
			\end{column}
			\begin{column}{0.75\linewidth}
				\begin{itemize}
					\item Міжатомні відстані
					      \begin{block}{}\scriptsize\justifying
						      Прообраз координати --- \emphs[red]{довжина хімічного зв'язку} між двома атомами молекулі. Необхідно вказати два атоми, між якими координати вказує відстань. Значення лежить у діапазоні від $0 \ldots +\infty$.
					      \end{block}
					\item Міжатомний кут
					      \begin{block}{}\scriptsize\justifying
						      Кут між трьома атомами. Змінюється в межах $\alpha = 0 \ldots \pi $.  Прообразом та природною реалізацією цього типу координат є кути між двома хімічними зв'язками -- \emphs[red]{валентні кути}.
					      \end{block}
					\item Диедральні (або торсіонні) кути
					      \begin{block}{}\scriptsize\justifying
						      Для обраних чотирьох атомів $A$, $B$, $C$, $D$ треба всього лише порахувати кут між площинами, утвореними атомами $C$, $A$, $B$ та $A$, $B$, $D$.
					      \end{block}
				\end{itemize}
			\end{column}
		\end{columns}
		\onslide<2>
		Приклад задавання внутрішніх координат атомів етану \ce{C2H6}
		\begin{minted}[mathescape,
        gobble=8,
        %        breaklines,
        fontsize=\scriptsize,
        ]
        {ruby}
        %coords
        CTyp internal # the type of coordinates = xyz or internal
        Charge 0      # the total charge of the molecule
        Mult 1        # the multiplicity = 2S+1 S = 0
        Units Angs    # the unit of length = angs or bohrs
        coords
           C     0     0     0        0.00000        0.00000        0.00000
           C     1     0     0        1.51920        0.00000        0.00000
           H     1     2     0        1.11038      109.82535        0.00000
           H     1     2     3        1.11038      109.82535      240.00000
           H     1     2     3        1.11038      109.82535      120.00000
           H     2     1     3        1.11038      109.82535       60.00000
           H     2     1     3        1.11038      109.82535      300.00000
           H     2     1     3        1.11038      109.82535      180.00000
        end
        end
        \end{minted}
	\end{overprint}
\end{frame}
%============================================================================

%============================================================================
\begin{frame}{Сили та потенціальна енергія}{}
	\begin{overprint}
		\onslide<1>
		\mol \hspace*{2cm}
		$
			\tcbhighmath{\vec{F} = - \vec{\nabla} U}%(x_1, y_1, y_1, \ldots, x_i, y_i, y_i, \ldots, x_N, y_N, y_N)
		$
		\vspace*{0.5cm}
		\begin{equation*}
			U_\text{chem} = \overbrace{
				\sum\limits_\text{bond} \frac{k_i}{2} (x_i - x_{0_i})^2}^{
			\tikz[]{
				\node[atom] (A) at (-2,0) {}; \node[atom] (B) at (-0.5,0) {};
				\draw[boundH] (A) -- node[above] {$x$} (B);
			}
			}
			+
			\overbrace{
				\sum\limits_\text{angle}\frac{b_i}{2} (\alpha_i - \alpha_{0_i})^2
			}^{
			\tikz[xshift=-1cm]{
				\node[atom] (O) at (0,0) {}; 	\node[atom] (A) at (135:1) {}; \node[atom] (B) at (45:1) {};
				\draw[] (O) -- (A) (O) -- (B); \draw[<->] (135:0.5) arc(135:45:0.5) node[above=-1pt, pos=0.5] {$\alpha$};

			}
			}
			+
			\overbrace{
				\sum\limits_\text{torsion} + \frac{V_i}{2} (1 - \cos\phi_i)
			}^{
			\tikz[xshift=-2cm]{
				\node[atom] (A) at (-0.5,-0.5) {}; \node[atom] (B) at (0,0) {}; \node[atom] (C) at (1,0) {}; \node[atom] (D) at (1.5,0.5) {};
				\draw[] (A) -- (B) -- (C) -- (D);
				\draw[->, thick] (0.5,-0.2) arc (-150:150:0.15 and 0.5) node[left] {$\phi$};
			}
			}
		\end{equation*}

		\vspace*{0.5cm}

		\begin{equation*}
			U_\text{phys} =
			\overbrace{
				\frac{q_iq_j}{r_{ij}}
			}^{
				\tikz[xshift=-2cm]{
					\node[atom, font=\scriptsize, inner sep=0pt] (A) at (-1,0) {$+$}; \node[atom, font=\scriptsize, inner sep=0.5pt] (B) at (-0.5,0) {$-$};
				}
			}
			+
			\overbrace{
				\sum\limits_i\sum\limits_j 4\epsilon_{ij}\left[ \left( \frac{\sigma_{ij}}{r_{ij}} \right)^{12} - \left( \frac{\sigma_{ij}}{r_{ij}} \right)^6 \right]
			}^{
				\tikz[xshift=-2cm]{
					\node[atom, font=\scriptsize] (A) at (0.5,0) {}; \node[atom, font=\scriptsize] (B) at (1.5,0) {};
					\draw[dashed] (A) circle (0.5); \draw[dashed] (B) circle (0.5);
				}
			}
		\end{equation*}

		\onslide<2>

		\framesubtitle<2>{Потенціал Морзе}

		\bigskip

		Потенціал Морзе --- модельна формула, що описує потенціальну енергію міжатомної взаємодії.

		\bigskip

		\begin{block}{}\scriptsize\justifying
			Наближення Морзе для визначення коливальної структури молекули краще, ніж квантовий осцилятор, оскільки воно точно враховує ефекти розриву зв'язків у молекулі, такі, як наприклад, існування незв'язаних станів. Також враховується ангармонізм для реальних зв'язків в молекулі і ненульова ймовірність переходу для вищих гармонік та комбінаційних частот.
		\end{block}

		\begin{columns}
			\begin{column}{0.4\linewidth}
				\begin{equation*}
					\tcbhighmath{U_\text{Morse} = D_e ( 1-e^{-a(r-r_e)} )^2}
				\end{equation*}
			\end{column}
			\begin{column}{0.6\linewidth}
				\begin{center}
					\begin{tikzpicture}[scale=2.5]
						\message{^^JMorse potential}
						\def\xmax{2} % max x axis
						\def\A{1}
						\def\b{2.3}
						\def\a{(0.26*\xmax)}

						\def\ymax{1.0}  % max y axis
						\draw[->,thick] (0,-0.2*\ymax) -- (0,1.1*\ymax) node[left] {$U$};
						\draw[->,thick] (-0.2*\ymax,0) -- (\xmax,0) node[below] {$r$};
						\draw[red, densely dashed,samples=100,smooth, variable=\x,domain=0.24*\a:1.76*\a]     plot(\x,{\A*\b^2*(\x-\a)^2});
						\node[below] at ({\a}, 0) {$r_0$};
						\draw[blue, samples=100,smooth,variable=\x,domain=0.1*\xmax:0.95*\xmax]
						plot(\x,{\A*(1-exp(-\b*(\x-\a)))^2});
						\node[atom] (A) at ({\a - \a/3}, {0.95*\ymax}) {};
						\node[atom] (B) at ({\a + \a/3}, {0.95*\ymax}) {};
						\draw[boundH] (A) -- node[above] {} (B);
					\end{tikzpicture}
				\end{center}
			\end{column}
		\end{columns}

		\onslide<3>

		\framesubtitle<3>{\href{https://avogadro.cc/docs/optimizing-geometry/molecular-mechanics/}{Molecular Mechanics \& Force Fields}}

		\bigskip
		\begin{enumerate}\small
			\item \href{https://en.wikipedia.org/wiki/Force_field_(chemistry)}{\emphz{UFF (Universal Force Field)}}
			      \begin{block}{}\footnotesize\justifying
				      Аналітичні вирази та силові константи вибираються таким чином, що цей метод може оптимізувати геометрію для всіх елементів і добре працює з неорганічними та металоорганічними матеріалами.
			      \end{block}
			\item \emphz{MMFF94 \& MMFF94s} (designed by Merck).
			      \begin{columns}
				      \begin{column}{0.5\linewidth}
					      \begin{block}{}\footnotesize\justifying
						      Особливо добре підходить для органічних сполук.
						      \emphs{MMFF94 і MMFF94s моделюють водневий зв'язок на відміну від інших методів}.
					      \end{block}
				      \end{column}
				      \begin{column}{0.5\linewidth}
					      \begin{center}
						      \includegraphics[height=2.25cm]{Hbond}
					      \end{center}
				      \end{column}
			      \end{columns}

			\item \href{https://en.wikipedia.org/wiki/AMBER}{\emphz{GAFF}} (General AMBER Force Field)
			      \begin{block}{}\footnotesize\justifying
				      Використовується для оптимізації геометрії молекул лікарських засобів. AMBER (Assisted Model Building with Energy Refinement) --- силове поле в білках.
			      \end{block}
		\end{enumerate}
	\end{overprint}
\end{frame}

%============================================================================

\begin{frame}[t]{Поверхня потенціальної енергії (ППЕ)}\small%
	\framesubtitle<1>{Стаціонарні точки}%
	\framesubtitle<2>{Матриця Гессе}%
	\framesubtitle<3>{Оптимізація геометрії}%
	\framesubtitle<4>{Частоти коливань молекули}%
	\vspace{-1.5em}
	\begin{overprint}
		\onslide<1>
		\begin{block}{}
			Стаціонарна точка на ППЕ --- точка, в якій градієнт потенціальної енергії дорівнює нулю:
			\begin{equation*}
				\vec{F} = -\vec{\nabla}U = 0.
			\end{equation*}
		\end{block}
		\onslide<2>
		\begin{block}{}\justifying
			Тип стаціонарної точки визначається матрицею других похідних функцій --- матрицею Гессе (визначник матриці наз. гесіаном).
		\end{block}
		\onslide<3>
		\begin{alertblock}{}
			Оптимізація молекулярної геометрії --- мінімізація енергії молекули  $U$ при варіації координат атомів.
		\end{alertblock}
		\onslide<4>
		\begin{alertblock}{}
			Матриця Гессе --- матриця силових констант. Діагоналізація матриці визначає нормальні частоти коливань молекули.
		\end{alertblock}
	\end{overprint}
	\begin{columns}
		\begin{column}{0.5\linewidth}
			\begin{overprint}
				\onslide<1>
				\begin{itemize}
					\item локальні мінімуми на потенціальної поверхні --- відповідають метастабільним конфігураціям молекули.
					\item абсолютний мінімум --- найстійкішою (стабільною) конфігурацією --- основного стану системи.
					\item сідлові точки --- відповідають перехідним станам.
				\end{itemize}
				\onslide<2>
				\begin{equation*}\label{}
					\begin{bmatrix}
						\dfrac{\partial^2 U}{\partial q_1^2}               & \dfrac{\partial^2 U_e}{\partial q_1\,\partial q_2} & \cdots & \dfrac{\partial^2 U_e}{\partial q_1\,\partial q_n} \\
						\dfrac{\partial^2 U_e}{\partial q_2\,\partial q_1} & \dfrac{\partial^2 U_e}{\partial q_2^2}             & \cdots & \dfrac{\partial^2 U_e}{\partial q_2\,\partial q_n} \\
						\vdots                                             & \vdots                                             & \ddots & \vdots                                             \\
						\dfrac{\partial^2 U_e}{\partial q_n\,\partial q_1} & \dfrac{\partial^2 U_e}{\partial q_n\,\partial q_2} & \cdots & \dfrac{\partial^2 U_e}{\partial q_n^2}
					\end{bmatrix}
				\end{equation*}
				\onslide<3>
				\begin{block}\justifying
					Залежність $U$ від координат ядер є поверхнею потенціальної енергії (ППЕ), оптимізація геометрії є пошуком точок локальних мінімумів на ППЕ.

					\bigskip

					Фактично алгоритми шукають рівноважні значення внутрішніх координат $x_{0_i}$, $\alpha_{0_i}$ та $\phi$ на основі емпіричних значень $k_i$, $b_i$ та $V_i$ варіюючи $x$, $\alpha$ та $\phi$.
				\end{block}
				\onslide<4>
				\begin{equation*}
					\begin{bmatrix}
						\dfrac{\partial^2 U}{\partial q_1^2} & 0                                      & \cdots & 0                                      \\
						0                                    & \dfrac{\partial^2 U_e}{\partial q_2^2} & \cdots & 0                                      \\
						\vdots                               & \vdots                                 & \ddots & \vdots                                 \\
						0                                    & 0                                      & \cdots & \dfrac{\partial^2 U_e}{\partial q_n^2}
					\end{bmatrix}
				\end{equation*}
				\[
					E_n = \left(n + \frac12\right)\hbar \omega
				\]

			\end{overprint}
		\end{column}
		\begin{column}{0.5\linewidth}
			\begin{center}
				\includegraphics[width=\linewidth]{ppe}
			\end{center}
		\end{column}
	\end{columns}
\end{frame}
%============================================================================


%============================================================================
\begin{frame}{Алгоритми пошуку стаціонарних точок на ППЕ}{Чисельні методи}
	\begin{enumerate}\small
		\item Градієнтний спуск (англ. \emphz{Steepest Descent}) --- це ітераційний алгоритм оптимізації першого порядку, в якому для знаходження локального мінімуму функції здійснюються кроки, пропорційні протилежному значенню градієнту (або наближеного градієнту) функції в поточній точці.
		\item Метод спряженого градієнта (англ. \emphz{Conjugate Gradient}) -- ітераційний метод для безумовної оптимізації у багатовимірному просторі Основною перевагою методу є те, що він вирішує квадратичну задачу оптимізації за кінцеве число кроків.
	\end{enumerate}
	\begin{center}
		\includegraphics[width=0.4\linewidth]{Gradient_descent}
	\end{center}
\end{frame}
%============================================================================

%============================================================================
\begin{frame}{\href{https://avogadro.cc/docs/tools/auto-optimize-tool/}{Оптимізація геометрії в Avogadro}}{\url{https://avogadro.cc/docs/tools/auto-optimize-tool/}}

	\tikz[remember picture,overlay] \node[opacity=0.3,inner sep=0pt, anchor=north east] at (current page.north east){\includegraphics[width=2cm]{avogadro_logo}};
	\begin{alertblock}{}\centering
		Інструмент Auto Optimize --- оптимізує молекулярну геометрію за допомогою молекулярної механіки.
	\end{alertblock}

	\begin{center}
		\includegraphics[width=0.6\linewidth]{Avogadro_Opt}
	\end{center}
\end{frame}
%============================================================================




%============================================================================
\begin{frame}[t]{Ізомери, таутомери та конформери}{}
	\begin{itemize}
		\item Одному і тому ж набору атомів можуть відповідати кілька мінімумів на ППЕ. Такі структури називаються \emphs[red]{ізомерами}.
		      \begin{onlyenv}<1>
			      \begin{center}
				      \includegraphics[width=0.65\linewidth]{isomers}\\
				      {\scriptsize Ізомери молекули азотистої кислоти }
				      \begin{block}{}\scriptsize\centering
					      \fullcite{Pradhan2017}
				      \end{block}
			      \end{center}
		      \end{onlyenv}
		\item<2-> \emphs[red]{Конформери} --- частковий випадок ізомерів, які отримуються один з одного без розриву хімічного зв'язку.
			\begin{onlyenv}<2>
				\begin{center}
					\includegraphics[width=0.75\linewidth]{conformers}\\
					{\scriptsize Поворотні конформери бутану}
				\end{center}
			\end{onlyenv}
		\item<3> \emphs[red]{Таутомери} ---ізомери, які легко переходять один в одного. Найчастіше це відбувається за рахунок переносу атома водню.
			\begin{onlyenv}<3>
				\begin{center}
					\includegraphics[width=0.5\linewidth]{Ethanal_Ethenol_Tautomerie}\\
					{\scriptsize Таутомерія етаналь—етенол}
				\end{center}
			\end{onlyenv}
	\end{itemize}
	\begin{center}
		\emphs[red]{Чим глибший мінімум на ППЕ, тим стабільніший ізомер.}
	\end{center}
\end{frame}
%============================================================================

%============================================================================
\begin{frame}{Недоліки методів молекулярної механіки}{}\small
	\emphs[red]{Переваги}:
	\begin{itemize}
		\item Прості методи і швиді розрахунки.
		\item Можуть працювати з доволі складними системами.
	\end{itemize}

	\emphs[red]{Недоліки}:
	\begin{itemize}
		\item Засновані на класичній фізиці систем багатьох частинок та не здатні описувати квантові ефекти (збуджені стани молекул).
		\item Розрахунок ППЕ заснований на формулах та параметрах, що мають емпіричну природу.
		\item Для врахування міжмолекулярних взаємодій необхідно знати ефективні заряди атомів, дипольні моменти груп атомів та хімічних зв'язків. \emphs[red]{Їх знаходження є самостійною задачею і може бути здійснено тільки квантово-хімічним методам}.
		\item Не застосовуються для моделювання систем, властивості яких визначаються електронними ефектами та у випадку розриву хімічних зв'язків.
	\end{itemize}
\end{frame}
%============================================================================

\end{document}
