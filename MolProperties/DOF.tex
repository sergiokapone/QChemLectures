\documentclass[a4paper,12pt]{article}
\usepackage{amsmath,amssymb}
\usepackage{physics}
\usepackage{geometry}
\geometry{margin=2.5cm}

\usepackage{fontspec}
\setsansfont{CMU Sans Serif}%{Arial}
\setmainfont{CMU Serif}%{Times New Roman}
\setmonofont{CMU Typewriter Text}%{Consolas}
\defaultfontfeatures{Ligatures={TeX}}
\usepackage[math-style=TeX]{unicode-math}
\usepackage[english, russian, ukrainian]{babel}

\usepackage[]{titlesec}



\title{Виведення щільності енергії електромагнітного поля через векторний потенціал}
\author{}
\date{}

\begin{document}
\maketitle

\section{Векторний потенціал хвилі}

Розглянемо плоску електромагнітну хвилю у вакуумі з векторним потенціалом:
\[
\vec{A}(\vec{r}, t) = \vec{A}_0 e^{i(\vec{k} \cdot \vec{r} - \omega t)} + \text{c.c.}
\]
де:
\begin{itemize}
  \item \( \vec{A}_0 \) — амплітуда векторного потенціалу,
  \item \( \vec{k} \) — хвильовий вектор,
  \item \( \omega = c|\vec{k}| \),
  \item \text{c.c.} — комплексно спряжене.
\end{itemize}

\section{Електричне і магнітне поля}

\[
\vec{E} = -\frac{1}{c} \frac{\partial \vec{A}}{\partial t} = i \frac{\omega}{c} \vec{A}_0 e^{i(\vec{k} \cdot \vec{r} - \omega t)} + \text{c.c.}
\]
\[
\Rightarrow \langle |\vec{E}|^2 \rangle = 2 \left( \frac{\omega}{c} \right)^2 |\vec{A}_0|^2
\]

\[
\vec{B} = \nabla \times \vec{A} = i (\vec{k} \times \vec{A}_0) e^{i(\vec{k} \cdot \vec{r} - \omega t)} + \text{c.c.}
\]
Оскільки \( \vec{A}_0 \perp \vec{k} \), маємо:
\[
\Rightarrow \langle |\vec{B}|^2 \rangle = 2 \left( \frac{\omega}{c} \right)^2 |\vec{A}_0|^2
\]

\section{Густина енергії електромагнітного поля}

\[
u = \frac{1}{8\pi} \left( \langle |\vec{E}|^2 \rangle + \langle |\vec{B}|^2 \rangle \right)
= \frac{1}{8\pi} \cdot 4 \left( \frac{\omega}{c} \right)^2 |\vec{A}_0|^2
= \frac{1}{2\pi} \left( \frac{\omega}{c} \right)^2 |\vec{A}_0|^2
\]

\[
\boxed{
\rho = \frac{\omega^2}{2\pi c^2} |\vec{A}_0|^2
}
\]

Це і є шуканий вираз для щільності енергії через амплітуду векторного потенціалу.

\end{document}
