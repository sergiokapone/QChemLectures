\documentclass{article}
\usepackage{fontspec}
\setsansfont{CMU Sans Serif}%{Arial}
\setmainfont{CMU Serif}%{Times New Roman}
\setmonofont{CMU Typewriter Text}%{Consolas}
\defaultfontfeatures{Ligatures={TeX}}


\usepackage[russian]{babel}
\usepackage{amsmath}
\usepackage{amssymb}
\usepackage{graphicx}

\begin{document}

\section*{Лекция 9. Молекулы во внешнем электромагнитном поле}

\subsection{Временное уравнение Шрёдингера для состояний системы во внешнем электромагнитном поле}

Пусть $\hat{H}_0$ — гамильтониан свободной частицы. При обсуждении гамильтониана молекулы во внешнем электромагнитном поле необходимо добавить к гамильтониану свободной частицы гамильтониан внешнего поля $\hat{H}_f$ и гамильтониан взаимодействия поля с молекулами $\hat{H}_{mf}$.

\begin{equation}
\hat{H} = \hat{H}_0 + \hat{H}_f + \hat{H}_{mf} = \hat{H}_{(0)} + \hat{H}'
\label{eq:83}
\end{equation}

Заметим, что электромагнитное поле в общем случае может меняться со временем, а значит, для решения задачи с таким гамильтонианом нужно воспользоваться временным уравнением Шрёдингера.

\begin{equation}
i\hbar \frac{\partial \tilde{\Psi}}{\partial t} = \hat{H} \tilde{\Psi}
\label{eq:84}
\end{equation}

Для свободной частицы временное уравнение Шрёдингера:

\begin{equation}
i\hbar \frac{\partial \tilde{\Psi}^{(0)}_{k}}{\partial t} = \hat{H}_0 \tilde{\Psi}^{(0)}_{k}
\label{eq:85}
\end{equation}

Учитывая, что оператор Гамильтона не зависит в явном виде от времени, решение уравнения можно искать в виде произведения двух волновых функций, одна из которых зависит только от координат частиц молекулярной системы, а вторая зависит только от времени:

\begin{equation}
\tilde{\Psi}^{(0)}_{k} = \Psi^{(0)}_{k}(\mathbf{r}, \mathbf{R}) f_k(t)
\label{eq:86}
\end{equation}

Тогда методом разделения переменных временное уравнение Шрёдингера можно свести к системе двух уравнений:

\begin{equation}
\begin{cases}
\hat{H}_{(0)} \Psi^{(0)}_{k} = E_k \Psi^{(0)}_{k} \\
i\hbar \frac{\partial f_k}{\partial t} = E_k f_k
\end{cases}
\label{eq:87}
\end{equation}

Решение второго уравнения — экспоненциальная функция от времени. Тогда решение временного уравнения Шрёдингера можно записать как:

\begin{equation}
\tilde{\Psi}^{(0)}_{k} = \Psi^{(0)}_{k} e^{-\frac{i}{\hbar} E_k t}
\label{eq:88}
\end{equation}

\subsection*{Теория возмущений}

Для упрощения задачи предположим, что в начальный момент времени система представлена одним состоянием $\tilde{\Psi}_{n}^{(0)}$. После выключения внешнего воздействия (рис. 9) молекула может оказаться в состоянии, которое можно записать как линейную комбинацию возможных исходных состояний:

\begin{equation}
\sum_{k} c_{nk}\tilde{\Psi}_{k}^{(0)}
\end{equation}

Вероятность перехода системы из состояния $\tilde{\Psi}_{n}^{(0)}$ в состояние $\tilde{\Psi}_{k}^{(0)}$ определяется квадратом коэффициента $c_{nk}$:

\begin{equation}
W_{nk} = |c_{nk}|^{2}
\label{eq:89}
\end{equation}

Подставим линейную комбинацию в уравнение \eqref{eq:85}:

\begin{equation}
i\hbar\frac{\partial}{\partial t}\left(\sum_{k} c_{nk}\tilde{\Psi}_{k}^{(0)}\right) = \hat{H}\left(\sum_{k} c_{nk}\tilde{\Psi}_{k}^{(0)}\right)
\label{eq:90}
\end{equation}

После дифференцирования и подстановки гамильтониана из уравнения \eqref{eq:83} получаем:

\begin{equation}
\sum_{k} c_{nk}i\hbar\frac{\partial\tilde{\Psi}_{k}^{(0)}}{\partial t} + \sum_{k} i\hbar\frac{\partial c_{nk}}{\partial t}\tilde{\Psi}_{k}^{(0)} = \sum_{k} c_{nk}\hat{H}_{0}\tilde{\Psi}_{k}^{(0)} + \sum_{k} c_{nk}\hat{H}'\tilde{\Psi}_{k}^{(0)}
\label{eq:91}
\end{equation}

Первые слагаемые в левой и правой частях согласно уравнению \eqref{eq:85} равны, поэтому:

\begin{equation}
\sum_{k} i\hbar\frac{\partial c_{nk}}{\partial t}\tilde{\Psi}_{k}^{(0)} = \sum_{k} c_{nk}\hat{H}'\tilde{\Psi}_{k}^{(0)}
\label{eq:92}
\end{equation}

\subsection*{Развитие теории возмущений}

Домножим уравнение \eqref{eq:92} на функцию $\tilde{\Psi}_{m}^{(0)}$, собственную для оператора $\hat{H}_{(0)}$, и проинтегрируем по всему конфигурационному пространству:

\begin{equation}
\sum_{k} i\hbar\frac{\partial c_{nk}}{\partial t}\langle\tilde{\Psi}_{m}^{(0)}|\Psi_{k}^{(0)}\rangle = \sum_{k} c_{nk}\langle\tilde{\Psi}_{m}^{(0)}|\hat{H}'|\tilde{\Psi}_{k}^{(0)}\rangle
\label{eq:93}
\end{equation}

Учитывая ортогональность и ортонормированность базиса, получим:

\begin{equation}
i\hbar\frac{\partial c_{nm}}{\partial t} = \sum_{k} c_{nk}\langle\tilde{\Psi}_{m}^{(0)}|\hat{H}'|\tilde{\Psi}_{k}^{(0)}\rangle
\label{eq:94}
\end{equation}

Перепишем это дифференциальное уравнение в интегральном виде:

\begin{equation}
c_{nm}(\tau) = -\frac{i}{\hbar}\int_{0}^{\tau}\sum_{k} c_{nk}\langle\tilde{\Psi}_{m}^{(0)}|\hat{H}'|\tilde{\Psi}_{k}^{(0)}\rangle dt
\label{eq:95}
\end{equation}

\subsection*{Приближенное решение}

Полученное уравнение (система уравнений) достаточно сложное. Вводят следующее приближение: в момент отключения внешнего поля система выбирает одно состояние, либо исходное, либо какое-то другое с вероятностью, равным поправке в теории возмущений. Решение ищут в виде:

\begin{equation}
\tilde{\Psi}_n = \tilde{\Psi}^{(0)}_n + \Delta; \quad \Delta = \sum_{k \neq n} c_{nk}(t) \tilde{\Psi}^{(0)}_k
\label{eq:96}
\end{equation}

Тогда уравнение \eqref{eq:94} можно переписать в виде:

\begin{equation}
i\hbar \frac{\partial}{\partial t} (c^{(1)}_{nm} + c^{(2)}_{nm} + \ldots) = \sum_k (\delta_{nk} + c^{(1)}_{nk} + c^{(2)}_{nk} + \ldots) \langle \tilde{\Psi}^{(0)}_m | \hat{H}'| \tilde{\Psi}^{(0)}_k \rangle
\label{eq:97}
\end{equation}

\subsubsection*{Теория возмущений первого порядка}

В первом порядке теории возмущений оставляем только первые слагаемые:

\begin{equation}
i\hbar \frac{\partial}{\partial t} c^{(1)}_{nm} = \langle \tilde{\Psi}^{(0)}_m | \hat{H}' | \tilde{\Psi}^{(0)}_n \rangle
\label{eq:98}
\end{equation}

\begin{equation}
c^{(1)}_{nm}(\tau) = -\frac{i}{\hbar} \int_0^{\tau} \langle \tilde{\Psi}^{(0)}_m | \hat{H}' | \tilde{\Psi}^{(0)}_n \rangle dt
\label{eq:99}
\end{equation}

\subsubsection*{Теория возмущений второго порядка}

Во втором порядке теории возмущений учитываем следующие слагаемые:

\begin{equation}
i\hbar \frac{\partial}{\partial t} c^{(2)}_{nm} = \sum_k c^{(1)}_{nk} \langle \tilde{\Psi}^{(0)}_m | \hat{H}' | \tilde{\Psi}^{(0)}_k \rangle
\label{eq:100}
\end{equation}

\begin{equation}
c^{(2)}_{nm}(\tau) = -\frac{i}{\hbar} \int_0^{\tau} \sum_k c^{(1)}_{nk} \langle \tilde{\Psi}^{(0)}_m | \hat{H}' | \tilde{\Psi}^{(0)}_k \rangle dt
\label{eq:101}
\end{equation}

\subsection*{Физическая интерпретация}

Физический смысл выражения \eqref{eq:99}: если оператор $\hat{H}'$ переводит функцию из состояния $\tilde{\Psi}^{(0)}_n$ в состояние $\tilde{\Psi}^{(0)}_m$, интеграл не равен нулю и получается ненулевой коэффициент вероятности. Такие переходы называются:

\begin{itemize}
\item одноквантовые процессы (поглощение и испускание)
\item двухквантовые процессы (стоксово рассеяние, антистоксово рассеяние, релеевское рассеяние)
\end{itemize}

\begin{figure}[h]
\centering
%\includegraphics[width=0.6\textwidth]{processes.png}
\caption{Одно- и двухквантовые процессы}
\label{fig:10}
\end{figure}

Физический смысл выражения \eqref{eq:101} — вероятность двух переходов: сначала в \textit{виртуальное состояние} $\tilde{\Psi}_k^{(0)}$, затем в состояние $\tilde{\Psi}_m^{(0)}$.

\end{document}