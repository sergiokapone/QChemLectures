%%============================ Compiler Directives =======================%%
%%                                                                        %%
% !TeX program = lualatex
% !TeX encoding = utf8
% !TeX spellcheck = uk_UA
%%                                                                        %%
%%============================== Клас документа ==========================%%
%%                                                                        %%
\documentclass[14pt]{extarticle}
\IfFileExists{ukrcorr.sty}{\usepackage{ukrcorr}}{}
%%                                                                        %%
%%========================== Мови, шрифти та кодування ===================%%
%%                                                                        %%
\usepackage{fontspec}
\setsansfont{CMU Sans Serif}%{Arial}
\setmainfont{CMU Serif}%{Times New Roman}
\setmonofont{CMU Typewriter Text}%{Consolas}
\defaultfontfeatures{Ligatures={TeX}}
\usepackage[math-style=TeX]{unicode-math}
\usepackage[english, russian, ukrainian]{babel}
\usepackage[most]{tcolorbox}

%%                                                                        %%
%%============================= Геометрія сторінки =======================%%
%%                                                                        %%
\usepackage[%
	a4paper,%
	footskip=1cm,%
	headsep=0.3cm,%
	top=2cm, %поле сверху
	bottom=2cm, %поле снизу
	left=2cm, %поле ліворуч
	right=2cm, %поле праворуч
    ]{geometry}
%%                                                                        %%
%%============================== Інтерліньяж  ============================%%
%%                                                                        %%
\renewcommand{\baselinestretch}{1}
%-------------------------  Подавление висячих строк  --------------------%%
\clubpenalty =10000
\widowpenalty=10000
%---------------------------------Інтервали-------------------------------%%
\setlength{\parskip}{0.5ex}%
\setlength{\parindent}{2.5em}%
%%                                                                        %%
%%=========================== Математичні пакети і графіка ===============%%
%%                                                                        %%
\usepackage{amsmath}
\usepackage{graphicx}
\usepackage{floatflt}
%%                                                                        %%
%%========================== Гіперпосилення (href) =======================%%
%%                                                                        %%
\usepackage[colorlinks=true,
	%urlcolor = blue, %Colour for external hyperlinks
	%linkcolor  = malina, %Colour of internal links
	%citecolor  = green, %Colour of citations
	bookmarks = true,
	bookmarksnumbered=true,
	unicode,
	linktoc = all,
	hypertexnames=false,
	pdftoolbar=false,
	pdfpagelayout=TwoPageRight,
	pdfauthor={Ponomarenko S.M. aka sergiokapone},
	pdfdisplaydoctitle=true,
	pdfencoding=auto
	]%
	{hyperref}
		\makeatletter
	\AtBeginDocument{
	\hypersetup{
		pdfinfo={
		Title={\@title},
		}
	}
	}
	\makeatother
%%                                                                        %%
%%============================ Заголовок та автори =======================%%
%%                                                                        %%
\title{Метод обчислення властивостей молекул}
\author{}
\date{}
%%                                                                        %%
%%========================================================================%%


\begin{document}
\maketitle

Після того, як електронна енергія отримана шляхом розв'язання електронного рівняння Шредінгера, можна визначити низку молекулярних властивостей,
серед яких, мабуть, найважливішою є рівноважна молекулярна геометрія. Розрахунок молекулярних структур є цінним доповненням до експериментальних
даних в таких галузях структурної хімії, як рентгенівська кристалографія, електронна дифракція електронів і мікрохвильова спектроскопія. Обчислення
похідних потенціальної енергії за ядерними координатами має вирішальне значення для ефективного визначення рівноважних структур.

Похідні можна обчислити чисельно, обчислюючи потенційну енергію в багатьох геометріях і визначивши зміну енергії при зміні кожної ядерної координати.
Однак градієнтні методи, які визначають похідні енергії аналітично, є обчислювально швидшими і точнішими, ніж чисельне диференціювання.


%% --------------------------------------------------------
\section{Похідні енергії та матриця Гессе}
%% --------------------------------------------------------

З 1969 року, коли П. Пулей написав першу комп'ютерну програму для аналітичного визначення перших похідних енергій SCF, градієнтні методи
перетворилися на одну з найбільш інтенсивно досліджуваних галузей сучасної квантової хімії. Вперше застосовані до розрахунків SCF для замкнених
оболонок, градієнтні методи згодом були узагальнені для розрахунків відкритих оболонок (методи ROHF і UHF). На додаток до розвитку градієнтних
методів для ab initio методик, заснованих на детермінантах Слейтера, були також отримані аналітичні вирази для DFT. Загалом, аналітичні перші та
другі похідні за енергії тепер доступні для ряду рівнів ab initio розрахунків. Для двоатомної молекули молекулярна потенціальна енергія, $E$,
залежить тільки від міжядерної відстані, $R$; отже, щоб знайти потенціальний мінімум (загалом, будь-яку стаціонарну точку), потрібно знайти нуль
похідної $dE/dR$. Пошук є складнішим для багатоатомних молекул, оскільки потенціальна енергія є функцією багатьох ядерних координат, $q_i$. У
рівноважній геометрії кожна з сил $f_i$, що діють на ядро з боку електронів та інших ядер, повинна зникнути:
\begin{equation}
    f_i = - \frac{\partial E}{\partial q_i} = 0.
\end{equation}

Тому, в принципі, рівноважну геометрію можна знайти, обчисливши всі сили при заданій молекулярній геометрії і подивившись, чи зникають вони. Якщо ні,
то геометрію змінюють доти, доки не знайдуть таку, що відповідає нульовим силам, тобто вектору градієнта нульової довжини. Обчислювально сили не
зникнуть однаково, але ми можемо зупинити ітераційний пошук рівноважної геометрії, коли величини сил будуть досить близькими до нуля (тобто, коли
величини будуть меншими за заздалегідь визначений рівень допуску).

Нульовий градієнт характеризує стаціонарну точку на поверхні, але не розрізняє мінімуми, максимуми і сідлові точки. Тому процедура пошуку дозволяє
знайти не тільки геометрію рівноваги стабільної молекули, але й перехідний стан хімічної реакції, який відповідає сідловій точці на поверхні
потенційної енергії. Щоб розрізнити типи стаціонарних точок, необхідно розглянути другі похідні енергії по відношенню до ядерних координат. Величини
$\frac{\partial^2 E}{\partial q_i \partial q_j}$ складають матрицю Гессе. Якщо мінімум (максимум) одновимірної потенціальної кривої відповідає
додатній (від'ємній) другій похідній, то мінімум (максимум) багатовимірної поверхні потенціальної енергії характеризується тим, що всі власні
значення матриці Гессе додатні (від'ємні). Перехідний стан (сідлова точка першого порядку) відповідає одному від'ємному власному значенню, а всі інші
додатні.


%% --------------------------------------------------------
\section{Аналітичні похідні та зв'язані збурені рівняння}
%% --------------------------------------------------------


Існує ряд алгоритмів для пошуку стаціонарних точок на потенціальній поверхні. Загалом, необхідно враховувати стабільність, надійність та
обчислювальну вартість алгоритму, а також швидкість його збіжності. Алгоритми можна умовно розділити на три групи. Ті, що використовують лише
енергію, збігаються найповільніше, але є корисними, якщо аналітичні похідні недоступні. Ті, що використовують як енергію, так і її аналітичні перші
похідні, значно ефективніші (майже на порядок). Крім того, швидкість їхньої збіжності можна покращити, якщо є хороша початкова оцінка матриці Гессе,
можливо, отримана з обчислень ab initio нижчого рівня (наприклад, тих, що використовують менші базисні набори). Алгоритми, які використовують енергію
разом з її аналітичними першою та другою похідними, є найбільш точними та ефективними методами. Який би алгоритм не використовувався, всі ядерні
координати повинні бути оптимізовані; ця оптимізація особливо важлива для перехідних станів, де оптимізація підмножини всіх ядерних координат може
виявити сідлову точку, яка суттєво змінюється, коли оптимізуються всі координати.

Похідні енергії також корисні для визначення інших молекулярних властивостей. Другі похідні енергії за ядерними координатами (елементи матриці
Гессе) є силовими константами для частот нормальної моди в межах гармонічного наближення. Третя, четверта і вищі похідні дають
ангармонічні поправки до частот коливань (Розділ 10.16). Енергетичні похідні не обов'язково повинні обмежуватися ядерними координатами; наприклад,
іноді корисно розглядати похідні за компонентами електричного поля. Змішані другі похідні відносно однієї ядерної координати та однієї компоненти
електричного поля дають похідні дипольного моменту, які використовують для визначення інтенсивності інфрачервоного випромінювання в гармонічному
наближенні.

Для обчислення аналітичних похідних енергії за ядерними координатами необхідно обчислити похідні одно- та двоелектронних інтегралів за базисними
функціями. Оскільки базисні функції зосереджені на ядрах атомів, при визначенні похідних інтегралів нам потрібні похідні функцій базисного набору за
ядерними координатами. Чи потрібні також похідні різних коефіцієнтів розкладу, залежить від того, чи були вони визначені варіаційно, і від порядку
похідної енергії, що розглядається.

Перша похідна енергії за ядерною координатою $q_i$ має вигляд:
\begin{equation}
    \frac{d E}{d q_i} = \frac{\partial E}{\partial q_i} + \sum_j \frac{\partial E}{\partial c_j} \frac{\partial c_j}{\partial q_i}.
\end{equation}

Однак для варіаційно визначених коефіцієнтів розкладу член $ \frac{\partial E}{\partial c_j}$ зникає; отже, маємо важливий результат, що для оцінки
градієнта (першої похідної) енергії, нам не потрібні похідні від варіаційно визначених коефіцієнтів. Як наслідок, у HF і MCSCF аналітичне визначення
енергії аналітичне визначення градієнта енергії потребує похідних лише одно- і двоелектронних інтегралів. Аналогічно, друга похідна від дається через:

\begin{equation}
    \frac{d^2 E}{d q_i^2} = \frac{\partial^2 E}{\partial q_i^2} + \sum_j \left( \frac{\partial^2 E}{\partial q_i \partial c_j} \frac{\partial
    c_j}{\partial q_i} +  \frac{\partial E}{\partial c_j} \frac{\partial^2 c_j}{\partial q_i^2} \right) .
\end{equation}

Оскільки член $\frac{\partial E}{\partial q_i}$ зникає, оцінка другої похідної енергії не потребує другої похідної варіаційно визначених коефіцієнтів:
\begin{equation}
    \frac{d^2 E}{d q_i^2} = \frac{\partial^2 E}{\partial q_i^2} + \sum_j  \frac{\partial^2 E}{\partial q_i \partial c_j} \frac{\partial
    c_j}{\partial q_i} .
\end{equation}

 Однак вона вимагає їхніх перших похідних $\frac{\partial c_j}{\partial q_i} $, які обчислюються \emph{методом зв'язаного збурення} на рівні рівнянь
 Гартрі-Фока (CPHF) для тих методів ab initio, які використовують один детермінант, або на рівні MCSCF для
 тих, які використовують декілька детермінантів. Взагалі, для обчислення похідних енергії порядку $2n + 1$ потрібні
 похідні варіаційно визначених коефіцієнтів порядку $n$. Отже, третя похідна енергії потребує перших похідних варіаційних коефіцієнтів. Ефективний
 метод розв'язання рівнянь CPHF був вперше розроблений Дж. Поплом та його співробітниками в 1979 році, що зробило обчислення другої похідної енергії
 практично можливим для RHF і UHF.

Похідна $\frac{\partial c_j}{\partial q_i} $ називається \emph{вектором хвильової функції лінійного відгуку} і містить інформацію про
те, як змінюється електронна структура під час збурення.

\end{document}



