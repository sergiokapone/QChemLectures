%%============================ Compiler Directives =======================%%
%%                                                                        %%
% !TeX program = lualatex
% !TeX encoding = utf8
% !TeX spellcheck = uk_UA
%%                                                                        %%
%%============================== Клас документа ==========================%%
%%                                                                        %%
\documentclass[14pt]{extarticle}

%%                                                                        %%
%%========================== Мови, шрифти та кодування ===================%%
%%                                                                        %%
\usepackage{fontspec}
\setsansfont{CMU Sans Serif}%{Arial}
\setmainfont{CMU Serif}%{Times New Roman}
\setmonofont{CMU Typewriter Text}%{Consolas}
\defaultfontfeatures{Ligatures={TeX}}
\usepackage[math-style=TeX]{unicode-math}
\usepackage[english, russian, ukrainian]{babel}
\usepackage[most]{tcolorbox}



%%                                                                        %%
%%============================= Геометрія сторінки =======================%%
%%                                                                        %%
\usepackage[%
	a4paper,%
	footskip=1cm,%
	headsep=0.3cm,%
	top=2cm, %поле сверху
	bottom=2cm, %поле снизу
	left=2cm, %поле ліворуч
	right=2cm, %поле праворуч
    ]{geometry}
%%                                                                        %%
%%============================== Інтерліньяж  ============================%%
%%                                                                        %%
\renewcommand{\baselinestretch}{1}
%-------------------------  Подавление висячих строк  --------------------%%
\clubpenalty =10000
\widowpenalty=10000
%---------------------------------Інтервали-------------------------------%%
\setlength{\parskip}{0.5ex}%
\setlength{\parindent}{2.5em}%
%%                                                                        %%
%%=========================== Математичні пакети і графіка ===============%%
%%                                                                        %%
\usepackage{amsmath}
\usepackage{graphicx}
\usepackage{floatflt}

\let\epsilon\varepsilon
%%                                                                        %%
%%========================== Гіперпосилення (href) =======================%%
%%                                                                        %%
\usepackage[colorlinks=true,
	%urlcolor = blue, %Colour for external hyperlinks
	%linkcolor  = malina, %Colour of internal links
	%citecolor  = green, %Colour of citations
	bookmarks = true,
	bookmarksnumbered=true,
	unicode,
	linktoc = all,
	hypertexnames=false,
	pdftoolbar=false,
	pdfpagelayout=TwoPageRight,
	pdfauthor={Ponomarenko S.M. aka sergiokapone},
	pdfdisplaydoctitle=true,
	pdfencoding=auto
	]%
	{hyperref}
		\makeatletter
	\AtBeginDocument{
	\hypersetup{
		pdfinfo={
		Title={\@title},
		}
	}
	}
	\makeatother
%%                                                                        %%
%%============================ Заголовок та автори =======================%%
%%                                                                        %%
\title{Метод обчислення властивостей молекул}
\author{}
\date{}
%%                                                                        %%
%%========================================================================%%


\begin{document}
\maketitle



\section{Взаємодія молекули з електромагнітним полем}

Усі молекулярні спектри є результатом взаємодії молекули з електромагнітним полем. Зовнішнє поле може не лише ініціювати переходи між електронно-коливально-обертальними станами молекул, але й змінювати саму структуру цих станів, зокрема розщеплювати вироджені рівні енергії.

Для опису таких процесів --- тобто для опису стану молекули у присутності зовнішнього електромагнітного поля — необхідно доповнити рівняння Шредінгера для вільної молекули відповідними доданками, які враховують поле та його взаємодію з молекулою. Повний гамільтоніан системи має вигляд:
\[
\hat{H} = \hat{H}_{\text{мол}} + \hat{H}_{\text{поля}} + \hat{H}_{\text{взаємодії}},
\]
де:
\begin{itemize}
  \item $\hat{H}_{\text{мол}}$ --- гамільтоніан молекули, для якого в були розглянуті наближені методи визначення власних функцій та власних значень;
  \item $\hat{H}_{\text{поля}}$ --- гамільтоніан самого електромагнітного поля;
  \item $\hat{H}_{\text{взаємодії}}$ --- оператор, що описує взаємодію між молекулою та полем.
\end{itemize}

У квантовій електродинаміці існує кілька способів опису електромагнітного поля. Один з поширених підходів — представлення поля як сукупності фотонів із різними частотами. У цьому випадку поглинання або випромінювання кванта енергії молекулою відповідає зникненню або виникненню фотона з відповідною енергією:
\[
E = h\nu,
\]
де $h$ — стала Планка, $\nu$ — частота фотона.

Однак точне квантово-польове моделювання взаємодії молекули з випромінюванням є надзвичайно складною задачею. Тому на практиці часто застосовують \textit{напівкласичний підхід}: стан молекули описується в рамках квантової механіки, тоді як електромагнітне поле моделюється класичними функціями напруженості $\vec{E}(t)$ та магнітної індукції $\vec{B}(t)$.

% В такому наближенні гамільтоніан взаємодії може бути поданий, наприклад, як:
%\[
%\hat{H}_{\text{взаємодії}} = -\hat{\vec{\mu}} \cdot \vec{E}(t),
%\]
%де $\hat{\vec{\mu}}$ — оператор електричного дипольного моменту молекули.


\section{Електричні та магнітні характеристики молекул у постійних полях}

\subsection*{Класична модель. Молекула в однорідному електричному полі}

Якщо зовнішнє електричне поле є однорідним і досить слабким, класичну енергію молекули $\epsilon$ можна розкласти в ряд Тейлора за степенями напруженості прикладеного поля. У кожному порядку розкладу реакція молекули на поле визначається певною її власною характеристикою.

У випадку електричного поля цей розклад має вигляд:
\[
\epsilon = \epsilon^{(0)} - \sum_{\rho} \left( \left. \frac{\partial \epsilon}{\partial E_\rho} \right|_{\vec{E}=0} \right) E_\rho
- \frac{1}{2} \sum_{\rho,\sigma} \left( \left. \frac{\partial^2 \epsilon}{\partial E_\rho \partial E_\sigma} \right|_{\vec{E}=0} \right) E_\rho E_\sigma + \ldots
\tag{IV.19}
\]

де:
\begin{itemize}
  \item $\vec{E}$ --- вектор напруженості електричного поля,
  \item $\epsilon^{(0)}$ --- енергія молекули у відсутності поля,
  \item $\rho, \sigma = x, y, z$ --- декартові компоненти.
\end{itemize}

За означенням, компонента вектора електричного дипольного моменту молекули дорівнює:
\[
d_\rho = \left. \frac{\partial e}{\partial E_\rho} \right|_{\vec{E}=0},
\tag{IV.20}
\]

а компонента тензора електричної поляризовності — це:
\[
\alpha_{\rho\sigma} = \left. \frac{\partial^2 e}{\partial E_\rho \partial E_\sigma} \right|_{\vec{E}=0}.
\tag{IV.21}
\]

Таким чином, повна енергія молекули в електричному полі може бути записана у вигляді:
\[
\epsilon = \epsilon^{(0)} - \vec{d} \cdot \vec{E} - \frac{1}{2} \vec{E}^\mathrm{T} \cdot \boldsymbol{\alpha} \cdot \vec{E} + \ldots,
\]

де:
\begin{itemize}
  \item $\vec{d} = (d_x, d_y, d_z)$ — вектор електричного дипольного моменту,
  \item $\boldsymbol{\alpha}$ — тензор поляризовності молекули.
\end{itemize}

\end{document}


