% !TeX program = lualatex
% !TeX encoding = utf8
% !TeX spellcheck = uk_UA

\documentclass[]{beamer}
\usetheme{QuantumChemistry}
\usepackage{QuantumChemistry}
\graphicspath{{pictures/}}
\addbibresource{../Bibliography/QuantumChemistry.bib}
\title[Лекції з квантової хімії]{\bfseries\huge Взаємодія електромагнітного поля з молекулами}
\subtitle{\bfseries Лекції з квантової хімії}
\author{Пономаренко С. М.}
\date{}
% https://www.youtube.com/watch?v=zv9Y7YmHyBk
\def\stext#1{\text{
        \makebox[0pt]{
            \tikz[baseline] {\node[text width=50pt, align=center, execute at begin node=\setlength{\baselineskip}{-0.5ex}, blue] {#1};}
        }
}}

\begin{document}

%============================================================================
\begin{frame}
	\thispagestyle{empty}
	\titlepage
\end{frame}
%============================================================================

% ============================== Слайд ## ===================================
\begin{frame}{}{}
	\begin{block}{}\justifying
		Рівняння Шредінгера для атомів та молекул показує, що атомно-молекулярна система має стаціонарні стани. Випромінювання та поглинання електромагнітної енергії відбувається в результаті переходу між цими станами, а величина кванта дорівнює різниці енергій цих рівнів:
		\begin{equation*}
			\hbar\omega = E_m - E_n.
		\end{equation*}
		Остання формула носить назву \emph{правила частот Бора}.
	\end{block}

	\begin{block}{}\justifying
		\emph{Правило частот Бора} --- умова, необхідна для переходу з одного стану в інший, проте вона не є достатньою: перехід може задовольняти цій вимозі, але бути малоймовірним. Відповідна лінія (або смуга) може мати дуже низьку інтенсивність або навіть не спостерігатися у спектрах. Такі переходи називаються \emph{забороненими}.

	\end{block}

\end{frame}
% ===========================================================================

% ============================== Слайд ## ===================================
%\begin{frame}{}{}
%	\begin{block}{}\justifying
%З класичної точки зору поглинання випромінювання можна розглядати як взаємодію електричної хвилі з електричним диполем молекули.
%
%
%З аналізу збуреного рівняння Шредінгера, що враховує дію зовнішнього електромагнітного поля, був встановлений критерій дозволеності переходів --- \emph{дипольний момент переходу}. Його значення визначається наступним інтегралом:
%
%\[
%\mathbf{p}_{12} = \int \psi_2^*(\mathbf{r}) \, q\hat{\mathbf{r}} \, \psi_1(\mathbf{r}) \, d^3\mathbf{r}
%\]
%
%де $\psi_1$ та $\psi_2$ --- хвильові функції початкового та кінцевого станів відповідно, а $q\hat{\mathbf{r}}$ --- оператор електричного дипольного моменту. Якщо цей інтеграл не дорівнює нулю, перехід є \emph{дозволеним} у дипольному наближенні. Якщо ж інтеграл обертається в нуль через симетрію хвильових функцій, перехід вважається \emph{забороненим}.
%
%
%%		Наявність або відсутність спектральної лінії визначається також, яким чином відбувається квантовий перехід. У квантовій механіці це залежить від форми та симетрії хвильових функцій ствнів, між якими відбувається перехід. Саме ці особливості визначають так звані \emph{правила відбору}, що встановлюють, чи є перехід дозволеним. Іншими словами, дозволений чи заборонений перехід визначається \emph{правилами відбору}, які враховують зміну симетрії хвильових функцій --- тобто просторової будови квантових станів.
%	\end{block}
%\end{frame}
% ===========================================================================

\begin{frame}{Дипольний момент переходу}
	\begin{block}{}\justifying
		Поглинання випромінювання --- це взаємодія \textbf{електричної хвилі} з \textbf{електричним диполем} молекули.

		\medskip

		\textbf{З квантової точки зору:} перехід можливий лише тоді, коли \textbf{дипольний момент переходу} не дорівнює нулю:

		\[
			\mathbf{p}_{12} = \langle \psi_1 | q\hat{\mathbf{r}} | \psi_2 \rangle = \int \psi_2^*(\mathbf{r}) \, q\hat{\mathbf{r}} \, \psi_1(\mathbf{r}) \, d^3\mathbf{r}
		\]
		де $\psi_1$, $\psi_2$ — хвильові функції початкового та кінцевого станів; $q\hat{\mathbf{r}}$ --- оператор електричного дипольного моменту.

		\medskip

		Цей інтеграл відображає середньозважене значення дипольного моменту для стану, що є суперпозицією \( \psi_1 \) та \( \psi_2 \).





	\end{block}
\end{frame}

\begin{frame}{Правила відбору та ймовірність переходу}

		\begin{block}{}\justifying


		\textbf{Дипольний момент переходу} \( | \mathbf{p} | \) можна розглядати як \emph{амплітуду ймовірності} переходу.

		\medskip

		\textbf{Його квадрат} \( | \mathbf{p} |^2 \) є \emph{ймовірністю} переходу.

		\medskip

		Таким чином, можна сформулювати \textbf{правила відбору}:

		\begin{itemize}
			\item Перехід \emph{заборонений}, якщо дипольний момент переходу \( \mathbf{p} = 0 \);
			\item Ймовірність \emph{дозволеного} переходу пропорційна квадрату величини дипольного моменту: \( P \propto |\mathbf{p}|^2 \).
		\end{itemize}
	\end{block}
\end{frame}

\end{document}
