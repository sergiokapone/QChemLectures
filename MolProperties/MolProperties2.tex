% !TeX program = lualatex
% !TeX encoding = utf8
% !TeX spellcheck = uk_UA

\documentclass[]{beamer}
\usetheme{QuantumChemistry}
\usepackage{QuantumChemistry}
\graphicspath{{pictures/}}
\addbibresource{\jobname.bib}
\title[Лекції з квантової хімії]{\bfseries\huge Властивості молекул}
\subtitle{\bfseries Лекції з квантової хімії}
\author{Пономаренко С. М.}
\date{}
% https://www.youtube.com/watch?v=zv9Y7YmHyBk



\begin{document}

%============================================================================
\begin{frame}
	\titlepage
\end{frame}
%============================================================================



\section{Аналіз електронної густини}

%============================================================================
\begin{frame}{Розподіл електронної густини}{}
	\only<1>{
		Розв'язок електронного рівняння Шредінґера $\Phi$ згідно інтерпретації Борна
		\begin{equation*}\label{}
			dW = \Phi^*(\vxi_1, \vxi_2, \ldots, \vxi_{N_e}) \Phi^*(\vxi_1, \vxi_2, \ldots, \vxi_{N_e}) d\xi_1 d\xi_2\ldots d\xi_{N_e} .
		\end{equation*}
		є імовірністю одночасного знаходження першого електрона в елементі $d\xi_1$ координатно-спінового простору, другого --- в елементі $d\xi_2$ і т. д. Для прикладних задач цікаво розподіл електронної густини в \alert{тривимірному просторі} що визначає різноманітні властивості системи і який можна \alert{визначити експериментально} --- наприклад методом електронографії.}
	\only<1-2>{
		\begin{equation*}\label{}
			\tcbhighmath{\rho(\vec{r}) = N_e\int |\Phi|^2 d\sigma_1 d\xi_2\ldots d\xi_{N_e}, \quad \int \rho(\vec{r}) d\vec{r}_1 = N_e}
		\end{equation*}
	}
	\only<2>{
		Для однодетермінантного наближення
		\begin{equation*}\label{}
			\Phi_0 = \mathrm{det}\{\vphi_1, \vphi_2, \ldots, \vphi_{N_e}\}.
		\end{equation*}
		електронна густина для базису $\phi_i =  \sum_{p=1}^M c_{iq} \chi_p$
		\begin{equation*}\label{}
			\tcbhighmath{\rho(\vec{r}) = \sum\limits_{m = 1}^{N_e} n_i |\phi_i|^2 = \sum_{p,q=1}^M \chi_p^*\chi_q\sum_{i = 1}^{N_e} n_i  c^*_{pi} c_{qi} = \sum_{p,q=1}^M \chi_p^*\chi_q R_{pq},}
		\end{equation*}
		$R_{pq}$~--- матриця густини.
	}
\end{frame}
%============================================================================





%============================================================================
%\begin{frame}{Карти електронної густини молекул}{}
%     \begin{center}
%        \includegraphics[width=0.5\linewidth]{Charge_in_water_molecule}
%        \includegraphics[width=0.5\linewidth]{Perfubron}
%    \end{center}
%\end{frame}
%============================================================================




%============================================================================
\tikzstyle{every picture}+=[remember picture]
\begin{frame}[t]{Аналіз атомних заселеностей}{}
	\framesubtitle<2>{Атомні заселеності}
	\framesubtitle<3>{Заселеність по Малікену}
	\begin{onlyenv}<1>
		\begin{equation*}\label{}
			\tcbhighmath{\rho(\vec{r}) = \sum\limits_{m = 1}^{N_e} n_i |\phi_i|^2 = \sum_{p,q=1}^M \chi_p^*\chi_q\sum_{i = 1}^{N_e} n_i  c^*_{pi} c_{qi} = \sum_{p,q=1}^M \chi_p^*\chi_q R_{pq},}
		\end{equation*}
		де $n_i$ --- число електронів на орбіталі $\phi_i$:
		\begin{itemize}
			\item $n_i = 1$ для \emphs{необмеженого методу Хартрі-Фока} (UHF);
			\item $n_i = 2$ для системи із \emphs{замкненими оболонками в обмеженому методі Хартрі-Фока} (RHF);
			\item $n_i = 2$ для спарених електронів і $n_1 = 1$ для неспарених \emphs{в методі Хартрі-Фока з відкритими оболонками} (ROHF).
		\end{itemize}

		\begin{center}\small
			Для визначення електричних властивостей молекули (дипольного, квадрупольного, і. т. д. моментів) необхідно знати, \alert{як розподілена електронна густина між атомами молекули}.
		\end{center}
	\end{onlyenv}
	\begin{onlyenv}<2>
		Вклади в АО для атомів $A$ та $B$
		\begin{equation*}\label{}
			\rho(\vec{r})_{A - B} = \sum_{p,q=1}^{M_A}  \sum_{p,q=1}^{M_B}
			\left(
			\highlight{cyan!50}{\chi_p^{*A}\chi_q^{A} R_{pq}^{A}}{A1} +
			\highlight{red!50}{ \chi_p^{*A}\chi_q^{B} R_{pq}^{AB}}{C1} +
			\highlight{cyan!50}{\chi_p^{*B}\chi_q^{B} R_{pq}^{B}}{B1}
			\right)
		\end{equation*}

		%        \highlighttext{cyan!50}{Густина на атомі $A$}{a1} \highlighttext{cyan!50}{Густина на атомі $B$}{b1} \highlighttext{red!50}{Густина між атомами $A - B$}{c1}
		\begin{tikzpicture}[
				overlay,
				%			arrows={-Latex},
				sig/.style = {anchor=north, font=\scriptsize},
			]
			%\draw<2>[]     (s1) to (pic cs:ge1);
			\draw<2>[->] (A1.south) to ++(-125:0.5cm) node[sig] {Густина на атомі $A$};
			\draw<2>[->] (C1.south) to ++(-90:1cm) node[sig] {Густина на між атомами $A - B$};
			\draw<2>[->] (B1.south) to ++(-55:0.5cm) node[sig] {Густина на атомі $B$};
			%		\draw[] (b1) edge [bend right] (B1);
			%		\draw[] (c1) edge [bend right] (C1);
		\end{tikzpicture}
		\vfill
		\begin{enumerate}
			\item Чиста заселеність атома $A$: \( N(A) = \int \rho^A(\vec r) dV \).
			\item Чиста заселеність атома $B$: \( N(B) = \int \rho^B(\vec r) dV \).
			\item Заселеність \emphs{перекривання}: \( N(AB) = \int \rho^{AB}(\vec r) dV \).
		\end{enumerate}

		\bigskip

		Заселеності між атомами $ N(AB) $ характеризують те, що у класичної теорії хімічної будови називають \emphs{хімічними зв'язками між атомами}.
	\end{onlyenv}
	\begin{onlyenv}<3>

		Чисті заселеності і заселеність перекривання
		\[
			N(A) = \int \rho^A(\vec r) dV, \quad N(B) = \int \rho^B(\vec r) dV, \quad N(AB) = \int \rho^{AB}(\vec r) dV.
		\]
		Заселеність атомів \emphs{по Малікену} (заселеність перекривання ділять навпіл і розкидують між атомами)
		\[
			\tilde{N}(A) = N(A) + \frac12 N(AB), \quad \tilde{N}(B) = N(B) + \frac12 N(AB).
		\]
		Заряди на атомах ($N^0$ --- заряд атома до утворення молекули):
		\[
			q^A = N^0(A) - \tilde{N}(A), \quad  q^B = N^0(B) - \tilde{N}(B).
		\]
	\end{onlyenv}
%	\begin{onlyenv}<4>
%		\begin{enumerate}
%			\item<+-> Заселеність по Льовдіну  (Lowdin Population Analysis --- LPA).
%
%					{\scriptsize Подібно до аналізу по Малікену, але виконується у ортогоналізованому базисі. Інтеграли перекриття між різними орбіталями дорівнюють нулю, зникають вклади заселеностей перекривання, а електронна густина атома дорівнює просто сумі діагональних елементів матриці густини за орбіталями даного атома.}
%
%				%\item Нормальні заселеності (Natural Population Analysis --- NPA).
%
%			\item<+-> Заселеність по Бейдеру (Atoms In Molecules --- AIM).
%
%					{\scriptsize Підхід до аналізу заселеностей заснований на повному ігноруванні орбітальної структури атома та інтегруванні всієї електронної густини, що припадає на об'єм атома.}
%
%			\item<+-> Підгонка\label{key} зарядів під електростатичний потенціал (MESP, CHELPG).
%		\end{enumerate}
%	\end{onlyenv}
\end{frame}
%============================================================================





%============================================================================
\begin{frame}{Класифікація зарядів на атомах}{}
{\scriptsize \fullcite{Cho2020}}

	\begin{enumerate}[Клас I]\small
		\item визначаються на основі вимірюваних властивостей молекули.
		      \begin{itemize}\scriptsize
			      \item з величини дипольного моменту молекул (молекули розглядаються як набір точкових зарядів, відстані між якими відомі),
			      \item моделі вирівнювання електронегативностей атомів у молекулі
			      \item спектроскопічних та діелектричних даних.
		      \end{itemize}
		\item обчислюються з методу МО ЛКАО.
            \begin{itemize}\scriptsize
                \item Заряди по Малікену, по Льовдіну.
            \end{itemize}
        \item виділяються з хвильової функції або електронної густини шляхом підгонки під певну фізичну величину (наприклад, електростатичним потенціалом), що отримується отриманий з них.
        \begin{itemize}\scriptsize
            \item Заряди по Бейдеру, обчислюються інтегруванням електронної густини молекули в межах атомного басейну, обмеженому поверхню нульового потоку вектора градієнта електронної густини.
            %\item Заряди по Хіршфельду, обчислюються інтегруванням деформаційної електронної густини по об'єму, пропорційному відповідним сферично-симетричним атомом в промолекулі.
        \end{itemize}
        \item отримують на основі зарядів класу II або III і підганяють до кращої відповідності з якоюсь фізичною величиною (наприклад, дипольним моментом).
	\end{enumerate}
\end{frame}
%============================================================================





%============================================================================
\begin{frame}{Величини, що залежать від електронної густини}{Дипольний та квадрупольний моменти молекул}
	Дипольний моменти молекул
	\begin{equation*}
		\vec{\mu} = \sum_A Z_AR_A - \int_V \rho(x,y,x) \vec{r} dxdydz.
	\end{equation*}
	В наближенні МО ЛКАО (RHF)
	\begin{equation*}
		\vec{\mu} = \sum_A Z_A \vec{R}_A - 2 \sum_p\sum_q P_{pq}\int_V \vec{r} \chi_p^* \chi_q  dxdydz.
	\end{equation*}
\end{frame}
%============================================================================






%============================================================================
\begin{frame}{Властивості молекул}{}

    \begin{myexample}{Властивості молекул}\scriptsize
        \begin{enumerate}[\faHandORight]
            \item Дипольний момент.
            \item Поляризовність.
            \item Хімічний зсув ЯМР.
            \item Частоти коливань.
            \item Інфра-червоні спектри.
            \item Раманівські спектри.
        \end{enumerate}
    \end{myexample}

    \begin{alertblock}{}\centering
        Фізичні властивості молекул --- характеризуються певною фізичною величиною.
    \end{alertblock}
    \begin{myexample}{Спостережувана величина}
        \begin{tblr}{X[j,m]Q}
            Кожна спостережувана фізична величина $\color{green!50!black}p$ системи в стані $\color{red}\Phi$ в квантовій механіці описується оператором $\color{blue}\hat{P}$ & ${\color{blue}\hat{P}}\,{\color{red}\Phi} = {\color{green!50!black}p}\, {\color{red}\Phi}$ \\
        \end{tblr}
    \end{myexample}
\end{frame}
%============================================================================




%============================================================================
\tikzstyle{every picture}+=[remember picture]
\begin{frame}[t]{Похідні енергії молекули --- властивості}{}
    \begin{columns}
        \begin{column}{0.5\linewidth}
            \begin{center}
                {Теорема Гельмана-Фейнмана}
                \[
                \frac{dE}{d\lambda} = \opbracket{\Phi}{\frac{d\hat{H}}{d\lambda}}{\Phi}
                \]
            \end{center}
        \end{column}
        \begin{column}{0.5\linewidth}
            \begin{alertblock}{}\scriptsize\centering
                Теорема Гельмана-Фейнмана не обов'язково виконується для наближених хвильових функцій, з якими ми зазвичай маємо справу в квантова хімії.
            \end{alertblock}
        \end{column}
    \end{columns}

    \begin{block}{}\centering\large
        {властивість} = \highlight[][red][c1]{відгук} молекули на \highlight[][blue][c2]{збурення ($\lambda$)}
    \end{block}
    \begin{tikzpicture}[overlay]
        \draw[->, red, font=\scriptsize] (c1.south) -- ++(0,-0.5) node[below] {зміна енергії};
        \draw[<-, blue] (c2.south) -- ++(0,-0.5) node[pos=1, below, text width=5.5cm, font=\scriptsize, align=center] {зміна геометрії молекули, \\ вплив електричного \\ або магнітного поля, тощо};
    \end{tikzpicture}

    \vspace{1.5em}

    %	Розкладання енергії в ряд Тейлора:
    \tikzset{prop/.style={fill=gray!15, draw=red}}
    \begin{equation*}
        E(\lambda) = E(0) +
        \highlight[math][prop][d1]{\left(\frac{\partial E}{\partial \lambda}\right)_{0}}\lambda +
        \frac{1}{2!}\highlight[math][prop][d2]{\left(\frac{\partial^2 E}{\partial \lambda^2}\right)_0}\lambda^2 +
        \frac{1}{3!}\highlight[math][prop][d3]{\left(\frac{\partial^3 E}{\partial \lambda^3}\right)_0}\lambda^3 +
        \ldots
    \end{equation*}
    \begin{tikzpicture}[overlay]
        \node[below=10pt of d2, blue] (p) {властивості};
        \draw[->] (d1.south) to[out=-90, in=90] (p.north west);
        \draw[->] (d2.south) to[out=-90, in=90] (p);
        \draw[->] (d3.south) to[out=-90, in=90] (p.north east);
    \end{tikzpicture}
\end{frame}
%============================================================================




%============================================================================
\begin{frame}{Порядок похідної}{}
    \begin{center}
        \begin{tblr}{
                cells = {font=\scriptsize},
                colspec={Q[c,m]Q[c,m]Q[c,m]Q[c,m]X[l,m]},
                cell{1}{1} = {c=4}{c},
                cell{1}{5} = {}{c,m},
                cell{2-Z}{1-4}={}{fg=blue, cmd=\bfseries},
                stretch = -1,
                %            rowsep=-2pt,
                row{3}  = {abovesep=5pt},
                row{Z}  = {belowsep=5pt},
                hline{2} = {2pt},
                hline{3} = {1pt},
            }
            $n$-та похідна енергії &       &       &       & Збурення ($q$ --- координати, $E$ --- зовнішнє електричне поле, $B$ --- зовнішнє магнітне поле, $I$ --- магнітне поле ядра) \\
            $n_R$                  & $n_E$ & $n_B$ & $n_I$ & Властивість                                                                                                                 \\
            1                      & 0     & 0     & 0     & Сила                                                                                                                        \\
            2                      & 0     & 0     & 0     & Частоти гармонічних коливань                                                                                                \\
            3                      & 0     & 0     & 0     & Ангармонічні поправки                                                                                                       \\
            0                      & 1     & 0     & 0     & Електричний дипольний момент $\vec \mu_e$                                                                                   \\
            0                      & 2     & 0     & 0     & Електрична поляризовність $\chi_e$                                                                                          \\
            0                      & 3     & 0     & 0     & Електрична надполяризовність                                                                                                \\
            0                      & 0     & 0     & 0     & Електричний дипольний момент $\vec \mu_m$                                                                                   \\
            0                      & 0     & 1     & 0     & Магнітна сприйнятливість $\chi_m$                                                                                           \\
            0                      & 0     & 2     & 1     & Константа надтонкого зв'язку ЕСР                                                                                            \\
            0                      & 0     & 0     & 2     & Стала спін-спінової взаємодії ядер                                                                                          \\
            1                      & 1     & 0     & 0     & Інтенсивності ІЧ спектрів                                                                                                   \\
            2                      & 1     & 0     & 0     & Інтенсивності обертонів ІЧ спектрах                                                                                         \\
            1                      & 2     & 0     & 0     & Інтенсивності Раманівських спектрів                                                                                         \\
            2                      & 2     & 0     & 0     & Інтенсивності обертонів в Раманівських спектрах                                                                             \\
            0                      & 1     & 1     & 0     & Циркулярний дихроїзм                                                                                                        \\
            0                      & 2     & 1     & 0     & Магнітний циркулярний дихроїзм                                                                                              \\
            0                      & 0     & 1     & 1     & Хімічний зсув                                                                                                               \\
        \end{tblr}
    \end{center}
\end{frame}
%============================================================================




%============================================================================
\begin{frame}{Частоти коливань молекули}{Збурення --- відхилення від рівноважної геометрії}

    \begin{block}{}\tiny
        \fullcite[in 15.10  <<Molecular Geometry>>, 15.11 <<Conformational Searching>>, 15.12 <<Molecular Vibrational Frequencies>>]{Levine}
    \end{block}
    \vspace*{-2em}
    \begin{equation*}
        E(q_1, q_2, \ldots, q_n) = E(0) +
        \sum_i^n
        \underbrace{\frac{\partial E}{\partial q_i}}_\stext{градієнти = сили} q_i +
        \frac{1}{2!}\sum_i^n\sum_j^n
        \underbrace{\frac{\partial^2 E}{\partial q_i q_j}}_\stext{силові константи $k_{ij}$} q_i q_j +
        \ldots
    \end{equation*}
    $q_i$ --- внутрішні координати молекули.
    \vspace*{-1em}
    \begin{equation*}\label{}
        \begin{+bmatrix}
            k_{11} & k_{12} & \cdots & k_{1n} \\
            k_{21} & k_{22} & \cdots & k_{2n} \\
            \vdots & \vdots & \ddots & \vdots \\
            k_{n1} & k_{n2} & \cdots & k_{nn}
        \end{+bmatrix}
        \tikz[baseline, thick] {\draw[->] (0,0) -- node[above, font=\tiny, red] {Діагоналізація} ++(0.75,0);}
        \begin{bmatrix}
            n_{11} & 0      & \cdots & 0      \\
            0      & n_{22} & \cdots & 0      \\
            \vdots & \vdots & \ddots & \vdots \\
            0      & 0      & \cdots & n_{nn}
        \end{bmatrix}
        \tikz[baseline, thick] {\draw[->] (0,0) -- node[above, font=\tiny, red] {маси ядер} ++(0.75,0);}
        \tikz[baseline] {
            \node[font=\tiny, blue, draw=blue, dashed] (A) {частоти коливань};
            \draw[->, overlay, thick] (A) -- ++(0,1) node[above, font=\tiny, text width = 2.25cm, align=center] {ІЧ та Раманівські спектри};
        }
    \end{equation*}

    \begin{exampleblock}{}\scriptsize\centering
        Розрахунок частот в \href{https://sites.google.com/site/orcainputlibrary/vibrational-frequencies-thermochemistry}{\button{ORCA}}
    \end{exampleblock}
\end{frame}
%============================================================================




%============================================================================
\begin{frame}{Електричні властивості молекул}{Збурення --- електричне поле $\vec{\mathcal{E}}$}
    \begin{block}{}\tiny\centering
        \fullcite[12  <<The electric properties of molecules>>]{AtkinsQM}
    \end{block}
    Доданок в гамільтоніані наявності електричного поля:
    \[
    \hat{H}^{(1)} = - \hat{\mu} \vec{\mathcal{E}}
    \]
    З теореми Гельмана-Фейнмана:
    \[
    \left.\frac{dE(\vec{\mathcal{E}})}{d\mathcal{E}_i}\right|_{ \vec{\mathcal{E}} = 0} = \opbracket{\Phi}{\frac{d\hat{H}^{(1)}(\vec{\mathcal{E}})}{d\mathcal{E}_i}}{\Phi} = -\mu_i
    \]


    %\begin{center}
    %            {\scriptsize Дипольний момент}
    %        \(
    %            \mu_i = - \dfrac{\partial E}{\partial \mathcal{E}_i},
    %        \)
    %        \hfill
    %        {\scriptsize Тензор поляризовності}
    %        \(
    %        \alpha_{ij} = - \dfrac{\partial^2 E}{\partial \mathcal{E}_i\mathcal{E}_j}
    %        \)
    %\end{center}

    \begin{equation*}
        E = E(0) +
        %\sum_i^3
        \underbrace{\frac{\partial E}{\partial \mathcal{E}_i}}_\stext{дипольний момент $-\mu_i$} \mathcal{E}_i +
        \frac{1}{2}%\sum_i^3\sum_j^3
        \underbrace{\frac{\partial^2 E}{\partial \mathcal{E}_i \mathcal{E}_j}}_\stext{тензор поляризовності $-\alpha_{ij}$} \mathcal{E}_i \mathcal{E}_j +
        \ldots = -\mu_i \mathcal{E}_i - \frac12 \alpha_{ij} \mathcal{E}_i \mathcal{E}_j + \ldots
    \end{equation*}
    \begin{exampleblock}{}\scriptsize\centering
        Розрахунок електричних властивостей в \href{https://sites.google.com/site/orcainputlibrary/molecular-properties}{\button{ORCA}}
    \end{exampleblock}
\end{frame}
%============================================================================
\end{document}
