% !TeX program = lualatex
% !TeX encoding = utf8
% !TeX spellcheck = uk_UA

\documentclass[]{beamer}
\usetheme{QuantumChemistry}
\usepackage{QuantumChemistry}
\graphicspath{{pictures/}}
\addbibresource{MolProperties.bib}
\title[Лекції з квантової хімії]{\bfseries\huge Властивості молекул}
\subtitle{\bfseries Лекції з квантової хімії}
\author{Пономаренко С. М.}
\date{}
% https://www.youtube.com/watch?v=zv9Y7YmHyBk



\begin{document}

%============================================================================
\begin{frame}
	\titlepage
\end{frame}
%============================================================================



\section{Аналіз електронної густини}

%============================================================================
\begin{frame}{Розподіл електронної густини}{}
	\only<1>{
		Розв'язок електронного рівняння Шредінґера $\Phi$ згідно інтерпретації Борна
		\begin{equation*}\label{}
			dW = \Phi^*(\vxi_1, \vxi_2, \ldots, \vxi_{N_e}) \Phi^*(\vxi_1, \vxi_2, \ldots, \vxi_{N_e}) d\xi_1 d\xi_2\ldots d\xi_{N_e} .
		\end{equation*}
		є імовірністю одночасного знаходження першого електрона в елементі $d\xi_1$ координатно-спінового простору, другого --- в елементі $d\xi_2$ і т. д. Для прикладних задач цікаво розподіл електронної густини в \alert{тривимірному просторі} що визначає різноманітні властивості системи і який можна \alert{визначити експериментально} --- наприклад методом електронографії.}
	\only<1-2>{
		\begin{equation*}\label{}
			\tcbhighmath{\rho(\vec{r}) = N_e\int |\Phi|^2 d\sigma_1 d\xi_2\ldots d\xi_{N_e}, \quad \int \rho(\vec{r}) d\vec{r}_1 = N_e}
		\end{equation*}
	}
	\only<2>{
		Для однодетермінантного наближення
		\begin{equation*}\label{}
			\Phi_0 = \mathrm{det}\{\vphi_1, \vphi_2, \ldots, \vphi_{N_e}\}.
		\end{equation*}
		електронна густина для базису $\phi_i =  \sum_{p=1}^M c_{iq} \chi_p$
		\begin{equation*}\label{}
			\tcbhighmath{\rho(\vec{r}) = \sum\limits_{m = 1}^{N_e} n_i |\phi_i|^2 = \sum_{p,q=1}^M \chi_p^*\chi_q\sum_{i = 1}^{N_e} n_i  c^*_{pi} c_{qi} = \sum_{p,q=1}^M \chi_p^*\chi_q R_{pq},}
		\end{equation*}
		$R_{pq}$~--- матриця густини.
	}
\end{frame}
%============================================================================





%============================================================================
%\begin{frame}{Карти електронної густини молекул}{}
%     \begin{center}
%        \includegraphics[width=0.5\linewidth]{Charge_in_water_molecule}
%        \includegraphics[width=0.5\linewidth]{Perfubron}
%    \end{center}
%\end{frame}
%============================================================================




%============================================================================
\tikzstyle{every picture}+=[remember picture]
\begin{frame}[t]{Аналіз атомних заселеностей}{}
	\framesubtitle<2>{Атомні заселеності}
	\framesubtitle<3>{Заселеність по Малікену}
	\begin{onlyenv}<1>
		\begin{equation*}\label{}
			\tcbhighmath{\rho(\vec{r}) = \sum\limits_{m = 1}^{N_e} n_i |\phi_i|^2 = \sum_{p,q=1}^M \chi_p^*\chi_q\sum_{i = 1}^{N_e} n_i  c^*_{pi} c_{qi} = \sum_{p,q=1}^M \chi_p^*\chi_q R_{pq},}
		\end{equation*}
		де $n_i$ --- число електронів на орбіталі $\phi_i$:
		\begin{itemize}
			\item $n_i = 1$ для \emphs{необмеженого методу Хартрі-Фока} (UHF);
			\item $n_i = 2$ для системи із \emphs{замкненими оболонками в обмеженому методі Хартрі-Фока} (RHF);
			\item $n_i = 2$ для спарених електронів і $n_1 = 1$ для неспарених \emphs{в методі Хартрі-Фока з відкритими оболонками} (ROHF).
		\end{itemize}

		\begin{center}\small
			Для визначення електричних властивостей молекули (дипольного, квадрупольного, і. т. д. моментів) необхідно знати, \alert{як розподілена електронна густина між атомами молекули}.
		\end{center}
	\end{onlyenv}
	\begin{onlyenv}<2>
		Вклади в АО для атомів $A$ та $B$
		\begin{equation*}\label{}
			\rho(\vec{r})_{A - B} = \sum_{p,q=1}^{M_A}  \sum_{p,q=1}^{M_B}
			\left(
			{\color{blue!50}{\chi_p^{*A}\chi_q^{A} R_{pq}^{A}}} +
			{\color{red!50}{ \chi_p^{*A}\chi_q^{B} R_{pq}^{AB}}} +
			{\color{blue!50}{\chi_p^{*B}\chi_q^{B} R_{pq}^{B}}}
			\right)
		\end{equation*}

		%        \highlighttext{cyan!50}{Густина на атомі $A$}{a1} \highlighttext{cyan!50}{Густина на атомі $B$}{b1} \highlighttext{red!50}{Густина між атомами $A - B$}{c1}
		%		\begin{tikzpicture}[
		%				overlay,
		%				%			arrows={-Latex},
		%				sig/.style = {anchor=north, font=\scriptsize},
		%			]
		%			%\draw<2>[]     (s1) to (pic cs:ge1);
		%			\draw<2>[->] (A1.south) to ++(-125:0.5cm) node[sig] {Густина на атомі $A$};
		%			\draw<2>[->] (C1.south) to ++(-90:1cm) node[sig] {Густина на між атомами $A - B$};
		%			\draw<2>[->] (B1.south) to ++(-55:0.5cm) node[sig] {Густина на атомі $B$};
		%			%		\draw[] (b1) edge [bend right] (B1);
		%			%		\draw[] (c1) edge [bend right] (C1);
		%		\end{tikzpicture}
		\vfill
		\begin{enumerate}
			\item Чиста заселеність атома $A$: \( N(A) = \int \rho^A(\vec r) dV \).
			\item Чиста заселеність атома $B$: \( N(B) = \int \rho^B(\vec r) dV \).
			\item Заселеність \emphs{перекривання}: \( N(AB) = \int \rho^{AB}(\vec r) dV \).
		\end{enumerate}

		\bigskip

		Заселеності між атомами $ N(AB) $ характеризують те, що у класичної теорії хімічної будови називають \emphs{хімічними зв'язками між атомами}.
	\end{onlyenv}
	\begin{onlyenv}<3>

		Чисті заселеності і заселеність перекривання
		\[
			N(A) = \int \rho^A(\vec r) dV, \quad N(B) = \int \rho^B(\vec r) dV, \quad N(AB) = \int \rho^{AB}(\vec r) dV.
		\]
		Заселеність атомів \emphs{по Малікену} (заселеність перекривання ділять навпіл і розкидують між атомами)
		\[
			\tilde{N}(A) = N(A) + \frac12 N(AB), \quad \tilde{N}(B) = N(B) + \frac12 N(AB).
		\]
		Заряди на атомах ($N^0$ --- заряд атома до утворення молекули):
		\[
			q^A = N^0(A) - \tilde{N}(A), \quad  q^B = N^0(B) - \tilde{N}(B).
		\]
	\end{onlyenv}
	%	\begin{onlyenv}<4>
	%		\begin{enumerate}
	%			\item<+-> Заселеність по Льовдіну  (Lowdin Population Analysis --- LPA).
	%
	%					{\scriptsize Подібно до аналізу по Малікену, але виконується у ортогоналізованому базисі. Інтеграли перекриття між різними орбіталями дорівнюють нулю, зникають вклади заселеностей перекривання, а електронна густина атома дорівнює просто сумі діагональних елементів матриці густини за орбіталями даного атома.}
	%
	%				%\item Нормальні заселеності (Natural Population Analysis --- NPA).
	%
	%			\item<+-> Заселеність по Бейдеру (Atoms In Molecules --- AIM).
	%
	%					{\scriptsize Підхід до аналізу заселеностей заснований на повному ігноруванні орбітальної структури атома та інтегруванні всієї електронної густини, що припадає на об'єм атома.}
	%
	%			\item<+-> Підгонка\label{key} зарядів під електростатичний потенціал (MESP, CHELPG).
	%		\end{enumerate}
	%	\end{onlyenv}
\end{frame}
%============================================================================





%============================================================================
\begin{frame}{Класифікація зарядів на атомах}{}
	{\scriptsize \fullcite{Cho2020}}

	\begin{enumerate}[Клас I]\small
		\item визначаються на основі вимірюваних властивостей молекули.
		      \begin{itemize}\scriptsize
			      \item з величини дипольного моменту молекул (молекули розглядаються як набір точкових зарядів, відстані між якими відомі),
			      \item моделі вирівнювання електронегативностей атомів у молекулі
			      \item спектроскопічних та діелектричних даних.
		      \end{itemize}
		\item обчислюються з методу МО ЛКАО.
		      \begin{itemize}\scriptsize
			      \item Заряди по Малікену, по Льовдіну.
		      \end{itemize}
		\item виділяються з хвильової функції або електронної густини шляхом підгонки під певну фізичну величину (наприклад, електростатичним потенціалом), що отримується отриманий з них.
		      \begin{itemize}\scriptsize
			      \item Заряди по Бейдеру, обчислюються інтегруванням електронної густини молекули в межах атомного басейну, обмеженому поверхню нульового потоку вектора градієнта електронної густини.
			            %\item Заряди по Хіршфельду, обчислюються інтегруванням деформаційної електронної густини по об'єму, пропорційному відповідним сферично-симетричним атомом в промолекулі.
		      \end{itemize}
		\item отримують на основі зарядів класу II або III і підганяють до кращої відповідності з якоюсь фізичною величиною (наприклад, дипольним моментом).
	\end{enumerate}
\end{frame}
%============================================================================





%============================================================================
\begin{frame}{Величини, що залежать від електронної густини}{Дипольний та квадрупольний моменти молекул}
	Дипольний моменти молекул
	\begin{equation*}
		\vec{\mu} = \sum_A Z_AR_A - \int_V \rho(x,y,x) \vec{r} dxdydz.
	\end{equation*}
	В наближенні МО ЛКАО (RHF)
	\begin{equation*}
		\vec{\mu} = \sum_A Z_A \vec{R}_A - 2 \sum_p\sum_q P_{pq}\int_V \vec{r} \chi_p^* \chi_q  dxdydz.
	\end{equation*}
\end{frame}
%============================================================================






%============================================================================
\begin{frame}{Властивості молекул}{}

	\begin{myexample}{Властивості молекул}\scriptsize
	\begin{enumerate}[\faHandORight]
		\item Дипольний момент.
		\item Поляризовність.
		\item Хімічний зсув ЯМР.
		\item Частоти коливань.
		\item Інфра-червоні спектри.
		\item Раманівські спектри.
	\end{enumerate}
	\end{myexample}

	\begin{alertblock}{}\centering
		Фізичні властивості молекул --- характеризуються певною фізичною величиною.
	\end{alertblock}
	\begin{myexample}{Спостережувана величина}
	\begin{tblr}{X[j,m]Q}
		Кожна спостережувана фізична величина $\color{green!50!black}p$ системи в стані $\color{red}\Phi$ в квантовій механіці описується оператором $\color{blue}\hat{P}$ & ${\color{blue}\hat{P}}\,{\color{red}\Phi} = {\color{green!50!black}p}\, {\color{red}\Phi}$ \\
	\end{tblr}
	\end{myexample}
\end{frame}
%============================================================================



%% --------------------------------------------------------
\section{Дипольний момент молекули}
%% --------------------------------------------------------


% ============================== Слайд ## ===================================
\begin{frame}{Дипольний момент молекули}{}
	\begin{block}{}
		Оператор дипольного моменту:
		\begin{equation*}
			\hat{\vec{\mu}}_e = \underbrace{\sum_a^{N_n} Z_a\vec{R}_a}_{\text{ядра}} +
			\underbrace{\sum_i^{N_e}-e\vec{r}_i}_{\text{електрони}}
		\end{equation*}

		Розрахунок \alert{постійного дипольного моменту} в квантовій механіці:

		\begin{equation*}
			\vec{\mu}_e = \bracket<\Phi_0|\hat{\vec{\mu}}_e|\Phi_0>.
		\end{equation*}
		Обчислення будуть точними, якщо $\Phi_0$ є точним розв'язком рівняння Шредінґера:

		\begin{equation*}
			\hat{H}_0\Phi_0 = E\Phi_0.
		\end{equation*}
	\end{block}


\end{frame}
% ===========================================================================

% ============================== Слайд ## ===================================
\begin{frame}{Молекула під впливом зовнішнього поля}{}
	\begin{block}{}\justifying
		В квантовій механіці, система, яка знаходиться під впливом зовнішніх полів описується гамільтоніаном:
		\begin{equation*}
			\hat{H} = \hat{H}_0 + \hat{H}_\text{field}.
		\end{equation*}
		Якщо \alert{нейтральна} молекула знаходиться в електричному полі, то гамільтоніан системи буде мати вигляд:
		\begin{equation*}
			\hat{H} = \hat{H}_0 - ({\hat{\vec{\mu}}_e}\cdot{\vec{\mathcal{E}}}),
		\end{equation*}
		де $\vec{\mathcal{E}}$ --- напруженість електричного поля.
		Теорема Гелмана-Фейнмана дає змогу в цьому випадку знайти дипольний момент як похідну енергії системи по
		електричному полю:
		\begin{equation*}
			\left.\frac{dE(\vec{\mathcal{E}})}{d\mathcal{E}_i}\right|_{\mathcal{E} = 0} = \bracket<\Phi_0|\frac{\partial
				\hat{H}}{\partial
				\mathcal{E}_i}|\Phi_0> = -
			\bracket<\Phi_0|\hat{\vec{\mu}}_e|\Phi_0> = - (\mu_e).
		\end{equation*}
	\end{block}
\end{frame}
% ===========================================================================

% ============================== Слайд ## ===================================
\begin{frame}{Молекула під впливом зовнішнього поля}{}
	\begin{block}{}\justifying
		При взаємодії молекули з зовнішнім електричним полем, розподіл електронної густини, а отже і сам дипольний
		момент молекули буде
		змінюватись. В результаті до постійного дипольного моменту додається наведений (індукований) момент:
		\begin{equation*}
			\mu_i(\vec{\mathcal{E}})      = \mu_i^\text{perm} + \mu_i^\text{ind}(\vec{\mathcal{E}})
		\end{equation*}
		Розкладання дипольного моменту в ряд
		\begin{equation*}\label{}
			\mu_i(\vec{\mathcal{E}}) = \mu_i^\text{perm} + \sum_j \chi_{ij}\mathcal{E}_j + \frac12\sum_{jk}
			\beta_{ijk} \mathcal{E}_j \mathcal{E}_k + \ldots,
		\end{equation*}
		Дипольні поляризовність та гіперполяризовність
		\begin{equation*}\label{}
			\tcbhighmath[title = {\scriptsize поляризовність}]{
			\chi_{ij} = \left.\frac{d\mu_i(\vec{\mathcal{E}})}{d\mathcal{E}_j}\right|_{\vec{\mathcal{E}} = 0}
			}
			\quad
			\tcbhighmath[title = {\scriptsize гіперполяризовність}]{
			\beta_{ijk} = \left.\frac{d\mu_i(\vec{\mathcal{E}})}{d\mathcal{E}_j
				d\mathcal{E}_k}\right|_{\vec{\mathcal{E}}
			= 0}
			}
		\end{equation*}
	\end{block}
\end{frame}
% ===========================================================================



% ============================== Слайд #7 ===================================
\begin{frame}{Електричні властивості як похідні енергії}{}
	\begin{itemize}
		\item Інфінітезимальна зміна енергії при інфінітезимальній зміні поля  $d\mathcal{E}_\alpha$ (дипольна
		      складова)
		      \begin{equation*}
			      dE = - \mu(\vec{\mathcal{E}}) d\vec{\mathcal{E}}
		      \end{equation*}
		\item Енергію можна отримати інтегруванням
		      \begin{multline*}
			      E(\mathcal{E}) - E_0 = -\int\limits_0^{\mathcal{E}} \mu(\vec{\mathcal{E}}) d\vec{\mathcal{E}} = \\
			      = -\sum_i \mu_i^\text{perm}\mathcal{E}_i - \frac12\sum_{ij}
			      \chi{ij}\mathcal{E}_i\mathcal{E}_j - \frac16\sum_{ijk} \beta_{ijk} =
			      \mathcal{E}_i\mathcal{E}_j\mathcal{E}_k + \ldots
		      \end{multline*}
		\item Властивості за відсутності поля ($\vec{\mathcal{E}} = 0$) можна отримати диференціюючи енергію:
	\end{itemize}
	\begin{equation*}
		\mu_i^\text{perm} = - \left.\frac{dE}{d\mathcal{E}_i}\right|_{\vec{\mathcal{E}}=0}
		\,\,
		\chi_{ij} = - \left.\frac{d^2E}{d\mathcal{E}_i d\mathcal{E}_j}\right|_{\vec{\mathcal{E}}=0}
		\,\,
		\beta_{ijk} = - \left.\frac{d^3E}{d\mathcal{E}_i d\mathcal{E}_j d\mathcal{E}_k}\right|_{\vec{\mathcal{E}}=0}
	\end{equation*}
\end{frame}
%============================================================================


% ============================== Слайд ## ===================================
\begin{frame}{Приклад теорії збурень}{}
	\begin{block}{}\justifying\
		Якщо вважати $\hat{H}^{(1)} = - ({\hat{\vec{\mu}}_e}\cdot{\vec{\mathcal{E}}}) $ збуренням, то
		використовуючи теорію збурень можна
		показати, що:
		\begin{multline*}
			E(\mathcal{E}) = E_0 + \bracket<\Phi_0|\hat{H}^{(1)}|\Phi_0>
			+ \sum_{n\neq 0}
			\frac{\bracket<\Phi_0|\hat{H}^{(1)}|\Phi_n>\bracket<\Phi_n|\hat{H}^{(1)}|\Phi_0>}{E_0 - E_n}+
			\ldots = \\
			= E_0 -  \bracket<\Phi_0|\hat{\vec{\mu}}|\Phi_0>\cdot \vec{\mathcal{E}} +
			\sum_{n\neq 0}
			\frac{\vec{\mathcal{E}}\cdot\bracket<\Phi_0|\hat{\vec{\mu}}|\Phi_n>\bracket<\Phi_n|\hat{\vec{\mu}}|\Phi_0>\cdot\vec{\mathcal{E}}}{E_0
				- E_n} + \ldots ,
		\end{multline*}
		де дипольний момент:
		\begin{equation*}
			\mu_i = \bracket<\Phi_0|\hat{\mu_i}|\Phi_0>,
		\end{equation*}
		поляризовність:
		\begin{equation*}
			\chi_{ij} = -2\sum_{n\neq 0} \frac{\bracket<\Phi_0|\hat{\mu}_i|\Phi_n>\bracket<\Phi_n|\hat{\mu}_j|\Phi_0>}{E_0
				- E_n}
		\end{equation*}
	\end{block}
\end{frame}
% ===========================================================================



% ============================== Слайд ## ===================================
\begin{frame}{Властивості, що обчислюються на основі дипольного моменту}{Інтенсивності дипольних переходів}

	\begin{block}{Золоте правило Фермі}
		Інтенсивності спектральних ліній пропорційні імовірність переходу в одиницю часу з вихідного стану $i$ в кінцевий
		стан $f$:
		\begin{equation*}
			I \sim \left|\bracket<\Phi_f|\hat{V}|\Phi_i>\right|^2\rho(E_f)|^2,
		\end{equation*}
		де $\bracket<\Phi_f|\hat{V}|\Phi_i>$ --- метричний елемент збурення між початковим і кінцевим станом, $\rho(E_f)$
		--- густина стані в околі енергії $E_f$.
	\end{block}

	\begin{block}{}\justifying
		Якщо в молекули є \alert{дипольний момент}, то при взаємодії з електричним полем, інтенсивність переходу можна
		записати як:
		\begin{equation*}
			I \sim \left|\bracket<\Phi_f|\hat{\vec{\mu}}|\Phi_i>\right|^2\mathcal{E}^2\rho(E_f)^2.
		\end{equation*}
	\end{block}
\end{frame}

% ============================== Слайд ## ===================================
\begin{frame}{Інтенсивності коливальних переходів}{Диатомні молекули}
	\begin{onlyenv}<1>
		\begin{block}{}\justifying
			Електричний дипольний момент переходу для з стану $i \to 0$:
			\begin{equation*}
				\bracket<i|\hat{\vec{\mu}}|0>
			\end{equation*}
			Переходи відбуваються в межах даного електронного стану $E_e$. Дипольний момент залежить від довжини зв'язку
			$R$,
			можна виразити його зміну зі зміщенням ядер від рівноваги як:
			\begin{equation*}
				\vec{\mu} = \left(\frac{d\vec{\mu}}{dR}\right)_{R=R_0} \Delta R +
				\frac12 \left(\frac{d^2\vec{\mu}}{dR^2}\right)_{R=R_0}
				\Delta R^2 + \ldots.
			\end{equation*}
		\end{block}
		Елементи матриці дипольних переходів:
		\begin{equation*}
			\bracket<\Phi_i|\hat{\vec{\mu}}|0> = \vec{\mu}_0\bracket<i|0> +
			\left.\frac{d\vec{\mu}}{dR}\right|_{0} \bracket<i|\Delta R|0> +
			\frac12 \left.\frac{d^2\vec{\mu}}{dR^2}\right|_{0} \bracket<i|\Delta R^2|0> +\ldots.
		\end{equation*}
		Оскільки $\bracket<i|0> = 0$ --- стани ортогональні, перший доданок дорівнює нулю.
	\end{onlyenv}
	\begin{onlyenv}<2>
		\begin{block}{}\justifying
			Інтенсивність коливальних дипольних переходів:
			\begin{equation*}
				I \sim \left(\frac{d\vec{\mu}}{dR}\right)^2 = \sum_i\left(\frac{d^2E}{dR d\mathcal{E}_i}\right)^2
			\end{equation*}
			Коливальні спектри мають лише ті диатомні молекули, які мають дипольний момент, який змінюється зі збільшенням
			відстані між ядрами. Гомоядерні двоатомні молекули не зазнають електричних
			дипольних коливальних переходів.
		\end{block}
	\end{onlyenv}
\end{frame}
% ===========================================================================

% ============================== Слайд ## ===================================
\begin{frame}{Рахманівські спектри}{}
	\begin{block}{Раманівський спектр}\justifying
		Спектр непружно розсіяного випромінення на частинках даної сполуки; становить систему супутніх ліній,
		розташованих симетрично відносно незміщеної лінії, частота якої збігається з частотою збуджуючого світла. Кожній
		супутній лінії з меншою частотою (червоній, стоксовій) відповідає фіолетова, чи антистоксова, з вищою частотою.
	\end{block}
	%---------------------------------------------------------
	\begin{figure}[h!]\centering
		\includegraphics[width=0.6\linewidth]{KIspectrum}
		\caption*{Раманівський спектр йодиту калію \ce{KI}}
	\end{figure}
	%---------------------------------------------------------
\end{frame}
% ===========================================================================

% ============================== Слайд ## ===================================
\begin{frame}{Рахманівські спектри}{Класична теорія}
	\begin{onlyenv}<1>
		\begin{block}{}
			Раманівські спектри виникають завдяки наявності в молекули поляризовності --- міри реакції молекули на
			електричне поле.

			Розгляненмо дипольний момент, індукований у молекулі залежним від часу електромагнітним полем $\mathcal{E} =
				\mathcal{E}_0\cos\omega t$:
			\begin{equation*}
				\mu(t) = \chi(t)\mathcal{E}
			\end{equation*}
			Якщо поляризованість молекули змінюється між $\chi_{\max}$ і $\chi_{\min}$ з частотою $\omega_0$ при
			обертанні або
			коливання за законом $\chi(t) = \chi_0 + \frac12\Delta\chi\cos(\omega_0 t)$, то
			\begin{multline*}
				\mu(t) = (\chi_0 + \frac12\Delta\chi\cos(\omega_0 t))\mathcal{E}_0\cos\omega t = \\
				= \chi_0 \mathcal{E}_0\cos\omega t +  \frac12\Delta\chi\mathcal{E}_0\cos\omega t \cos(\omega_0 t) = \\
				= \chi_0 \mathcal{E}_0\cos\omega t  + \frac14\Delta\chi\mathcal{E}_0(\cos(\omega + \omega_0)t +
				\cos(\omega
				- \omega_0) t).
			\end{multline*}
		\end{block}
	\end{onlyenv}
	\begin{onlyenv}<2>
		\begin{block}{}
			\begin{equation*}
				\mu(t) = \chi_0 \mathcal{E}_0\cos\omega t  + \frac14\Delta\chi\mathcal{E}_0(\cos(\omega + \omega_0)t +
				\cos(\omega
				- \omega_0) t).
			\end{equation*}
			Індукований дипольний момент має три компоненти:
			\begin{enumerate}
				\item Перша $\chi_0 \mathcal{E}_0\cos\omega t$ --- має частоту падаючого випромінювання і спричиняє незсунуту
				      \alert{релеєвську лінію} в спектрі.
				\item Друга з частотою $\omega + \omega_0$ \alert{антистоксова лінія} і
				\item третя з частотою $\omega - \omega_0$ \alert{стоксова лінія}
			\end{enumerate}
			Таким чином, коли молекула опромінюється монохроматичним світлом із частотою $\omega$
			у результаті індукованої електронної поляризації вона розсіює випромінювання як із частотою $\omega + \omega_0$,
			так і з частотою $\omega - \omega_0$. При квантовому розгляді таких стає більше.
		\end{block}
	\end{onlyenv}
\end{frame}
% ===========================================================================


%\begin{block}{}
%    Компонента сили осцилятора:
%    \begin{multline*}
%        (f_{n})_i =
%\frac{2}{3}(E_n - E_0)^2|\bracket<\Phi_n|\hat{\mu}_i|\Phi_0>|^2 = \\
%    = \frac{2m_e}{\hbar^2e^2}(E_n - E_0)\bracket<\Phi_n|\hat{\mu}_i|\Phi_0>\bracket<\Phi_0|\hat{\mu}_i|\Phi_n> = \\
%    =  \frac{2}{3}(E_n - E_0)^2 \chi_{ii} = \frac{1}{3}(E_n - E_0)^2 \left. \frac{d\mu_i
%    }{d\mathcal{E}_i}\right|_{\vec{\mathcal{E}} = 0}.
%    \end{multline*}
%
%\end{block}

%\end{frame}
% ===========================================================================


\section{Теорія відгуку}





% ============================== Слайд #8 ===================================
\begin{frame}{Теорія відгуку}{Відгук --- реакція на збурення}

	\begin{alertblock}{}\centering
		Молекулярні властивості можуть бути отримані з використанням похідних електронної енергії або молекулярних
		моментів за збуренням!
	\end{alertblock}

	\begin{block}{}\centering\large
		{властивість} = \highlight[][red][c1]{відгук} молекули на \highlight[][blue][c2]{збурення
			($\vec{\mathcal{F}}$)}
	\end{block}
	\begin{tikzpicture}[overlay]
		\draw[->, red, font=\scriptsize] (c1.south) -- ++(0,-0.5) node[below] {зміна енергії};
		\draw[<-, blue] (c2.south) -- ++(0,-0.5) node[pos=1, below, text width=5.5cm, font=\scriptsize,
			align=center] {зміна геометрії молекули, \\
			вплив електричного \\ або магнітного поля, тощо};
	\end{tikzpicture}

	\vspace{1.5em}

	%	Розкладання енергії в ряд Тейлора:
	\begin{equation*}\small
		\tcbset{highlight math/.append style={left=0mm,right=0mm,top=0mm,bottom=0mm}}
		E(\vec{\mathcal{F}}) = E(0) +
		\tcbhighmath[remember as=d1]{
		\left(\frac{d E}{d \vec{\mathcal{F}}}\right)_{\vec{\mathcal{F}}=0}
		}\vec{\mathcal{F}} + \frac{1}{2!}
		\tcbhighmath[remember as=d2]{
		\left(\frac{d^2 E}{d \vec{\mathcal{F}}^2}\right)_{\vec{\mathcal{F}}=0}
		}\vec{\mathcal{F}}^2 + \frac{1}{3!}
		\tcbhighmath[remember as=d3]{
		\left(\frac{d^3 E}{d \vec{\mathcal{F}}^3}\right)_{\vec{\mathcal{F}}=0}
		}\vec{\mathcal{F}}^3 +
		\ldots
	\end{equation*}
	\begin{tikzpicture}[overlay]
		\node[below=10pt of d2, blue] (p) {властивості};
		\draw[->] (d1.south) to[out=-90, in=90] (p.north west);
		\draw[->] (d2.south) to[out=-90, in=90] (p);
		\draw[->] (d3.south) to[out=-90, in=90] (p.north east);
	\end{tikzpicture}
\end{frame}
%============================================================================



%============================================================================
\begin{frame}{Порядок похідної}{}
	\begin{center}
		\begin{tblr}{
			cells = {font=\scriptsize},
			colspec={Q[c,m]Q[c,m]Q[c,m]Q[c,m]X[l,m]},
			cell{1}{1} = {c=4}{c},
			cell{1}{5} = {}{c,m},
			cell{2-Z}{1-4}={}{fg=blue, cmd=\bfseries},
			stretch = -1,
			%            rowsep=-2pt,
			row{3}  = {abovesep=5pt},
			row{Z}  = {belowsep=5pt},
			hline{2} = {2pt},
			hline{3} = {1pt},
			}
			$n$-та похідна енергії &       &       &       & Збурення ($q$ --- координати, $E$ --- зовнішнє електричне поле, $B$ --- зовнішнє магнітне поле, $I$ --- магнітне поле ядра) \\
			$n_R$                  & $n_E$ & $n_B$ & $n_I$ & Властивість                                                                                                                 \\
			1                      & 0     & 0     & 0     & Сила                                                                                                                        \\
			2                      & 0     & 0     & 0     & Частоти гармонічних коливань                                                                                                \\
			3                      & 0     & 0     & 0     & Ангармонічні поправки                                                                                                       \\
			0                      & 1     & 0     & 0     & Електричний дипольний момент $\vec \mu_e$                                                                                   \\
			0                      & 2     & 0     & 0     & Електрична поляризовність
			$\chi_e$                                                                                                                                                                     \\
			0                      & 3     & 0     & 0     & Електрична надполяризовність                                                                                                \\
			0                      & 0     & 0     & 0     & Електричний дипольний момент $\vec \mu_m$                                                                                   \\
			0                      & 0     & 1     & 0     & Магнітна сприйнятливість $\chi_m$                                                                                           \\
			0                      & 0     & 2     & 1     & Константа надтонкого зв'язку ЕСР                                                                                            \\
			0                      & 0     & 0     & 2     & Стала спін-спінової взаємодії ядер                                                                                          \\
			1                      & 1     & 0     & 0     & Інтенсивності ІЧ спектрів                                                                                                   \\
			2                      & 1     & 0     & 0     & Інтенсивності обертонів ІЧ спектрах                                                                                         \\
			1                      & 2     & 0     & 0     & Інтенсивності Раманівських спектрів                                                                                         \\
			2                      & 2     & 0     & 0     & Інтенсивності обертонів в Раманівських спектрах                                                                             \\
			0                      & 1     & 1     & 0     & Циркулярний дихроїзм                                                                                                        \\
			0                      & 2     & 1     & 0     & Магнітний циркулярний дихроїзм                                                                                              \\
			0                      & 0     & 1     & 1     & Хімічний зсув                                                                                                               \\
		\end{tblr}
	\end{center}
\end{frame}
%============================================================================




%============================================================================
%\begin{frame}{Частоти коливань молекули}{Збурення --- відхилення від рівноважної геометрії}
%
%    \begin{block}{}\tiny
%        \fullcite[in 15.10  <<Molecular Geometry>>, 15.11 <<Conformational Searching>>, 15.12 <<Molecular
%Vibrational Frequencies>>]{Levine}
%    \end{block}
%    \vspace*{-2em}
%    \begin{equation*}
%        E(q_1, q_2, \ldots, q_n) = E(0) +
%        \sum_i^n
%        \underbrace{\frac{\partial E}{\partial q_i}}_\stext{градієнти = сили} q_i +
%        \frac{1}{2!}\sum_i^n\sum_j^n
%        \underbrace{\frac{\partial^2 E}{\partial q_i q_j}}_\stext{силові константи $k_{ij}$} q_i q_j +
%        \ldots
%    \end{equation*}
%    $q_i$ --- внутрішні координати молекули.
%    \vspace*{-1em}
%    \begin{equation*}\label{}
%        \begin{+bmatrix}
%            k_{11} & k_{12} & \cdots & k_{1n} \\
%            k_{21} & k_{22} & \cdots & k_{2n} \\
%            \vdots & \vdots & \ddots & \vdots \\
%            k_{n1} & k_{n2} & \cdots & k_{nn}
%        \end{+bmatrix}
%        \tikz[baseline, thick] {\draw[->] (0,0) -- node[above, font=\tiny, red] {Діагоналізація} ++(0.75,0);}
%        \begin{bmatrix}
%            n_{11} & 0      & \cdots & 0      \\
%            0      & n_{22} & \cdots & 0      \\
%            \vdots & \vdots & \ddots & \vdots \\
%            0      & 0      & \cdots & n_{nn}
%        \end{bmatrix}
%        \tikz[baseline, thick] {\draw[->] (0,0) -- node[above, font=\tiny, red] {маси ядер} ++(0.75,0);}
%        \tikz[baseline] {
%            \node[font=\tiny, blue, draw=blue, dashed] (A) {частоти коливань};
%            \draw[->, overlay, thick] (A) -- ++(0,1) node[above, font=\tiny, text width = 2.25cm, align=center] {ІЧ
%та Раманівські спектри};
%        }
%    \end{equation*}
%
%    \begin{exampleblock}{}\scriptsize\centering
%        Розрахунок частот в
%\href{https://sites.google.com/site/orcainputlibrary/vibrational-frequencies-thermochemistry}{\button{ORCA}}
%    \end{exampleblock}
%\end{frame}
%============================================================================




%============================================================================
%\begin{frame}{Електричні властивості молекул}{Збурення --- електричне поле $\vec{\mathcal{E}}$}
%    \begin{block}{}\tiny\centering
%        \fullcite[12  <<The electric properties of molecules>>]{AtkinsQM}
%    \end{block}
%    Доданок в гамільтоніані наявності електричного поля:
%    \[
%    \hat{H}^{(1)} = - \hat{\mu} \vec{\mathcal{E}}
%    \]
%    З теореми Гельмана-Фейнмана:
%    \[
%    \left.\frac{dE(\vec{\mathcal{E}})}{d\mathcal{E}_i}\right|_{ \vec{\mathcal{E}} = 0} =
%\opbracket{\Phi}{\frac{d\hat{H}^{(1)}(\vec{\mathcal{E}})}{d\mathcal{E}_i}}{\Phi} = -\mu_i
%    \]


%\begin{center}
%            {\scriptsize Дипольний момент}
%        \(
%            \mu_i = - \dfrac{\partial E}{\partial \mathcal{E}_i},
%        \)
%        \hfill
%        {\scriptsize Тензор поляризовності}
%        \(
%        \alpha_{ij} = - \dfrac{\partial^2 E}{\partial \mathcal{E}_i\mathcal{E}_j}
%        \)
%\end{center}

%    \begin{equation*}
%        E = E(0) +
%        %\sum_i^3
%        \underbrace{\frac{\partial E}{\partial \mathcal{E}_i}}_\stext{дипольний момент $-\mu_i$} \mathcal{E}_i +
%        \frac{1}{2}%\sum_i^3\sum_j^3
%        \underbrace{\frac{\partial^2 E}{\partial \mathcal{E}_i \mathcal{E}_j}}_\stext{тензор поляризовності
%$-\alpha_{ij}$} \mathcal{E}_i
%\mathcal{E}_j +
%        \ldots = -\mu_i \mathcal{E}_i - \frac12 \alpha_{ij} \mathcal{E}_i \mathcal{E}_j + \ldots
%    \end{equation*}
%    \begin{exampleblock}{}\scriptsize\centering
%        Розрахунок електричних властивостей в
%\href{https://sites.google.com/site/orcainputlibrary/molecular-properties}{\button{ORCA}}
%    \end{exampleblock}
%\end{frame}
%============================================================================
\end{document}
