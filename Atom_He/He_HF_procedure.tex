% !TeX program = lualatex
% !TeX encoding = utf8
% !TeX spellcheck = uk_UA

\documentclass[]{article}
\usepackage[fontsize=14pt]{fontsize}

\usepackage{fontspec}
\setsansfont{CMU Sans Serif}%{Arial}
\setmainfont{CMU Serif}%{Times New Roman}
\setmonofont{CMU Typewriter Text}%{Consolas}


\defaultfontfeatures{Ligatures={TeX}}
\usepackage[math-style=TeX]{unicode-math}
\usepackage[russian, english, ukrainian]{babel}

\usepackage[autolang=other]{biblatex}

\usepackage{microtype}
\usepackage{amsmath}
\usepackage{indentfirst}

\usepackage[most]{tcolorbox}
\tcbset{highlight math style={enhanced,
  colframe=red,colback=white,arc=0pt,boxrule=1pt}}
\tcbset{
    mybox/.style={
        enhanced,
        colback=gray!5,
        fonttitle=\bfseries,
        sharp corners,
        boxrule=0pt,
        frame empty
    }
}
\usepackage{subcaption}
\captionsetup[subfigure]{justification=centering}
\usepackage{hyperref}

\usepackage{titlesec}
\titlelabel{\thetitle.\space}
\usepackage{minted}
\usepackage{tikz, calc}
\usetikzlibrary{backgrounds}

\usepackage{enumitem}


\addbibresource{d:/Projects/LaTeX/QChem/A_Documents/Syllabus_QChem/Syllabus_QChem.bib}

%%                                                                        %%
%%============================= Геометрія сторінки =======================%%
%%                                                                        %%
\usepackage[%
	a4paper,%
	footskip=1cm,%
	headsep=0.3cm,%
	top=2cm, %поле сверху
	bottom=2cm, %поле снизу
	left=2cm, %поле ліворуч
	right=2cm, %поле праворуч
    ]{geometry}
%%                                                                        %%
%%============================== Інтерліньяж  ============================%%

\usepackage{tikz}
\usepackage{pgfplots}
\usepgfplotslibrary{units}
\pgfplotsset{compat=newest}
\usepackage{pgfplotstable}
\usepgfplotslibrary{groupplots}
\usepackage{xifthen}



\def\C#1{{\color{teal}#1}}
\def\a#1{{\color{magenta}#1}}
\def\sgto#1{\left(\frac{2\cdot\a{#1}}{\pi}\right)^{3/4} e^{-\a{#1}\cdot r^2}}
\NewDocumentCommand{\He}{}{
\tikz[baseline=-5pt]{
    \coordinate (dv) at (0,0);
    \coordinate (base) at (50pt, 0pt);
    \coordinate (height) at (0pt,50pt);
    \coordinate (diag) at ($(base)+(height)$);
    \path[draw=teal!80!black, rounded corners=2pt, fill=teal, thick] ($(dv)-.5*(diag)$) rectangle +(diag);
    \node[white] at (dv) {\sffamily\huge He};
    \node[white, inner sep=2pt] (dvtext) at ($(dv)-.5*(height)$) [anchor=south] {\sffamily\tiny Helium};
    \node[white, inner sep=2pt] (dvnum) at ($(dv)+.5*(height)-.5*(base)$) [anchor=north west] {\sffamily\tiny 2};
    \node[white, inner sep=2pt] (elestruct) at ($(dv)+.5*(height)+.1*(base)$) [anchor=north west] {\sffamily\tiny $1s^2$};
}
}

\title{\bfseries Процедура розв'язку рівнянь Хартрі-Фока для атома \He}
\date{}

%% --------------------------------------------------------
\def\P#1{%
% Parity operator
\ifnum#1=1
    2
\else
    1
\fi
}

\NewDocumentCommand{\J}{E_{\space} m E_{\space} d()}{
\ifthenelse{\equal{#2}{\phi}}{\edef\V{V}}{\edef\V{\mathcal{V}}}
% #1 - operator index
% #2 - function name
% #3 - function index
% #4 - electron number
\ensuremath{%
    \int \frac{%
               #2^*_{#1}(\P{#4})%
               #2_{#1}(\P{#4})
              }%
               {r_{12}}%
               d\V_{\P{#4}}\ #2_{#3}(#4)
}
}

\NewDocumentCommand{\K}{E_{\space} m E_{\space} d()}{
\ifthenelse{\equal{#2}{\phi}}{\edef\V{V}}{\edef\V{\mathcal{V}}}
% #1 - operator index
% #2 - function name
% #3 - function index
% #4 - electron number
\ensuremath{%
    \int \frac{%
                #2^*_{#1}(\P{#4})%
                #2_{#3}(\P{#4})%
              }%
              {r_{12}}%
             d\V_{\P{#4}}\ #2_{#1}(#4)
}
}

\ExplSyntaxOn
\NewDocumentCommand{\bracket}{r<>}
  {
    \tl_set:Nn \l_tmpa_tl { #1 }
    \tl_replace_all:Nnn \l_tmpa_tl { | } { \middle | }
    \left\langle \l_tmpa_tl \right\rangle
  }

\ExplSyntaxOff


\def\braces(#1#2|#3#4){
    \ensuremath{
        \int \frac{\chi_#1(1)\chi_#2(1)\chi_#3(2)\chi_#4(2)}{r_{12}} dV_1 dV_2
    }
}

%% --------------------------------------------------------


%===============================================================================
\begin{document}
\maketitle


%% --------------------------------------------------------
\section{Оператори кулонівсьої та обмінної взаємодії}
%% --------------------------------------------------------

\begin{align*}
    \hat{J}_j\varphi_i(1) &= \J_j\varphi_i(1),\\
    \hat{K}_j\varphi_i(1) &= \K_j\varphi_i(1).
\end{align*}

%% --------------------------------------------------------
\section{Рівняння Хартрі-Фока}
%% --------------------------------------------------------

Рівняння Хартрі-Фока для для гелію:

\begin{align*}
    \hat{h}(1)\varphi_1(1) +
%    \left( \J<1>\varphi<1>(1) -  \K<1>\varphi<1>(1) \right) + \\
    \left( \J_2\varphi_1(1) -  \K_2\varphi_1(1) \right)
     = \varepsilon_1\varphi_1(1), \\
    \hat{h}(2)\varphi_2(2) +
    \left( \J_1\varphi_2(2) -  \K_1\varphi_2(2) \right)
%    \left( \J<2>\varphi<1>(1) -  \K<2>\varphi<1>(1) \right)
     = \varepsilon_2\varphi_2(2), \\
\end{align*}

Якщо оболонки замкнені, то $\varphi_1 = \phi\alpha$, $\varphi_2 = \phi\beta$, то рівняння приймуть вигляд:

\begin{align*}
    \hat{h}(1)\phi(1) + \J\phi(1) = \varepsilon_1\phi(1), \\
    \hat{h}(2)\phi(2) + \J\phi(2) = \varepsilon_2\phi(2). \\
\end{align*}

Оскільки електрони не розрізненні, то нам фактично доводиться розв'язувати одне рівняння:

\begin{equation}\label{eq:parahelium_HF_equation}
    \hat{h}(1)\phi(1) + \J\phi(1) = \varepsilon_1\phi(1).
\end{equation}

Розв'язок другого рівняння аналогічний, за винятком номера електрона.

Представимо орбіталь у вигляді:
\begin{equation*}
    \phi = \sum_{s=1}^{M}c_s\chi_s,
\end{equation*}
і підставимо у рівняння~\eqref{eq:parahelium_HF_equation}:
\begin{equation*}
    \sum_{s=1}^{M}c_s \hat{h}(1)\chi_s(1) +
    \sum_{t=1}^{M} \sum_{u=1}^{M}c^*_t c_u\int \frac{\chi^*_t(2)\chi_u(2)}{r_{12}} d2\ \sum_{s=1}^{M}c_s\chi_s(1)
     = \sum_{s=1}^{M}c_s \varepsilon \chi_s(1).
\end{equation*}

Домножаємо останнє рівняння на $\chi^*_r(1)$ і інтегруємо:
\begin{equation*}
    \sum_{s=1}^{M}c_s h_{rs} +
    \sum_{s=1}^{M}c_s \sum_{t=1}^{M} \sum_{u=1}^{M}c^*_t c_u\int \frac{\chi^*_r(1) \chi_s(1) \chi^*_t(2)\chi_u(2)}{r_{12}} d1 d2
     = \sum_{s=1}^{M}c_s \varepsilon S_{rs}.
\end{equation*}

\begin{equation*}
    \sum_{s=1}^{M}c_s \left( h_{rs} +
     \sum_{t=1}^{M} \sum_{u=1}^{M}c^*_t c_u\int \frac{\chi^*_r(1) \chi_s(1) \chi^*_t(2)\chi_u(2)}{r_{12}} d1 d2
     -  \varepsilon S_{rs} \right) = 0.
\end{equation*}

\begin{equation*}
    \sum_{s=1}^{M}c_s \left( h_{rs} +
     \sum_{t=1}^{M} \sum_{u=1}^{M}c^*_t c_u (rs|tu)
     - \varepsilon S_{rs} \right) = 0.
\end{equation*}

\begin{equation*}
    \sum_{s=1}^{M}c_s \left( h_{rs} +
     \sum_{t=1}^{M} \sum_{u=1}^{M} P_{tu} (rs|tu)
     - \varepsilon S_{rs} \right) = 0.
\end{equation*}

\begin{equation}\label{eq:HF}
    \sum_{s=1}^{M}c_s \left( F_{rs} - \varepsilon S_{rs} \right) = 0.
\end{equation}


Введені традиційні позначення:

\begin{equation*}
    h_{rs} = \bracket<\chi_r(1)|\hat{h}(1)|\chi_s(1)>,
\end{equation*}

\begin{equation*}
    (rs|tu) = \int \frac{\chi^*_r(1) \chi_s(1) \chi^*_t(2)\chi_u(2)}{r_{12}} d1 d2.
\end{equation*}

\begin{equation*}
    P_{tu} = c^*_t c_u.
\end{equation*}

\begin{equation*}
    F_{rs} = h_{rs} + \sum_{t=1}^{M} \sum_{u=1}^{M} P_{tu} (rs|tu).
\end{equation*}

\begin{equation*}
    S_{rs} = \bracket<\chi_r(1)|\chi_s(1)>,
\end{equation*}

Для нетривіальних розв'язків, повинні виконуватись рівняння:
\begin{equation*}
    \det\left( F_{rs} - \varepsilon S_{rs} \right) = 0
\end{equation*}


Це секулярне рівняння, корені якого дають орбітальні енергії $\varepsilon$.

\clearpage
\nocite{Levine}
\printbibliography


\end{document}
