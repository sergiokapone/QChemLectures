% !TeX program = lualatex
% !TeX encoding = utf8
% !TeX spellcheck = uk_UA

\documentclass{article}
\usepackage[fontsize=14pt]{fontsize}
\usepackage{fontsetup}
\usepackage[english, russian, ukrainian]{babel}
\usepackage{microtype}

\usepackage{tikz}
\usepackage{tikz-3dplot}
\usetikzlibrary{3d}



\usepackage[%
	a4paper,%
	footskip=1cm,%
	headsep=0.3cm,%
	top=2cm, %поле сверху
	bottom=2cm, %поле снизу
	left=2cm, %поле ліворуч
	right=2cm, %поле праворуч
    ]{geometry}

\usepackage{amsmath}
\usepackage{graphicx}
\usepackage{floatflt}

\usepackage[%colorlinks=true,
	%urlcolor = blue, %Colour for external hyperlinks
	%linkcolor  = malina, %Colour of internal links
	%citecolor  = green, %Colour of citations
	bookmarks = true,
	bookmarksnumbered=true,
	unicode,
	linktoc = all,
	hypertexnames=false,
	pdftoolbar=false,
	pdfpagelayout=TwoPageRight,
	pdfauthor={Ponomarenko S.M. aka sergiokapone},
	pdfdisplaydoctitle=true,
	pdfencoding=auto
	]%
	{hyperref}


\usepackage{amsmath}
\usepackage{titlesec}
\titlelabel{\thetitle.\space}


\NewDocumentCommand{\He}{}{
\tikz[baseline=-5pt]{
    \coordinate (dv) at (0,0);
    \coordinate (base) at (50pt, 0pt);
    \coordinate (height) at (0pt,50pt);
    \coordinate (diag) at ($(base)+(height)$);
    \path[draw=teal!80!black, rounded corners=2pt, fill=teal, thick] ($(dv)-.5*(diag)$) rectangle +(diag);
    \node[white] at (dv) {\sffamily\huge He};
    \node[white, inner sep=2pt] (dvtext) at ($(dv)-.5*(height)$) [anchor=south] {\sffamily\tiny Helium};
    \node[white, inner sep=2pt] (dvnum) at ($(dv)+.5*(height)-.5*(base)$) [anchor=north west] {\sffamily\tiny 2};
    \node[white, inner sep=2pt] (elestruct) at ($(dv)+.5*(height)+.1*(base)$) [anchor=north west] {\sffamily\tiny $1s^2$};
}
}

\title{\bfseries \Large Кулонівський інтеграл для $1s$-стану \He}
\date{}
\begin{document}
\maketitle

\section{Суть задачі}




Розглянемо двоелектронну систему в полі ядра. Загальний гамільтоніан у
атомних одиницях має вигляд:
\begin{equation}
\hat{H} = \hat{h}(1) + \hat{h}(2) + \frac{1}{r_{12}}, \qquad
\hat{h}(i) = -\tfrac{1}{2}\nabla_i^2 - \frac{Z}{r_i},
\end{equation}
де $r_{12}=|\mathbf r_1 - \mathbf r_2|$~--- відстань між електронами.
Останній доданок $\tfrac{1}{r_{12}}$ відповідає за кулонівське
взаємодіювання електрон--електрон.

У наближенні незалежних орбіталей просторову хвильову функцію
записують як добуток
\begin{equation}
\Phi(\mathbf r_1,\mathbf r_2) = \psi_a(\mathbf r_1)\,\psi_b(\mathbf r_2).
\end{equation}
Тоді математичне сподівання кулонівського оператора має вигляд
\begin{equation}
\langle \Phi | \tfrac{1}{r_{12}} | \Phi \rangle =
\iint \psi_a^*(\mathbf r_1)\psi_b^*(\mathbf r_2)\,
\frac{1}{r_{12}}\,
\psi_a(\mathbf r_1)\psi_b(\mathbf r_2)\,dV_1\,dV_2.
\end{equation}

Цей вираз називається \emph{прямим кулонівським інтегралом} і позначається $J_{ab}$:
\begin{equation}
J_{ab} = \iint \frac{|\psi_a(\mathbf r_1)|^2 \, |\psi_b(\mathbf r_2)|^2}{r_{12}}
\, dV_1\, dV_2.
\end{equation}

У випадку, коли обидва електрони перебувають в одній і тій самій
орбіталі $1s$, тобто $\psi_a=\psi_b=\Psi_{100}$, отримуємо формулу
\begin{equation}
J = \iint \frac{|\Psi_{100}(\mathbf r_1)|^2\,|\Psi_{100}(\mathbf r_2)|^2}
{|\mathbf r_1-\mathbf r_2|}\, dV_1\,dV_2,
\label{eq:J_def}
\end{equation}
що й є предметом подальшого розгляду.

%Потрібно знайти кулонівський інтеграл для електрона в стані $1s$:
%\begin{equation}
%J = \iint \frac{|\Psi_{100}(\mathbf r_1)|^2\,|\Psi_{100}(\mathbf r_2)|^2}{|\mathbf r_1-\mathbf r_2|}\, dV_1 dV_2.
%\label{eq:J_def}
%\end{equation}
%Цей вираз описує середню кулонівську взаємодію між двома електронами, якщо обидва перебувають у стані $1s$.

\section{Хвильова функція $1s$}

У атомних одиницях (тобто $a_0=1$) для ядра із зарядом $Z$ маємо
\begin{equation*}
\Psi_{100}(\mathbf r) = \sqrt{\tfrac{Z^3}{\pi}}\, e^{-Zr},
\qquad
|\Psi_{100}(\mathbf r)|^2 = \tfrac{Z^3}{\pi} e^{-2Zr}.
\end{equation*}

\section{Геометрія інтеграла}


\begin{figure}[h!]
\centering
% 3D AXIS with spherical coordinates
\tdplotsetmaincoords{60}{110}
\begin{tikzpicture}[scale=2,tdplot_main_coords, >=latex]

  % VARIABLES
  \def\rvec{3}
  \def\thetavec{30}
  \def\phivec{60}

  % AXES
  \coordinate (O) at (0,0,0);
  \draw[thick,->] (0,0,0) -- (2,0,0) node[below left=-3]{$x$};
  \draw[thick,->] (0,0,0) -- (0,2,0) node[right=-1]{$y$};
  \draw[thick,->] (0,0,0) -- (0,0,2) node[above=-1]{$z$};

  % VECTORS
  \tdplotsetcoord{P}{\rvec}{\thetavec}{\phivec}
    \node[inner sep=2pt, ball color=red, circle] (2) at (P) {};
    \node[inner sep=2pt, ball color=red, circle] (1) at (0, 0, 1) {};
  \draw[->] (O)  -- node[right] {$\vec{r}_2$} (2);
\draw[->] (O)  -- node[left] {$\vec{r}_1$} (1);
  \draw[dashed,red]   (O)  -- (Pxy);
  \draw[dashed,red]   (P)  -- (Pxy);
%  \draw[dashed,red]   (Py) -- (Pxy);
%\draw[dashed,red]   (Px) -- (Pxy);

  % ARCS
  \tdplotdrawarc[->]{(O)}{0.5}{0}{\phivec}
    {anchor=north}{$\varphi_2$}
  \tdplotsetthetaplanecoords{\phivec}
  \tdplotdrawarc[->,tdplot_rotated_coords]{(0,0,0)}{0.75}{0}{\thetavec}
    {anchor=south west}{\hspace{-1mm}$\theta_2$}


     \node[inner sep=2pt, ball color=red, circle] at (P) {};
    \draw (1) -- node[above, sloped] {$r_{12}$} (2);
\end{tikzpicture}

\end{figure}

%Щоб спростити, беремо вісь $z$ вздовж вектора $\mathbf r_1$. Тоді $\varphi_2$ зникає, і лишається лише $\theta_2$.
%





Щоб спростити інтеграл, беремо вісь $z$ вздовж вектора $\mathbf r_1$. Знаменник має вигляд
\begin{equation*}
|\mathbf r_1-\mathbf r_2| = \sqrt{r_1^2+r_2^2-2r_1r_2\cos\theta_2}.
\end{equation*}

% Тоді інтеграл по азимутальному куту $\varphi_2$ тривіальний і дає множник $2\pi$, лишаючи інтеграл по полярному куту $\theta_2$ і радіусу $r_2$.

\begin{equation}
J=\int dV_1\,|\Psi_{100}(\mathbf r_1)|^2
\int |\Psi_{100}(\mathbf r_2)|^2
\frac{r_2^2\sin\theta_2\,dr_2\,d\theta_2\,d\varphi_2}
{\sqrt{r_1^2+r_2^2-2r_1r_2\cos\theta_2}}.
\label{eq:J_split}
\end{equation}

Тепер інтеграл «складається в купу»: спочатку інтегруємо по $\theta_2$ (з використанням тотожності для похідної), потім по $r_2$, і нарешті по $r_1$.

\section{Інтегрування по \texorpdfstring{$\theta_2$}{theta2} }

Використаємо тотожність:
\begin{equation*}
\frac{\sin\theta_2}{\sqrt{r_1^2+r_2^2-2r_1r_2\cos\theta_2}}
=\frac{1}{r_1r_2}\frac{d}{d\theta_2}\sqrt{r_1^2+r_2^2-2r_1r_2\cos\theta_2}.
\end{equation*}

Тоді
\begin{equation*}
\int_0^\pi\frac{\sin\theta_2\,d\theta_2}{\sqrt{r_1^2+r_2^2-2r_1r_2\cos\theta_2}}
=\frac{\sqrt{r_1^2+r_2^2+2r_1r_2}-\sqrt{r_1^2+r_2^2-2r_1r_2}}{r_1r_2}.
\end{equation*}

Але
\begin{equation*}
\sqrt{r_1^2+r_2^2+2r_1r_2} = |r_1+r_2| = r_1+r_2 \quad (\text{бо завжди додатне}),
\end{equation*}
\begin{equation*}
\sqrt{r_1^2+r_2^2-2r_1r_2} = |r_1-r_2|.
\end{equation*}

Тому
\begin{equation*}
\int_0^\pi\frac{\sin\theta_2\,d\theta_2}{\sqrt{r_1^2+r_2^2-2r_1r_2\cos\theta_2}}
=\frac{(r_1+r_2)-|r_1-r_2|}{r_1r_2}.
\end{equation*}

Тепер розглянемо два можливих випадки:

\begin{itemize}
\item Якщо $r_1>r_2$, тоді $|r_1-r_2|=r_1-r_2$, і
\begin{equation*}
\frac{(r_1+r_2)-(r_1-r_2)}{r_1r_2} = \frac{2r_2}{r_1r_2}=\frac{2}{r_1}.
\end{equation*}

\item Якщо $r_2>r_1$, тоді $|r_1-r_2|=r_2-r_1$, і
\begin{equation*}
\frac{(r_1+r_2)-(r_2-r_1)}{r_1r_2} = \frac{2r_1}{r_1r_2}=\frac{2}{r_2}.
\end{equation*}
\end{itemize}

Отже остаточно:
\begin{equation*}
\int_0^\pi\frac{\sin\theta_2\,d\theta_2}{\sqrt{r_1^2+r_2^2-2r_1r_2\cos\theta_2}}
=
\begin{cases}
\dfrac{2}{r_1}, & r_1 > r_2, \\
\dfrac{2}{r_2}, & r_2 > r_1.
\end{cases}
\end{equation*}

Таким чином після інтегрування по куту залишаються лише радіальні інтеграли.

\section{Розбиття інтеграла}

Отримаємо
\begin{multline}
J = \int dV_1 \, |\Psi_{100}(\mathbf r_1)|^2
\left( \int_0^{r_1} dr_2 \, r_2^2 e^{-2Z r_2}\frac{2}{r_1}
+ \int_{r_1}^\infty dr_2 \, r_2^2 e^{-2Z r_2}\frac{2}{r_2} \right) \\
=\int dV_1 \, |\Psi_{100}(\mathbf r_1)|^2
\left( \frac{\left(- Z r_{1} + e^{2 Z r_{1}} - 1\right) e^{- 2 Z r_{1}}}{r_{1}} \right).
\label{eq:J_after_r2}
\end{multline}

\section{Остаточне інтегрування}

Тепер виконуємо інтеграл по $r_1$:
\begin{equation}
J = \int\limits_{0}^{\infty} r_1^2  \sin\theta_1 dr_1 d\theta_1 d\phi_1
\frac{Z^{3} e^{- 2 Z r_1}}{\pi}
\left( \frac{\left(- Z r_{1} + e^{2 Z r_{1}} - 1\right) e^{- 2 Z r_{1}}}{r_{1}} \right).
\end{equation}

Інтегрування по кутум дає множник $4\pi$, після чого залишковий інтеграл легко обчислити:
%\begin{equation}
%J = 4Z^3\int\limits_{0}^{\infty} r_1 dr_1
% e^{- 2 Z r_1}
%\left( - Z r_1 + e^{2 Z r_1} - 1\right) e^{- 2 Z r_1} .
%\end{equation}


% Після кутової інтеграції (множник 4\pi) маємо
\begin{align}
J &= 4\pi \int_0^\infty r_1^2 \frac{Z^3}{\pi} e^{-2Z r_1}
\left( \frac{(-Z r_1 + e^{2Z r_1} -1)e^{-2Z r_1}}{r_1} \right) dr_1 \\
&= 4 Z^3 \int_0^\infty r_1^2 \frac{(-Z r_1 + e^{2Z r_1} -1) e^{-4Z r_1}}{r_1}\, dr_1 \\
&= 4 Z^3 \int_0^\infty \bigl(-Z r_1^2 + r_1 e^{2Z r_1} - r_1\bigr) e^{-4Z r_1}\, dr_1 \\
&= 4 Z^3 \left[ -Z \int_0^\infty r_1^2 e^{-4Z r_1}\,dr_1
+ \int_0^\infty r_1 e^{-2Z r_1}\,dr_1
- \int_0^\infty r_1 e^{-4Z r_1}\,dr_1 \right].
\end{align}

\noindent Використовуючи формулу
\[
\int_0^\infty r^n e^{-a r}\,dr=\frac{n!}{a^{\,n+1}},
\qquad a>0,
\]
маємо
\begin{align*}
\int_0^\infty r_1^2 e^{-4Z r_1}\,dr_1 = \frac{2!}{(4Z)^3} &= \frac{1}{32 Z^3}, \quad
\int_0^\infty r_1 e^{-2Z r_1}\,dr_1 = \frac{1!}{(2Z)^2} = &\frac{1}{4 Z^2},\\
\int_0^\infty r_1 e^{-4Z r_1}\,dr_1=\frac{1!}{(4Z)^2} = &\frac{1}{16 Z^2}.
\end{align*}

\noindent Підставляємо:
\begin{multline*}
J = 4 Z^3 \left[ -Z\cdot\frac{1}{32 Z^3} + \frac{1}{4 Z^2} - \frac{1}{16 Z^2} \right]
= 4 Z^3 \left[ \frac{-1}{32 Z^2} + \frac{1}{4 Z^2} - \frac{1}{16 Z^2} \right] = \\
= 4 Z^3 \cdot \frac{5}{32 Z^2} = \frac{20}{32}\,Z = \frac{5}{8}\,Z.
\end{multline*}

\noindent Отже,
\[
\boxed{J=\tfrac{5}{8}\,Z.}
\]


\end{document}
