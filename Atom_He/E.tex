\documentclass{article}
\usepackage[fontsize=14pt]{fontsize}

\usepackage{fontsetup}
\usepackage[english, russian, ukrainian]{babel}


\usepackage[dvipsnames]{xcolor}
\newcommand{\psii}{\textcolor{blue}{\psi_i}}
\newcommand{\phii}{\textcolor{blue}{\phi_i}}
\newcommand{\psij}{\textcolor{red}{\psi_j}}
\newcommand{\phij}{\textcolor{red}{\phi_j}}

\renewcommand{\i}{\textcolor{blue}{i}}
\renewcommand{\j}{\textcolor{red}{j}}

\newcommand\I{\textcolor{blue}{(1)}}
\newcommand\II{\textcolor{red}{(2)}}


\allowdisplaybreaks
\usepackage{cancel}

\def\ket#1{\left| #1 \right\rangle }
\def\bra#1{\left\langle #1 \right| }
\def\bracket#1#2{\left\langle #1\left| #2 \right. \right\rangle }
\def\opbracket#1#2#3{\left\langle #1\left| \textcolor{OliveGreen}{#2}\right|  #3 \right\rangle}
\def\eff{e\kern-2.4pt{f\kern-2.65pt f}}
\def\vxi{\vec{\xi}}
\def\vr{\vec{r}}
\let\vphi\varphi
\let\ve\varepsilon

\def\f{(\textcolor{red}{1})}
\def\s{(\textcolor{blue}{2})}

\def\F{\textcolor{red}{\mathbf{1}}}
\def\S{\textcolor{blue}{\mathbf{2}}}

\def\oa{\textcolor{red}{ \vphi_{\mathbf{1}}} }
\def\ob{\textcolor{blue}{ \vphi_{\mathbf{2}}} }
\def\cPhi{\oa\f \ob\s - \oa\s \ob\f} % лінійна Комбінація
\def\h{\hat{h}}
\def\ha{\h\f}
\def\hb{\h\s}
\def\V{\frac{1}{r_{12}}}
\def\a{\oa\f \ob\s}
\def\b{\ob\f \oa\s}
\def\dt{\,.\,}
%\def\sp{\textvisiblespace}

\def\hr{\hrulefill\\*[10pt]}
\def\hhr{\hrulefill\\[-0.9\baselineskip]\hrulefill\\*[10pt]}

\usepackage{titlesec}

% Налаштування для section
\titleformat{\section}
  {\normalfont\large\bfseries}  % формат тексту (розмір і жирність)
  {\thesection.}                % формат номеру з точкою
  {1em}                         % відступ між номером і текстом
  {}
  [\hrulefill]

% Налаштування для section
\titleformat{\subsection}
  {\normalfont\bfseries}  % формат тексту (розмір і жирність)
  {\thesubsection.}                % формат номеру з точкою
  {1em}                         % відступ між номером і текстом
  {}

\usepackage[%
	a4paper,%
	footskip=1cm,%
	headsep=0.3cm,%
	top=2cm, %поле сверху
	bottom=2cm, %поле снизу
	left=2cm, %поле ліворуч
	right=2cm, %поле праворуч
    ]{geometry}

\usepackage{amsmath}

\begin{document}

\title{Енергія атома гелію}
\author{}
\date{}
\maketitle

\section{Постановка задачі}

\textbf{Гамільтоніан атома гелію:}
\begin{equation}
\hat{H} = -\frac{1}{2}\nabla_1^2 - \frac{1}{2}\nabla_2^2 - \frac{Z}{r_1} - \frac{Z}{r_2} + \frac{1}{r_{12}}
\end{equation}

\section{Антисиметризована пробна функція}

\begin{equation}
\psi(1,2) = \frac{1}{\sqrt{2}}[\psii\I\psij\II - \psii\II\psij\I]
\end{equation}

де
%$\psii(i) = \varphi_i(i)\gamma_i(i)$, $\psij(i) = \varphi_j(i)\gamma_j(i)$
%
повна спін-орбіталь:
\[
\psi_p(\mathbf{x}) = \phi_p(\mathbf{r}) \, \gamma_p(\sigma),
\qquad
\gamma_p(\sigma) \in \{\alpha(\sigma), \beta(\sigma)\}.
\]

\section{Обчислення енергії}

\begin{equation}
E = \langle \psi | \hat{H} | \psi \rangle
\end{equation}

\begin{equation}
E = \frac{1}{2}\langle \psii\I\psij\II - \psii\II\psij\I | \hat{H} | \psii\I\psij\II - \psii\II\psij\I \rangle
\end{equation}

Щоб не рябило в очах, вводимо нотацію:

\begin{equation}
E = \frac{1}{2}\langle \i\j - \j\i | \hat{H} | \i\j - \j\i \rangle
\end{equation}


Розкриваємо добуток:
\begin{align*}
E &= \frac{1}{2}[\langle \psii\I\psij\II | \hat{H} | \psii\I\psij\II \rangle - \langle \psii\I\psij\II | \hat{H} | \psii\II\psij\I \rangle \nonumber \\
&\quad - \langle \psii\II\psij\I | \hat{H} | \psii\I\psij\II \rangle + \langle \psii\II\psij\I | \hat{H} | \psii\II\psij\I \rangle]
\end{align*}

або:

\begin{align}
E &= \frac{1}{2}[\langle \i\j | \hat{H} | \i\j \rangle - \langle \i\j | \hat{H} | \j\i \rangle \nonumber \\
&\quad - \langle \j\i | \hat{H} | \i\j \rangle + \langle \j\i | \hat{H} | \j\i \rangle]
\end{align}




\section{Використання симетрії гамільтоніана}

Перший і четвертий терми однакові, другий і третій теж однакові:

\begin{equation}
E = \langle \psii\I\psij\II | \hat{H} | \psii\I\psij\II \rangle - \langle \psii\I\psij\II | \hat{H} | \psii\II\psij\I \rangle,
\end{equation}
або в зручній нотації:
\begin{equation}
E = \langle \i\j   | \hat{H} |\i\j  \rangle - \langle \i\j | \hat{H} | \j\i \rangle,
\end{equation}

\section{Двоелектронні інтеграли зі спіновими функціями}

\subsection{Спін-орбіталі}


Ортонормованість спінових функцій:
\begin{align*}
\langle \alpha | \alpha \rangle = \int \alpha^*(\sigma)\alpha(\sigma)\,d\sigma = 1,
\\
\langle \beta | \beta \rangle = \int \beta^*(\sigma)\beta(\sigma)\,d\sigma = 1,
\\
\langle \alpha | \beta \rangle = \int \alpha^*(\sigma)\beta(\sigma)\,d\sigma = 0.
\end{align*}


\subsection{Двоелектронний інтеграл}

Визначення двоелектронних інтегралів:
\[
\langle pq|rs \rangle
= \iint \psi_p^*(\mathbf{x}_1)\psi_q(\mathbf{x}_2)\,
\frac{1}{r_{12}}\,
\psi_r^*(\mathbf{x}_1)\psi_s(\mathbf{x}_2)\,
d\mathbf{x}_1 d\mathbf{x}_2.
\]
де:
\begin{itemize}
\item $p, q, r, s$ --- \textbf{індекси квантових станів} (орбіталей)
\item $\mathbf{x}_1, \mathbf{x}_2$ --- \textbf{номери електронів} (координати  першого та другого електронів)
\item $r_{12} = |r_1 - r_2|$ --- відстань між електронами
\end{itemize}

\subsubsection*{Позиційна нотація}

\begin{equation*}
    \langle pq | rs \rangle = \langle \underbrace{p}_{1}\underbrace{q}_{2} | \underbrace{r}_{1}\underbrace{s}_{2} \rangle
\end{equation*}


\begin{itemize}
\item \textbf{Позиція 1:} електрон 1 --- орбіталі $i$ та $k$
\item \textbf{Позиція 2:} електрон 2 --- орбіталі $j$ та $l$
\end{itemize}

Інтеграл описує ймовірність кулонівської взаємодії між двома електронами при переході з початкового стану $(p,q)$
 в кінцевий стан $(r,s)$.

\section{Двоелектронні інтеграли: кулонівський та обмінний}

Розкладаємо $\mathbf{x}=(\mathbf{r},\gamma)$:
\begin{multline*}
\langle pq|rs \rangle =
\underbrace{\iint \phi_p(\mathbf{r}_1)\phi_q(\mathbf{r}_2)
\frac{1}{r_{12}}
\phi_r(\mathbf{r}_1)\phi_s(\mathbf{r}_2)
\,d\mathbf{r}_1 d\mathbf{r}_2}_{(pq|rs)}
\cdot
\underbrace{\int \gamma_p^*(\sigma_1)\gamma_r(\sigma_1)\,d\sigma_1}_{\delta_{pr}}\cdot\\
\cdot
\underbrace{\int \gamma_q^*(\sigma_2)\gamma_s(\sigma_2)\,d\sigma_2}_{\delta_{qs}}.
\end{multline*}

Отже:
\[
\langle pq|rs \rangle = (pq|rs)\,\delta_{pr}\,\delta_{qs}.
\]



\subsection{Кулонівський інтеграл $J$}

Кулонівський інтеграл виникає тоді, коли перший електрон знаходиться на орбіталі $\i$, а другий електрон на орбіталі $\j$, тоді індекси групуються по парах:
\begin{equation}
    J_{\i\j} = \langle \i\j|\i\j \rangle = (\i\j|\i\j) \, \underbrace{\delta_{ii}}_{=1} \, \underbrace{\delta_{jj}}_{=1} = (\i\j|\i\j),
\end{equation}
або

\begin{multline*}
    J_{\i\j} =  \iint \phii^*(\mathbf{r}_1)\phij^*(\mathbf{r}_2) \frac{1}{r_{12}} \phii(\mathbf{r}_1)\phij(\mathbf{r}_2) \,d\mathbf{r}_1 d\mathbf{r}_2 = \\ =
\iint \phii(\mathbf{r}_1)\phii(\mathbf{r}_1) \frac{1}{r_{12}} \phij(\mathbf{r}_2)\phij(\mathbf{r}_2) \,d\mathbf{r}_1 d\mathbf{r}_2 = \\ =
\iint \frac{|\phii(\mathbf{r}_1)|^2 |\phij(\mathbf{r}_2)|^2}{r_{12}} \,d\mathbf{r}_1 d\mathbf{r}_2
\end{multline*}

Фізичний сенс ---  класичне електростатичне відштовхування між електронними хмарами.

Він описує класичне кулонівське відштовхування між електронною густиною орбіталі $i$ та густиною орбіталі $j$.
Спінові дельти тут завжди дорівнюють $1$, отже $J_{\i\j}$ не залежить від спіну.


\subsection{Обмінний інтеграл $K$}

\begin{equation}
    K_{\i\j} = \langle \i\j|\j\i \rangle = (\i\j|\j\i)\,\delta_{\i\j} \delta_{\j\i} = (\i\j|\j\i)\,\delta_{\i\j}.
\end{equation}

Цей інтеграл існує лише для паралельних спінів, тобто коли $\gamma_i = \gamma_j$. Тобто, обмінна взаємодія можлива лише між електронами з паралельними спінами, що є прямим наслідком антисиметрії хвильової функції.

Розпишемо:
\begin{multline*}
    K_{\i\j} =  \iint \phii^*(\mathbf{r}_1)\phij^*(\mathbf{r}_2) \frac{1}{r_{12}} \phij(\mathbf{r}_1)\phii(\mathbf{r}_2) \,d\mathbf{r}_1 d\mathbf{r}_2 = \\ =
\iint \phii(\mathbf{r}_1)\phij(\mathbf{r}_1) \frac{1}{r_{12}} \phij(\mathbf{r}_2)\phii(\mathbf{r}_2) \,d\mathbf{r}_1 d\mathbf{r}_2
\end{multline*}



Обмінна енергія виникає через ідентичність електронів та можливість їх обміну між станами $i$ і $j$. Кожен електрон частково перебуває в стані $i$ і частково в стані $j$. <<Різні частини>> одного електрона взаємодіють між собою через {кулонівське відштовхування}, що дає \textbf{енергію взаємодії}, яку називають {обмінною енергією}.

Ця енергія {не має класичного аналога} і є {чисто квантовим ефектом обміну електронів} між різними станами.





\section{Детальний аналіз термів}

\subsection{Перший терм (пряма взаємодія)}

\begin{equation}
\langle \psii\I\psij\II | \hat{H} | \psii\I\psij\II \rangle = \langle \i\j | \hat{H} | \i\j \rangle
\end{equation}

Розкладаємо гамільтоніан:
\begin{multline}
= \langle \psii\I\psij\II | \hat{h}\I + \hat{h}\II + \frac{1}{r_{12}} | \psii\I\psij\II \rangle = \\
= \langle \i\j | \left(\hat{h}\I + \hat{h}\II + \frac{1}{r_{12}}\right) | \i\j \rangle
\end{multline}

\textbf{Одноелектронні терми:}
\begin{multline}
\langle \psii\I\psij\II | \hat{h}\I | \psii\I\psij\II \rangle = \langle \psii\I | \hat{h}\I | \psii\I \rangle \langle \psij\II | \psij\II \rangle = \\ = \langle \psii\I | \hat{h}\I | \psii\I \rangle = \\
= \int \gamma_i^*\I \gamma_i\I \,  d\omega_1 \int \phii^*\I \hat{h}\I \phii\I\  dV_1\\
=  1 \cdot ( i | \hat{h} | i ) = ( i | \hat{h} | i ) =  h_{ii}
\end{multline}

%\begin{multline}
%\langle \psii\I\psij\II | \hat{h}\II | \psii\I\psij\II \rangle = \langle \psii\I | \psii\I \rangle \langle \psij\II | \hat{h}\II | \psij\II \rangle = \\
%= \langle \i\j | \hat{h}\II |\i\j \rangle = \langle i | i \rangle \langle j | \hat{h}\II | j \rangle
%= h_{jj}
%\end{multline}


%\begin{multline}
%h_{jj} = \langle j | \hat{h} | j \rangle = \int dr_j \, \varphi_j^*(r_j) \hat{h}_j \varphi_j(r_j) \\
%= \int d\omega_j \, \sigma_j^*(\omega_j) \sigma_j(\omega_j) \int dr_j \, \psi_{1/2}^*(r_j) \hat{h}_j \psi_{1/2}(r_j) \\
%= \delta_{\sigma_j, \sigma_j} \langle j | \hat{h} | j \rangle = \langle j | \hat{h} | j \rangle
%\end{multline}




\textbf{Двоелектронний терм:}

\begin{equation}
\langle \psii\I\psij\II | \frac{1}{r_{12}} | \psii\I\psij\II \rangle = \langle \i\j|\i\j \rangle = (\i\j|\i\j) = J_{\i\j}
\end{equation}


\textbf{Підсумок:}
\begin{equation}
\langle \psii\I\psij\II | \hat{H} | \psii\I\psij\II \rangle = h_{ii} + h_{jj} + J_{\i\j}
\end{equation}

\subsection{Другий терм (обмінна взаємодія)}

\begin{equation}
\langle \psii\I\psij\II | \hat{H} | \psii\II\psij\I \rangle = \langle \i\j | \hat{H} | \j\i \rangle
\end{equation}

Розкладаємо гамільтоніан:
\begin{equation}
= \langle \psii\I\psij\II | \hat{h}\I + \hat{h}\II + \frac{1}{r_{12}} | \psii\II\psij\I \rangle
\end{equation}

\textbf{Одноелектронні терми:}
\begin{multline}
\langle \psii\I\psij\II | \hat{h}\I | \psii\II\psij\I \rangle = \langle \psii\I | \hat{h}\I | \psii\II \rangle \langle \psij\II | \psij\I \rangle = \\
\langle \i\j | \hat{h}\I | \j\i \rangle = \langle i\cdot | \hat{h}\I | \cdot i \rangle \langle \cdot j |j \cdot \rangle
\end{multline}

Оскільки $\hat{h}\I$ діє тільки на координати електрона 1:
\begin{multline}
= \langle \varphi_i\I | \hat{h}\I | \varphi_i\II \rangle \langle \gamma_i\I | \gamma_i\II \rangle \langle \varphi_j\II | \varphi_j\I \rangle \langle \gamma_j\II | \gamma_j\I \rangle = \\
=
\left( i\cdot | \hat{h}\I | \cdot i \right) ( \cdot j | j \cdot ) \langle \gamma_i | \gamma_j \rangle  \langle \gamma_j | \gamma_i \rangle
\end{multline}

Але $\left( i\cdot | \hat{h}\I | \cdot i \right) = 0$ (різні координати), тому:
\begin{equation}
\langle \i\j | \hat{h}\I | \j\i \rangle = 0
\end{equation}

Аналогічно:
\begin{equation}
\langle \i\j | \hat{h}\II | \j\i \rangle = 0
\end{equation}

\textbf{Двоелектронний терм:}
\begin{equation}
\langle \psii\I\psij\II | \frac{1}{r_{12}} | \psii\II\psij\I \rangle = \langle \i\j | \frac{1}{r_{12}} | \j\i \rangle = \langle \i\j | \j\i \rangle  = (\i\j|\j\i) \delta_{\i\j} = K_{\i\j}
\end{equation}

%
%
%
%
%\textbf{Підсумок:}
%\begin{equation}
%\langle \psii\I\psij\II | \hat{H} | \psii\II\psij\I \rangle = \left( \i\j | \j\i \right) \delta_{\i\j}
%\end{equation}
%
%де $\delta_{\i\j} = 1$ якщо спіни однакові, $\delta_{\i\j} = 0$ якщо спіни різні.



\section{Підсумковий результат}

\begin{equation}
E = h_{ii} + h_{jj} + J_{\i\j} - K_{\i\j} = \varepsilon_1 + \varepsilon_2 + + J_{\i\j} - K_{\i\j}
\end{equation}

Результат ми можемо записати у вигляді суми:
\begin{equation}
    E = \sum\limits_{i=1}^2 \varepsilon_i + \frac12  \sum\limits_{i=1}^2 \sum\limits_{j=1}^2(J_{\i\j} - K_{\i\j}), \quad J_{\i\i} = 0, \ K_{\i\j} = 0
\end{equation}

\section{Отримання рівнянь Хартрі-Фока з варіаційного принципу}

\begin{equation*}
    F = E - \sum_{i = 1}^{2} \sum_{j = 1}^{2} (\ve_{ij} \bracket{\vphi_i}{\vphi_j} - \delta_{ij})
\end{equation*}


\begin{multline*}
    \delta F =  \opbracket{\delta\F}{\h}{\F} + \opbracket{\delta\S}{\h}{\S} + \\
    % for J
    + \frac12\opbracket{\delta\F\S}{\V}{\F\S} + \frac12\opbracket{\F\delta\S}{\V}{\F\S} +
    \frac12\opbracket{\delta\S\F}{\V}{\S\F} + \frac12\opbracket{\S\delta\F}{\V}{\S\F} - \\
    % for K
    - \frac12\opbracket{\delta\F\S}{\V}{\S\F} - \frac12\opbracket{\delta\S\F}{\V}{\F\S}
    - \frac12\opbracket{\F\delta\S}{\V}{\S\F} - \frac12\opbracket{\S\delta\F}{\V}{\F\S} - \\
    - \ve_{11} \bracket{\delta\F}{\F}
    - \ve_{12} \bracket{\delta\F}{\S}
    - \ve_{21} \bracket{\delta\S}{\F}
    - \ve_{22} \bracket{\delta\S}{\S} + \\
    + \text{c. c.} = 0
\end{multline*}

\begin{align*}
    \left( \h + \frac12\opbracket{\dt\S}{\V}{\dt\S} + \frac12\opbracket{\S\dt}{\V}{\S\dt}
    - \frac12 \opbracket{\dt\S}{\V}{\S\dt} - \frac12\opbracket{\S\,.}{\V}{\dt\S}
    \right) \ket{\F} &= \ve_{11} \ket{\F} + \ve_{12}\ket{\S}
    \\[1.1\baselineskip]%
    \left( \h + \frac12\opbracket{.\,\F}{\V}{.\,\F} + \frac12\opbracket{\F\dt}{\V}{\F\dt}
    - \frac12 \opbracket{\dt\F}{\V}{\F\dt} - \frac12\opbracket{\F\dt}{\V}{\dt\F}
    \right) \ket{\S} &= \ve_{21} \ket{\F} + \ve_{22}\ket{\S}
\end{align*}

Canonical HF-Equations

\begin{align*}
    \left( \h + \opbracket{\dt\S}{\V}{\dt\S}
    - \opbracket{\dt\S}{\V}{\S\dt}
    \right) \ket{\F} &= \ve_{11} \ket{\F}
    \\[1.1\baselineskip]%
    \left( \h + \opbracket{\dt\F}{\V}{\dt\F}
    -  \opbracket{\dt\F}{\V}{\F\dt}
    \right) \ket{\S} &= \ve_{22} \ket{\S}
\end{align*}

Columb operator

\begin{equation*}
   \hat{J}_{\S}\ket{\F} = \opbracket{\dt\S}{\V}{\dt\S} \ket{\F} = \int \frac{\dt\ob\s\dt\ob\s}{r_{12}} d\s \cdot \oa\f =  \int \frac{\dt\ob\s\oa\f\ob\s}{r_{12}} d\s
\end{equation*}

Columb integral

\begin{equation*}
   \bra{\F}\hat{J}_{\S}\ket{\F} = \bra{\F}\opbracket{\dt\S}{\V}{\dt\S} \ket{\F} = \iint \frac{\oa\f\ob\s\oa\f\ob\s}{r_{12}} d\s d\f
\end{equation*}

Exchange operator

\begin{equation*}
    \hat{K}_{\S}\ket{\F} = \opbracket{\dt\S}{\V}{\S\dt}\ket{\F} = \int \frac{\dt\ob\s \ob\f\dt}{r_{12}} d\s \cdot \oa\f = \int \frac{\dt\ob\s \ob\f\oa\s}{r_{12}} d\s
\end{equation*}

Exchange integral

\begin{equation*}
    \bra{\F}\hat{K}_{\S}\ket{\F} = \bra{\F}\opbracket{\dt\S}{\V}{\S\dt}\ket{\F} = \iint \frac{\oa\f\ob\s \ob\f\oa\s}{r_{12}} d\s d\f
\end{equation*}



\end{document}