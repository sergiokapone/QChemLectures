%%============================ Compiler Directives =======================%%
%%                                                                        %%
% !TeX program = lualatex
% !TeX encoding = utf8
% !TeX spellcheck = uk_UA
%%                                                                        %%
%%============================== Клас документа ==========================%%
%%                                                                        %%
\documentclass[]{article}
\usepackage[fontsize=14pt]{fontsize}
%%                                                                        %%
%%========================== Мови, шрифти та кодування ===================%%
%%                                                                        %%
\usepackage{fontspec}
\setsansfont{CMU Sans Serif}%{Arial}
\setmainfont{CMU Serif}%{Times New Roman}
\setmonofont{CMU Typewriter Text}%{Consolas}
\defaultfontfeatures{Ligatures={TeX}}
\usepackage[math-style=TeX]{unicode-math}
\usepackage[russian, english, ukrainian]{babel}
\usepackage{biblatex}
\usepackage{microtype}
\usepackage{xurl}
\usepackage{indentfirst}
\usepackage[most]{tcolorbox}
\tcbset{highlight math style={enhanced,
  colframe=red,colback=white,arc=0pt,boxrule=1pt}}
\tcbset{
    mybox/.style={
        enhanced,
        colback=gray!5, % Цвет фона
%        colframe=white, % Цвет рамки
        fonttitle=\bfseries, % Жирный шрифт для заголовка
        sharp corners, % Углы box острые
        boxrule=0pt % Ширина линии рамки
    }
}
\usepackage{subcaption}
\captionsetup[subfigure]{justification=centering}
\usepackage{hyperref}
\usepackage{titlesec}
\titlelabel{\thetitle.\space}
\usepackage{minted}
\usepackage{tikz, calc}
\usetikzlibrary{backgrounds}
\usepackage{enumitem}


\addbibresource{d:/Projects/LaTeX/QChem/A_Documents/Syllabus_QChem/Syllabus_QChem.bib}

%%                                                                        %%
%%============================= Геометрія сторінки =======================%%
%%                                                                        %%
\usepackage[%
	a4paper,%
	footskip=1cm,%
	headsep=0.3cm,%
	top=2cm, %поле сверху
	bottom=2cm, %поле снизу
	left=2cm, %поле ліворуч
	right=2cm, %поле праворуч
    ]{geometry}
%%                                                                        %%
%%============================== Інтерліньяж  ============================%%

\usepackage{tikz}
\usepackage{pgfplots}
\usepgfplotslibrary{units}
\pgfplotsset{compat=newest}
\usepackage{pgfplotstable}
\usepgfplotslibrary{groupplots}



\def\C#1{{\color{teal}#1}}
\def\a#1{{\color{magenta}#1}}
\def\sgto#1{\left(\frac{2\cdot\a{#1}}{\pi}\right)^{3/4} e^{-\a{#1}\cdot r^2}}
\NewDocumentCommand{\He}{}{
\tikz[baseline=-5pt]{
    \coordinate (dv) at (0,0);
    \coordinate (base) at (50pt, 0pt);
    \coordinate (height) at (0pt,50pt);
    \coordinate (diag) at ($(base)+(height)$);
    \path[draw=teal!80!black, rounded corners=2pt, fill=teal, thick] ($(dv)-.5*(diag)$) rectangle +(diag);
    \node[white] at (dv) {\sffamily\huge He};
    \node[white, inner sep=2pt] (dvtext) at ($(dv)-.5*(height)$) [anchor=south] {\sffamily\tiny Helium};
    \node[white, inner sep=2pt] (dvnum) at ($(dv)+.5*(height)-.5*(base)$) [anchor=north west] {\sffamily\tiny 2};
    \node[white, inner sep=2pt] (elestruct) at ($(dv)+.5*(height)+.1*(base)$) [anchor=north west] {\sffamily\tiny $1s^2$};
}
}

\title{\bfseries Базисні набори для атома \He}
\date{}
%===============================================================================
\begin{document}
\maketitle

\section{Що таке базис?}

Базисний набір --- функції, що слугують для представлення наближеної хвильової функції атома (чи молекули).  Чим точніше цей набір обрано, тим точніше можна отримати результати при розв'язанні рівнянь Хартрі-Фока, що визначають електронну структуру системи.

Розкладання орбіталі $\phi$ через базисні функції $\chi$:

\begin{equation*}
    \phi(x, y, z) = \sum_{i=1}^M c_i \chi_i(x, y, z | \zeta).
\end{equation*}

\section{Базисні функції}

\begin{itemize}
\item \href{https://en.wikipedia.org/wiki/Slater-type_orbital}{Орбіталі слейтерового типу (STO)}:
\begin{equation}\label{STO}
	\tcbhighmath{\mathrm{STO} = \frac{(2\zeta)^{n+\frac12}}{\sqrt{(2n)!}} r^{n - 1}e^{-\zeta r}\times\text{Linear combination of $Y$}.
	}
\end{equation}

\item \href{https://en.wikipedia.org/wiki/Gaussian_orbital}{Орбіталі гаусового типу (GTOs)}:

\begin{equation}\label{GTO}
\tcbhighmath{\mathrm{GTO}(x,y,z;\alpha,i,j,k) = \left(\frac{2\alpha}{\pi}\right)^{3/4}\sqrt{\frac{(8\alpha)^{i + j + k} i!j!k!}{(2i)!(2j)!(2k)!}} x^{i} y^{j} z^{k} e^{-\alpha r^2}
}
\end{equation}

\begin{itemize}
\item Якщо $i + j + k = 0$ (такі що, $i = 0$, $j = 0$, $k = 0$), GTO називається $s$-типу.
\item Якщо $i + j + k = 1$, маємо гауссову функцію $p$-типу, яка містить множник $x$, $y$, або $z$.
\item Якщо $i + j + k = 2$, ми маємо $d$-тип GTO і так далі \ldots
\end{itemize}
\end{itemize}

У GTO підносить $r$ підноситься до квадрату, так що добуток гауссових функцій є іншою гауссовою функцією. Цим самим, ми отримуємо інтеграли, з яким легше працювати, а також прискорюємо їх обрахунки. Однак, ціна, яку ми платимо --- це втрата точності. Щоб компенсувати цю втрату, ми виявили, що чим більше гауссових рівнянь ми комбінуємо, тим точніше наше рівняння.

Усі рівняння базисного набору у формі STO-nG (де $n$ --- кількість GTO, об'єднаних для апроксимації STO) вважаються <<мінімальними>> базисними наборами. <<Розширені>> базисні набори --- це ті, що враховують вищі орбіталі молекули і враховують розмір і форму розподілу молекулярного заряду.

%% --------------------------------------------------------
\section{Мінімальний базис STO-3G}
%% --------------------------------------------------------


Базис STO-3G з \cite{BSE}:

\begin{minted}[mathescape,
        fontsize=\small,
        bgcolor=gray!5,
        ]
        {ruby}
%basis
    NewGTO He
    S   3
    1         0.6362421394E+01       0.1543289673E+00
    2         0.1158922999E+01       0.5353281423E+00
    3         0.3136497915E+00       0.4446345422E+00
    end
end
\end{minted}

Ці числа означають, що ми конструюємо на їх основі базисні функції за наступною формулою:

\begin{equation*}
	\text{STO-3G} =
    {\color{blue} c_1 \cdot  \left(\frac{2\alpha_1}{\pi}\right)^{3/4}e^{-\alpha_1r^2}
    }
	+
	{\color{magenta}
	c_2 \cdot  \left(\frac{2\alpha_2}{\pi}\right)^{3/4}e^{-\alpha_2r^2}
    }
	+
	{\color{teal}
	c_3 \cdot \left(\frac{2\alpha_3}{\pi}\right)^{3/4}e^{-\alpha_3r^2}
    }.
\end{equation*}


Підставимо коефіцієнти:


\begin{tcolorbox}[mybox]
\begin{multline*}
	\text{STO-3G} =
    \C{0.1543289673} \cdot  \left(\frac{2\cdot\a{0.6362421394\cdot10^1}}{\pi}\right)^{3/4}e^{-\a{0.6362421394\cdot10^1}\cdot r^2}
	+ \\ +
	\C{0.5353281423} \cdot  \left(\frac{2\cdot\a{0.1158922999\cdot10^1}}{\pi}\right)^{3/4}e^{-\a{0.1158922999}\cdot10^1\cdot r^2}
	+ \\ +
	\C{0.4446345422} \cdot \left(\frac{2\cdot\a{0.3136497915}}{\pi}\right)^{3/4}e^{-\a{0.3136497915}\cdot r^2}
\end{multline*}
\end{tcolorbox}

Базис можна зобразити на двовимірному графіку (рис.~\ref{pig:STO-3G}).

\begin{figure}[!h]\centering
	\begin{tikzpicture}[
        declare function = {
            gto(\c,\a,\r)=\c*(2*\a/pi)^(3/4)*exp(-\a*\r^2);
            sto(\z,\r)=sqrt((2*\z)^3/(8*pi))*exp(-\z*\r);
        },
        background rectangle/.style={fill=gray!5}, show background rectangle,
        ]
		\begin{axis}[
                width=0.7\linewidth,
                height=0.46\linewidth,
                legend style={draw=none, fill=white, column sep=5pt, row sep=10pt},
				axis lines=left,
				xlabel={$r$, bohr},
				ylabel=$\chi$,
                samples=200,
                domain={0:3},
                smooth,
			]

			\addplot [dashed]
			{sto(1.7,x)} node [pos=0.1, pin=0:STO] (s1) {};
            \addlegendentry{$\sqrt{\frac1{4\pi}}\cdot\frac{(2\cdot1.7)^{\frac32}}{\sqrt{2}} e^{-1.7\cdot r}$}


			\addplot [blue]
			{gto(0.154, 0.636e1, x)} node [pos=0.1] (g1) {};
            \addlegendentry{$c_1 \cdot  \left(\frac{2\alpha_1}{\pi}\right)^{3/4}e^{-\alpha_1r^2}$}

			\addplot [magenta]
			{gto(0.535, 0.116e1, x)} node [pos=0.2] (g2) {};
            \addlegendentry{$c_2 \cdot  \left(\frac{2\alpha_2}{\pi}\right)^{3/4}e^{-\alpha_2r^2}$}

			\addplot [teal]
			{gto(0.445, 0.314e0, x)} node [pos=0.2] (g3) {};
            \addlegendentry{$c_3 \cdot  \left(\frac{2\alpha_3}{\pi}\right)^{3/4}e^{-\alpha_3r^2}$}

			\addplot [thick, red]
			{%
                gto(0.154, 0.636e1, x) +
                gto(0.535, 0.116e1, x)+
                gto(0.445, 0.314e0, x)
            } node [pos=0.2, pin=0:STO-3G] (gs) {};

		\end{axis}
	\end{tikzpicture}
\caption{Представлення STO трьома GTO}\label{pig:STO-3G}
\end{figure}


%% --------------------------------------------------------
%\clearpage
\section{Розширений базис 6-31G}
%% --------------------------------------------------------


Базис 6-31G з \cite{BSE}:

\begin{minted}[mathescape,
        fontsize=\small,
        bgcolor=gray!5,
        ]
        {ruby}
%basis
 NewGTO He
 S 3
   1      38.3549367370      0.0401838903
   2       5.7689081479      0.2613913445
   3       1.2399407035      0.7930391578
 S 1
   1       0.2975781595      1.0000000000
  end
end
\end{minted}


Побудова STO на обнові базису:
\begin{tcolorbox}[mybox]
\begin{multline*}
	\mathrm{STO}_{1s} = \sum_{i = 1}^3 C_i \cdot GTO(\alpha_i) = \\
	= \C{0.0401838903}\cdot\sgto{38.3549367370} +\\
	+ \C{0.2613913445}\cdot\sgto{5.7689081479} +\\
	+ \C{0.7930391578}\cdot\sgto{1.2399407035}
\end{multline*}
\end{tcolorbox}

\begin{tcolorbox}[mybox]
\begin{multline*}
	\mathrm{STO}_{2s} =  C_i \cdot GTO(\alpha_i) = \\ = \C{1.0000000000}\cdot\sgto{0.2975781595}.
\end{multline*}
\end{tcolorbox}

%% --------------------------------------------------------
%\clearpage
\section{Розрахунки орбіталей в ORCA}
%% --------------------------------------------------------

Файл \texttt{.inp}:

\begin{minted}[mathescape,
        fontsize=\small,
        bgcolor=gray!5,
        ]
        {ruby}
!RHF SP 6-31G
!Printbasis
!PrintMOs

* xyz 0 1
  He        0.0        0.0        0.00000
*
\end{minted}

Метод розрахунку --- \texttt{RHF} (Restricted Hartree-Fock), тип обчислень --- \texttt{SP} (Single Point), базис --- \texttt{6-31G}.

Виведення енергії атома та її складових в ORCA:

\begin{minted}[mathescape,
        fontsize=\small,
        bgcolor=gray!5,
        ]
        {ruby}
----------------
TOTAL SCF ENERGY
----------------

Total Energy       :           -2.85516048 Eh             -77.69287 eV

Components:
Nuclear Repulsion  :            0.00000000 Eh               0.00000 eV
Electronic Energy  :           -2.85516048 Eh             -77.69287 eV
One Electron Energy:           -3.88201948 Eh            -105.63512 eV
Two Electron Energy:            1.02685900 Eh              27.94225 eV

Virial components:
Potential Energy   :           -5.71032096 Eh            -155.38573 eV
Kinetic Energy     :            2.85516048 Eh              77.69287 eV
Virial Ratio       :            2.00000000
\end{minted}

Виведення орбіталей в ORCA:

\begin{minted}[mathescape,
        fontsize=\small,
        bgcolor=gray!5,
        ]
        {ruby}
                    0         1
                -0.91413   1.39986
                 2.00000   0.00000
                --------  --------
0He  1s         0.592081 -1.149818
0He  2s         0.513586  1.186959
\end{minted}

Побудуємо орбіталі на основі розрахунків:

\begin{align*}
	\phi_{0} & = \C{0.592081} \cdot \mathrm{STO}_{1s}  + \C{0.513586} \cdot \mathrm{STO}_{2s}, \quad \text{двічі заселена орбіталь}           \\
	\phi_{1} & = \C{-1.149818} \cdot \mathrm{STO}_{1s}  + \C{1.186959} \cdot \mathrm{STO}_{2s}. \quad \text{віртуальна (незаселена) орбіталь}
\end{align*}

Детермінант Слейтера (\emph{\color{red}будується лише із заселених орбіталей}):
\begin{equation*}
	\Phi(\vec\xi_1, \vec\xi_2) = \phi_{0}(\vec{r}_1)\phi_{0}(\vec{r}_2)\left[\alpha(1)\beta(2) - \alpha(2)\beta(1)\right] .
\end{equation*}

Електронна густина (\emph{\color{red}рахується лише по заселеним орбіталям}):
\begin{equation*}
	\rho =  2\cdot|\phi_{0}|^2 .
\end{equation*}

Незаселені орбіталі не мають фізичного сенсу (артефакт методу).


%=================== Побудова засобами pgfplots ===================

\tikzset{
	declare function ={
			a11  = 38.421634000000; C11  =  0.0401397393;
			a21  =  5.778030000000; C21  =  0.2612460970;
			a31  =  1.241774000000; C31  =  0.7931846246;
			%            a51  =    3.564152000; C51  =  0.16336284;
			%            a61  =    1.240443000; C61  =  0.33133146;
			%            a71  =     0.44731600; C71  =  0.41429728;
			%            a81  =     0.16420600; C81  =  0.18903228;
			%            a91  =     0.05747200; C91  =  0.00515606;
			%
			a12  =    0.2979640000; C12  =    1.0000000000;
			%            a22  =  5.778030000000; C22  =    0.300385478532;
			%            a32  =  1.241774000000; C32  =    0.912018000393;
			%            a42  =  0.297964000000; C42  =   -1.186958795989;
			%
			zeta1 = 1.45363;
			zeta2 = 2.91093;
			%            zeta2 = 1.45;
			c1 = 0.592081;
			c2 = 0.513586;
			sto1s(\x) = sqrt(zeta1^3/pi)*exp(-zeta1*\x);
			sto2s(\x) = sqrt(zeta2^3/pi)*exp(-zeta2*\x);
			%
			g1s1(\x) = C11*(2*a11/pi)^(3/4)*exp(-a11*\x^2);
			g1s2(\x) = C21*(2*a21/pi)^(3/4)*exp(-a21*\x^2);
			g1s3(\x) = C31*(2*a31/pi)^(3/4)*exp(-a31*\x^2);
			g1s4(\x) = C41*(2*a41/pi)^(3/4)*exp(-a41*\x^2);
			g1s5(\x) = C51*(2*a51/pi)^(3/4)*exp(-a51*\x^2);
			g1s6(\x) = C61*(2*a61/pi)^(3/4)*exp(-a61*\x^2);
			g1s7(\x) = C71*(2*a71/pi)^(3/4)*exp(-a71*\x^2);
			g1s8(\x) = C81*(2*a81/pi)^(3/4)*exp(-a81*\x^2);
			g1s9(\x) = C91*(2*a91/pi)^(3/4)*exp(-a91*\x^2);
			%
			g2s1(\x) = C12*(2*a12/pi)^(3/4)*exp(-a12*\x^2);
			g2s2(\x) = C22*(2*a22/pi)^(3/4)*exp(-a22*\x^2);
			g2s3(\x) = C32*(2*a32/pi)^(3/4)*exp(-a32*\x^2);
			g2s4(\x) = C42*(2*a42/pi)^(3/4)*exp(-a42*\x^2);
			%
			phi(\x) =  c1*(g1s1(\x) + g1s2(\x) + g1s3(\x)) + c2*(g2s1(\x));
		},
}

%=========================================================
\begin{figure}[h!]\centering
	%---------------------------------------------------------
	\begin{subfigure}[t]{0.45\linewidth}\centering
		\begin{tikzpicture}[background rectangle/.style={fill=gray!5}, show background rectangle]
			\begin{axis}[
%					ymax=2.2,
					ymin=-0,
					xmax=4,
					axis lines=left,
					xlabel={$r$, bohr},
					ylabel=$\phi_0$,
					width=0.95\linewidth,
				]
				%			\addplot [dashed, domain={0:4}, smooth]     {sto1s(x)};

				%			\addplot [cyan, domain={0:4}, smooth]       {g1s1(x)} ;
				%			\addplot [magenta, domain={0:4}, smooth]    {g1s2(x)}  ;
				%			\addplot [green, domain={0:4}, smooth]      {g1s3(x) } ;
				%			\addplot [green, domain={0:4}, smooth]      {g1s4(x) } ;

				%            \addplot [thick, domain={0:4}, smooth, dashed, samples=1500] {-0.592081*(g1s1(x) + g1s2(x) + g1s3(x) + g1s4(x))} ;
				%            \addplot [thick, domain={0:4}, smooth, dashed, samples=1500] {- 0.513586*(g2s1(x) + g2s2(x) + g2s3(x) + g2s4(x))} ;
				\addplot [thick, domain={0:4}, smooth, red, samples=1500] {phi(x)} ;
%				\addplot [thick, domain={0:4}, smooth, dashed, samples=1500] {c1 * sto1s(x) + c2 * sto2s(x)} ;
				%						\legend{STO,STO-4G}
			\end{axis}
		\end{tikzpicture}
		\caption{Орбіталь}
	\end{subfigure}
	\qquad%---------------------------------------------------------
	\begin{subfigure}[t]{0.45\linewidth}\centering
		\begin{tikzpicture}[background rectangle/.style={fill=gray!5}, show background rectangle]
			\begin{axis}[
					ymax=1.8,
					xmax=4,
					axis lines=left,
					xlabel={$r$, bohr},
					ylabel=$4\pi r^2 \rho(r)$,
					width=0.95\linewidth,
				]
				%			\addplot [dashed, domain={0:4}, smooth]     {sto2s(x)};
				%
				%%			\addplot [cyan, domain={0:4}, smooth]       {g1s1(x)} ;
				%%			\addplot [magenta, domain={0:4}, smooth]    {g1s2(x)}  ;
				%%			\addplot [green, domain={0:4}, smooth]      {g1s3(x) } ;
				%%			\addplot [green, domain={0:4}, smooth]      {g1s4(x) } ;

				\addplot [thick, domain={0:4}, smooth, red] {4*pi*x^2*2*(phi(x))^2} ;
			\end{axis}
		\end{tikzpicture}
		\caption{Електронна густина}
	\end{subfigure}
	\caption{Радіальні розподіли}
	%---------------------------------------------------------
\end{figure}
%=========================================================

%
%\begin{multline*}\label{}
%	\chi_{1s} = \sqrt{\frac{\zeta_1^3}{\pi}} e^{-\zeta_1 r} \approx \\ \approx  C_1 \left(\frac{2\alpha_1}{\pi}\right)^{3/4}e^{-\alpha_1 r^2} +
%	C_2 \left(\frac{2\alpha_2}{\pi}\right)^{3/4}e^{-\alpha_2 r^2} +
%	C_3 \left(\frac{2\alpha_3}{\pi}\right)^{3/4}e^{-\alpha_3 r^2}
%\end{multline*}
%
%\begin{center}
%	\begin{tikzpicture}
%		\begin{axis}[
%				ymax=0.3,
%				xmax=6,
%				axis lines=left,
%				xlabel=$r$,
%				ylabel=$2s$,
%			]
%			\addplot [dashed, domain={0:4}, smooth]     {sto2s(x)} ;
%
%			\addplot [cyan, domain={0:6}, smooth]       {g2s4(x)} ;
%			\addplot [magenta, domain={0:6}, smooth]    {g2s5(x)} ;
%			\addplot [green, domain={0:6}, smooth]      {g2s6(x) } ;
%
%			\addplot [thick, domain={0:6}, smooth, red] {g2s4(x) + g2s5(x) + g2s6(x) };
%
%			\legend{STO,,,,STO-3G}
%		\end{axis}
%	\end{tikzpicture}
%\end{center}
%
%\begin{multline*}\label{}
%	\chi_{2s} = \sqrt{\frac{\zeta_2^5}{3\pi}} r e^{-\zeta_2 r} \approx \\ \approx C_4 \left(\frac{2\alpha_1}{\pi}\right)^{3/4}e^{-\alpha_4 r^2} +
%	C_5 \left(\frac{2\alpha_2}{\pi}\right)^{3/4}e^{-\alpha_5 r^2} +
%	C_6 \left(\frac{2\alpha_3}{\pi}\right)^{3/4}e^{-\alpha_6 r^2}
%\end{multline*}
%
%\begin{center}
%	\begin{tikzpicture}
%		\begin{axis}[
%				ymax=0.3,
%				xmax=6,
%				axis lines=left,
%				xlabel=$r$,
%				ylabel=$2p$,
%			]
%			\addplot [dashed, domain={0:6}, smooth]     {sto2p(x)} ;
%
%			\addplot [cyan, domain={0:6}, smooth]       {g2p1(x)} ;
%			\addplot [magenta, domain={0:6}, smooth]    {g2p2(x)}  ;
%			\addplot [green, domain={0:6}, smooth]      {g2p3(x) };
%
%			\addplot [thick, domain={0:6}, smooth, red] {g2p1(x) + g2p2(x) + g2p3(x) };
%
%			\legend{STO,,,,STO-3G}
%		\end{axis}
%	\end{tikzpicture}
%\end{center}
%
%\begin{multline*}
%	\chi_{2p} = \sqrt{\frac{\zeta_2^5}{\pi}} x e^{-\zeta_2 r} \approx \\ D_1 \left(\frac{128\alpha_4^5}{\pi^3}\right)^{1/4} xe^{-\alpha_4 r^2} +
%	D_2 \left(\frac{128\alpha_5^5}{\pi^3}\right)^{1/4} xe^{-\alpha_5 r^2} +
%	D_3 \left(\frac{128\alpha_6^5}{\pi^3}\right)^{1/4} xe^{-\alpha_6 r^2}
%\end{multline*}

\clearpage
\nocite{BSE, EGMI1, EGMI2, EGMI3}
\printbibliography


\end{document}
