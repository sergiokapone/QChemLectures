% !TeX program = lualatex
% !TeX encoding = utf8
% !TeX spellcheck = uk_UA

% !TeX program = lualatex
% !TeX encoding = utf8
% !TeX spellcheck = uk_UA

\documentclass[]{article}
\usepackage[fontsize=14pt]{fontsize}

\usepackage{fontspec}
\setsansfont{CMU Sans Serif}%{Arial}
\setmainfont{CMU Serif}%{Times New Roman}
\setmonofont{CMU Typewriter Text}%{Consolas}

\defaultfontfeatures{Ligatures={TeX}}
\usepackage[math-style=TeX]{unicode-math}
\usepackage[russian, english, ukrainian]{babel}

\usepackage{biblatex}

\usepackage{microtype}
\usepackage{xurl}
\usepackage{indentfirst}

\usepackage[most]{tcolorbox}
\tcbset{highlight math style={enhanced,
  colframe=red,colback=white,arc=0pt,boxrule=1pt}}
\tcbset{
    mybox/.style={
        enhanced,
        colback=gray!5,
        fonttitle=\bfseries,
        sharp corners,
        boxrule=0pt,
        frame empty
    }
}
\usepackage{subcaption}
\captionsetup[subfigure]{justification=centering}
\usepackage{hyperref}

\usepackage{titlesec}

% Налаштування для section
\titleformat{\section}
  {\normalfont\large\bfseries}  % формат тексту (розмір і жирність)
  {\thesection.}                % формат номеру з точкою
  {1em}                         % відступ між номером і текстом
  {}
  [\hrulefill]

\titleformat{\subsection}
  {\normalfont\bfseries}  % формат тексту (розмір і жирність)
  {\thesubsection.}                % формат номеру з точкою
  {1em}                         % відступ між номером і текстом
  {}


\usepackage{minted}
\usepackage{tikz, calc}
\usetikzlibrary{backgrounds}

\usepackage{enumitem}


\addbibresource{../Bibliography/QuantumChemistry.bib}

%%                                                                        %%
%%============================= Геометрія сторінки =======================%%
%%                                                                        %%
\usepackage[%
	a4paper,%
	footskip=1cm,%
	headsep=0.3cm,%
	top=2cm, %поле сверху
	bottom=2cm, %поле снизу
	left=2cm, %поле ліворуч
	right=2cm, %поле праворуч
    ]{geometry}
%%                                                                        %%
%%============================== Інтерліньяж  ============================%%

\usepackage{tikz}
\usepackage{pgfplots}
\usepgfplotslibrary{units}
\pgfplotsset{compat=newest}
\usepackage{pgfplotstable}
\usepgfplotslibrary{groupplots}




\title{Атом гелію}
\author{}
\date{}

\begin{document}

\maketitle

\section{Атом гелію без міжелектронної взаємодії}

\subsection{Гамільтоніан системи}

Розглянемо атом гелію ($Z = 2$) з двома електронами. В атомній системі одиниць повний гамільтоніан має вигляд:

\begin{equation}
\hat{H} = -\frac{1}{2}\nabla_1^2 - \frac{1}{2}\nabla_2^2 - \frac{Z}{r_1} - \frac{Z}{r_2} + \frac{1}{r_{12}}
\end{equation}

де:
\begin{itemize}
\item перші два доданки --- кінетична енергія електронів
\item третій і четвертий доданки --- притягання електронів до ядра ($Z = 2$)
\item останній доданок --- кулонівська взаємодія між електронами
\end{itemize}

\subsection{Наближення незалежних електронів}

\textbf{Припущення:} знехтуємо міжелектронною взаємодією $1/r_{12}$.

Тоді гамільтоніан спрощується:
\begin{equation}
\hat{H}_0 = -\frac{1}{2}\nabla_1^2 - \frac{1}{2}\nabla_2^2 - \frac{Z}{r_1} - \frac{Z}{r_2}
\end{equation}

Це можна переписати як суму незалежних одноелектронних гамільтоніанів:
\begin{equation}
\hat{H}_0 = \hat{h}_1 + \hat{h}_2
\end{equation}

де:
\begin{align}
\hat{h}_1 &= -\frac{1}{2}\nabla_1^2 - \frac{Z}{r_1} \\
\hat{h}_2 &= -\frac{1}{2}\nabla_2^2 - \frac{Z}{r_2}
\end{align}

\subsection{Розділення змінних та факторизація}

\textbf{Ключовий момент:} оскільки гамільтоніан розділяється на суму $\hat{H}_0 = \hat{h}_1 + \hat{h}_2$, хвильова функція може бути факторизована:

\begin{equation}
\Psi(\mathbf{r}_1, \mathbf{r}_2) = \psi_1(\mathbf{r}_1) \psi_2(\mathbf{r}_2)
\end{equation}

\textbf{Фізичний сенс:} електрони рухаються незалежно один від одного, кожен у полі ядра.

\subsection{Розв'язання рівняння Шредінгера}

Рівняння Шредінгера:
\begin{equation}
\hat{H}_0 \Psi = E \Psi
\end{equation}

Підставляючи факторизовану функцію:
\begin{equation}
(\hat{h}_1 + \hat{h}_2) \psi_1(\mathbf{r}_1) \psi_2(\mathbf{r}_2) = E \psi_1(\mathbf{r}_1) \psi_2(\mathbf{r}_2)
\end{equation}

Ділимо на $\Psi = \psi_1 \psi_2$:
\begin{equation}
\frac{\hat{h}_1 \psi_1(\mathbf{r}_1)}{\psi_1(\mathbf{r}_1)} + \frac{\hat{h}_2 \psi_2(\mathbf{r}_2)}{\psi_2(\mathbf{r}_2)} = E
\end{equation}

\textbf{Ключове спостереження:} перший доданок залежить тільки від $\mathbf{r}_1$, другий --- тільки від $\mathbf{r}_2$. Щоб їх сума була константою, кожен доданок повинен дорівнювати константі:

\begin{align}
\hat{H}_1 \psi_1(\mathbf{r}_1) &= E_1 \psi_1(\mathbf{r}_1) \\
\hat{H}_2 \psi_2(\mathbf{r}_2) &= E_2 \psi_2(\mathbf{r}_2)
\end{align}

де $E = E_1 + E_2$.

\subsection{Водневоподібні атоми}

Кожне з рівнянь описує водневоподібний атом з ядерним зарядом $Z = 2$:

\textbf{Енергії:}
\begin{equation}
E_n = -\frac{Z^2}{2n^2} = -\frac{2}{n^2} \text{ а.о.}
\end{equation}

\textbf{Хвильові функції:}
\begin{equation}
\psi_{nlm}(r,\theta,\phi) = R_{nl}(r) Y_l^m(\theta,\phi)
\end{equation}

\textbf{Основний стан атома гелію (наближення):}
\begin{align}
\Psi_{gs} &= \psi_{1s}(\mathbf{r}_1) \psi_{1s}(\mathbf{r}_2) \\
E_{gs} &= 2 \times E_{1s} = 2 \times \left(-\frac{4}{2}\right) = -4 \text{ а.о.}
\end{align}

\section{Атом гелію з міжелектронною взаємодією}

\subsection{Повний гамільтоніан}

Повертаємося до повного гамільтоніана з взаємодією:
\begin{equation}
\hat{H} = \hat{h}_1 + \hat{h}_2 + \frac{1}{r_{12}}
\end{equation}

\subsection{Неможливість точної факторизації}

\textbf{Проблема:} доданок $1/r_{12}$ не може бути розділений на суму функцій від $\mathbf{r}_1$ та $\mathbf{r}_2$ окремо:

\begin{equation}
\frac{1}{r_{12}} = \frac{1}{|\mathbf{r}_1 - \mathbf{r}_2|} \neq f(\mathbf{r}_1) + g(\mathbf{r}_2)
\end{equation}

\textbf{Математичний доказ неможливості факторизації:}

Припустимо, що $\Psi(\mathbf{r}_1, \mathbf{r}_2) = \psi_1(\mathbf{r}_1) \psi_2(\mathbf{r}_2)$. Тоді:

\begin{equation}
\hat{H}_{full} \Psi = \left(\hat{h}_1 + \hat{h}_2 + \frac{1}{r_{12}}\right) \psi_1(\mathbf{r}_1) \psi_2(\mathbf{r}_2)
\end{equation}

\begin{equation}
= \hat{h}_1 \psi_1(\mathbf{r}_1) \psi_2(\mathbf{r}_2) + \psi_1(\mathbf{r}_1) \hat{h}_2 \psi_2(\mathbf{r}_2) + \frac{1}{r_{12}} \psi_1(\mathbf{r}_1) \psi_2(\mathbf{r}_2)
\end{equation}

Ділячи на $\psi_1 \psi_2$:
\begin{equation}
\frac{\hat{h}_1 \psi_1(\mathbf{r}_1)}{\psi_1(\mathbf{r}_1)} + \frac{\hat{h}_2 \psi_2(\mathbf{r}_2)}{\psi_2(\mathbf{r}_2)} + \frac{1}{r_{12}} = E
\end{equation}

\textbf{Висновок:} доданок $1/r_{12}$ залежить від обох координат одночасно, тому неможливо розділити змінні. Факторизація не працює!

\subsection{Фізичні наслідки}

\textbf{Кореляція електронів:} через взаємодію $1/r_{12}$ рух електронів стає скорельованим --- позиція одного електрона впливає на ймовірність знаходження іншого.

\textbf{Квантова заплутаність:} електрони утворюють заплутаний квантовий стан, який не може бути описаний добутком індивідуальних хвильових функцій.

\section{Теорія збурень для атома гелію}

\subsection{Чому теорія збурень?}

З попереднього розділу ми побачили, що:
\begin{itemize}
\item \textbf{Без взаємодії} $1/r_{12}$: можемо точно розв'язати (факторизація працює)
\item \textbf{З взаємодією} $1/r_{12}$: точний розв'язок неможливий (факторизація не працює)
\end{itemize}

\textbf{Ідея:} якщо взаємодія між електронами ``не дуже сильна'' порівняно з взаємодією з ядром, то можна використати теорію збурень:
\begin{itemize}
\item ``Головна'' частина: електрони в полі ядра (знаємо точний розв'язок)
\item ``Мале'' збурення: взаємодія між електронами
\end{itemize}

\textbf{Обґрунтування:} в атомі гелію $Z = 2$, тому притягання до ядра сильніше за відштовхування між електронами. Це робить теорію збурень розумним наближенням.

\subsection{Постановка задачі}

Представимо повний гамільтоніан у вигляді:
\begin{equation}
\hat{H} = \hat{H}_0 + \hat{V}
\end{equation}

де:
\begin{align}
\hat{H}_0 &= \hat{h}_1 + \hat{h}_2 \quad \text{(незбурений гамільтоніан)} \\
\hat{V} &= \frac{1}{r_{12}} \quad \text{(збурення)}
\end{align}

\subsection{Незбурені власні функції та енергії}

З попереднього розділу знаємо:
\begin{align}
\hat{H}_0 \Psi_n^{(0)} &= E_n^{(0)} \Psi_n^{(0)} \\
\Psi_n^{(0)} &= \psi_{n_1}(\mathbf{r}_1) \psi_{n_2}(\mathbf{r}_2) \\
E_n^{(0)} &= E_{n_1} + E_{n_2}
\end{align}

\subsection{Основний стан}

Для основного стану ($1s^2$ конфігурація):
\begin{align}
\Psi_{gs}^{(0)} &= \psi_{1s}(\mathbf{r}_1) \psi_{1s}(\mathbf{r}_2) \\
E_{gs}^{(0)} &= 2E_{1s} = 2 \times \left(-\frac{Z^2}{2}\right) = -Z^2 = -4 \text{ а.о.}
\end{align}

\subsection{Перша поправка до енергії}

Згідно з теорією збурень першого порядку:
\begin{equation}
E_{gs}^{(1)} = \langle \Psi_{gs}^{(0)} | \hat{V} | \Psi_{gs}^{(0)} \rangle
\end{equation}

Підставляючи:
\begin{equation}
E_{gs}^{(1)} = \left\langle \psi_{1s}(\mathbf{r}_1) \psi_{1s}(\mathbf{r}_2) \left| \frac{1}{r_{12}} \right| \psi_{1s}(\mathbf{r}_1) \psi_{1s}(\mathbf{r}_2) \right\rangle
\end{equation}

\begin{equation}
E_{gs}^{(1)} = \iint \psi_{1s}^*(\mathbf{r}_1) \psi_{1s}^*(\mathbf{r}_2) \frac{1}{r_{12}} \psi_{1s}(\mathbf{r}_1) \psi_{1s}(\mathbf{r}_2) d^3r_1 d^3r_2
\end{equation}

Оскільки $\psi_{1s}$ --- дійсна функція:
\begin{equation}
E_{gs}^{(1)} = \iint \psi_{1s}^2(\mathbf{r}_1) \psi_{1s}^2(\mathbf{r}_2) \frac{1}{r_{12}} d^3r_1 d^3r_2
\end{equation}

\subsection{Кулонівський інтеграл}

Позначимо цей інтеграл як кулонівський інтеграл $J$:
\begin{equation}
J = E_{gs}^{(1)} = \iint \psi_{1s}^2(\mathbf{r}_1) \psi_{1s}^2(\mathbf{r}_2) \frac{1}{r_{12}} d^3r_1 d^3r_2
\end{equation}

\textbf{Фізичний сенс кулонівського інтеграла:}

Інтеграл $J$ представляє електростатичну взаємодію між двома електронними зарядовими хмарами:
\begin{itemize}
\item $|\psi_{1s}(\mathbf{r}_1)|^2$ --- густина ймовірності першого електрона
\item $|\psi_{1s}(\mathbf{r}_2)|^2$ --- густина ймовірності другого електрона
\item $1/r_{12}$ --- кулонівська взаємодія між елементами зарядових хмар
\end{itemize}

\textbf{Інтерпретація:} $J$ --- це енергія відштовхування між двома сферично симетричними електронними хмарами в $1s$ стані.

\subsection{Загальна енергія основного стану}

Енергія основного стану в першому порядку теорії збурень:
\begin{equation}
E_{gs} = E_{gs}^{(0)} + E_{gs}^{(1)} = -Z^2 + J = -4 + J
\end{equation}

\textbf{Чисельне значення:} точне обчислення дає $J = 5/8 Z = 1.25$ а.о.

Таким чином:
\begin{equation}
E_{gs} = -4 + 1.25 = -2.75 \text{ а.о.}
\end{equation}

\textbf{Порівняння з експериментом:}
\begin{itemize}
\item Теорія збурень: $E_{gs} = -2.75$ а.о.
\item Експеримент: $E_{gs} = -2.90$ а.о.
\item Похибка: $\sim 5\%$
\end{itemize}

\subsection{Фізичні висновки}

\begin{enumerate}
\item Міжелектронна взаємодія \textbf{підвищує} загальну енергію (робить атом менш зв'язаним)
\item Кулонівський інтеграл $J > 0$ представляє енергію відштовхування електронів
\item Теорія збурень дає розумне наближення для основного стану
\end{enumerate}

\subsection{Недоліки теорії збурень}


\begin{enumerate}
\item  Відсутність кореляційних ефектів:
 Теорія збурень першого порядку не враховує:
 \begin{itemize}
 \item \textbf{Динамічні кореляції:} електрони ``уникають'' один одного в реальному часі
 \item \textbf{Кутові кореляції:} електрони мають тенденцію знаходитися на протилежних сторонах ядра
 \end{itemize}

\item Математичні проблеми:
 \begin{itemize}
 \item Ряд теорії збурень може розходитися для сильних збурень
 \item Вищі порядки стають все складнішими для обчислення
 \end{itemize}

\end{enumerate}
\textbf{Висновок:} теорія збурень --- хороший початок, але потрібні більш досконалі методи для точного опису багатоелектронних атомів.


\section{Метод Хартрі}

\subsection{Мотивація методу}

Після аналізу недоліків теорії збурень постає питання: чи можна знайти кращий спосіб врахувати міжелектронну взаємодію, зберігши при цьому можливість факторизації?

\textbf{Ідея Хартрі:} замінити точну миттєву взаємодію між електронами на взаємодію кожного електрона з усередненим електростатичним полем, створеним усіма іншими електронами.

\subsection{Основна концепція}

\textbf{Фізична картина:}
\begin{itemize}
\item Кожен електрон рухається незалежно
\item Але в ефективному полі, створеному ``розмазаними'' іншими електронами
\item Розподіл інших електронів визначається їх хвільовими функціями
\end{itemize}

\textbf{Математично:} шукаємо розв'язок у факторизованому вигляді:
\begin{equation}
\Psi(\mathbf{r}_1, \mathbf{r}_2) = \psi_1(\mathbf{r}_1) \psi_2(\mathbf{r}_2)
\end{equation}

але тепер $\psi_i$ визначаються самоузгоджено з урахуванням взаємодії.

%\subsection{Вивід рівнянь Хартрі}
%
%\textbf{Крок 1:} Енергія системи як функціонал орбіталей.
%
%Середня енергія системи:
%\begin{equation}
%E[\psi_1, \psi_2] = \langle \Psi | \hat{H} | \Psi \rangle
%\end{equation}
%
%\begin{align}
%E &= \int \psi_1^*(\mathbf{r}_1) \psi_2^*(\mathbf{r}_2) \left[ \hat{h}_1 + \hat{h}_2 + \frac{1}{r_{12}} \right] \psi_1(\mathbf{r}_1) \psi_2(\mathbf{r}_2) d^3r_1 d^3r_2
%\end{align}
%
%\begin{align}
%E &= \int |\psi_1(\mathbf{r}_1)|^2 \hat{h}_1 d^3r_1 + \int |\psi_2(\mathbf{r}_2)|^2 \hat{h}_2 d^3r_2 \\
%&\quad + \iint |\psi_1(\mathbf{r}_1)|^2 |\psi_2(\mathbf{r}_2)|^2 \frac{1}{r_{12}} d^3r_1 d^3r_2
%\end{align}
%
%\textbf{Крок 2:} Варіаційний принцип.
%
%Мінімізуємо енергію за умови нормування орбіталей:
%\begin{equation}
%\frac{\delta}{\delta \psi_i^*} \left[ E - \varepsilon_i \left( \int |\psi_i|^2 d^3r - 1 \right) \right] = 0
%\end{equation}

\subsection{Рівняння Хартрі}

Рівняння Хартрі:

\textbf{Для першого електрона:}
\begin{equation}
\left[ \hat{h}_1 + \int \frac{|\psi_2(\mathbf{r}_2)|^2}{r_{12}} d^3r_2 \right] \psi_1(\mathbf{r}_1) = \varepsilon_1 \psi_1(\mathbf{r}_1)
\end{equation}

\textbf{Для другого електрона:}
\begin{equation}
\left[ \hat{h}_2 +  \int \frac{|\psi_1(\mathbf{r}_1)|^2}{r_{12}} d^3r_1 \right] \psi_2(\mathbf{r}_2) = \varepsilon_2 \psi_2(\mathbf{r}_2)
\end{equation}

де хартрівські потенціали:
\begin{align}
V_H^{(2)}(\mathbf{r}_1) &= \int \frac{|\psi_2(\mathbf{r}_2)|^2}{r_{12}} d^3r_2 \\
V_H^{(1)}(\mathbf{r}_2) &= \int \frac{|\psi_1(\mathbf{r}_1)|^2}{r_{12}} d^3r_1
\end{align}

\subsection{Фізичний сенс і одноелектронне наближення}

\textbf{Хартрівський потенціал $V_H^{(2)}(\mathbf{r}_1)$:}
\begin{itemize}
\item Представляє електростатичний потенціал, створений зарядовою густиною $-|\psi_2(\mathbf{r}_2)|^2$ другого електрона
\item Перший електрон ``бачить'' другий як розмазану хмару
\item Статичний потенціал (не залежить від часу)
\end{itemize}

\textbf{Ключове спостереження:} хоча електрони реально взаємодіють, рівняння Хартрі дозволяють розглядати кожен електрон окремо.

\textbf{Як це працює:}
\begin{itemize}
\item Кожен електрон рухається в індивідуальному одноелектронному рівнянні
\item Взаємодія з іншими електронами ``захована'' в хартрівському потенціалі $V_H$
\item Електрон ``не знає'' точних координат інших електронів
\item Відчуває тільки усереднене електростатичне поле від них
\end{itemize}

\textbf{Фізична аналогія:} електрон рухається як ``одинока'' частинка в ефективному полі, створеному розмазаними зарядовими хмарами інших електронів.

\textbf{Математичні наслідки:}
\begin{equation}
\Psi(\mathbf{r}_1, \mathbf{r}_2) = \psi_1(\mathbf{r}_1) \psi_2(\mathbf{r}_2)
\end{equation}

Факторизація знову стає можливою! Але тепер орбіталі $\psi_i$ визначаються самоузгоджено, а не як прості водневоподібні функції.

\textbf{Зв'язок з попередніми методами:}
\begin{itemize}
\item \textbf{Незалежні електрони:} $V_H = 0$, факторизація точна
\item \textbf{Теорія збурень:} $V_H$ замінюється на ``малу поправку''
\item \textbf{Метод Хартрі:} $V_H$ визначається самоузгоджено з орбіталей
\end{itemize}

\subsection{Самоузгодженість --- ітераційна процедура}

\textbf{Проблема:} щоб знайти $\psi_1$, потрібно знати $\psi_2$, і навпаки!

\textbf{Розв'язок --- ітераційна процедура:}
\begin{enumerate}
\item Задати початкові наближення $\psi_1^{(0)}$, $\psi_2^{(0)}$
\item Обчислити $V_H^{(2)}$ з $\psi_2^{(0)}$ та розв'язати рівняння для $\psi_1^{(1)}$
\item Обчислити $V_H^{(1)}$ з $\psi_1^{(1)}$ та розв'язати рівняння для $\psi_2^{(1)}$
\item Повторювати до збіжності: $\psi_i^{(n+1)} \approx \psi_i^{(n)}$
\end{enumerate}

\textbf{Самоузгодженість означає:} кожен електрон створює електростатичне поле, в якому рухається інший електрон, і це поле узгоджується з розподілом електронів.


\section{Принцип Паулі та недоліки методу Хартрі}

\subsection{Проблема з факторизацією в методі Хартрі}

Хоча метод Хартрі покращує теорію збурень, він має принципову проблему: \textbf{не враховує принцип Паулі}.

\subsection{Принцип Паулі}

\textbf{Формулювання:} хвильова функція системи ферміонів (електронів) повинна бути антисиметричною відносно перестановки будь-яких двох частинок.

Для двох електронів:
\begin{equation}
\Psi(\mathbf{r}_1, \mathbf{r}_2) = -\Psi(\mathbf{r}_2, \mathbf{r}_1)
\end{equation}

\subsection{Проблема простої факторизації}

У методі Хартрі ми використовуємо:
\begin{equation}
\Psi_{Hartree}(\mathbf{r}_1, \mathbf{r}_2) = \psi_1(\mathbf{r}_1) \psi_2(\mathbf{r}_2)
\end{equation}

\textbf{Перевірка антисиметрії:}
\begin{align}
\Psi_{Hartree}(\mathbf{r}_2, \mathbf{r}_1) &= \psi_1(\mathbf{r}_2) \psi_2(\mathbf{r}_1) \\
&\neq -\psi_1(\mathbf{r}_1) \psi_2(\mathbf{r}_2) = -\Psi_{Hartree}(\mathbf{r}_1, \mathbf{r}_2)
\end{align}

\textbf{Висновок:} проста факторизація Хартрі \textbf{не є антисиметричною} і порушує принцип Паулі!

\subsection{Приклад: основний стан гелію}

Для основного стану гелію ($1s^2$) обидва електрони мають протилежні спіни, тому принцип Паулі дозволяє їм займати одну і ту ж орбіталь $1s$.

\textbf{Правильна антисиметрична функція:}
\begin{equation}
\Psi(\mathbf{r}_1, \mathbf{r}_2) = \frac{1}{\sqrt{2}} \left[ \phi_{1s}(\mathbf{r}_1)\alpha(1) \phi_{1s}(\mathbf{r}_2)\beta(2) - \phi_{1s}(\mathbf{r}_2)\alpha(2) \phi_{1s}(\mathbf{r}_1)\beta(1) \right]
\end{equation}

де $\alpha$, $\beta$ --- спінові функції.

\textbf{Спрощено (просторова частина):}
\begin{equation}
\Phi_\text{spatial} = \phi_{1s}(\mathbf{r}_1) \phi_{1s}(\mathbf{r}_2)
\end{equation}

У цьому випадку метод Хартрі дає правильний результат, бо спінова частина автоматично антисиметрична.

\subsection{Коли метод Хартрі особливо неточний}

\textbf{Збуджені стани з однаковими спінами:}
Наприклад, стан $1s^1 2s^1$ з паралельними спінами. Тут принцип Паулі критично важливий, а метод Хартрі дає принципово неправильні результати.

\textbf{Висновок:} потрібен метод, який поєднує переваги Хартрі (самоузгодженість) з правильною антисиметрією (принцип Паулі).

\section{Метод Хартрі-Фока}

\subsection{Мотивація}

Метод Хартрі-Фока поєднує переваги методу Хартрі (самоузгодженість, одноелектронне наближення) з правильним врахуванням принципу Паулі через антисиметричну хвильову функцію.

\subsection{Антисиметрична хвильова функція --- детермінант Слейтера}

Для системи двох електронів правильна антисиметрична хвильова функція будується у вигляді детермінанта Слейтера:

\begin{equation}
\Psi(\mathbf{r}_1, \mathbf{r}_2) = \frac{1}{\sqrt{2}} \begin{vmatrix}
\psi_1(\mathbf{r}_1) & \psi_2(\mathbf{r}_1) \\
\psi_1(\mathbf{r}_2) & \psi_2(\mathbf{r}_2)
\end{vmatrix}
\end{equation}

\begin{equation}
= \frac{1}{\sqrt{2}} \left[ \chi_1(\mathbf{r}_1)\chi_2(\mathbf{r}_2) - \chi_2(\mathbf{r}_1)\chi_1(\mathbf{r}_2) \right]
\end{equation}

де $\psi_i(\mathbf{r})$ --- спін-орбіталі, які включають і просторову, і спінову частини:
\begin{equation}
\psi_i(\mathbf{r}) = \phi_i(\mathbf{r}) \gamma_i
\end{equation}

$\gamma_i$ --- спінові функції $\alpha$ або $\beta$.

\subsection{Перевірка антисиметрії}

При перестановці електронів 1 ↔ 2:
\begin{align}
\Psi(\mathbf{r}_2, \mathbf{r}_1) &= \frac{1}{\sqrt{2}} \left[ \psi_1(\mathbf{r}_2)\psi_2(\mathbf{r}_1) - \psi_2(\mathbf{r}_2)\psi_1(\mathbf{r}_1) \right] \\
&= -\frac{1}{\sqrt{2}} \left[ \psi_1(\mathbf{r}_1)\psi_2(\mathbf{r}_2) - \psi_2(\mathbf{r}_1)\psi_1(\mathbf{r}_2) \right] \\
&= -\Psi(\mathbf{r}_1, \mathbf{r}_2)
\end{align}

\textbf{Антисиметрія виконується!}

%\subsection{Основний стан гелію}
%
%Для основного стану ($1s^2$) з протилежними спінами:
%\begin{align}
%\chi_1(\mathbf{r}) &= \psi_{1s}(\mathbf{r}) \alpha \\
%\chi_2(\mathbf{r}) &= \psi_{1s}(\mathbf{r}) \beta
%\end{align}
%
%Детермінант Слейтера:
%\begin{equation}
%\Psi(\mathbf{r}_1, \mathbf{r}_2) = \frac{1}{\sqrt{2}} \left[ \psi_{1s}(\mathbf{r}_1)\alpha(1) \psi_{1s}(\mathbf{r}_2)\beta(2) - \psi_{1s}(\mathbf{r}_2)\alpha(2) \psi_{1s}(\mathbf{r}_1)\beta(1) \right]
%\end{equation}
%
%\textbf{Спрощення:} для основного стану гелію спінова частина автоматично антисиметрична, тому просторова частина залишається симетричною:
%\begin{equation}
%\Psi_{spatial} = \psi_{1s}(\mathbf{r}_1) \psi_{1s}(\mathbf{r}_2)
%\end{equation}
%
%\textbf{Висновок:} для основного стану гелію методи Хартрі та Хартрі-Фока дають однакові результати!
%
%\subsection{Збуджені стани --- де проявляється різниця}
%
%Розглянемо збуджений стан $1s^1 2s^1$ з паралельними спінами:
%\begin{align}
%\chi_1(\mathbf{r}) &= \psi_{1s}(\mathbf{r}) \alpha \\
%\chi_2(\mathbf{r}) &= \psi_{2s}(\mathbf{r}) \alpha
%\end{align}
%
%\textbf{Детермінант Слейтера:}
%\begin{equation}
%\Psi(\mathbf{r}_1, \mathbf{r}_2) = \frac{1}{\sqrt{2}} \left[ \psi_{1s}(\mathbf{r}_1)\psi_{2s}(\mathbf{r}_2) - \psi_{2s}(\mathbf{r}_1)\psi_{1s}(\mathbf{r}_2) \right] \alpha(1)\alpha(2)
%\end{equation}
%
%\textbf{Ключова різниця:} просторова частина тепер антисиметрична, на відміну від простої факторизації Хартрі.
%
%\subsection{Рівняння Хартрі-Фока}

Застосовуючи варіаційний принцип до детермінанта Слейтера, отримуємо рівняння Хартрі-Фока:

\begin{equation}
\hat{f}_i \chi_i(\mathbf{r}) = \varepsilon_i \chi_i(\mathbf{r})
\end{equation}

де $\hat{f}_i$ --- оператор Фока:
\begin{equation}
\hat{f}_i = \hat{h}_i + \sum_{j \neq i} \left( \hat{J}_j - \hat{K}_j \right)
\end{equation}

\textbf{Компоненти оператора Фока:}
\begin{itemize}
\item $\hat{h}_i$ --- одноелектронний гамільтоніан (кінетична енергія + ядерне притягання)
\item $\hat{J}_j$ --- кулонівський (хартрівський) оператор
\item $\hat{K}_j$ --- обмінний оператор (новий!)
\end{itemize}

\subsection{Кулонівський та обмінний оператори}

\textbf{Кулонівський оператор:}
\begin{equation}
\hat{J}_j \chi_i(\mathbf{r}_1) = \left[ \int \frac{|\chi_j(\mathbf{r}_2)|^2}{r_{12}} d\mathbf{r}_2 \right] \chi_i(\mathbf{r}_1)
\end{equation}

\textbf{Обмінний оператор:}
\begin{equation}
\hat{K}_j \chi_i(\mathbf{r}_1) = \left[ \int \frac{\chi_j^*(\mathbf{r}_2) \chi_i(\mathbf{r}_2)}{r_{12}} d\mathbf{r}_2 \right] \chi_j(\mathbf{r}_1)
\end{equation}

\subsection{Фізичний сенс обмінної взаємодії}

\textbf{Кулонівська взаємодія $\hat{J}_j$:}
\begin{itemize}
\item Класична електростатична відштовхування
\item Електрон відчуває усереднене поле зарядової густини інших електронів
\item Аналогічно до методу Хартрі
\end{itemize}

\textbf{Обмінна взаємодія $\hat{K}_j$:}
\begin{itemize}
\item Чисто квантово-механічний ефект
\item Виникає з антисиметрії хвильової функції
\item Діє тільки між електронами з паралельними спінами
\item Створює ефективне ``відштовхування'' без класичного аналогу
\end{itemize}

\textbf{Обмінна енергія завжди від'ємна}, тому знижує загальну енергію системи.

\end{document}