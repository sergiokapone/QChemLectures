% !TeX program = lualatex
% !TeX encoding = utf8
% !TeX spellcheck = uk_UA
% !BIB program = biber
% !TeX root =../QChemBook.tex
\graphicspath{ {\currfiledir/Pictures/} }




%% --------------------------------------------------------
\chapter{Багатоелектронні атоми}
%% --------------------------------------------------------

Атом Гідрогену, гідрогеноподібні атоми, а також молекула $\mathrm{H_2^+}$ —
нечисленні електрон-ядерні системи, для яких вдається отримати
достатньо точні розв’язки, засновані на добре обґрунтованих наближеннях.
В атомах, де число електронів $N \ge 2$, зробити це вже надзвичайно складно.
Річ у тім, що якщо електронів більше одного, то кожен із них рухається
не лише в електричному полі ядра, а й у полі, створюваному ядром і
рештою електронів.

Щоб зрозуміти проблеми, які виникають при цьому, розглянемо
гамільтоніан нерухомого $N$-електронного атома, вибравши початок
координат у ядрі з зарядом $Z|e|$:

\begin{equation}
\hat{H} =
\sum_{i=1}^{N} \left( -\frac{\hbar^2}{2m_e}\nabla_i^2
- \frac{Ze^2}{r_i} \right)
+ \sum_{i<j}^{N} \frac{e^2}{r_{ij}} ,
\label{eq:H_multi}
\end{equation}

де $r_i$ — радіус-вектор, проведений від ядра до електрона $i$,
а $r_{ij} = |\mathbf{r}_i - \mathbf{r}_j|$ — відстань між електронами $i$ і $j$.
Індекс $i = 1, 2, \ldots, N$ пробігає всі електрони атома.
Умова $i < j$ запобігає подвійному врахуванню взаємодії між парами
електронів $i$ та $j$.
У цьому гамільтоніані ми не враховуємо спін-орбітальну взаємодію
та інші релятивістські ефекти.

Вираз~(\ref{eq:H_multi}) відрізняється від гамільтоніана атома Гідрогену
лише наявністю членів, пропорційних $1/r_{ij}$, які описують
міжелектронну взаємодію. Саме цей факт є джерелом основних труднощів:
члени, що залежать від $r_{ij}$, не дозволяють розділити змінні в
сферичних координатах у рівнянні Шредингера й представити хвильову
функцію у вигляді добутку радіальної та кутової частин.
Крім того, врахувати явно електронну взаємодію, яка залежить від
відстані $r_{ij} = |\mathbf{r}_i - \mathbf{r}_j|$ і є функцією кутів,
можна лише в найпростіших випадках.

Розглянемо атом Гелію, що має два електрони у стані $1s$.
Гамільтоніан~(\ref{eq:H_multi}), записаний явно для електронів цього атома,
яким умовно надамо номери 1 та 2, має вигляд:

\begin{equation}
\hat{H} = T_{e,1}(\mathbf{r}_1) + T_{e,2}(\mathbf{r}_2)
+ V_{eN,1}(\mathbf{r}_1) + V_{eN,2}(\mathbf{r}_2)
+ V_{ee}(\mathbf{r}_1, \mathbf{r}_2).
\label{eq:H_He}
\end{equation}

Електрони не можна відрізнити один від одного, отже
\[
T_{e,1}(\mathbf{r}) = T_{e,2}(\mathbf{r}), \qquad
V_{eN,1}(\mathbf{r}) = V_{eN,2}(\mathbf{r}).
\]

Представимо хвильову функцію атома Гелію у вигляді
\[
\Psi = \varphi_1(\mathbf{r}_1)\,\varphi_2(\mathbf{r}_2),
\]
де $\varphi_i(\mathbf{r}_i)$ — одноелектронні хвильові функції $1s$-електронів.
Як початкове наближення для $\varphi_i$ логічно використовувати водневі
$1s$-функції (див. табл.~1.5).
Відмінність хвильових функцій атома Гелію від водневих
урахуємо заміною $Z$ в експоненті на варіаційний параметр $z$.

Запишемо тепер середні значення складових гамільтоніана~(\ref{eq:H_He})
з урахуванням зроблених припущень, додаючи і віднімаючи в операторі
потенціальної енергії члени, що містять $z$:

\begin{equation}
E(z)
= 2\langle T_e \rangle
+ 2\langle V_{eN} \rangle
+ \langle V_{ee} \rangle,
\label{eq:E_He}
\end{equation}
де середнє береться за хвильовою функцією
$\Psi = \varphi_1(\mathbf{r}_1)\varphi_2(\mathbf{r}_2)$.

Після обчислень отримаємо, що перші два члени суми в~(\ref{eq:E_He})
дають енергію воднеподібного атома (див. формулу~(1.141)):

\[
E_1 = - \frac{z^2 e^2}{a_0},
\]
а додатковий член, що враховує міжелектронне відштовхування, дорівнює
\[
E_{ee} = \frac{5}{8}\frac{z e^2}{a_0}.
\]
Таким чином, повна енергія атома Гелію в даному наближенні:

\begin{equation}
E(z) = -2 \frac{z^2 e^2}{a_0}
+ \frac{5}{4}\frac{z e^2}{a_0}
- 2(Z - z)\frac{z e^2}{a_0}.
\label{eq:E_total}
\end{equation}

Мінімізуємо цю енергію за параметром $z$:
\[
\frac{dE}{dz} = 0.
\]
Звідси маємо
\[
z = Z - \frac{5}{16} = 1{,}6875.
\]

Підставивши це значення в~(\ref{eq:E_total}), знаходимо:
\[
E = -77{,}49~\text{еВ}.
\]
Точне експериментальне значення енергії основного стану атома Гелію
становить $E = -79{,}014~\text{еВ}$, тобто похибка нашого наближення
близько 2~\%.
З огляду на простоту моделі, така точність цілком прийнятна.

Фізично це означає, що дія одного електрона на інший зменшує
ефективний вплив позитивного заряду ядра.
Можна говорити про \textbf{екранування заряду ядра електронами}
або про \textbf{ефективний заряд ядра} $z$.

\begin{figure}[h!]
\centering
\begin{tikzpicture}[scale=1.1]
\draw[->] (0,0) -- (6,0) node[right] {$r,\,\text{ат. од.}$};
\draw[->] (0,0) -- (0,3) node[above] {$\psi_{1s}(r)$};
\draw[domain=0:5,smooth,variable=\x,thick,blue] plot ({\x},{2.5*exp(-\x)});
\draw[domain=0:5,smooth,variable=\x,thick,red,dashed] plot ({\x},{2.5*exp(-1.6875*\x/2)});
\node[blue] at (4.5,1.2) {$Z=2$};
\node[red] at (4.5,0.6) {$z=1.6875$};
\end{tikzpicture}
\caption{Залежність хвильової функції $1s$-електронів атома Гелію
від відстані до ядра: (1) неекрановане ядро ($Z=2$),
(2) екрановане ядро ($z=1.6875$).}
\label{fig:He_wave}
\end{figure}

\section{Підхід Гіллерааса}

Гіллераас~\cite{Hylleraas1929} використав інший підхід:
він явно ввів у хвильову функцію атома відстань між електронами
$r_{12}$ і представив її у вигляді
\[
\Psi = N\,\varphi_1(\mathbf{r}_1)\,\varphi_2(\mathbf{r}_2)\,[1 + b\,r_{12}],
\]
де $N$ — нормувальний множник,
а $\varphi_i(\mathbf{r}_i)$ — модифіковані хвильові функції $1s$-електронів.
В якості початкового наближення для $\varphi_i$
також використовували водневоподібні функції із параметром $z$.

Записавши середнє значення енергії основного стану атома Гелію
у вигляді інтеграла
\[
E(z, b) = \frac{\langle \Psi | \hat{H} | \Psi \rangle}
{\langle \Psi | \Psi \rangle},
\]
і мінімізувавши $E(z,b)$ відносно параметрів $z$ і $b$
(в еліптичних координатах),
Гіллераас отримав:
\[
z = 1{,}849, \qquad b = 0{,}364.
\]
Це дає енергію
\[
E = -78{,}673~\text{еВ},
\]
що відрізняється від точного значення лише на $0{,}4\%$.

Метод Гіллерааса придатний для гелійподібних атомів
типу $\mathrm{H^-}$, $\mathrm{Li^+}$, $\mathrm{Be^{2+}}$ тощо.
У загальному випадку його реалізувати складно, тому доводиться
шукати інші наближені методи розв’язання рівняння Шредінгера.

\section{Варіаційний принцип і розв’язання рівняння Шредінгера}

Щоб вивчити електронні властивості будь-якої хімічної системи, потрібно
розв’язати рівняння Шредінгера для всіх можливих електронних станів,
зазначивши вид потенціальної енергії та граничні умови.

Будь-яка квантова система прагне до стану з мінімальною енергією.
Отже, наближені розв’язки рівняння Шредінгера можна знайти за допомогою
\textbf{варіаційного принципу}, мінімізуючи енергію системи та визначаючи
оптимальну форму хвильової функції.

Нехай $\hat{H}$ — оператор повної енергії системи. Визначимо
функціонал енергії:
\begin{equation}
E[\Psi] =
\frac{\langle \Psi | \hat{H} | \Psi \rangle}
{\langle \Psi | \Psi \rangle}.
\label{eq:variational_principle}
\end{equation}
Відповідно до варіаційного принципу:
\[
E[\Psi] \ge E_0,
\]
де $E_0$ — енергія основного стану системи, а рівність досягається
тільки тоді, коли $\Psi$ є точною хвильовою функцією цього стану.

\subsection{Сутність варіаційного методу}

Варіаційний метод полягає в тому, що ми обираємо пробну функцію
$\Psi(a_1, a_2, \ldots, a_n)$, яка містить деяку кількість варіаційних
параметрів $a_i$, і мінімізуємо функціонал~(\ref{eq:variational_principle})
щодо цих параметрів. Значення параметрів, при яких $E[\Psi]$ досягає
мінімуму, вважаються найкращими, а знайдене $E$ — наближеним
значенням енергії основного стану.

Варіаційний принцип можна записати у вигляді
\[
\delta E[\Psi] = 0,
\]
що еквівалентно стаціонарності функціонала енергії.

Якщо пробна функція правильно відтворює основні фізичні властивості
системи (симетрію, поведінку при $r \to 0$ і $r \to \infty$, тощо),
то отриманий результат буде дуже близьким до точного значення енергії.
Завдяки цьому варіаційний принцип є одним із найважливіших інструментів
квантової механіки, особливо у випадку складних систем, для яких точні
розв’язки рівняння Шредінгера відсутні.

\subsection{Варіаційний метод для атомів}

У випадку атомів із багатьма електронами хвильову функцію можна подати
у вигляді добутку одноелектронних функцій:
\[
\Psi = \varphi_1(\mathbf{r}_1)\,\varphi_2(\mathbf{r}_2)\,\ldots\,\varphi_N(\mathbf{r}_N),
\]
де $\varphi_i(\mathbf{r}_i)$ описують рух кожного електрона в деякому
ефективному полі, створюваному ядром і рештою електронів.

Мінімізація середньої енергії
\[
E = \langle \Psi | \hat{H} | \Psi \rangle
\]
за кожною з одноелектронних функцій призводить до системи рівнянь,
у яких кожен електрон рухається в середньому полі інших електронів.
Це і є суть \textbf{методу Гартрі}.

\section{Метод Гартрі}

Метод Гартрі є одним із найважливіших наближених способів
розв’язання рівняння Шредінгера для багатоелектронних атомів.
Його ідея полягає в тому, щоб розглядати кожен електрон як такий,
що рухається в деякому \textbf{середньому полі}, створеному ядром і
іншими електронами.

\subsection{Основні припущення методу}

У цьому методі хвильова функція всього атома записується як
\textbf{добуток одноелектронних функцій}:
\begin{equation}
\Psi(\mathbf{r}_1, \mathbf{r}_2, \ldots, \mathbf{r}_N)
= \varphi_1(\mathbf{r}_1)\,\varphi_2(\mathbf{r}_2)\,\ldots\,\varphi_N(\mathbf{r}_N),
\label{eq:hartree_product}
\end{equation}
де $\varphi_i(\mathbf{r}_i)$ — нормовані орбіталі кожного електрона.
Це наближення нехтує \textbf{обмінною симетрією} хвильової функції
щодо перестановки електронів, тому його часто називають
\textbf{неспіновим наближенням} або \textbf{методом Гартрі без обміну}.

Кожен електрон у цьому наближенні відчуває потенціал
\textbf{середнього електростатичного поля}:
\[
V_{\text{еф}}(\mathbf{r}_i)
= -\frac{Z e^2}{r_i}
+ \sum_{j \ne i}
\int \frac{e^2\,|\varphi_j(\mathbf{r}_j)|^2}{|\mathbf{r}_i - \mathbf{r}_j|}\,d\tau_j,
\]
яке створюється ядром і середнім розподілом заряду решти електронів.

\subsection{Виведення рівнянь Гартрі}

Варіаційний принцип~(\ref{eq:variational_principle}) вимагає мінімуму
функціонала енергії:
\[
E[\Psi]
= \int \Psi^*(\mathbf{r}_1, \ldots, \mathbf{r}_N)
\hat{H}
\Psi(\mathbf{r}_1, \ldots, \mathbf{r}_N)\,d\tau_1\ldots d\tau_N,
\]
де гамільтоніан $\hat{H}$ має вигляд~(\ref{eq:H_multi}).
Мінімізуючи $E$ відносно кожної з одноелектронних функцій $\varphi_i^*$,
одержуємо систему рівнянь Гартрі:
\begin{equation}
\left[
-\frac{\hbar^2}{2m_e}\nabla_i^2
- \frac{Ze^2}{r_i}
+ V_i(\mathbf{r}_i)
\right]
\varphi_i(\mathbf{r}_i)
= \varepsilon_i \varphi_i(\mathbf{r}_i),
\label{eq:hartree_equation}
\end{equation}
де $V_i(\mathbf{r}_i)$ — середній потенціал, створений усіма іншими
електронами, крім $i$-го:
\begin{equation}
V_i(\mathbf{r}_i)
= \sum_{j \ne i}
\int \frac{e^2\,|\varphi_j(\mathbf{r}_j)|^2}
{|\mathbf{r}_i - \mathbf{r}_j|}\,d\tau_j.
\label{eq:hartree_potential}
\end{equation}

Система рівнянь~(\ref{eq:hartree_equation}) є \textbf{нелінійною},
оскільки потенціал $V_i(\mathbf{r}_i)$ сам залежить від орбіталей
$\varphi_j$.
Тому її розв’язують ітераційно:
\begin{enumerate}
\item задають початкові орбіталі $\varphi_i^{(0)}$;
\item обчислюють потенціал $V_i^{(0)}$ за формулою~(\ref{eq:hartree_potential});
\item розв’язують рівняння~(\ref{eq:hartree_equation}) для нових орбіталей;
\item повторюють кроки, доки зміни енергії не стануть меншими за обрану точність.
\end{enumerate}
Цей процес відомий як \textbf{самоузгоджена процедура} (\emph{self-consistent field}, SCF).

\subsection{Властивості енергії в методі Гартрі}

Загальна енергія системи, отримана з рівнянь Гартрі, обчислюється як
\begin{equation}
E = \sum_i \varepsilon_i
- \frac{1}{2} \sum_i \int \varphi_i^*(\mathbf{r})
V_i(\mathbf{r})\,
\varphi_i(\mathbf{r})\,d\tau,
\label{eq:hartree_energy}
\end{equation}
де другий доданок усуває подвійний облік електростатичної взаємодії.

Енергії $\varepsilon_i$, що з’являються в рівняннях~(\ref{eq:hartree_equation}),
не є енергіями окремих електронів; вони лише параметри, які з’являються
внаслідок ортонормування орбіталей:
\[
\langle \varphi_i | \varphi_j \rangle = \delta_{ij}.
\]

Розв’язки рівнянь~(\ref{eq:hartree_equation}) для кожного електрона
називають \textbf{атомними орбіталями}, а потенціал~(\ref{eq:hartree_potential})
— \textbf{потенціалом середнього поля}.

\subsection{Обмеження методу}

Метод Гартрі нехтує квантовою \textbf{обмінною симетрією},
оскільки хвильова функція~(\ref{eq:hartree_product}) не антисиметрична
щодо перестановки координат електронів.
Тому метод Гартрі не може пояснити такі явища, як:
\begin{itemize}
\item обмінна енергія та виродження спінових станів,
\item правила заповнення орбіталей (принцип Паулі),
\item розщеплення тонкої структури спектрів.
\end{itemize}

Щоб урахувати ці ефекти, необхідно використовувати
\textbf{антисиметризовану хвильову функцію}, тобто детермінант Слейтера.
Це приводить до вдосконаленої версії методу, відомої як
\textbf{метод Гартрі–Фока}, який розглянемо далі.

\section{Метод Гартрі–Фока}

Метод Гартрі–Фока є розвитком методу Гартрі й усуває його головний
недолік — відсутність антисиметрії хвильової функції відносно
перестановки електронів.
Згідно з принципом Паулі, повна хвильова функція системи електронів
повинна змінювати знак при перестановці будь-якої пари частинок:
\[
\Psi(\ldots,\mathbf{r}_i,\sigma_i,\ldots,\mathbf{r}_j,\sigma_j,\ldots)
= -\Psi(\ldots,\mathbf{r}_j,\sigma_j,\ldots,\mathbf{r}_i,\sigma_i,\ldots).
\]
Тут $\sigma_i$ і $\sigma_j$ — спінові координати електронів.

\subsection{Детермінант Слейтера}

Антисиметрична хвильова функція для системи $N$ електронів може бути
записана у вигляді \textbf{детермінанта Слейтера}:
\begin{equation}
\Psi(1,2,\ldots,N)
= \frac{1}{\sqrt{N!}}
\begin{vmatrix}
\psi_1(1) & \psi_2(1) & \ldots & \psi_N(1) \\
\psi_1(2) & \psi_2(2) & \ldots & \psi_N(2) \\
\vdots    & \vdots    & \ddots & \vdots    \\
\psi_1(N) & \psi_2(N) & \ldots & \psi_N(N)
\end{vmatrix},
\label{eq:slater_determinant}
\end{equation}
де $\psi_i(k)$ — одноелектронні спін-орбіталі, які залежать
від просторових координат $\mathbf{r}_k$ і спіну $\sigma_k$.

Антисиметричність детермінанта забезпечує автоматичне виконання
принципу Паулі: якщо дві орбіталі збігаються, детермінант дорівнює нулю.

\subsection{Середнє значення енергії}

Середня енергія системи, що описується хвильовою функцією
(\ref{eq:slater_determinant}), дорівнює
\begin{equation}
E = \sum_i h_{ii}
+ \frac{1}{2} \sum_{i,j}
\left( J_{ij} - K_{ij} \right),
\label{eq:HF_energy}
\end{equation}
де
\[
h_{ii} =
\int \psi_i^*(1)\,\hat{h}(1)\,\psi_i(1)\,d\tau_1
\]
— середня одноелектронна енергія ($\hat{h}$ — оператор кінетичної
енергії й взаємодії з ядром),
а $J_{ij}$ і $K_{ij}$ — інтеграли кулонівської та обмінної взаємодії:
\begin{align}
J_{ij} &= \iint
\psi_i^*(1)\psi_j^*(2)
\frac{e^2}{r_{12}}
\psi_i(1)\psi_j(2)\,
d\tau_1 d\tau_2, \label{eq:coulomb_integral} \\
K_{ij} &= \iint
\psi_i^*(1)\psi_j^*(2)
\frac{e^2}{r_{12}}
\psi_j(1)\psi_i(2)\,
d\tau_1 d\tau_2.
\label{eq:exchange_integral}
\end{align}

Перший інтеграл $J_{ij}$ має класичний електростатичний зміст:
він описує середню кулонівську взаємодію між електронами, розподіленими
в орбіталях $\psi_i$ і $\psi_j$.
Другий інтеграл $K_{ij}$ не має класичного аналога й відображає чисто
квантове явище — \textbf{обмінну взаємодію}, що виникає з
антисиметричності хвильової функції.

\subsection{Рівняння Гартрі–Фока}

Застосування варіаційного принципу до функціонала
енергії~(\ref{eq:HF_energy}) за умовою ортонормування орбіталей
\[
\langle \psi_i | \psi_j \rangle = \delta_{ij}
\]
призводить до системи рівнянь Гартрі–Фока:
\begin{equation}
\hat{F}(1)\,\psi_i(1)
= \varepsilon_i\,\psi_i(1),
\label{eq:HF_equation}
\end{equation}
де оператор Фока $\hat{F}(1)$ має вигляд
\begin{equation}
\hat{F}(1)
= \hat{h}(1)
+ \sum_j
\left[
\hat{J}_j(1) - \hat{K}_j(1)
\right].
\label{eq:Fock_operator}
\end{equation}

Оператори $\hat{J}_j$ і $\hat{K}_j$ діють на $\psi_i(1)$ таким чином:
\begin{align}
\hat{J}_j(1)\psi_i(1)
&= \left[
\int \psi_j^*(2)
\frac{e^2}{r_{12}}
\psi_j(2)\,d\tau_2
\right]
\psi_i(1),
\label{eq:J_operator}\\
\hat{K}_j(1)\psi_i(1)
&= \int \psi_j^*(2)
\frac{e^2}{r_{12}}
\psi_i(2)\,d\tau_2\,
\psi_j(1).
\label{eq:K_operator}
\end{align}

Таким чином, рівняння~(\ref{eq:HF_equation}) мають форму власних рівнянь
для кожної орбіталі $\psi_i$, але всі вони пов’язані між собою через
оператор Фока, який залежить від усіх орбіталей системи.
Розв’язок цієї системи рівнянь здійснюється
\textbf{самоузгодженою процедурою} — аналогічно до методу Гартрі.

\subsection{Фізичний зміст методу Гартрі–Фока}

Порівняно з методом Гартрі, метод Гартрі–Фока враховує:
\begin{itemize}
\item антисиметрію хвильової функції,
\item обмінну взаємодію між електронами з однаковими спінами,
\item принцип Паулі й правильне виродження спінових станів.
\end{itemize}

Енергія системи, обчислена в цьому наближенні, завжди менша
(або рівна) енергії, отриманій у методі Гартрі, тобто
\[
E_{\text{HF}} \le E_{\text{H}}.
\]
Однак навіть метод Гартрі–Фока не враховує \textbf{кореляцію електронів},
пов’язану з узгодженим миттєвим рухом частинок.
Для точнішого опису енергії атомів і молекул розробляються методи, що
виходять за межі наближення Гартрі–Фока (методи після Гартрі–Фока).

\subsection{Ітераційна процедура розв’язання}

Розв’язання рівнянь Гартрі–Фока здійснюється послідовно:
\begin{enumerate}
\item вибирають початковий набір орбіталей $\psi_i^{(0)}$;
\item обчислюють оператор Фока $\hat{F}^{(0)}$;
\item розв’язують рівняння~(\ref{eq:HF_equation}) для нового набору орбіталей;
\item повторюють кроки, доки зміна енергії не стане меншою за обрану точність.
\end{enumerate}
Після досягнення збіжності одержують набір орбіталей
$\{\psi_i\}$ і власних значень $\varepsilon_i$, що характеризують
ефективні одноелектронні рівні системи.

\subsection{Загальна оцінка}

Метод Гартрі–Фока є основою сучасних обчислювальних методів
у квантовій хімії та фізиці атомів.
Його результати для енергій, потенціалів і розподілу електронної густини
зазвичай досить точні, щоб описувати більшість атомних і молекулярних
властивостей без залучення складніших кореляційних поправок.

\section{Кореляційна енергія і методи після Гартрі–Фока}

Метод Гартрі–Фока враховує кулонівську та обмінну взаємодії електронів,
але нехтує \textbf{кореляцією їхнього миттєвого руху}.
У дійсності рух електронів не є незалежним: коли один електрон
наближається до певної ділянки простору, інші миттєво реагують, змінюючи
свій розподіл, щоб уникнути взаємного відштовхування.

\subsection{Поняття кореляційної енергії}

\textbf{Кореляційна енергія} визначається як різниця між точною
енергією основного стану $E_{\text{exact}}$ і енергією,
отриманою в наближенні Гартрі–Фока $E_{\text{HF}}$:
\begin{equation}
E_{\text{corr}} = E_{\text{exact}} - E_{\text{HF}}.
\label{eq:correlation_energy}
\end{equation}

Оскільки метод Гартрі–Фока завжди дає енергію вищу за істинну
(через варіаційний принцип), то $E_{\text{corr}}$ завжди від’ємна:
\[
E_{\text{corr}} < 0.
\]
За величиною кореляційна енергія становить кілька відсотків від
повної енергії атома, але саме вона визначає тонкі ефекти —
точні значення іонізаційних потенціалів, енергій зв’язку молекул,
а також спектральні розщеплення.

\subsection{Типи кореляційних ефектів}

Кореляцію електронів поділяють на два основні типи:
\begin{itemize}
\item \textbf{динамічна кореляція} — враховує швидкі флуктуації
положень електронів навколо середнього поля;
\item \textbf{нединамічна (статична) кореляція} — проявляється у
системах, де декілька конфігурацій мають близькі енергії,
тобто хвильова функція повинна бути лінійною комбінацією
кількох детермінантів.
\end{itemize}

Методи після Гартрі–Фока спрямовані на кількісне врахування обох типів
кореляції за допомогою розширення базису хвильових функцій.

\subsection{Метод конфігураційної взаємодії (CI)}

Одним із найпряміших способів урахування кореляції є
\textbf{метод конфігураційної взаємодії} (Configuration Interaction, CI).
У цьому методі хвильова функція системи розглядається як
лінійна комбінація кількох детермінантів Слейтера:
\begin{equation}
\Psi = c_0 \Psi_0 + \sum_{a,r} c_a^r \Psi_a^r
+ \sum_{a<b,r<s} c_{ab}^{rs} \Psi_{ab}^{rs} + \ldots,
\label{eq:CI_expansion}
\end{equation}
де $\Psi_0$ — детермінант Гартрі–Фока (основна конфігурація),
а $\Psi_a^r$, $\Psi_{ab}^{rs}$ тощо — збуджені конфігурації,
що отримуються заміною однієї або кількох орбіталей $\psi_a$
на вищі орбіталі $\psi_r$.

Підставляючи (\ref{eq:CI_expansion}) у рівняння Шредінгера та
застосовуючи варіаційний принцип, одержують систему лінійних рівнянь
для коефіцієнтів $c_i$.
Розв’язання цієї системи дозволяє отримати енергію та хвильову функцію,
які враховують міжелектронну кореляцію.

Найточнішим є \textbf{повний CI} (Full CI), який включає всі можливі
збудження. Однак кількість конфігурацій росте комбінаційно з числом
електронів і орбіталей, тому цей метод застосовний лише до найменших систем.

\subsection{Метод збурень Меллера–Плессета}

Іншим шляхом урахування кореляції є застосування
\textbf{теорії збурень Меллера–Плессета} (Møller–Plesset, MP).
Гамільтоніан системи розбивають на частину Гартрі–Фока
та збурення:
\[
\hat{H} = \hat{H}_0 + \lambda \hat{V},
\]
де $\hat{H}_0$ — ефективний гамільтоніан Фока, а $\hat{V}$ —
залишковий оператор, який описує відхилення від середнього поля.

Енергію системи розкладають у ряд за степенями $\lambda$:
\[
E = E^{(0)} + E^{(1)} + E^{(2)} + \ldots,
\]
де $E^{(0)} = E_{\text{HF}}$,
а член $E^{(2)}$ відповідає першому наближенню до кореляційної енергії.
Найчастіше використовується \textbf{друге наближення MP2},
яке дає досить точні результати для багатьох атомів і молекул.

\subsection{Методи багатоконфігураційного типу}

Для систем з істотною нединамічною кореляцією застосовують
\textbf{багатоконфігураційні} варіаційні методи, наприклад:
\begin{itemize}
\item багатоконфігураційний метод Гартрі–Фока (MCHF),
\item метод активного простору (CASSCF),
\item метод багатоконфігураційних збурень (CASPT2).
\end{itemize}
У цих методах одночасно оптимізуються коефіцієнти конфігурацій і
форма орбіталей, що забезпечує високу точність для складних систем.

\subsection{Метод функціоналу густини (DFT)}

Сучасною альтернативою методам після Гартрі–Фока є
\textbf{метод функціонала електронної густини} (Density Functional Theory, DFT).
Згідно з теоремою Гоенберга–Кона, усі властивості
електронної системи визначаються лише її густиною $\rho(\mathbf{r})$.
Функціонал енергії має вигляд:
\begin{equation}
E[\rho] =
T[\rho] + V_{ne}[\rho] + J[\rho] + E_{xc}[\rho],
\label{eq:DFT_energy}
\end{equation}
де $T[\rho]$ — кінетична енергія, $V_{ne}[\rho]$ — взаємодія з ядрами,
$J[\rho]$ — класична кулонівська енергія,
а $E_{xc}[\rho]$ — обмінно-кореляційний функціонал,
який містить усю складність методу.

Завдяки простоті формалізму та високій точності,
метод DFT став основним інструментом квантової хімії та фізики твердого тіла.

\subsection{Підсумок}

Методи після Гартрі–Фока дозволяють послідовно враховувати ефекти
міжелектронної кореляції з будь-яким бажаним ступенем точності.
Найчастіше на практиці використовують комбінації методів,
зокрема MP2, CI, CASSCF і DFT.

Таким чином, розвиток варіаційних і післяваріаційних методів дозволяє
здійснювати високоточні квантово-хімічні розрахунки енергій,
спектрів і властивостей складних атомних і молекулярних систем.

\section{Підсумки і висновки}

У цій главі розглянуто основні методи квантово-механічного опису
багатоелектронних атомів.
Починаючи з гамільтоніана $N$-електронної системи, ми побачили, що
міжелектронна взаємодія робить рівняння Шредінгера нероздільним
і вимагає використання наближених методів.

\begin{enumerate}
\item
Для атомів із двома електронами (гелій, іон $\mathrm{H^-}$ тощо)
наближення незалежних частинок дозволяє пояснити основні риси енергетичного спектра.
Використання варіаційного методу із простими пробними функціями
дає енергії з похибкою лише кілька відсотків від точних значень.

\item
\textbf{Метод Гартрі} узагальнює ідею незалежних електронів,
вводячи поняття середнього електростатичного поля.
Кожен електрон рухається в полі ядра та усередненому полі решти електронів.
Рівняння Гартрі розв’язуються самоузгоджено, поки орбіталі
та потенціал не стабілізуються.

\item
\textbf{Метод Гартрі–Фока} вдосконалює попередній підхід, забезпечуючи
антисиметричність хвильової функції й правильне врахування обмінної
взаємодії.
Він приводить до рівнянь із оператором Фока, що визначає набір
спін-орбіталей — базис для опису електронної структури атомів і молекул.

\item
Попри значну точність, метод Гартрі–Фока не враховує
\textbf{кореляцію електронів}, тобто миттєву узгодженість їхнього руху.
Різниця між точною енергією й енергією Гартрі–Фока визначає
\textbf{кореляційну енергію}.

\item
Для поліпшення результатів розроблено \textbf{методи після Гартрі–Фока}:
метод конфігураційної взаємодії (CI), збурень Меллера–Плессета (MP),
багатоконфігураційні варіаційні підходи (MCHF, CASSCF) тощо.
Вони дозволяють кількісно врахувати як динамічну, так і статичну
кореляцію електронів.

\item
Альтернативний сучасний напрям — \textbf{метод функціонала густини} (DFT),
у якому всі властивості системи визначаються лише електронною густиною
$\rho(\mathbf{r})$.
Завдяки ефективності та відносній простоті, DFT став основним
інструментом обчислювальної квантової хімії.
\end{enumerate}

Таким чином, варіаційний принцип і методи, що з нього випливають,
утворюють основу квантового опису атомної будови речовини.
Вони забезпечують зв’язок між фундаментальними рівняннями квантової
механіки та реальною структурою й властивостями атомів і молекул.

\section*{Контрольні запитання}

\begin{enumerate}
\item Запишіть гамільтоніан $N$-електронного атома та поясніть
фізичний зміст кожного його члена.
\item Чому для багатоелектронних атомів не вдається розділити змінні
в рівнянні Шредінгера?
\item У чому полягає фізичний зміст варіаційного принципу?
\item Сформулюйте основну ідею методу Гартрі та поясніть, що означає
самоузгоджене поле.
\item Чим метод Гартрі–Фока відрізняється від методу Гартрі?
\item Що таке кулонівські й обмінні інтеграли?
\item Як визначається кореляційна енергія?
\item Назвіть основні методи після Гартрі–Фока та коротко охарактеризуйте
їхні особливості.
\item У чому полягає принцип методу функціонала густини?
\item Які наближення використовуються для побудови функціонала
обмінно-кореляційної енергії?
\end{enumerate}

