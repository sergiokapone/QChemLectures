% !TeX program = lualatex
% !TeX encoding = utf8
% !TeX spellcheck = uk_UA
% !BIB program = biber
% !TeX root =../QChemBook.tex
\graphicspath{ {\currfiledir/Pictures/} }

\chapter{Молекули та їх спектри}

Вивчення атомних та молекулярних спектрів є ключем до пізнання їх внутрішньої структури.

Спектри молекул значно складніше лінійних спектрів атомів і мають характерний вигляд. Так, в ультрафіолетовій, видимій та інфрачервоній областях спектри складаються з окремих смуг, які в свою чергу представляють сукупності близько розташованих ліній. У далекій інфрачервоній та мікрохвильовій області спектри молекул складаються з окремих ліній. Такий вид спектрів обумовлений тим, що на відміну від атома в молекулі є три види рухів: електронний (рух електронів в області знаходження ядер), коливальний (коливання ядер відносно їх положень рівноваги) і обертальне (обертання окремих груп атомів і молекули як цілого у просторі).

Класифікацію спектрів можна здійснити за величинами енергії електронних рівнів та переходами між ними.

\begin{enumerate}
\item \emph{Електронні рівні} пов'язані з рухом електронів. Переходи між електронними рівнями внутрішніх електронів (оболонок) з енергіями зв'язку рівним десяткам тисяч еВ дають рентгенівські спектри. У свою чергу, переходи між рівнями зовнішніх електронів (оболонок) в атомах і молекул з енергіями зв'язку порядку декількох еВ дають спектри у видимій та ультрафіолетовій області.
% ---------
\item \emph{Коливальні рівні} молекул пов'язані з коливальними рухами ядер у молекулі при рівноважних положеннях. Частоти коливань лежать в діапазоні від $ 0.025 $~еВ до $ 0.5 $~еВ. Переходи між такими рівнями дають спектри в інфрачервоній області електромагнітного випромінювання.
% ---------
\item \emph{Обертальні рівні} молекул пов'язані з обертальним рухом. Зазвичай для розгляду обертального руху молекули використовують модель твердого тіла, у якій довжини зв'язку в молекулах розглядаються жорсткими та незмінними. Різниці енергій між обертальними рівнями становлять соті частки еВ для легких молекул і стотисячні еВ для важких молекул.
% ---------
\item У молекулах також є розташовані дуже близько електронні рівні, пов'язані з наявністю у електрона спіну. Такі рівні називаються рівнями \emph{тонкою структурою}, а їх енергія становить стотисячні частки еВ для легких атомів та десяті частки для важких атомів та молекул.
% ---------
\item Виділяють також дуже близькі рівні енергії, пов'язані з наявністю у ядер спинів --- \emph{рівні надтонкої структури}. Різниці енергій цих рівнів становлять від десятимільйонних до стотисячних еВ.
% ---------
\item У зовнішньому магнітному полі рівні енергії можуть розщеплюватися. Такі рівні (розщеплені) називаються \emph{рівнями магнітної структури}. Розщеплюються електронні, обертальні рівні та рівні надтонкої структури. Розщеплення електронних рівнів дорівнює десятитисячних часток еВ, обертальних десятимільйонним часткам еВ. Розщеплення рівнів енергій у магнітному полі називають ефектом Зеємана для слабких магнітних полів та ефектом Пашена-Бака для сильних магнітних полів.
% ---------
\item Розщеплюватись рівні можуть і в електричному полі. Такі розщеплені рівні називаються \emph{рівнями електричної структури}. Розщеплюються електронні та
обертальні рівні молекул, які мають дипольний електричний момент. Величина розщеплення електронних рівнів може становити десятитисячні та тисячні еВ, а обертальних мільйонні частки еВ. Розщеплення рівнів енергій у електричному полі називають ефектом Штарка.
\end{enumerate}