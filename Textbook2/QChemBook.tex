% !TeX program = lualatex
% !TeX encoding = utf8
% !TeX spellcheck = uk_UA
% !BIB program = biber

\documentclass[%
biblatex,
%marginversioninfo
]{QchemBook}

%========================================================================================================
%
%									      Вихідні дані докумета
%
%========================================================================================================

\title{Квантова хімія та квантово-механічні обчислення}
\def\subtitle{Конспект лекцій}


%======================================== Бібліографія ===================================================
%
\addbibresource{QchemBook.bib}
%\addbibresource{Kvant.bib}
%
%========================================================================================================

%========================================================================================================
%
% !TeX program = lualatex
% !TeX encoding = utf8
% !TeX spellcheck = uk_UA
% !TeX root =../FTProblems.tex
\def\methodcomdate{}
\def\methodcomnum{}
\def\iptmethodcomdate{}
\def\iptmethodcomnum{}
%========================================================================================================
%
%									      Палітурка
%
%========================================================================================================

\newcommand{\CoverPage}{
	\begin{alwayssingle}
		\begin{center}
			\begin{flushright}\bfseries\sffamily
				\MakeUppercase{Міністерство освіти і науки України}\\
				КПІ ім. Ігоря Сікорського\\
			\end{flushright}
			\begin{tcolorbox}[titlepagestyle,
					toprule=0.10cm,
					bottomrule=0.10cm,
					overlay={%
						\node (picture) at ([xshift=4cm]frame.west) {\includegraphics{logo_PTI}};
					}
			]%
			\begin{flushright}
				\large\bfseries\color{white}Фізико-технічний інститут
			\end{flushright}
			\end{tcolorbox}
			\vspace*{-2em}
		 	\begin{Large}\color{themecolordark!90!black}
			\begin{align*}
                \hat{F}\psi_n &= \epsilon_n\psi_n \\
                \hat{F} &= \hat{h} + \sum\limits_{i = 1}^N (\hat{J}_i - \hat{K}_i)
			\end{align*}
			\end{Large}
			\vspace*{2em}
			\begin{tcolorbox}[
				titlepagestyle,
				toprule=0.15cm,
				bottomrule=0.15cm,
				top=1.3cm,
				bottom=0.7cm,
				overlay={%
				\node[%
							fill=white,
							rounded corners = 15pt,	
							draw=themecolorlight,
							line width=0.15cm,
							inner sep=0pt,
							text width=17cm,
							minimum height=2cm,
							align=center,
							%anchor=east,
							font=\sffamily\bfseries\large
						] (title) at (frame.north) {С.~М. Пономаренко
					};
				}
			]
			\centering
			\Huge\sffamily\bfseries\textcolor{white}{\realtitle}\\
			\huge\sffamily\bfseries\textcolor{white}{\subtitle}
			\end{tcolorbox}	
			\vfill
		 	\begin{Large}\color{themecolordark!90!black}
			\begin{equation*}
                \hat{H}_\mathrm{el} = -\frac12 \sum\limits_{i = 1}^N  \nabla^2_i
                + \frac12 \sum\limits_{\alpha \neq \beta}^K \frac{Z_\alpha Z_\beta}{R_{\alpha\beta}}
                - \sum\limits_{\alpha = 1}^K  \sum\limits_{i = 1}^N \frac{Z_\alpha}{R_{\alpha i}}
                + \frac12 \sum\limits_{i = 1}^N \sum\limits_{j \neq j}^N \frac{1}{r_{ij}} 
			\end{equation*}
			\end{Large}
			\vfill
			\begin{tcolorbox}[titlepagestyle,
					toprule=0.10cm,
					bottomrule=0.10cm]
				\begin{center}\color{white}\bfseries\normalsize
					\MakeUppercase{Київ~\the\year} \\
%					КПІ ім. Ігоря Сікорського \\
%					\the\year
				\end{center}			       	
			\end{tcolorbox}
		\end{center}
		\clearpage
	\end{alwayssingle}
\setcounter{page}{1}	
}


%========================================================================================================
%
%									      Титульна сторінка
%
%========================================================================================================

\renewcommand\maketitle{
	\begin{alwayssingle}
		\begin{center}
				\MakeUppercase{Міністерство освіти і науки України}
				
				\bigskip
				\MakeUppercase{Національний технічний університет України}\\
				<<КИЇВСЬКИЙ ПОЛІТЕХНІЧНИЙ ІНСТИТУТ \\ імені ІГОРЯ СІКОРСЬКОГО>>
				\vspace*{100pt}
		
				{\large С.~М.~Пономаренко}
				\vspace*{50pt}
			
				{\Huge\sffamily\bfseries\realtitle}\\[1em]
				{\huge\sffamily\bfseries\subtitle}	
			
			\vspace*{50pt}
			\begin{center}\itshape
			Рекомендовано Методичною радою КПІ ім. Ігоря Сікорського як навчальний посібник для здобувачів ступеня бакалавра за спеціальностями 105 <<Прикладна фізика та наноматеріали>>
			\end{center}

			\vfill
			\begin{center}
				\MakeUppercase{Київ} \\
				КПІ ім. Ігоря Сікорського \\
				\the\year
			\end{center}			       	
		\end{center}
		\clearpage
	\end{alwayssingle}	
}


%========================================================================================================
%
%									      Друга сторінка
%
%========================================================================================================
\newcommand\makeinfopage{
	\begin{alwayssingle}
		\noindent%	
		\begin{minipage}[t]{0.5\textwidth}
				\begin{flushleft}
					УДК  544.18\\
					П 563
				\end{flushleft}
		\end{minipage}


		\bigskip\noindent%
        \begin{minipage}[t]{0.2\linewidth}
            	\begin{flushleft}
                    Рецензенти:
                \end{flushleft}
        \end{minipage}\hfill
        \begin{minipage}[t]{0.78\linewidth}
%                \href{http://www.nas.gov.ua/UA/PersonalSite/Pages/default.aspx?PersonID=0000006576}{В.~О.~Кочелап}, д.ф.-м.н., професор, член-кореспондент НАН України, завідувач відділу теоретичної фізики Інституту фізики напівпровідників ім. В.~Є. Лашкарьова\\[1ex]
%                \href{http://apd.ipt.kpi.ua/blog/author/19}{Я.~Д.~Кривенко-Еметов}, к.ф.-м.н., доцент кафедри прикладної фізики, Фізико-технічного інституту КПІ ім. Ігоря Сікорського
        \end{minipage}

		\bigskip\noindent%
        \begin{minipage}[t]{0.2\linewidth}
            	\begin{flushleft}
                    Відповідальний редактор:
                \end{flushleft}
        \end{minipage}\hfill
        \begin{minipage}[t]{0.78\linewidth}
                \href{http://ipt.kpi.ua/litvinova}{С.~О.~Смирнов}, к.ф.-м.н., доцент
        \end{minipage}

		\begin{center}\itshape\small
				Гриф надано Методичною радою КПІ ім. Ігоря Сікорського (протокол №~\methodcomnum~від \methodcomdate~р.) за поданням Вченої ради Фізико-технічного інституту (протокол №~\iptmethodcomnum від \iptmethodcomdate~р.)
		\end{center}
		\begin{center}
			\ifelectronic Електронне мережне навчальне видання \fi
			%\par {Версія від~\href{http://www.istpravda.com.ua/dates}{\today}} \par\else \par  \fi
		\end{center}
		\begin{center}
			\href{http://phes.ipt.kpi.ua/ponomarenko-sergij-mikolajovich}{\itshape Пономаренко Сергій Миколайович}, к.ф.-м.н., доцент 
		\end{center}
%			\vspace*{1em}%
		\begin{center}\bfseries
			\LARGE\sffamily\realtitle \\
			\Large\sffamily\subtitle
		\end{center}
        \noindent%
        \begin{minipage}[t]{\textwidth}\small
                \realtitle: \subtitle\ [Електронний ресурс] : навч. посіб. для студ. спеціальностей
                105 <<Прикладна фізика та наноматеріали>> /  С.~М. Пономаренко ; КПІ ім. Ігоря Сікорського.~--- Електронні текстові дані
            (1 файл: 570~кБ). – Київ : КПІ ім. Ігоря Сікорського, \the\year. --- \the\numexpr\getpagerefnumber{LastPage}-1\relax~с.
        \end{minipage}

%		\noindent%
%		\begin{flushleft}
%			\begin{tabular}{lp{0.9\textwidth}}
%				     & В.~І. Жданов, С.~М. Пономаренко, В.~Б. Долгошей                                                                                                                                                                        \\
%				Ж 42 & \hspace*{3ex} \realtitle : \subtitle{} [Електронний ресурс] : навчальний посібник / %
%				В.~І. Жданов, %
%				С.~М. Пономаренко%,
%				В.~Б. Долгошей %
%				--- К.:~КПІ ім. Ігоря Сікорського, \the\year. --~\the\numexpr\getpagerefnumber{LastPage}-1\relax~с. -- Бібліогр.: с.~\pageref{BibPage}\relax. \ifelectronic\relax\else-- 80~прим.\fi
%			\end{tabular}
%		\end{flushleft} 
		\vfill

		Квантова використовує засади квантової механіки для інтерпретації всіх явищ, що протікають в атомах, молекулах та твердих тілах і є  фундаментом теоретичних уявлень сучасної хімії. Методи, розроблені в квантовій хімії є універсальним і застосовується для опису будови речовин та пояснення їх властивостей і в наш час використовуються не лише в хімії, а і в прикладній науці в цілому. 

        Даний навчальний посібник має на меті донесення знань про принципи і методи квантової хімії до фахівця в галузі прикладної фізики, що дасть йому змогу застосовувати їх при створенні речовин і матеріалів з наперед заданими властивостями та у інших наукоємних технологіях. 

		Для студентів фізико-технічного інституту КПІ ім. Ігоря Сікорського, які навчаються за спеціальностями 105~<<Прикладна фізика та наноматеріали>>.
		
		\vfill
				
%		\begin{flushleft}\small
%			Ілюстративний матеріал підручника підготовлений за допомогою пакету \href{http://pgf.sourceforge.net}{TikZ/Pgf}. Верстка тексту проведена в видавничій системі \LaTeXe{} (компілятор Lua\LaTeX) на базі системи комп'ютерної верстки \TeX{} (Збірка  \href{https://www.tug.org/texlive/}{\TeX Live~\the\year}) з використанням оболонки \href{https://www.texstudio.org}{\TeX Studio}.
%		\end{flushleft}	
	\hfill
	\begin{minipage}[t]{0.45\linewidth}\small
        \textcopyright{} С.~М. Пономаренко \the\year\,р. \\
        \textcopyright{} КПІ ім. Ігоря Сікорського (ФТІ), \the\year~р.
    \end{minipage}
		\newpage%
	\end{alwayssingle}
}

%%%%%%%%%%%%%%%%%%%%%%%%%%%%%%%%%%%%%%%%%%%%%%%%%%%%%%%%%%%%%%%%%%%%%%%%%%%%
%%                                                                        %%
%%                              Last Page                                 %%
%%                                                                        %%
%%%%%%%%%%%%%%%%%%%%%%%%%%%%%%%%%%%%%%%%%%%%%%%%%%%%%%%%%%%%%%%%%%%%%%%%%%%% 	
\newcommand{\makelastpage}{%
\clearpage%
\thispagestyle{empty}%
\vspace*{0.4\textheight}
\begin{center}
	    \textbf{Пономаренко} Сергій Миколайович
\end{center}

\begin{center}\bfseries
    \Large\sffamily\realtitle \\
    \large\sffamily\subtitle
\end{center}

\begin{center}\itshape
    Комп'ютерне верстання в системі \LaTeXe{} С.\ М. Пономаренко
\end{center}

\vspace*{1em}
\begin{center}\small
Національний технічний університет України \\
<<Київський політехнічний інститут імені Ігоря Сікорського>> \\
Свідоцтво про державну реєстрацію: серія ДК № 5354 від 25.05.2017 р.\\
просп. Перемоги, 37, м. Київ, 03056
\end{center}

%\vfill
%
%\begin{center}
%    Видавництво <<Політехніка>> \kpishort \\
%    вул. Політехнічна, 14, корп. 15 \\
%    м. Київ, 03056 \\
%    Тел. (044) 204-81-78
%\end{center}
}	


%					      Титульна сторінка
%
%========================================================================================================

\begin{document}
\pagestyle{empty}
%\ifelectronic\CoverPage\setcounter{page}{0}\fi
%\CoverPage
\pagenumbering{arabic}
%\maketitle
%\makeinfopage
%========================================================================================================
%
\clearpage\pagestyle{plain}
\tableofcontents%				               Зміст
%
%========================================================================================================

%========================================================================================================
%
\chapter*{Передмова}
\addcontentsline{toc}{chapter}{Передмова}

Цей методичний посібник є практичним введенням у квантово-хімічні розрахунки атомних систем з використанням програмного пакету PySCF (Python-based Simulations of Chemistry Framework). Посібник призначений для студентів хімічних, фізичних та матеріалознавчих спеціальностей, які вивчають квантову хімію, обчислювальну хімію та суміжні дисципліни.

\section*{Передумови}

Для успішного опанування матеріалу посібника бажано мати:

\paragraph{Обов'язкові знання:}
\begin{itemize}
    \item Базові знання квантової механіки (хвильова функція, оператори, власні значення)
    \item Основи хімії (періодична система, електронні конфігурації, хімічний зв'язок)
    \item Елементарне програмування на Python (змінні, цикли, функції)
    \item Робота з командним рядком (термінал/консоль)
\end{itemize}

\paragraph{Бажано знати:}
\begin{itemize}
    \item Лінійну алгебру (матриці, власні вектори, діагоналізація)
    \item Основи чисельних методів
    \item Роботу з NumPy та Matplotlib
    \item Jupyter Notebook
\end{itemize}



\paragraph{Кожен розділ містить:}
\begin{itemize}
    \item Теоретичне пояснення методу
    \item Детальні приклади коду з коментарями
    \item Практичні завдання для самостійної роботи
    \item Контрольні запитання
    \item Посилання на додаткову літературу
\end{itemize}

\section*{Організація навчальних матеріалів}

Всі коди, що наведені в посібнику, організовані в окремих папках відповідно до структури розділів:

\begin{verbatim}
pyscf_atomic_tutorial/
├── chapter_02/
│   ├── 01_installation.py
│   ├── 02_basic_structure.py
│   └── examples/
├── chapter_03/
│   ├── 01_mole_object.py
│   ├── 02_basis_sets.py
│   └── examples/
├── chapter_04/
│   ├── 01_hf_hydrogen.py
│   ├── 02_hf_helium.py
│   ├── 03_second_period.py
│   └── examples/
├── chapter_05/
│   ├── 01_dft_basics.py
│   ├── 02_functionals.py
│   └── examples/
├── chapter_06/
│   ├── 01_mp2.py
│   ├── 02_ccsd.py
│   ├── 03_casscf.py
│   └── examples/
├── chapter_07/
│   ├── 01_ionization_energies.py
│   └── examples/
├── notebooks/
│   ├── Chapter_02_Interactive.ipynb
│   ├── Chapter_03_Interactive.ipynb
│   └── ...
└── README.md
\end{verbatim}

\subsection*{Робота з кодом}

Код з посібника можна використовувати кількома способами:

\paragraph{1. Jupyter Notebook (рекомендовано для навчання)}

Jupyter Notebook --- інтерактивне середовище, ідеальне для навчання та експериментування:

\begin{minted}{bash}
# Встановлення Jupyter
pip install jupyter

# Запуск
cd pyscf_atomic_tutorial/notebooks
jupyter notebook
\end{minted}

\textbf{Переваги Jupyter:}
\begin{itemize}
    \item Виконання коду по частинах (комірками)
    \item Миттєвий перегляд результатів
    \item Збереження графіків безпосередньо в notebook
    \item Можливість додавати власні нотатки
    \item Легко ділитися з колегами
\end{itemize}

\paragraph{2. Окремі Python скрипти}

Всі приклади можна виконувати як звичайні Python скрипти:

\begin{minted}{bash}
# Виконання окремого скрипта
python chapter_04/01_hf_hydrogen.py

# Або з додатковими опціями
python -u chapter_04/02_hf_helium.py > output.log
\end{minted}

Це зручно для:
\begin{itemize}
    \item Автоматизації серії розрахунків
    \item Виконання на кластерах та серверах
    \item Інтеграції у власні pipeline
\end{itemize}

\paragraph{3. Інтерактивний Python (iPython)}

Для швидких тестів та експериментів:

\begin{minted}{python}
ipython
>>> from pyscf import gto, scf
>>> mol = gto.M(atom='H 0 0 0', basis='sto-3g', spin=1)
>>> mf = scf.UHF(mol)
>>> mf.kernel()
\end{minted}

\paragraph{4. IDE (PyCharm, VS Code, Spyder)}

Всі приклади сумісні з популярними IDE. Рекомендуємо налаштувати:
\begin{itemize}
    \item Автодоповнення для PySCF
    \item Відлагодження (debugging)
    \item Інтеграцію з Git для контролю версій
\end{itemize}

\section*{Вимоги до системи}

\subsection*{Мінімальні вимоги:}
\begin{itemize}
    \item \textbf{ОС:} Linux, macOS, або Windows 10/11
    \item \textbf{Python:} версія 3.7 або новіша
    \item \textbf{Оперативна пам'ять:} 4 ГБ (8 ГБ рекомендовано)
    \item \textbf{Вільне місце:} 2 ГБ
    \item \textbf{Процесор:} будь-який сучасний (багатоядерний краще)
\end{itemize}

\subsection*{Рекомендовані вимоги:}
\begin{itemize}
    \item \textbf{Оперативна пам'ять:} 16+ ГБ
    \item \textbf{Процесор:} 4+ ядра
    \item \textbf{SSD:} для швидкого читання/запису великих файлів
\end{itemize}

\textbf{Примітка:} Складні розрахунки (CCSD, CASSCF для великих систем) можуть потребувати значних ресурсів. Для навчальних прикладів з посібника достатньо звичайного ноутбука.

\section*{Встановлення програмного забезпечення}

\subsection*{Швидкий старт (Linux/macOS):}

\begin{minted}{bash}
# Створення віртуального середовища (рекомендовано)
python3 -m venv pyscf_env
source pyscf_env/bin/activate

# Встановлення PySCF та залежностей
pip install --upgrade pip
pip install pyscf
pip install numpy scipy matplotlib jupyter

# Перевірка встановлення
python -c "import pyscf; print(pyscf.__version__)"
\end{minted}

\subsection*{Швидкий старт (Windows):}

\begin{minted}{bash}
# Створення віртуального середовища
python -m venv pyscf_env
pyscf_env\Scripts\activate

# Встановлення
pip install --upgrade pip
pip install pyscf numpy scipy matplotlib jupyter

# Перевірка
python -c "import pyscf; print(pyscf.__version__)"
\end{minted}

Детальні інструкції з встановлення наведені в Розділі 2.

\section*{Як користуватися цим посібником}

\subsection*{Для самостійного навчання:}

\begin{enumerate}
    \item \textbf{Читайте послідовно} --- матеріал побудований від простого до складного
    \item \textbf{Виконуйте всі приклади} --- просте читання коду не замінить практики
    \item \textbf{Експериментуйте} --- змінюйте параметри, пробуйте інші атоми
    \item \textbf{Виконуйте завдання} --- вони закріплюють матеріал
    \item \textbf{Веніть до складних місць} --- деякі концепції потребують часу
\end{enumerate}


\section*{Зворотний зв'язок}

Ваші коментарі та пропозиції допоможуть покращити цей посібник! Якщо ви знайшли:
\begin{itemize}
    \item Помилки в коді або тексті
    \item Незрозумілі пояснення
    \item Теми, які потребують більш детального розгляду
    \item Інші проблеми
\end{itemize}

Будь ласка, повідомте про це через:
\begin{itemize}
    \item GitHub Issues (якщо матеріал на GitHub)
    \item Email викладачу/автору
    \item Форум курсу
\end{itemize}

\section*{Ліцензія та використання}

Цей методичний посібник розповсюджується для освітніх цілей. Ви можете:
\begin{itemize}
    \item Використовувати матеріали для навчання
    \item Модифікувати приклади коду для власних потреб
    \item Ділитися посібником з колегами та студентами
\end{itemize}

При використанні матеріалів просимо посилатися на цей посібник.


\vspace{1cm}

\begin{center}
\textit{Бажаємо успіхів у вивченні квантової хімії!}
\end{center}

\vspace{0.5cm}

\begin{flushright}
С. М. Пономаренко\\
\today
\end{flushright}

\clearpage%                          Вступ та передмова
%
%========================================================================================================

%========================================================================================================
%
%									      Вставка файлів розділів
%
%========================================================================================================

\newcommand{\ChaptersOne}{
    ManyElectronsAtom,
    %MolecularSpectra,
    %AdiabaticApprox,
    %HartreeFockMethod,
    %EMstatics,
    %Radiation,
    %SpecialRelativity
}
%\newcommand{\ChaptersTwo}{
%	EMstaticsInMedia,
%	QuasiStationarMedia,
%	WawesInLinearMedia
%}

\pagestyle{main}

%\part{Мікроскопічна теорія}
%\epigraph{\Annabelle  Неможливо позбутися відчуття, що ці математичні формули існують незалежно від нас і володіють власним розумом, що вони мудріші за нас, мудріше навіть тих, хто їх відкрив, і що ми дістаємо з них більше, ніж спочатку було закладено\ldots}{Heinrich Hertz}
%
%\input{Additions/MaxwellEquation}
\multiinclude{\ChaptersOne}[]
%
%\part{Електродинаміка суцільних середовищ}
%\epigraph{\Annabelle  Вважаю, що більш приземлені та матеріальні науки аж ніяк не можуть бути зневажені у порівнянні з піднесеним вивченням розуму і духу \ldots}{James Clerk Maxwell}
%\input{Additions/MaxwellEquationMacro}
%\multiinclude{\ChaptersTwo}[]

%========================================================================================================
%
%									       Вставка відповідей
%
%========================================================================================================
%\bookmarksetup{startatroot}
%\answers
%\multiinclude{\ChaptersOne}[-Answers]
%\multiinclude{\ChaptersTwo}[-Answers]
%========================================================================================================
%
%									             Додатки
%
%========================================================================================================
%\appendix


%%========================================================================================================
%
%									       Вставка бібліографії
%
%=========================================================================================================
\clearpage\pagestyle{bibliography}

\nocite{*}
\printbibliography[title={Література}]

\makelastpage
\end{document}
