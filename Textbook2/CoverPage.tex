% !TeX program = lualatex
% !TeX encoding = utf8
% !TeX spellcheck = uk_UA
% !TeX root =./QChemBook.tex

%===============================================================================
%
%									      Палітурка
%
%===============================================================================

\def\curryear{\the\year}
\newcommand{\CoverPage}{
	\begin{alwayssingle}
		\begin{center}
			\begin{flushright}\bfseries\sffamily
				\MakeUppercase{Міністерство освіти і науки України}\\
				КПІ ім. Ігоря Сікорського\\
			\end{flushright}
			\begin{tcolorbox}[titlepagestyle,
					toprule=0.10cm,
					bottomrule=0.10cm,
%					overlay={%
%						\node (picture) at ([xshift=2.5cm]frame.west) {\includegraphics[width=4cm]{logo_IPT}};
%					}
			]%
			\begin{flushright}
				\large\bfseries\color{white}Навчально-науковий фізико-технічний інститут
			\end{flushright}
			\end{tcolorbox}
			\vspace*{-2em}
		 	\begin{Large}\color{themecolordark!90!black}
			\begin{alignat*}{3}
				\vect{\nabla}&\times&&\Efield &&= -\dfrac{1}{c}\dfrac{\partial\Bfield}{\partial t} \\
				\vect{\nabla}&\,\cdot&&\Bfield &&= 0 \\
				\vect{\nabla}&\,\cdot&&\Dfield &&= 4\pi\rho \\
				\vect{\nabla}&\times&&\Hfield &&= \dfrac{4\pi}{c} \vect{j}+\dfrac{1}{c}\dfrac{\partial\Dfield}{\partial t}
			\end{alignat*}
			\end{Large}
			\vspace*{2em}
			\begin{tcolorbox}[
				titlepagestyle,
				toprule=0.15cm,
				bottomrule=0.15cm,
				top=1.3cm,
				bottom=0.7cm,
				overlay={%
				\node[%
							fill=white,
							rounded corners = 15pt,
							draw=themecolorlight,
							line width=0.15cm,
							inner sep=0pt,
							text width=8cm,
							minimum height=2cm,
							align=center,
							%anchor=east,
							font=\sffamily\bfseries\Large
						] (title) at (frame.north) {\authors};
				}
			]
			\centering
			\Huge\sffamily\bfseries\textcolor{white}{\realtitle}\\
			\huge\sffamily\bfseries\textcolor{white}{\subtitle}
			\end{tcolorbox}
			\vfill
		 	\begin{Large}\color{themecolordark!90!black}
			\begin{alignat*}{4}
					&\oiint\limits_S \Dfield\cdot d\vect{S} = 4\pi\iiint\limits_V\rho dV  \quad
					&&\oint\limits_L \Efield\cdot d\vect{r} &&&= - \frac1c \iint\limits_S \frac{\partial\Bfield}{\partial t}\cdot d\vect{S} \\
					&\oiint\limits_S \Bfield\cdot d\vect{S} = 0
					 \quad
					&&\oint\limits_L \Hfield\cdot d\vect{r} &&& =\dfrac{4\pi}{c} \iint\limits_S \vect{j}\cdot d\vect{S} \,+ \\
					& && &&& +\,\frac{1}{c} \iint\limits_S  \frac{\partial\Dfield}{\partial t}\cdot d\vect{S}
			\end{alignat*}
			\end{Large}
			\vfill
			\begin{tcolorbox}[titlepagestyle,
					toprule=0.10cm,
					bottomrule=0.10cm]
				\begin{center}\color{white}\bfseries\normalsize
					\MakeUppercase{Київ~\curryear} \\
%					КПІ ім. Ігоря Сікорського \\
%					2021
				\end{center}
			\end{tcolorbox}
		\end{center}
		\clearpage
	\end{alwayssingle}
\setcounter{page}{1}
}


%===============================================================================
%
%									      Титульна сторінка
%
%===============================================================================

\renewcommand\maketitle{
	\begin{alwayssingle}
		\begin{center}
				\MakeUppercase{Міністерство освіти і науки України}

				\bigskip
				\MakeUppercase{Національний технічний університет України}\\
				<<КИЇВСЬКИЙ ПОЛІТЕХНІЧНИЙ ІНСТИТУТ \\ імені ІГОРЯ СІКОРСЬКОГО>>
				\vspace*{100pt}

				{\Large\authors}
				\vspace*{50pt}

				{\Huge\sffamily\bfseries\realtitle}\\
				{\huge\sffamily\bfseries\subtitle}

			\vspace*{50pt}
			\begin{center}\itshape

				Рекомендовано Методичною радою КПІ ім. Ігоря Сікорського як навчальний посібник для здобувачів ступеня бакалавра за освітньою програмою
				<<Прикладна фізика>> спеціальності 105 <<Прикладна фізика та наноматеріали>>
			\end{center}

			\vfill
			\begin{center}
				\MakeUppercase{Київ} \\
				КПІ ім. Ігоря Сікорського \\
				\curryear
			\end{center}
		\end{center}
		\clearpage
	\end{alwayssingle}
}


%========================================================================================================
%
%									      Друга сторінка
%
%========================================================================================================
\newcommand\makeinfopage{
	\begin{alwayssingle}
		\noindent%
		\begin{minipage}[t]{0.5\textwidth}
				\begin{flushleft}
					УДК  537\\
					ББК  22.3\\
					П 563
				\end{flushleft}
		\end{minipage}

%		\begin{minipage}[t]{0.55\textwidth}
%				\begin{flushleft}
%					Гриф надано Методичною радою КПІ ім. Ігоря Сікорського (протокол № 9/2019~від 30.05.2019~р.) за поданням Вченої ради Фізико-технічного інституту (протокол № 5/2019 від 24.04.2019 р.)
%				\end{flushleft}
%		\end{minipage}
		\bigskip\noindent%
		\begin{tabular}[t]{p{4.2cm}p{0.7\textwidth}}
	%				Автор та укладач:                        & \href{http://phes.ipt.kpi.ua/ponomarenko-sergij-mikolajovich}{\authors}, к.ф.-м.н., доцент кафедри  фізики енергетичних систем ФТІ \\
	%				                                         &                                                                                                                                   \\
			Рецензенти:                              & \href{http://imfn.lviv.ua/zf/?page_id=569}{І.~Р.~Зачек}, д.ф.-м.н., професор  кафедри загальної фізики Інституту прикладної математики та фундаментальних наук Національного університету <<Львівська політехніка>>                            \\
											 & \href{https://www.nas.gov.ua/UA/PersonalSite/Pages/default.aspx?PersonID=0000006279}{О.~М.~Кордюк}, д.ф.-м.н., член-кореспондент НАН України, директор Київського академічного університету                            \\
	                                     &                                                                                                                                   \\
			Відповідальний редактор: & \href{http://ipt.kpi.ua/litvinova}{С.~А.~Смирнов}, к.ф.-м..н., доцент, голова методичної ради ФТІ
		\end{tabular}
		\vfill
		\begin{center}\itshape\small
				Гриф надано Методичною радою КПІ ім. Ігоря Сікорського (протокол №~9/2019~від 30.05.2019~р.) за поданням Вченої ради Фізико-технічного інституту (протокол №~5/2019 від 24.04.2019 р.)
		\end{center}
		\begin{center}
			\ifelectronic Електронне мережне видання \fi
			%\par {Версія від~\href{http://www.istpravda.com.ua/dates}{\today}} \par\else \par  \fi
		\end{center}
			\vfill
		\begin{center}
			\href{http://phes.ipt.kpi.ua/ponomarenko-sergij-mikolajovich}{\itshape Пономаренко Сергій Миколайович}, к.ф.-м.н., доцент
		\end{center}
			\vfill
		\begin{center}
			\LARGE\sffamily\realtitle \\
			\Large\sffamily\subtitle
		\end{center}
			\vfill
		\bigskip\noindent%
		\begin{flushleft}
			\begin{tabular}{lp{0.9\textwidth}}
				     & \textbf{\href{http://phes.ipt.kpi.ua/ponomarenko-sergij-mikolajovich}{\authors}}                                                                                                                                                                        \\
				П 563 & \hspace*{3ex} \realtitle : \subtitle{} [Електронний ресурс] : навчальний посібник / \authors{} --- К.:~КПІ ім. Ігоря
				Сікорського, \curryear. --~\the\numexpr\getpagerefnumber{LastPage}-1\relax~с. -- Бібліогр.: с.~\pageref{BibPage}.
				\ifelectronic\relax\else-- 80~прим.\fi
			\end{tabular}
		\end{flushleft}
		\vfill

        Представлено конспект лекцій з освітньої компоненти <<Електрика та магнетизм>>, у якому розглядаються фізичні та математичні
        підґрунтя теорії електромагнітного поля, основні визначення, рівняння та
        характеристики електромагнітного поля.

		Для студентів фізико-технічного інституту КПІ ім. Ігоря Сікорського, які навчаються за спеціальністю 105~<<Прикладна фізика та наноматеріали>>.

		\vfill

%		\begin{flushleft}\small
%			Ілюстративний матеріал підручника підготовлений за допомогою пакету \href{http://pgf.sourceforge.net}{TikZ/Pgf}. Верстка тексту проведена в видавничій системі \LaTeXe{} (компілятор Lua\LaTeX) на базі системи комп'ютерної верстки \TeX{} (Збірка  \href{https://www.tug.org/texlive/}{\TeX Live~2021}) з використанням оболонки \href{https://www.texstudio.org}{\TeX Studio}.
%		\end{flushleft}
%		\vfill

		\begin{flushleft}
			\begin{tabular}{p{\textwidth - 45ex}l}
				& \small\textcopyright\quad \authors, \curryear~р. \\
				& \small \textcopyright\quad КПІ ім. Ігоря Сікорського (ФТІ), \curryear~р.
			\end{tabular}
		\end{flushleft}
		\newpage%
	\end{alwayssingle}
}

%%%%%%%%%%%%%%%%%%%%%%%%%%%%%%%%%%%%%%%%%%%%%%%%%%%%%%%%%%%%%%%%%%%%%%%%%%%%
%%                                                                        %%
%%                              Last Page                                 %%
%%                                                                        %%
%%%%%%%%%%%%%%%%%%%%%%%%%%%%%%%%%%%%%%%%%%%%%%%%%%%%%%%%%%%%%%%%%%%%%%%%%%%%
\newcommand{\makelastpage}{%
\clearpage%
\thispagestyle{empty}%
\vspace*{0.4\textheight}
\begin{center}
	    \textbf{Пономаренко} Сергій Миколайович
\end{center}

\begin{center}\bfseries
    \Large\sffamily\realtitle \\
    \large\sffamily\subtitle
\end{center}

\begin{center}\itshape
    Комп'ютерне верстання в системі \LaTeXe{} С.\ М. Пономаренко
\end{center}

\vspace*{1em}
\begin{center}\small
Національний технічний університет України \\
<<Київський політехнічний інститут імені Ігоря Сікорського>> \\
Свідоцтво про державну реєстрацію: серія ДК № 5354 від 25.05.2017 р.\\
просп. Перемоги, 37, м. Київ, 03056
\end{center}

%\vfill
%
%\begin{center}
%    Видавництво <<Політехніка>> \kpishort \\
%    вул. Політехнічна, 14, корп. 15 \\
%    м. Київ, 03056 \\
%    Тел. (044) 204-81-78
%\end{center}
}






