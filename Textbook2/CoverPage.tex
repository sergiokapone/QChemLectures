% !TeX program = lualatex
% !TeX encoding = utf8
% !TeX spellcheck = uk_UA
% !TeX root =../FTProblems.tex
\def\methodcomdate{}
\def\methodcomnum{}
\def\iptmethodcomdate{}
\def\iptmethodcomnum{}
%========================================================================================================
%
%									      Палітурка
%
%========================================================================================================

\newcommand{\CoverPage}{
	\begin{alwayssingle}
		\begin{center}
			\begin{flushright}\bfseries\sffamily
				\MakeUppercase{Міністерство освіти і науки України}\\
				КПІ ім. Ігоря Сікорського\\
			\end{flushright}
			\begin{tcolorbox}[titlepagestyle,
					toprule=0.10cm,
					bottomrule=0.10cm,
					overlay={%
						\node (picture) at ([xshift=4cm]frame.west) {\includegraphics{logo_PTI}};
					}
			]%
			\begin{flushright}
				\large\bfseries\color{white}Фізико-технічний інститут
			\end{flushright}
			\end{tcolorbox}
			\vspace*{-2em}
		 	\begin{Large}\color{themecolordark!90!black}
			\begin{align*}
                \hat{F}\psi_n &= \epsilon_n\psi_n \\
                \hat{F} &= \hat{h} + \sum\limits_{i = 1}^N (\hat{J}_i - \hat{K}_i)
			\end{align*}
			\end{Large}
			\vspace*{2em}
			\begin{tcolorbox}[
				titlepagestyle,
				toprule=0.15cm,
				bottomrule=0.15cm,
				top=1.3cm,
				bottom=0.7cm,
				overlay={%
				\node[%
							fill=white,
							rounded corners = 15pt,	
							draw=themecolorlight,
							line width=0.15cm,
							inner sep=0pt,
							text width=17cm,
							minimum height=2cm,
							align=center,
							%anchor=east,
							font=\sffamily\bfseries\large
						] (title) at (frame.north) {С.~М. Пономаренко
					};
				}
			]
			\centering
			\Huge\sffamily\bfseries\textcolor{white}{\realtitle}\\
			\huge\sffamily\bfseries\textcolor{white}{\subtitle}
			\end{tcolorbox}	
			\vfill
		 	\begin{Large}\color{themecolordark!90!black}
			\begin{equation*}
                \hat{H}_\mathrm{el} = -\frac12 \sum\limits_{i = 1}^N  \nabla^2_i
                + \frac12 \sum\limits_{\alpha \neq \beta}^K \frac{Z_\alpha Z_\beta}{R_{\alpha\beta}}
                - \sum\limits_{\alpha = 1}^K  \sum\limits_{i = 1}^N \frac{Z_\alpha}{R_{\alpha i}}
                + \frac12 \sum\limits_{i = 1}^N \sum\limits_{j \neq j}^N \frac{1}{r_{ij}} 
			\end{equation*}
			\end{Large}
			\vfill
			\begin{tcolorbox}[titlepagestyle,
					toprule=0.10cm,
					bottomrule=0.10cm]
				\begin{center}\color{white}\bfseries\normalsize
					\MakeUppercase{Київ~\the\year} \\
%					КПІ ім. Ігоря Сікорського \\
%					\the\year
				\end{center}			       	
			\end{tcolorbox}
		\end{center}
		\clearpage
	\end{alwayssingle}
\setcounter{page}{1}	
}


%========================================================================================================
%
%									      Титульна сторінка
%
%========================================================================================================

\renewcommand\maketitle{
	\begin{alwayssingle}
		\begin{center}
				\MakeUppercase{Міністерство освіти і науки України}
				
				\bigskip
				\MakeUppercase{Національний технічний університет України}\\
				<<КИЇВСЬКИЙ ПОЛІТЕХНІЧНИЙ ІНСТИТУТ \\ імені ІГОРЯ СІКОРСЬКОГО>>
				\vspace*{100pt}
		
				{\large С.~М.~Пономаренко}
				\vspace*{50pt}
			
				{\Huge\sffamily\bfseries\realtitle}\\[1em]
				{\huge\sffamily\bfseries\subtitle}	
			
			\vspace*{50pt}
			\begin{center}\itshape
			Рекомендовано Методичною радою КПІ ім. Ігоря Сікорського як навчальний посібник для здобувачів ступеня бакалавра за спеціальностями 105 <<Прикладна фізика та наноматеріали>>
			\end{center}

			\vfill
			\begin{center}
				\MakeUppercase{Київ} \\
				КПІ ім. Ігоря Сікорського \\
				\the\year
			\end{center}			       	
		\end{center}
		\clearpage
	\end{alwayssingle}	
}


%========================================================================================================
%
%									      Друга сторінка
%
%========================================================================================================
\newcommand\makeinfopage{
	\begin{alwayssingle}
		\noindent%	
		\begin{minipage}[t]{0.5\textwidth}
				\begin{flushleft}
					УДК  544.18\\
					П 563
				\end{flushleft}
		\end{minipage}


		\bigskip\noindent%
        \begin{minipage}[t]{0.2\linewidth}
            	\begin{flushleft}
                    Рецензенти:
                \end{flushleft}
        \end{minipage}\hfill
        \begin{minipage}[t]{0.78\linewidth}
%                \href{http://www.nas.gov.ua/UA/PersonalSite/Pages/default.aspx?PersonID=0000006576}{В.~О.~Кочелап}, д.ф.-м.н., професор, член-кореспондент НАН України, завідувач відділу теоретичної фізики Інституту фізики напівпровідників ім. В.~Є. Лашкарьова\\[1ex]
%                \href{http://apd.ipt.kpi.ua/blog/author/19}{Я.~Д.~Кривенко-Еметов}, к.ф.-м.н., доцент кафедри прикладної фізики, Фізико-технічного інституту КПІ ім. Ігоря Сікорського
        \end{minipage}

		\bigskip\noindent%
        \begin{minipage}[t]{0.2\linewidth}
            	\begin{flushleft}
                    Відповідальний редактор:
                \end{flushleft}
        \end{minipage}\hfill
        \begin{minipage}[t]{0.78\linewidth}
                \href{http://ipt.kpi.ua/litvinova}{С.~О.~Смирнов}, к.ф.-м.н., доцент
        \end{minipage}

		\begin{center}\itshape\small
				Гриф надано Методичною радою КПІ ім. Ігоря Сікорського (протокол №~\methodcomnum~від \methodcomdate~р.) за поданням Вченої ради Фізико-технічного інституту (протокол №~\iptmethodcomnum від \iptmethodcomdate~р.)
		\end{center}
		\begin{center}
			\ifelectronic Електронне мережне навчальне видання \fi
			%\par {Версія від~\href{http://www.istpravda.com.ua/dates}{\today}} \par\else \par  \fi
		\end{center}
		\begin{center}
			\href{http://phes.ipt.kpi.ua/ponomarenko-sergij-mikolajovich}{\itshape Пономаренко Сергій Миколайович}, к.ф.-м.н., доцент 
		\end{center}
%			\vspace*{1em}%
		\begin{center}\bfseries
			\LARGE\sffamily\realtitle \\
			\Large\sffamily\subtitle
		\end{center}
        \noindent%
        \begin{minipage}[t]{\textwidth}\small
                \realtitle: \subtitle\ [Електронний ресурс] : навч. посіб. для студ. спеціальностей
                105 <<Прикладна фізика та наноматеріали>> /  С.~М. Пономаренко ; КПІ ім. Ігоря Сікорського.~--- Електронні текстові дані
            (1 файл: 570~кБ). – Київ : КПІ ім. Ігоря Сікорського, \the\year. --- \the\numexpr\getpagerefnumber{LastPage}-1\relax~с.
        \end{minipage}

%		\noindent%
%		\begin{flushleft}
%			\begin{tabular}{lp{0.9\textwidth}}
%				     & В.~І. Жданов, С.~М. Пономаренко, В.~Б. Долгошей                                                                                                                                                                        \\
%				Ж 42 & \hspace*{3ex} \realtitle : \subtitle{} [Електронний ресурс] : навчальний посібник / %
%				В.~І. Жданов, %
%				С.~М. Пономаренко%,
%				В.~Б. Долгошей %
%				--- К.:~КПІ ім. Ігоря Сікорського, \the\year. --~\the\numexpr\getpagerefnumber{LastPage}-1\relax~с. -- Бібліогр.: с.~\pageref{BibPage}\relax. \ifelectronic\relax\else-- 80~прим.\fi
%			\end{tabular}
%		\end{flushleft} 
		\vfill

		Квантова використовує засади квантової механіки для інтерпретації всіх явищ, що протікають в атомах, молекулах та твердих тілах і є  фундаментом теоретичних уявлень сучасної хімії. Методи, розроблені в квантовій хімії є універсальним і застосовується для опису будови речовин та пояснення їх властивостей і в наш час використовуються не лише в хімії, а і в прикладній науці в цілому. 

        Даний навчальний посібник має на меті донесення знань про принципи і методи квантової хімії до фахівця в галузі прикладної фізики, що дасть йому змогу застосовувати їх при створенні речовин і матеріалів з наперед заданими властивостями та у інших наукоємних технологіях. 

		Для студентів фізико-технічного інституту КПІ ім. Ігоря Сікорського, які навчаються за спеціальностями 105~<<Прикладна фізика та наноматеріали>>.
		
		\vfill
				
%		\begin{flushleft}\small
%			Ілюстративний матеріал підручника підготовлений за допомогою пакету \href{http://pgf.sourceforge.net}{TikZ/Pgf}. Верстка тексту проведена в видавничій системі \LaTeXe{} (компілятор Lua\LaTeX) на базі системи комп'ютерної верстки \TeX{} (Збірка  \href{https://www.tug.org/texlive/}{\TeX Live~\the\year}) з використанням оболонки \href{https://www.texstudio.org}{\TeX Studio}.
%		\end{flushleft}	
	\hfill
	\begin{minipage}[t]{0.45\linewidth}\small
        \textcopyright{} С.~М. Пономаренко \the\year\,р. \\
        \textcopyright{} КПІ ім. Ігоря Сікорського (ФТІ), \the\year~р.
    \end{minipage}
		\newpage%
	\end{alwayssingle}
}

%%%%%%%%%%%%%%%%%%%%%%%%%%%%%%%%%%%%%%%%%%%%%%%%%%%%%%%%%%%%%%%%%%%%%%%%%%%%
%%                                                                        %%
%%                              Last Page                                 %%
%%                                                                        %%
%%%%%%%%%%%%%%%%%%%%%%%%%%%%%%%%%%%%%%%%%%%%%%%%%%%%%%%%%%%%%%%%%%%%%%%%%%%% 	
\newcommand{\makelastpage}{%
\clearpage%
\thispagestyle{empty}%
\vspace*{0.4\textheight}
\begin{center}
	    \textbf{Пономаренко} Сергій Миколайович
\end{center}

\begin{center}\bfseries
    \Large\sffamily\realtitle \\
    \large\sffamily\subtitle
\end{center}

\begin{center}\itshape
    Комп'ютерне верстання в системі \LaTeXe{} С.\ М. Пономаренко
\end{center}

\vspace*{1em}
\begin{center}\small
Національний технічний університет України \\
<<Київський політехнічний інститут імені Ігоря Сікорського>> \\
Свідоцтво про державну реєстрацію: серія ДК № 5354 від 25.05.2017 р.\\
просп. Перемоги, 37, м. Київ, 03056
\end{center}

%\vfill
%
%\begin{center}
%    Видавництво <<Політехніка>> \kpishort \\
%    вул. Політехнічна, 14, корп. 15 \\
%    м. Київ, 03056 \\
%    Тел. (044) 204-81-78
%\end{center}
}	


