%%============================ Compiler Directives =======================%%
%%                                                                        %%
% !TeX program = lualatex							    	
% !TeX encoding = utf8
% !TeX spellcheck = uk_UA
%%                                                                        %%
%%============================== Клас документа ==========================%%
%%                                                                        %%
\documentclass[14pt]{extarticle}
\IfFileExists{ukrcorr.sty}{\usepackage{ukrcorr}}{}
\usepackage{ifluatex}
%%                                                                        %%
%%========================== Мови, шрифти та кодування ===================%%
%%
\ifluatex                                                                        %%
	\usepackage{fontspec}
	\setsansfont{CMU Sans Serif}%{Arial}
	\setmainfont{CMU Serif}%{Times New Roman}
	\setmonofont{CMU Typewriter Text}%{Consolas}
	\defaultfontfeatures{Ligatures={TeX}}
	\usepackage[math-style=TeX]{unicode-math}
\else
	\usepackage[utf8]{inputenc}
	\usepackage[T2A,T1]{fontenc}
	\usepackage{amsmath}
	%\usepackage{pscyr}
	\usepackage{cmap}
\fi
\usepackage[english, russian, ukrainian]{babel}
%%                                                                        %%
%%============================= Геометрія сторінки =======================%%
%%                                                                        %%
\usepackage[%
	a4paper,%
	footskip=1cm,%
	headsep=0.3cm,% 
	top=2cm, %поле сверху
	bottom=2cm, %поле снизу
	left=2cm, %поле ліворуч
	right=2cm, %поле праворуч
    ]{geometry}                          
%%                                                                        %%        
%%============================== Інтерліньяж  ============================%%
%%                                                                        %%
\renewcommand{\baselinestretch}{1}
%-------------------------  Подавление висячих строк  --------------------%%
\clubpenalty =10000
\widowpenalty=10000
%---------------------------------Інтервали-------------------------------%%
\setlength{\parskip}{0.5ex}%
\setlength{\parindent}{2.5em}%
%%                                                                        %%
%%=========================== Математичні пакети і графіка ===============%%
%%                                                                        %%
\usepackage{amsmath}
\usepackage{graphicx}
\usepackage{tikz}
\usetikzlibrary{calc}
\usetikzlibrary{shapes.geometric}
\usetikzlibrary{arrows.meta}
\usetikzlibrary{positioning}
%%                                                                        %%
%%========================== Гіперпосилення (href) =======================%%
%%                                                                        %%
\usepackage[%colorlinks=true,
	%urlcolor = blue, %Colour for external hyperlinks
	%linkcolor  = malina, %Colour of internal links
	%citecolor  = green, %Colour of citations
	bookmarks = true,
	bookmarksnumbered=true,
	unicode,
	linktoc = all,
	hypertexnames=false,
	pdftoolbar=false,
	pdfpagelayout=TwoPageRight,
	pdfauthor={Ponomarenko S.M. aka sergiokapone},
	pdfdisplaydoctitle=true,
	pdfencoding=auto
	]%
	{hyperref}
		\makeatletter
	\AtBeginDocument{
	\hypersetup{
		pdfinfo={
		Title={\@title},
		}
	}
	}
	\makeatother 
%%                                                                        %%
%%============================ Заголовок та автори =======================%%
%%                                                                        %%
\title{}
\author{}                                   
%%                                                                        %%
%%========================================================================%%


\begin{document}


  \begin{tikzpicture}[
    info/.style={font=\small\ttfamily, text width=5cm},
    tharrow/.style={thick, arrows=-Stealth},
    startstop/.style={rectangle, rounded corners, text width=7cm,  text centered, draw=black, fill=red!30},
    process/.style={rectangle, text width=11cm, minimum height=1cm, text centered, draw=black, fill=orange!30},
    io/.style={trapezium, trapezium left angle=70, trapezium right angle=110, text width=6.7cm, text centered, draw=black, fill=blue!30},
    decision/.style={diamond,  aspect=2, text width=6cm, text centered, draw=black, fill=green!30},
    ]

    \node (basis) [io]  {$\{\chi_q(\vec{r})\}$}; \node[right=1cm of basis, info] {Вибір базисних функцій};
    \node (coeff) [startstop,below = 0.5cm of basis] {$c^{(n)}_{kq}$}; \node[right=1cm of coeff, info] {Вибір початкових коефіцієнтів};
    \node (integ) [process,below = 0.5cm of coeff]   {$F^{(n)}_{pq}$, $S^{(n)}_{pq}$}; \node[right=1cm of integ, info] {Обчислення інтегралів та формування матриць};
    \node (HFR)   [process,below = 0.5cm of integ]   {$\sum\limits_{q = 1}^{N_b}  c^{(n + 1)}_{kq} \left(F^{(n)}_{pq} - \varepsilon^{(n + 1)}_k   S^{(n)}_{pq} \right) = 0$};  \node[right=1cm of HFR, info] {Розв'язок рівнянь Хартрі-Фока};
    \node (Energy) [process,below = 0.5cm of HFR]  { $E^{(n + 1)} = \sum\limits_{k = 1}^{N_e} \varepsilon_k^{(n+1)} - \frac12\sum\limits_{k = 1}^{N_e}\sum\limits_{j = 1, j \neq i}^{N_e} \left( {J}^{(n)}_{kj} - {K}^{(n)}_{kj}\right)$}; \node[right=1cm of Energy, info] {Розрахунок енегрії};
    \node (Condition) [decision,below = 0.5cm of Energy] {$|E^{(n)} - E^{(n)}| \le \delta $};

    \draw [tharrow] (basis) -- (coeff) ;
    \draw [tharrow] (coeff) -- (integ);
    \draw [tharrow] (integ) -- (HFR);
    \draw [tharrow] (HFR) -- (Energy);
    \draw [tharrow] (Energy) -- (Condition);
    \draw [tharrow] (Condition.west)  -| node[above, pos=0.25] {no} ++(-2,0) |- (coeff.west);

  \end{tikzpicture}


\end{document}


