%%============================ Compiler Directives =======================%%
%%                                                                        %%
% !TeX program = xelatex
% !TeX encoding = utf8
% !TeX spellcheck = uk_UA
%%                                                                        %%
%%============================== Клас документа ==========================%%
%%                                                                        %%
\documentclass[14pt]{article}
\IfFileExists{ukrcorr.sty}{\usepackage{ukrcorr}}{}
\usepackage{ifxetex}
%%                                                                        %%
%%========================== Мови, шрифти та кодування ===================%%
%%
\ifxetex                                                                        %%
	\usepackage{fontspec}
	\setsansfont{CMU Sans Serif}%{Arial}
	\setmainfont{CMU Serif}%{Times New Roman}
	\setmonofont{CMU Typewriter Text}%{Consolas}
	\defaultfontfeatures{Ligatures={TeX}}
	\usepackage[math-style=TeX]{unicode-math}
\else
	\usepackage[utf8]{inputenc}
	\usepackage[T2A,T1]{fontenc}
	\usepackage{amsmath}
	%\usepackage{pscyr}
	\usepackage{cmap}
\fi
\usepackage[english, russian, ukrainian]{babel}
%%                                                                        %%
%%============================= Геометрія сторінки =======================%%
%%                                                                        %%
\usepackage[%
	a4paper,%
	footskip=1cm,%
	headsep=0.3cm,%
	top=2cm, %поле сверху
	bottom=2cm, %поле снизу
	left=2cm, %поле ліворуч
	right=2cm, %поле праворуч
    ]{geometry}
%%                                                                        %%
%%============================== Інтерліньяж  ============================%%
%%                                                                        %%
\renewcommand{\baselinestretch}{1}
%-------------------------  Подавление висячих строк  --------------------%%
\clubpenalty =10000
\widowpenalty=10000
%---------------------------------Інтервали-------------------------------%%
\setlength{\parskip}{0.5ex}%
\setlength{\parindent}{2.5em}%
%%                                                                        %%
%%=========================== Математичні пакети і графіка ===============%%
%%                                                                        %%
\usepackage{amsmath}
\usepackage{graphicx}
\usepackage{tikz}
\usetikzlibrary{decorations.pathreplacing}

%%                                                                        %%
%%========================== Гіперпосилення (href) =======================%%
%%                                                                        %%
\usepackage[%colorlinks=true,
	%urlcolor = blue, %Colour for external hyperlinks
	%linkcolor  = malina, %Colour of internal links
	%citecolor  = green, %Colour of citations
	bookmarks = true,
	bookmarksnumbered=true,
	unicode,
	linktoc = all,
	hypertexnames=false,
	pdftoolbar=false,
	pdfpagelayout=TwoPageRight,
	pdfauthor={Ponomarenko S.M. aka sergiokapone},
	pdfdisplaydoctitle=true,
	pdfencoding=auto
	]%
	{hyperref}
		\makeatletter
	\AtBeginDocument{
	\hypersetup{
		pdfinfo={
		Title={\@title},
		}
	}
	}
	\makeatother
%%                                                                        %%
%%============================ Заголовок та автори =======================%%
%%                                                                        %%
\title{}
\author{}
%%                                                                        %%
%%========================================================================%%


\begin{document}

\begin{tikzpicture}[
spin/.pic = {
                \draw[-stealth] (0,-0.25) -- (0,0.25);
            }
]


    \draw[-latex] (-1.5,-4.5) -- ++(0,{2*8}) node[left] {$E$};

%    \draw[ultra thick, red] foreach \i [count=\ni] in {-4,-3,...,4} {(-1,\i) coordinate(a\ni) node[left, black] { $\left|\phi_0\right\rangle_\ni$} -- (1,\i) coordinate(A\ni) };

%    \foreach \i in {1,2,...,9} {\pic at ([xshift=0.9cm]a\i) {spin}; \pic[rotate=180] at ([xshift=1.1cm]a\i) {spin};}

    \draw[ultra thick, blue] foreach \i [count=\ni] in {-4.5,-3.5,...,3.5} {(-1,\i)  coordinate(B\ni) -- (1,\i) coordinate(b\ni)  node[right, black] { $\phi_\ni$}};

    \foreach \i in {1,2,...,9} {\pic at ([xshift=0.9cm]B\i) {spin}; \pic[rotate=180] at ([xshift=1.1cm]B\i) {spin};}

    \draw[ultra thick, dashed, blue] foreach \i [count=\ni] in {4.5,5.5,...,10.5} {(-1,\i)   -- (1,\i) coordinate(V\ni) };

    \node [right] at (V1) {$\phi_{10}$};    \node [right] at (V7) {$|\phi_{\infty}$};

%    \draw[dashed] foreach \i in {1,...,9} {(A\i) -- (B\i) };

    \draw [decorate,decoration={brace,amplitude=5pt,mirror,raise=8ex}]       (V1) -- (V7)  node[midway, right=2cm, text width = 5cm]{Virtual orbitals as an eigenvectors of final\\ $\hat F_{n}$ operator};

%    \draw [decorate,decoration={brace,amplitude=5pt,mirror,raise=4ex}]       (a1) --  (A1)
%    node[below=1cm, midway, text width=6cm, align=center]{ Initial RHF equations \\
%    $\hat F_{0n} \left|\phi_0\right\rangle_n = \varepsilon_n \left|\phi_0\right\rangle_n $};

    \draw [decorate,decoration={brace,amplitude=5pt,mirror,raise=4ex}]       (B1) --  (b1)
    node[below=1cm, midway, text width=6cm, align=center]{
    $\hat F_{n} \phi_n = \varepsilon_n \phi_n $};

%    \draw [decorate,decoration={brace,amplitude=5pt,raise=8ex}] (A9|-B9) -- node[above=10ex] {Iteration procedure} (B9);

\end{tikzpicture}


\end{document}


