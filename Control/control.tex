% !TeX spellcheck = uk_UA
% !TeX encoding = utf-8
% !TeX program = lualatex

\documentclass[]{article}
\usepackage[fontsize=14pt]{fontsize}
%packages-------------------------------------------------------------------------------
\let\phi\varphi

\usepackage{fontspec}
\setsansfont{CMU Sans Serif}%{Arial}
\setmainfont{CMU Serif}%{Times New Roman}
\setmonofont{CMU Typewriter Text}%{Consolas}
\defaultfontfeatures{Ligatures={TeX}}
\usepackage[math-style=TeX]{unicode-math}
\usepackage[english, russian, ukrainian]{babel}


\usepackage[%
	a4paper,%
	footskip=1cm,%
	headsep=0.3cm,%
	top=2cm, %поле сверху
	bottom=2cm, %поле снизу
	left=2cm, %поле ліворуч
	right=2cm, %поле праворуч
    ]{geometry}

\usepackage{amsmath}
\usepackage{microtype}

%Розмір шрифта та міжстрочний інтервал---------------------------------------------------

\renewcommand{\baselinestretch}{1.5}


\def\Cut{\noindent\dotfill}
\newcounter{nom}
\setcounter{nom}{1}
\newcommand{\problem}{\textbf{Задача \No \thenom} \stepcounter{nom}}
\newcommand{\signat}{ \raisebox{-\baselineskip}{\shortstack{\underline{\hspace{5cm}}}}}
\newcounter{var}
\setcounter{var}{1}
\newcommand{\variant}{\Cut
\begin{flushleft}
\textbf{Варіант \Roman{var}} \addtocounter{var}{1}
%\hfill \textit{Група ФБ --} \underline{\hspace{1cm}} \, \textit{Студент:} \underline{\hspace{8cm}}
\end{flushleft}
}
\let\phi\varphi
\def\vidp#1#2{\textbf{Іродов #1} -- \textit{ Відповідь: #2}}
%\def\vidp#1#2{}
\usepackage[shortlabels]{enumitem}



\begin{document}
\clearpage
\thispagestyle{empty}


\section{Аналітичні розрахунки}

\problem Після опромінювання атома гідрогену, який перебуває в основному стані, світлом з довжиною
хвилі $\lambda$ електрон переходить у збуджений стан з $n = 3$. Знайти значення $\lambda$.

\problem Довести, що середня потенціальна енергія гідрогеноподібного
атома в основному стані становить $\left\langle U\right\rangle = - Z^2$ (в Хартрі).

\problem Довести, що середня кінетична енергія гідрогеноподібного
атома в основному стані становить $\left\langle T\right\rangle = \frac{Z^2}{2}$.

\problem Для основного стану атома гідрогену знайти ймовірність
знаходження електрона усередині сфери радіусом $0.53$ пм.

\problem Енергія іонізації атома гелію дорівнює $24,58$~еВ. Знайдіть
константу екранування $\sigma$ одного електрона іншим.

\section{Комп'ютерні розрахунки}

\problem Використайте метод RHF та базис STO-3G. Побудуйте орбіталі атома Ne (Неону) та запишіть їх
аналітичний вигляд через базисні функції. Знайдіть RDF, запишіть його аналітичний вигляд через
знайдені орбіталі та наведіть графік.

\end{document}
