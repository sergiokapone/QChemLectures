\documentclass{article}

\usepackage[fontsize=14pt]{fontsize}

\usepackage{fontspec}
\setsansfont{CMU Sans Serif}%{Arial}
\setmainfont{CMU Serif}%{Times New Roman}
\setmonofont{CMU Typewriter Text}%{Consolas}
\defaultfontfeatures{Ligatures={TeX}}
\usepackage[math-style=TeX]{unicode-math}
\usepackage[english, russian, ukrainian]{babel}


\usepackage{amsmath}
\usepackage{mhchem}[version=4]
\usepackage[a4paper, margin=2.5cm]{geometry}

\usepackage{titlesec}

% Настройка стиля разделов
\titleformat{\section}
    {\normalfont\normalsize\bfseries} % стиль заголовка (здесь полужирный)
    {Завдання \thesection.}                    % формат номера (с точкой)
    {0.5em}                           % расстояние между номером и текстом
    {}                                 % код перед заголовком


\title{Розрахункові завдання з квантової хімії в ORCA}
\date{}
\begin{document}
\maketitle

\section{Оптимізація геометрії та визначення дипольних моментів молекул}

Виконати оптимізацію геометрії молекули методом HF з базисним набором cc-pVDZ. Після оптимізації дипольні моменти молекул.
\begin{enumerate}
\item аміаку \ce{NH3};
\item вуглекислого газу \ce{CO2};
\item монооксиду вуглецю \ce{CO}.
\end{enumerate}

\vspace*{1em}
\textbf{Очікувані результати:}
\begin{itemize}
  \item Результати записані в таблицю.
\end{itemize}

\section{Розрахунок УФ-спектру формальдегіду  методом EOM-CCSD}

Виконати розрахунок збуджених електронних станів молекули \ce{H2CO} (формальдегід) методом EOM-CCSD, використовуючи попередньо оптимізовану геометрію. Визначити положення смуг в УФ-видимому спектрі.

\vspace*{1em}
\textbf{Очікувані результати:}
\begin{itemize}
  \item Оптимізована геометрія молекули.
  \item Енергії збуджених станів (у еВ).
  \item Осциляторні сили.
\end{itemize}

\section{Оптимізація та розрахунок ІЧ-спектрів для молекули ацетилену}

Провести повну геометричну оптимізацію молекули \ce{C2H2} методом HF, а також розрахувати її ІЧ-спектр.

\vspace*{1em}
\textbf{Очікувані результати:}
\begin{itemize}
  \item Оптимізована геометрія молекули.
  \item Визначені частоти ІЧ-переходів (у см\(^{-1}\)) та нм та інтенсивності ІЧ-переходів.
  \item Результати записані в таблицю. Наведений спектр у вигляді графіка.
\end{itemize}

\end{document}
