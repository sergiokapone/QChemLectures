%%============================ Compiler Directives =======================%%
%%                                                                        %%
% !TeX program = lualatex							    	
% !TeX encoding = utf8
% !TeX spellcheck = uk_UA
%%                                                                        %%
%%============================== Клас документа ==========================%%
%%                                                                        %%
\documentclass[14pt]{extarticle}
%%                                                                        %%
%%========================== Мови, шрифти та кодування ===================%%
%%                                                                        %%
\usepackage{fontspec}
\setsansfont{CMU Sans Serif}%{Arial}
\setmainfont{CMU Serif}%{Times New Roman}
\setmonofont{CMU Typewriter Text}%{Consolas}
\defaultfontfeatures{Ligatures={TeX}}
\usepackage[math-style=TeX]{unicode-math}
\usepackage[russian, ukrainian, english]{babel}
\usepackage{biblatex}
\usepackage{xurl}
\usepackage{hyperref}
\addbibresource{e:/Projects/LaTeX/QChem/A_Documents/Syllabus_QChem/Syllabus_QChem.bib}

%%                                                                        %%
%%============================= Геометрія сторінки =======================%%
%%                                                                        %%
\usepackage[%
	a4paper,%
	footskip=1cm,%
	headsep=0.3cm,% 
	top=2cm, %поле сверху
	bottom=2cm, %поле снизу
	left=2cm, %поле ліворуч
	right=2cm, %поле праворуч
    ]{geometry}                          
%%                                                                        %%        
%%============================== Інтерліньяж  ============================%%

\usepackage{tikz}
\usepackage{pgfplots}
\usepgfplotslibrary{units}
\pgfplotsset{compat=newest}
\usepackage{pgfplotstable}
\usepgfplotslibrary{groupplots}

\title{Basis sets for Hydrogen and Lithium}
\date{}

%===============================================================================
\begin{document}
\maketitle


\section{STO-3G for Lithium}

\subsection{Orbital structure and Radial Charge Distribution}

A minimum basis set is one in which a single basis function is used for each orbital in a Hartree-Fock calculation on the atom. However, for atoms such as lithium, basis functions of p type are added to the basis functions corresponding to the 1s and 2s orbitals of each atom. For example, each atom in the first row of the periodic system (Li - Ne) would have a basis set of five functions (two s functions and three p functions).

{\small \begin{verbatim}
------------------
MOLECULAR ORBITALS Li ROHF STO-3G
------------------
                      0         1         2         3         4   
                  -2.35349  -0.18017   0.13013   0.13013   0.13013
                   2.00000   1.00000   0.00000   0.00000   0.00000
                  --------  --------  --------  --------  --------
  0Li  1s        -0.991218  0.281468 -0.000000 -0.000000 -0.000000
  0Li  2s        -0.034144 -1.029840  0.000000  0.000000  0.000000
  0Li  1pz        0.000000 -0.000000  0.237039 -0.963695  0.122904
  0Li  1px        0.000000 -0.000000  0.106503 -0.099970 -0.989274
  0Li  1py        0.000000 -0.000000 -0.965645 -0.247586 -0.0789407
\end{verbatim}}
\tikzset{
	declare function ={
			zeta = 2.69;
			zeta2 = 0.81;
			sto1s(\x) = sqrt(zeta^3/pi)*exp(-zeta*\x);
			sto2s(\x) = sqrt(zeta2^5/(3*pi))*\x*exp(-zeta2*\x);
			sto2p(\x) = sqrt(zeta2^5/pi)*\x*exp(-zeta2*\x);
		},
}

\tikzset{
	declare function =
		{
			C11 = -0.991218; C12 = -0.034144; C21 =  0.281468; C22 = -1.029840;
			f(\x) = \x;
		}
}

Orbitals
\begin{align*}\label{}
	\phi_{1s} = C_{11}\chi_{1s} + C_{12} \chi_{2s} + C_{13} \chi_{2p_x} + C_{14} \chi_{2p_y} + C_{15} \chi_{2p_z}, \\
	\phi_{2s} = C_{21}\chi_{1s} + C_{22} \chi_{2s} + C_{23} \chi_{2p_x} + C_{24} \chi_{2p_y} + C_{25} \chi_{2p_z}.
\end{align*}


\begin{center}
	\begin{tikzpicture}
		\begin{axis}[
				ymax=0.5,
				ymin=-1,
				xmax=6,
				axis lines=left,
				xlabel=$r$,
				ylabel=Orbitals,
				axis lines = middle,
			]

			\addplot [thick, domain={0:7}, smooth, red,samples=100] {
				sto1s(x)*C11 + sto2s(x)*C12 % 1s                 
			} node [pin=1:1s, pos=0.2] {};
			\addplot [thick, domain={0:7}, smooth, blue,samples=100] {
				sto1s(x)*C21 + sto2s(x)*C22  % 2s                   
			} node [pin=1:2s, pos=0.1] {};
		\end{axis}
	\end{tikzpicture}
\end{center}

Radial Density Distribution $4\pi r^2 \rho(r) = 4\pi r^2 \left(2|\phi_{1s}|^2 + |\phi_{2s}|^2\right)$.

\begin{center}
	\begin{tikzpicture}
		\begin{axis}[
				ymax=3,
				xmax=6,
				axis lines=left,
				xlabel=$r$,
				ylabel=$4\pi r^2 \rho(r)$,
			]

			\addplot [thick, domain={0:7}, smooth, red,samples=100] {
				4*pi*x^2*(
				2*(sto1s(x)*C11 + sto2s(x)*C12)^2  + % 1s
				(sto1s(x)*C21 + sto2s(x)*C22)^2  % 2s                   
				)
			};
		\end{axis}
	\end{tikzpicture}
\end{center}


\end{document}
