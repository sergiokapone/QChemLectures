% !TeX program = lualatex
% !TeX encoding = utf8
% !TeX spellcheck = uk_UA

\documentclass[]{article}
\usepackage[fontsize=14pt]{fontsize}

\usepackage{fontspec}
\setsansfont{CMU Sans Serif}%{Arial}
\setmainfont{CMU Serif}%{Times New Roman}
\setmonofont{CMU Typewriter Text}%{Consolas}

\defaultfontfeatures{Ligatures={TeX}}
\usepackage[math-style=TeX]{unicode-math}
\usepackage[russian, english, ukrainian]{babel}

\usepackage[autolang=other, bibstyle=gost-numeric]{biblatex}

\usepackage{microtype}
\usepackage{xurl}
\usepackage{indentfirst}

\usepackage[most]{tcolorbox}
\tcbset{highlight math style={enhanced,
  colframe=red,colback=white,arc=0pt,boxrule=1pt}}
\tcbset{
    mybox/.style={
        enhanced,
        colback=gray!5,
        fonttitle=\bfseries,
        sharp corners,
        boxrule=0pt,
        frame empty
    }
}
\usepackage{subcaption}
\captionsetup[subfigure]{justification=centering}
\usepackage{hyperref}

\usepackage{titlesec}
\titlelabel{\thetitle.\space}
\usepackage{minted}
\usepackage{tikz, calc}
\usetikzlibrary{backgrounds}

\usepackage{enumitem}


\addbibresource{../../Bibliography/QuantumChemistry.bib}


\usepackage[%
	a4paper,%
	footskip=1cm,%
	headsep=0.3cm,%
	top=2cm,
	bottom=2cm,
	left=2cm,
	right=2cm,
    ]{geometry}


\usepackage{tikz}
\usepackage{pgfplots}
\usepgfplotslibrary{units}
\pgfplotsset{compat=newest}
\usepackage{pgfplotstable}
\usepgfplotslibrary{groupplots}



\def\C#1{{\color{teal}#1}}
\def\a#1{{\color{magenta}#1}}
\def\sgto#1{\left(\frac{2\cdot\a{#1}}{\pi}\right)^{3/4} e^{-\a{#1}\cdot r^2}}
\NewDocumentCommand{\Li}{}{
\tikz[baseline=-5pt]{
    \coordinate (dv) at (0,0);
    \coordinate (base) at (50pt, 0pt);
    \coordinate (height) at (0pt,50pt);
    \coordinate (diag) at ($(base)+(height)$);
    \path[draw=teal!80!black, rounded corners=2pt, fill=teal, thick] ($(dv)-.5*(diag)$) rectangle +(diag);
    \node[white] at (dv) {\sffamily\huge Li};
    \node[white, inner sep=2pt] (dvtext) at ($(dv)-.5*(height)$) [anchor=south] {\sffamily\tiny Lithium};
    \node[white, inner sep=2pt] (dvnum) at ($(dv)+.5*(height)-.5*(base)$) [anchor=north west] {\sffamily\tiny 3};
    \node[white, inner sep=2pt] (elestruct) at ($(dv)+.5*(height)-.1*(base)$) [anchor=north west] {\sffamily\tiny $1s^22s^1$};
}
}

\title{\bfseries Розрахунок атома \Li в \texttt{ORCA}}
\date{}
%===============================================================================
\begin{document}
\maketitle


%% --------------------------------------------------------
%\clearpage
\section{Розрахунки орбіталей в ORCA}
%% --------------------------------------------------------

%% --------------------------------------------------------
\subsection*{Файл \texttt{.inp}}
%% --------------------------------------------------------

\begin{minted}[mathescape,
        fontsize=\small,
        bgcolor=gray!5,
        ]
        {ruby}
!ROHF SP STO-3G
!Printbasis
!PrintMOs

%coords
    CTyp xyz       # the type of coordinates = xyz or internal
    Charge 0       # the total charge of the molecule
    Mult 2         # the multiplicity = 2S+1 ; S = +1/2 - 1/2  = 1/2
    Units Borhs     # the unit of length = angs or borhs
    coords
        Li        0.000000       0.00000        0.00000
    end
end
\end{minted}

%% --------------------------------------------------------
%\clearpage
\subsection*{Базисні функції}
%% --------------------------------------------------------

\begin{minted}[mathescape,
        fontsize=\small,
        bgcolor=gray!5,
        ]
        {ruby}
 NewGTO Li
 S 3
   1      16.1195750000      0.1543289703
   2       2.9362007000      0.5353281412
   3       0.7946505000      0.4446345410
 S 3
   1       0.6362897000     -0.0999672298
   2       0.1478601000      0.3995128292
   3       0.0480887000      0.7001154686
 P 3
   1       0.6362897000      0.1559162650
   2       0.1478601000      0.6076837007
   3       0.0480887000      0.3919573775
  end;
\end{minted}

%% --------------------------------------------------------
%\clearpage
\subsection*{Виведення орбіталей в ORCA}
%% --------------------------------------------------------


\begin{minted}[mathescape,
        fontsize=\small,
        bgcolor=gray!5,
        ]
        {ruby}
----------------
ORBITAL ENERGIES
----------------

  NO   OCC          E(Eh)            E(eV)
   0   2.0000      -2.353491       -64.0417
   1   1.0000      -0.180172        -4.9027
   2   0.0000       0.130126         3.5409
   3   0.0000       0.130126         3.5409
   4   0.0000       0.130126         3.5409

------------------
MOLECULAR ORBITALS
------------------

                      0         1         2         3         4
                  -2.35349  -0.18017   0.13013   0.13013   0.13013
                   2.00000   1.00000   0.00000   0.00000   0.00000
                  --------  --------  --------  --------  --------
  0Li  1s        -0.991218  0.281468 -0.000000 -0.000000 -0.000000
  0Li  2s        -0.034144 -1.029840  0.000000  0.000000  0.000000
  0Li  1pz       -0.000000  0.000000  0.212225  0.933091  0.290348
  0Li  1px       -0.000000  0.000000 -0.089877 -0.277220  0.956594
  0Li  1py       -0.000000  0.000000  0.973079 -0.229108  0.025030
\end{minted}

%% --------------------------------------------------------
\subsection*{Побудова орбіталей та детермінанта Слейтера}
%% --------------------------------------------------------
%\clearpage

Орбіталі:
\begin{tcolorbox}[mybox]
	\abovedisplayskip=0pt%
	\begin{align*}
		\phi_{0} & = \C{-0.991218} \cdot \mathrm{STO}_{1s}  + (\C{-0.034144}) \cdot \mathrm{STO}_{2s},                                                       \\
		\phi_{1} & = \C{0.281468} \cdot \mathrm{STO}_{1s}  + (\C{-1.029840}) \cdot \mathrm{STO}_{2s},                                                        \\
		\phi_{2} & = \C{0.212225} \cdot \mathrm{STO}_{1p_{z}}  + (\C{-0.089877}) \cdot \mathrm{STO}_{1p_{x}}  + \C{0.973079} \cdot \mathrm{STO}_{1p_{y}},    \\
		\phi_{3} & = \C{0.933091} \cdot \mathrm{STO}_{1p_{z}}  + (\C{-0.277220}) \cdot \mathrm{STO}_{1p_{x}}  + (\C{-0.229108}) \cdot \mathrm{STO}_{1p_{y}}, \\
		\phi_{4} & = \C{0.290348} \cdot \mathrm{STO}_{1p_{z}}  + \C{0.956594} \cdot \mathrm{STO}_{1p_{x}}  + \C{0.025030} \cdot \mathrm{STO}_{1p_{y}}.
	\end{align*}
\end{tcolorbox}

Орбіталь $\phi_0$ --- двічі заселена, орбіталь $\phi_1$ --- заселена одним електроном. Інші орбіталі не заселені (\emph{віртуальні}).

\clearpage
Детермінант Слейтера (\emph{\color{red}будується лише із заселених орбіталей}):
\begin{tcolorbox}[mybox]
	\begin{equation*}
		\Phi(\vec\xi_1, \vec\xi_2, \vec\xi_3) = \frac1{\sqrt6}
		\begin{vmatrix}
			\phi_0(1)\alpha(1) & \phi_0(1)\beta(1) & \phi_1(1)\alpha(1) \\
			\phi_0(2)\alpha(2) & \phi_0(2)\beta(2) & \phi_1(2)\alpha(2) \\
			\phi_0(3)\alpha(3) & \phi_0(3)\beta(3) & \phi_1(3)\alpha(3)
		\end{vmatrix}.
	\end{equation*}
\end{tcolorbox}

%\clearpage
Електронна густина (\emph{\color{red}рахується лише по заселеним орбіталям}):
\begin{tcolorbox}[mybox]
	\begin{equation*}
		\rho =  2\cdot|\phi_{0}|^2 + 1\cdot |\phi_1|^2.
	\end{equation*}
\end{tcolorbox}

Функція радіального розподілу $4\pi r^2 \rho(r)$.

%=================== Побудова засобами pgfplots ===================

\tikzset{
	declare function = {
			gtoS(\c,\a,\r)=\c*(2*\a/pi)^(3/4)*exp(-\a*\r^2);
			gtoPx(\c,\a,\x,\y,\z)=\c*(2*\a/pi)^(3/4)*sqrt(4*\a)\x*exp(-\a*(\x^2+\y^2+\z^2);
			gtoPy(\c,\a,\x,\y,\z)=\c*(2*\a/pi)^(3/4)*sqrt(4*\a)\y*exp(-\a*(\x^2+\y^2+\z^2);
			gtoPz(\c,\a,\x,\y,\z)=\c*(2*\a/pi)^(3/4)*sqrt(4*\a)\z*exp(-\a*(\x^2+\y^2+\z^2);
			sto1s(\r)=gtoS(0.15433,16.11,\r)+gtoS(0.53533,2.94,\r)+gtoS(0.445,0.79,\r);
			sto2s(\r)=gtoS(-0.09997,0.6363,\r)+gtoS(0.39951,0.148,\r)+gtoS(0.7,0.048,\r);
			sto1px(\x,\y,\z)=gtoPx(0.1559, 0.6363,\x,\y,\z)+gtoPx(0.6077, 0.148,\x,\y,\z)+
			gtoPx(0.392, 0.048,\x,\y,\z);
			sto1py(\x,\y,\z)=gtoPy(0.1559, 0.6363,\x,\y,\z)+gtoPy(0.6077, 0.148,\x,\y,\z)+
			gtoPy(0.392, 0.048,\x,\y,\z);
			sto1pz(\x,\y,\z)=gtoPz(0.1559, 0.6363,\x,\y,\z)+gtoPz(0.6077, 0.148,\x,\y,\z)+
			gtoPz(0.392, 0.048,\x,\y,\z);
		},
}

%=========================================================
\begin{figure}[h!]\centering
	%---------------------------------------------------------
	\begin{subfigure}[t]{0.45\linewidth}\centering
		\begin{tikzpicture}[background rectangle/.style={fill=gray!5}, show background rectangle]
			\begin{axis}[
					ymin=-1.5,
					ymax=0.5,
					xmax=5,
					axis lines=left,
					xlabel={$r$, bohr},
					ylabel=$\phi(r)$,
					width=0.95\linewidth,
					legend style={at={(1,0.1)}, draw=none, fill=white, anchor=south east},
				]

				\addplot [thick, domain={0:5}, smooth, red, samples=1500] {-0.991218*sto1s(x) - 0.034144*sto2s(x)} ;
				\addlegendentry{$\phi_0$}

				\addplot [thick, domain={0:5}, smooth, blue, samples=1500] {0.281468*sto1s(x) - 1.029840*sto2s(x)} ;
				\addlegendentry{$\phi_1$}

			\end{axis}
		\end{tikzpicture}
		\caption{Зайняті орбіталі}
	\end{subfigure}
	\qquad%---------------------------------------------------------
	\begin{subfigure}[t]{0.45\linewidth}\centering
		\begin{tikzpicture}[background rectangle/.style={fill=gray!5}, show background rectangle]
			\begin{axis}[
					ymax=3,
					xmax=5,
					axis lines=left,
					xlabel={$r$, bohr},
					ylabel=$4\pi r^2 \rho(r)$,
					width=0.95\linewidth,
				]
				\addplot [thick, domain={0:5}, smooth, red, samples=1500] { 4*pi*x^2*(2*(-0.991218*sto1s(x) - 0.034144*sto2s(x))^2+
					1*(0.281468*sto1s(x) - 1.029840*sto2s(x))^2)};

			\end{axis}
		\end{tikzpicture}
		\caption{Електронна густина}
	\end{subfigure}
	\caption{Радіальні розподіли}
	%---------------------------------------------------------
\end{figure}
%=========================================================


Радіальний розподіл електронної густини атома показує, що атом має оболонкову структуру.

%% --------------------------------------------------------
\clearpage
\section{Візуалізація орбіталей та радіального розподілу в \texttt{Multiwfn}}
%% --------------------------------------------------------

%---------------------------------------------------------
\begin{center}
	\begin{tikzpicture}[background rectangle/.style={fill=gray!5}, show background rectangle]
		\begin{axis}[
				ymax=3,
				xmax=5,
				axis lines=left,
				xlabel={$r$, bohr},
				ylabel=$4\pi r^2 \rho(r)$,
				width=1\linewidth,
			]
			\addplot[mark=none, red, smooth, thick] table[x expr=2*\thisrowno{1},y index=2] {RDF.txt};

		\end{axis}
	\end{tikzpicture}
	\captionof{figure}{Радіальний розподіл електронної густини}
\end{center}
%---------------------------------------------------------


%=========================================================
\begin{figure}[h!]\centering
	%---------------------------------------------------------
	\begin{subfigure}[t]{0.4\linewidth}\centering
		\includegraphics[width=\linewidth]{pict/1.png}
		\caption{Орбіталь $\phi_0$}
	\end{subfigure}
	\qquad%---------------------------------------------------------
	\begin{subfigure}[t]{0.4\linewidth}\centering
		\includegraphics[width=\linewidth]{pict/2.png}
		\caption{Орбіталь $\phi_1$}
	\end{subfigure}
	\qquad%---------------------------------------------------------
	\begin{subfigure}[t]{0.4\linewidth}\centering
		\includegraphics[width=\linewidth]{pict/3.png}
		\caption{Орбіталь $\phi_2$}
	\end{subfigure}
	\qquad%---------------------------------------------------------
	\begin{subfigure}[t]{0.4\linewidth}\centering
		\includegraphics[width=\linewidth]{pict/4.png}
		\caption{Орбіталь $\phi_3$}
	\end{subfigure}
	\qquad%---------------------------------------------------------
	\begin{subfigure}[t]{0.4\linewidth}\centering
		\includegraphics[width=\linewidth]{pict/5.png}
		\caption{Орбіталь $\phi_4$}
	\end{subfigure}
	\caption{Орбіталі}
	%---------------------------------------------------------
\end{figure}
%=========================================================

\clearpage
\nocite{BSE,  Multiwfn, Lu2011, Bartell1953}
\printbibliography[title=Перелік використаних джерел]


\end{document}
