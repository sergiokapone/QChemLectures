\chapter*{Передмова}
\addcontentsline{toc}{chapter}{Передмова}

Цей методичний посібник є практичним введенням у квантово-хімічні розрахунки атомних систем з використанням програмного пакету PySCF (Python-based Simulations of Chemistry Framework). Посібник призначений для студентів хімічних, фізичних та матеріалознавчих спеціальностей, які вивчають квантову хімію, обчислювальну хімію та суміжні дисципліни.

\section*{Передумови}

Для успішного опанування матеріалу посібника бажано мати:

\paragraph{Обов'язкові знання:}
\begin{itemize}
    \item Базові знання квантової механіки (хвильова функція, оператори, власні значення)
    \item Основи хімії (періодична система, електронні конфігурації, хімічний зв'язок)
    \item Елементарне програмування на Python (змінні, цикли, функції)
    \item Робота з командним рядком (термінал/консоль)
\end{itemize}

\paragraph{Бажано знати:}
\begin{itemize}
    \item Лінійну алгебру (матриці, власні вектори, діагоналізація)
    \item Основи чисельних методів
    \item Роботу з NumPy та Matplotlib
    \item Jupyter Notebook
\end{itemize}



\paragraph{Кожен розділ містить:}
\begin{itemize}
    \item Теоретичне пояснення методу
    \item Детальні приклади коду з коментарями
    \item Практичні завдання для самостійної роботи
    \item Контрольні запитання
    \item Посилання на додаткову літературу
\end{itemize}

\section*{Організація навчальних матеріалів}

Всі коди, що наведені в посібнику, організовані в окремих папках відповідно до структури розділів:

\begin{verbatim}
pyscf_atomic_tutorial/
├── chapter_02/
│   ├── 01_installation.py
│   ├── 02_basic_structure.py
│   └── examples/
├── chapter_03/
│   ├── 01_mole_object.py
│   ├── 02_basis_sets.py
│   └── examples/
├── chapter_04/
│   ├── 01_hf_hydrogen.py
│   ├── 02_hf_helium.py
│   ├── 03_second_period.py
│   └── examples/
├── chapter_05/
│   ├── 01_dft_basics.py
│   ├── 02_functionals.py
│   └── examples/
├── chapter_06/
│   ├── 01_mp2.py
│   ├── 02_ccsd.py
│   ├── 03_casscf.py
│   └── examples/
├── chapter_07/
│   ├── 01_ionization_energies.py
│   └── examples/
├── notebooks/
│   ├── Chapter_02_Interactive.ipynb
│   ├── Chapter_03_Interactive.ipynb
│   └── ...
└── README.md
\end{verbatim}

\subsection*{Робота з кодом}

Код з посібника можна використовувати кількома способами:

\paragraph{1. Jupyter Notebook (рекомендовано для навчання)}

Jupyter Notebook --- інтерактивне середовище, ідеальне для навчання та експериментування:

\begin{minted}{bash}
# Встановлення Jupyter
pip install jupyter

# Запуск
cd pyscf_atomic_tutorial/notebooks
jupyter notebook
\end{minted}

\textbf{Переваги Jupyter:}
\begin{itemize}
    \item Виконання коду по частинах (комірками)
    \item Миттєвий перегляд результатів
    \item Збереження графіків безпосередньо в notebook
    \item Можливість додавати власні нотатки
    \item Легко ділитися з колегами
\end{itemize}

\paragraph{2. Окремі Python скрипти}

Всі приклади можна виконувати як звичайні Python скрипти:

\begin{minted}{bash}
# Виконання окремого скрипта
python chapter_04/01_hf_hydrogen.py

# Або з додатковими опціями
python -u chapter_04/02_hf_helium.py > output.log
\end{minted}

Це зручно для:
\begin{itemize}
    \item Автоматизації серії розрахунків
    \item Виконання на кластерах та серверах
    \item Інтеграції у власні pipeline
\end{itemize}

\paragraph{3. Інтерактивний Python (iPython)}

Для швидких тестів та експериментів:

\begin{minted}{python}
ipython
>>> from pyscf import gto, scf
>>> mol = gto.M(atom='H 0 0 0', basis='sto-3g', spin=1)
>>> mf = scf.UHF(mol)
>>> mf.kernel()
\end{minted}

\paragraph{4. IDE (PyCharm, VS Code, Spyder)}

Всі приклади сумісні з популярними IDE. Рекомендуємо налаштувати:
\begin{itemize}
    \item Автодоповнення для PySCF
    \item Відлагодження (debugging)
    \item Інтеграцію з Git для контролю версій
\end{itemize}

\section*{Вимоги до системи}

\subsection*{Мінімальні вимоги:}
\begin{itemize}
    \item \textbf{ОС:} Linux, macOS, або Windows 10/11
    \item \textbf{Python:} версія 3.7 або новіша
    \item \textbf{Оперативна пам'ять:} 4 ГБ (8 ГБ рекомендовано)
    \item \textbf{Вільне місце:} 2 ГБ
    \item \textbf{Процесор:} будь-який сучасний (багатоядерний краще)
\end{itemize}

\subsection*{Рекомендовані вимоги:}
\begin{itemize}
    \item \textbf{Оперативна пам'ять:} 16+ ГБ
    \item \textbf{Процесор:} 4+ ядра
    \item \textbf{SSD:} для швидкого читання/запису великих файлів
\end{itemize}

\textbf{Примітка:} Складні розрахунки (CCSD, CASSCF для великих систем) можуть потребувати значних ресурсів. Для навчальних прикладів з посібника достатньо звичайного ноутбука.

\section*{Встановлення програмного забезпечення}

\subsection*{Швидкий старт (Linux/macOS):}

\begin{minted}{bash}
# Створення віртуального середовища (рекомендовано)
python3 -m venv pyscf_env
source pyscf_env/bin/activate

# Встановлення PySCF та залежностей
pip install --upgrade pip
pip install pyscf
pip install numpy scipy matplotlib jupyter

# Перевірка встановлення
python -c "import pyscf; print(pyscf.__version__)"
\end{minted}

\subsection*{Швидкий старт (Windows):}

\begin{minted}{bash}
# Створення віртуального середовища
python -m venv pyscf_env
pyscf_env\Scripts\activate

# Встановлення
pip install --upgrade pip
pip install pyscf numpy scipy matplotlib jupyter

# Перевірка
python -c "import pyscf; print(pyscf.__version__)"
\end{minted}

Детальні інструкції з встановлення наведені в Розділі 2.

\section*{Як користуватися цим посібником}

\subsection*{Для самостійного навчання:}

\begin{enumerate}
    \item \textbf{Читайте послідовно} --- матеріал побудований від простого до складного
    \item \textbf{Виконуйте всі приклади} --- просте читання коду не замінить практики
    \item \textbf{Експериментуйте} --- змінюйте параметри, пробуйте інші атоми
    \item \textbf{Виконуйте завдання} --- вони закріплюють матеріал
    \item \textbf{Веніть до складних місць} --- деякі концепції потребують часу
\end{enumerate}


\section*{Зворотний зв'язок}

Ваші коментарі та пропозиції допоможуть покращити цей посібник! Якщо ви знайшли:
\begin{itemize}
    \item Помилки в коді або тексті
    \item Незрозумілі пояснення
    \item Теми, які потребують більш детального розгляду
    \item Інші проблеми
\end{itemize}

Будь ласка, повідомте про це через:
\begin{itemize}
    \item GitHub Issues (якщо матеріал на GitHub)
    \item Email викладачу/автору
    \item Форум курсу
\end{itemize}

\section*{Ліцензія та використання}

Цей методичний посібник розповсюджується для освітніх цілей. Ви можете:
\begin{itemize}
    \item Використовувати матеріали для навчання
    \item Модифікувати приклади коду для власних потреб
    \item Ділитися посібником з колегами та студентами
\end{itemize}

При використанні матеріалів просимо посилатися на цей посібник.


\vspace{1cm}

\begin{center}
\textit{Бажаємо успіхів у вивченні квантової хімії!}
\end{center}

\vspace{0.5cm}

\begin{flushright}
С. М. Пономаренко\\
\today
\end{flushright}

\clearpage