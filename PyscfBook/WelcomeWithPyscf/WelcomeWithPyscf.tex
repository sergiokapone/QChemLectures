% !TeX program = lualatex
% !TeX encoding = utf8
% !TeX spellcheck = uk_UA
% !TeX root =../PyscfBook.tex

%=========================================================
\Opensolutionfile{answer}[\currfilebase/\currfilebase-Answers]
\chapter{Знайомство з PySCF}\label{\currfilebase}
%=========================================================

%% --------------------------------------------------------
\section{Що таке PySCF: можливості та переваги}
%% --------------------------------------------------------

PySCF (Python-based Simulations of Chemistry Framework) --- це сучасний програмний пакет з відкритим вихідним кодом для квантово-хімічних розрахунків, написаний переважно на мові Python з критичними для продуктивності частинами на C. Розроблений групою під керівництвом професора Garnet Chan, PySCF надає потужний та гнучкий інструментарій для проведення ab initio розрахунків молекулярних та атомних систем.

%% --------------------------------------------------------
\subsection{Основні можливості PySCF}
%% --------------------------------------------------------

PySCF підтримує широкий спектр квантово-хімічних методів:

\begin{itemize}
    \item \emph{Метод Хартрі-Фока (HF):} restricted (RHF), unrestricted (UHF), та restricted open-shell (ROHF) варіанти
    \item \emph{Теорія функціоналу густини (DFT):} з великою бібліотекою функціоналів обміну-кореляції через інтеграцію з libxc
    \item \emph{Post-Hartree-Fock методи:} MP2, CCSD, CCSD(T), CASSCF, CASPT2, NEVPT2
    \item \emph{Configuration Interaction (CI):} CISD, full CI для малих систем
    \item \emph{Явна кореляція:} F12 методи
    \item \emph{Багаторівневі методи:} ONIOM, embedding
    \item \emph{Періодичні системи:} через модуль PBC (Periodic Boundary Conditions)
\end{itemize}


%% --------------------------------------------------------
\subsection{Переваги використання PySCF}
%% --------------------------------------------------------

\paragraph{Відкритий код та безкоштовність.}
PySCF розповсюджується під ліцензією Apache 2.0, що робить його повністю безкоштовним для академічного та комерційного використання. Доступ до вихідного коду дозволяє глибоко зрозуміти реалізацію методів та за потреби модифікувати програму.

\paragraph{Гнучкість та розширюваність.}
Завдяки використанню Python як основної мови, PySCF надзвичайно гнучкий. Користувачі можуть легко:
\begin{itemize}
    \item Комбінувати різні методи в одному скрипті.
    \item Створювати власні workflow для складних розрахунків.
    \item Інтегрувати PySCF з іншими Python бібліотеками (NumPy, SciPy, matplotlib).
    \item Розробляти власні модулі та методи.
\end{itemize}

\paragraph{Сучасна архітектура.}
PySCF використовує об'єктно-орієнтований підхід, що робить код зрозумілим та легким для модифікації. Модульна структура дозволяє використовувати тільки потрібні компоненти.

\paragraph{Активна розробка.}
Проект активно розвивається, регулярно додаються нові функції та покращується продуктивність. Спільнота користувачів надає підтримку через GitHub та форуми.

\paragraph{Продуктивність.}
Незважаючи на використання Python, критичні обчислювальні частини написані на C та оптимізовані. PySCF ефективно використовує сучасні бібліотеки лінійної алгебри (BLAS, LAPACK) та може працювати на багатоядерних системах.

\subsection{Порівняння з іншими програмами}

У таблиці~\ref{tab:comparison} наведено порівняння PySCF з іншими популярними квантово-хімічними пакетами.

\begin{table}[h]
\centering
\caption{Порівняння квантово-хімічних програм}
\label{tab:comparison}
\begin{tblr}{lccccc}
\hline
\textbf{Програма} & \textbf{Ліцензія} & \textbf{Мова} & \textbf{Гнучкість} & \textbf{Навчання} \\
\hline
PySCF & Відкрита & Python/C & Висока & Середнє \\
Gaussian & Комерційна & Fortran & Низька & Легке \\
ORCA & Академічна & C++ & Середня & Легке \\
Q-Chem & Комерційна & C/C++ & Середня & Середнє \\
Psi4 & Відкрита & C++/Python & Висока & Середнє \\
\hline
\end{tblr}
\end{table}


%% --------------------------------------------------------
\section{Встановлення PySCF}
%% --------------------------------------------------------



%% --------------------------------------------------------
\subsection{Системні вимоги}
%% --------------------------------------------------------

Для роботи з PySCF необхідно:
\begin{itemize}
    \item Python 3.7 або новіший
    \item NumPy версії 1.13.0 або новішої
    \item SciPy версії 1.0.0 або новішої
    \item H5py (для збереження великих масивів даних)
    \item 4--8 ГБ оперативної пам'яті (мінімум)
    \item Сучасний процесор (бажано багатоядерний)
\end{itemize}

%% --------------------------------------------------------
\subsection{Встановлення в Linux}
%% --------------------------------------------------------

Найпростіший спосіб встановлення PySCF у Linux --- використання pip:

\begin{minted}{bash}
# Оновлення pip
pip install --upgrade pip

# Встановлення PySCF
pip install pyscf
\end{minted}

Для встановлення з додатковими можливостями:

\begin{minted}{bash}
# З підтримкою XC функціоналів
pip install pyscf[geomopt,dftd3,dmrgscf]
\end{minted}

Альтернативно, можна встановити з вихідного коду:

\begin{minted}{bash}
git clone https://github.com/pyscf/pyscf.git
cd pyscf
pip install -e .
\end{minted}

%% --------------------------------------------------------
\subsection{Встановлення в Windows}
%% --------------------------------------------------------

Для Windows рекомендується використання Anaconda:

\begin{minted}{bash}
# Створення нового середовища
conda create -n pyscf_env python=3.10
conda activate pyscf_env

# Встановлення PySCF
conda install -c pyscf pyscf
\end{minted}

%% --------------------------------------------------------
\subsection{Встановлення в macOS}
%% --------------------------------------------------------

Для macOS процес аналогічний до Linux:

\begin{minted}{bash}
# Встановлення через pip
pip3 install pyscf

# Або через Homebrew + pip
brew install python3
pip3 install pyscf
\end{minted}

%% --------------------------------------------------------
\subsection{Перевірка встановлення}
%% --------------------------------------------------------

Після встановлення перевіримо, чи PySCF працює коректно:

\inputcode{code_1.py}

Якщо програма виводить версію та обчислює енергію молекули H$_2$, встановлення виконано успішно.

\section{Базова структура програми на PySCF}

\subsection{Загальна схема розрахунку}

Типовий розрахунок у PySCF складається з трьох основних етапів:

\begin{enumerate}
    \item \textbf{Створення молекулярного об'єкта} --- визначення геометрії системи, базисного набору, заряду та спіну
    \item \textbf{Вибір та налаштування методу} --- вибір квантово-хімічного методу (HF, DFT, CC тощо)
    \item \textbf{Виконання розрахунку} --- запуск обчислень та отримання результатів
\end{enumerate}

%% --------------------------------------------------------
\subsection{Мінімальний приклад}
%% --------------------------------------------------------

Розглянемо найпростіший приклад розрахунку атома Гідрогену:

\inputcode{code_2.py}

%% --------------------------------------------------------
\subsection{Структура для атома Гелію}
%% --------------------------------------------------------

Для атома з двома електронами у замкненій оболонці:

\inputcode{code_3.py}

%% --------------------------------------------------------
\subsection{Розрахунок з аналізом результатів}
%% --------------------------------------------------------

Більш детальний приклад з виведенням додаткової інформації:

\inputcode{code_4.py}

%% --------------------------------------------------------
\section{Система одиниць та конвенції}
%% --------------------------------------------------------

%% --------------------------------------------------------
\subsection{Атомні одиниці}
%% --------------------------------------------------------

PySCF використовує атомну систему одиниць (atomic units, a.u.), де:

\begin{align*}
\hbar &= 1 \\
m_e &= 1 \quad \text{(маса електрона)} \\
e &= 1 \quad \text{(елементарний заряд)} \\
4\pi\epsilon_0 &= 1 \quad \text{(електрична константа)}
\end{align*}

У цій системі:
\begin{itemize}
    \item Енергія вимірюється в Hartree (Ha): 1 Ha = 27.211386 eV = 627.509 kcal/mol
    \item Відстань вимірюється в Bohr (a$_0$): 1 Bohr = 0.529177 Å
    \item Час вимірюється у a.u.: 1 a.u. = 2.4189 $\times$ 10$^{-17}$ s
\end{itemize}

%% --------------------------------------------------------
\subsection{Конвертація одиниць}
%% --------------------------------------------------------

PySCF надає модуль для конвертації:

\inputcode{code_5.py}

%% --------------------------------------------------------
\subsection{Системні константи}
%% --------------------------------------------------------

Доступ до фізичних констант:

\inputcode{code_6.py}

%% --------------------------------------------------------
\subsection{Конвенції для атомів}
%% --------------------------------------------------------

При визначенні атомів у PySCF:

\paragraph{Координати.}
Координати задаються у формі рядка або списку. За замовчуванням використовуються ангстреми (Ångström), але можна вказати одиниці:

\inputcode{code_7.py}

\paragraph{Спін.}
Параметр \texttt{spin} визначає $2S = N_\alpha - N_\beta$, де $N_\alpha$ та $N_\beta$ --- кількість альфа та бета електронів:

\inputcode{code_8.py}

\paragraph{Симетрія.}
За замовчуванням PySCF використовує повну точкову симетрію системи:

\inputcode{code_9.py}

%% --------------------------------------------------------
\section{Документація та ресурси}
%% --------------------------------------------------------


%% --------------------------------------------------------
\subsection{Офіційна документація}
%% --------------------------------------------------------

\begin{center}
\begin{tblr}{
colspec={lX[l,m, font=\small]},
row{1} = {c, m, font=\bfseries},
}
\hline
{Ресурс} & {URL} \\
\hline
Головний сайт & \url{https://pyscf.org} \\
GitHub репозиторій & \url{https://github.com/pyscf/pyscf} \\
Документація API & \url{https://pyscf.org/pyscf_api_docs/pyscf.html} \\
Приклади коду & \url{https://github.com/pyscf/pyscf/tree/master/examples} \\
\hline
\end{tblr}
\end{center}

%% --------------------------------------------------------
\subsection{Основні публікації}
%% --------------------------------------------------------

Ключові статті для цитування при використанні PySCF:

\addtocategoryNoCi{pyscf}{%
sun2018pyscf, sun2020pyscf,
}

\printbibliography[category=pyscf, heading=none]

%% --------------------------------------------------------
\section{Резюме}
%% --------------------------------------------------------

У цьому розділі ми познайомились з PySCF --- потужним інструментом для квантово-хімічних розрахунків. Основні моменти:

\begin{itemize}
    \item PySCF є сучасним, відкритим та гнучким програмним пакетом
    \item Встановлення здійснюється просто через pip або conda
    \item Базова структура програми включає три етапи: створення системи, вибір методу, виконання розрахунку
    \item PySCF використовує атомну систему одиниць
    \item Доступна велика кількість документації та навчальних матеріалів
\end{itemize}

У наступному розділі ми детально розглянемо базові об'єкти PySCF та навчимось налаштовувати параметри розрахунків для атомних систем.