% !TeX program = lualatex
% !TeX encoding = utf8
% !TeX spellcheck = uk_UA
% !TeX root =../PyscfBook.tex

%% ========================================================
\section{Розрахунок атома Гелію}
%% ========================================================

%% --------------------------------------------------------
\subsection{Двоелектронна система}
%% --------------------------------------------------------

Атом гелію є фундаментальним прикладом для демонстрації методів \textbf{Хартрі–Фока (HF)}. Два електрони гелію мають протилежні спіни й заповнюють одну й ту саму орбіталь $1s$, тому це \emph{замкнена оболонка}
зі спіном $S = 0$ (синглет, $M = 1$).

%Це перша система, де проявляється \textbf{електрон–електронна кореляція}, відсутня в атомі водню.

Після завершення SCF-розрахунку \texttt{PySCF} надає доступ до одноелектронних інтегралів
(\inlinecode{int1e\_kin}, \inlinecode{int1e\_nuc}) та до двоелектронної частини (\inlinecode{get\_veff}).
Це дозволяє безпосередньо обчислити всі компоненти енергії та перевірити віральне співвідношення.

\inputcode{HeHF.py}

\begin{commentbox}
Якщо розрахунок збіжний і фізично коректний, то відношення
\(\langle V \rangle / \langle T \rangle \approx -2.00 \pm 0.01\),
що підтверджує виконання віральної теореми.
Будь-яке суттєве відхилення (наприклад, $-1.7$ чи $-2.3$) свідчить про помилку збіжності або неадекватність базисного набору.
\end{commentbox}

%
%
%
%\inputcode{code_4.py}
%
%\textbf{Коментар.}
%Метод RHF нехтує кореляцією між електронами, тобто вважає, що хвильова функція є добутком одноелектронних орбіталей.
%Тому енергія HF завжди \emph{вище} (менш від’ємна), ніж точна. Для гелію похибка складає близько \(\sim 1.7\%\).
%Ця різниця --- це \textbf{кореляційна енергія}:
%\[
%E_{\text{corr}} = E_{\text{exact}} - E_{\text{HF}}
%\]


%% --------------------------------------------------------
\subsection{Порівняння RHF та UHF}
%% --------------------------------------------------------

Оскільки гелій має замкнену оболонку ($N_\alpha = N_\beta$), обидва методи --- \texttt{RHF}
і \texttt{UHF}  --- дають ідентичні результати.
UHF дозволяє $\alpha$ та $\beta$ орбіталям відрізнятися, але тут ця свобода не використовується.

\inputcode{code_5.py}

\begin{commentbox}
UHF може порушувати спінову симетрію (\textit{spin conta\-mination}), особливо для відкрито-оболонкових систем.
Тут же $S^2 = 0$, тобто спінова симетрія зберігається, і RHF $\equiv$ UHF.
\end{commentbox}


%% --------------------------------------------------------
\subsection{Збуджені стани Гелію}
%% --------------------------------------------------------

У збуджених станах один електрон переходить на більш високий рівень (наприклад, $2s$),
а система може існувати в двох спінових конфігураціях:

\begin{itemize}
    \item \textbf{Синглет:} $S=0$ --- антисиметрична за спіном, симетрична за просторовою координатою.
    \item \textbf{Триплет:} $S=1$ --- симетрична за спіном, антисиметрична за просторовою частиною.
\end{itemize}

У PySCF ці стани можна моделювати за допомогою відповідного значення параметра \texttt{spin}
та методу \texttt{ROHF} для частково заповнених оболонок.

\inputcode{code_6.py}

\begin{commentbox}
Метод \texttt{MOM (Maximum Overlap Method)} --- це модифікація самозгодженої процедури HF, у якій вибір зайнятих орбіталей на кожній ітерації здійснюється не за найменшими енергіями, а за критерієм \textbf{максимального перекриття} з орбіталями попередньої ітерації:
\[
O_{ij} = \left| \langle \phi_i^{(n)} \mid \phi_j^{(n-1)} \rangle \right|^2.
\]
Таким чином, MOM «закріплює» бажану орбітальну конфігурацію (наприклад, $1s2s$) і не дозволяє хвильовій функції сповзати назад до основного стану ($1s^2$), що часто трапляється у звичайному HF.

Попри те, що MOM стабілізує збуджені конфігурації, він залишається одно-детермінантним наближенням і не враховує електронної кореляції, тому не належить до post-HF методів.
Отримані стани є лише самозгодженими розв’язками рівнянь HF, а не справжніми власними станами гамільтоніана.
Для точного опису енергій і спектрів збудження слід застосовувати багатоконфігураційні або збурювальні підходи (\texttt{CASSCF}, \texttt{CI}, \texttt{TD-DFT}).
\end{commentbox}



\textbf{Підсумок:}
\begin{itemize}
    \item Гелій --- базовий тест для валідації HF-методу.
    \item Різниця між RHF і експериментом демонструє значення кореляційної енергії.
    \item Збуджені стани потребують спеціальних методів (MOM, CASSCF, TD-DFT).
\end{itemize}