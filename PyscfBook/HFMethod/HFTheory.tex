%% --------------------------------------------------------
\section{Теоретичні основи методу Хартрі-Фока}
%% --------------------------------------------------------

%% --------------------------------------------------------
\subsection{Рівняння Хартрі-Фока}
%% --------------------------------------------------------

Метод Хартрі-Фока (HF) є наближеним методом розв'язання рівняння Шредінгера для багатоелектронних систем. Основна ідея полягає у представленні багатоелектронної хвильової функції у вигляді єдиного детермінанта Слейтера:

\begin{equation}
\Psi(\mathbf{r}_1, \mathbf{r}_2, \ldots, \mathbf{r}_N) = \frac{1}{\sqrt{N!}}
\begin{vmatrix}
\psi_1(\mathbf{r}_1) & \psi_2(\mathbf{r}_1) & \cdots & \psi_N(\mathbf{r}_1) \\
\psi_1(\mathbf{r}_2) & \psi_2(\mathbf{r}_2) & \cdots & \psi_N(\mathbf{r}_2) \\
\vdots & \vdots & \ddots & \vdots \\
\psi_1(\mathbf{r}_N) & \psi_2(\mathbf{r}_N) & \cdots & \psi_N(\mathbf{r}_N)
\end{vmatrix}
\end{equation}

де $\psi_i(\mathbf{r})$ --- спін-орбіталі, які є добутком просторової частини та спінової функції.

Канонічні рівняння Хартрі-Фока мають вигляд:

\begin{equation}
\hat{f} \psi_i = \varepsilon_i \psi_i,
\end{equation}
%
де $\hat{f}$ --- оператор Фока:
\begin{equation}
\hat{f} = \hat{h} + \sum_{j=1}^{N} (\hat{J}_j - \hat{K}_j),
\end{equation}
Тут $\hat{h}$ --- одноелектронний оператор (кінетична енергія + притягання до ядра),
$\hat{J}_j$ --- кулонівський оператор,
$\hat{K}_j$ --- обмінний оператор.

%% --------------------------------------------------------
\subsection{Енергія Хартрі-Фока}
%% --------------------------------------------------------

Повна енергія системи у методі HF:
\begin{equation}
E_{HF} = \sum_{i=1}^{N} h_{ii} + \frac{1}{2} \sum_{i,j}^{N} (J_{ij} - K_{ij}) + V_{NN},
\end{equation}
де:
\begin{itemize}
    \item $h_{ii}$ --- одноелектронні інтеграли;
    \item $J_{ij}$ --- кулонівські інтеграли;
    \item $K_{ij}$ --- обмінні інтеграли;
    \item $V_{NN}$ --- енергія міжядерного відштовхування (для атомів $V_{NN}=0$).
\end{itemize}

\paragraph{Віральне співвідношення.}
Для будь-якої стаціонарної хвильової функції (зокрема, у наближенні HF) повинно виконуватись \textbf{віральне співвідношення}:
\[
2 \langle T \rangle + \langle V \rangle = 0,
\]
або еквівалентно:
\[
\frac{\langle V \rangle}{\langle T \rangle} = -2.
\]
Цей критерій є важливим тестом коректності збіжного SCF-розв’язку.

%% --------------------------------------------------------
\subsection{Матриця густини та енергетичні складові}
%% --------------------------------------------------------

Для практичної реалізації HF-методу у базисному представленні вводиться \textbf{матриця густини}:
\begin{equation}
D_{\mu\nu} = 2 \sum_{i}^{\text{occ}} C_{\mu i} C_{\nu i},
\end{equation}
де $C_{\mu i}$ --- коефіцієнти розкладу $i$-ї молекулярної орбіталі за базисом атомних орбіталей,
а множник $2$ враховує заповнення двома спінами (для RHF).

Оператор середнього значення будь-якої одноелектронної величини $\hat{O}$ можна записати у матричній формі:
\begin{equation}
\langle \hat{O} \rangle = \mathrm{Tr}(D\, O),
\end{equation}
що значно спрощує обчислення середніх значень у базисі атомних орбіталей.

Одноелектронний гамільтоніан:
\begin{equation}
h_{\mu\nu} = T_{\mu\nu} + V_{\mu\nu}^{\text{nuc}},
\end{equation}
де $T_{\mu\nu}$ --- інтеграли кінетичної енергії,
а $V_{\mu\nu}^{\text{nuc}}$ --- інтеграли притягання електрона до ядра.

Матриця Фока визначається як:
\begin{equation}
F_{\mu\nu} = h_{\mu\nu} + \sum_{\lambda\sigma} D_{\lambda\sigma}
\left[ (\mu\nu|\lambda\sigma) - \tfrac{1}{2} (\mu\lambda|\nu\sigma) \right].
\end{equation}

Тоді загальна енергія Гартрі–Фока може бути записана компактно:
\begin{equation}
E_{\text{HF}} = \sum_{\mu\nu} D_{\mu\nu} \left( h_{\mu\nu} + \tfrac{1}{2} F_{\mu\nu} \right).
\end{equation}

У розгорнутому вигляді можна виділити окремі внески:
\[
\begin{aligned}
E_{\text{kin}} &= \sum_{\mu\nu} D_{\mu\nu} T_{\mu\nu}, \\[4pt]
E_{\text{nuc}} &= \sum_{\mu\nu} D_{\mu\nu} V_{\mu\nu}^{\text{nuc}}, \\[4pt]
E_{\text{ee}}  &= \tfrac{1}{2} \sum_{\mu\nu} D_{\mu\nu} (F_{\mu\nu} - h_{\mu\nu}),
\end{aligned}
\]
і, відповідно, повна енергія:
\[
E_{\text{tot}} = E_{\text{kin}} + E_{\text{nuc}} + E_{\text{ee}} + V_{NN}.
\]

Таким чином, матриця густини $D$ є центральним об’єктом, що зберігає всю інформацію про одноелектронні властивості системи.
У коді \texttt{PySCF} вона створюється викликом \inlinecode{dm = mf.make\_rdm1()}, а середні значення обчислюються як добутки $D$ на відповідні матриці інтегралів.


%% --------------------------------------------------------
\subsection{Варіанти методу Хартрі-Фока}
%% --------------------------------------------------------

%% --------------------------------------------------------
\subsubsection{Restricted Hartree-Fock (RHF)}
%% --------------------------------------------------------

RHF використовується для систем з замкненими оболонками, де всі електрони спарені:

\begin{equation}
\psi_i^\alpha(\mathbf{r}) = \psi_i^\beta(\mathbf{r}) = \phi_i(\mathbf{r})
\end{equation}

Підходить для: He, Be, Ne, Mg, Ar, Ca, Zn (у синглетних станах).

%% --------------------------------------------------------
\subsubsection{Unrestricted Hartree-Fock (UHF)}
%% --------------------------------------------------------

UHF дозволяє різні просторові орбіталі для альфа та бета спінів:

\begin{equation}
\psi_i^\alpha(\mathbf{r}) \neq \psi_i^\beta(\mathbf{r})
\end{equation}

Підходить для: всіх відкритих систем (H, Li, B, C, N, O, F та їх іони).


\subsubsection{Restricted Open-shell Hartree-Fock (ROHF)}

ROHF --- компроміс між RHF та UHF. Спарені електрони описуються однаковими орбіталями, неспарені --- різними:

\begin{equation}
\begin{cases}
\psi_i^\alpha = \psi_i^\beta & \text{для спарених} \\
\psi_i^\alpha \neq \psi_i^\beta & \text{для неспарених}
\end{cases}
\end{equation}