%% --------------------------------------------------------
\section{Розрахунок атома Гідрогену}
%% --------------------------------------------------------

Атом Гідрогену є найпростішим квантовим об'єктом, який можна описати в рамках не лише хімії, а й фундаментальної квантової механіки. Його Гамільтоніан має вигляд:
\[
\hat{H} = -\frac{1}{2}\nabla^2 - \frac{1}{r},
\]
де \(-\frac{1}{2}\nabla^2\) --- оператор кінетичної енергії електрона, а \(-\frac{1}{r}\) --- потенціал кулонівської взаємодії між електроном і ядром.
Ця система має аналітичне рішення, і енергія основного стану дорівнює
\[
E_1 = -\frac{1}{2}\ \text{Ha}.
\]
Для атома Гідрогену це рішення можна отримати навіть чисельно в PySCF, що дозволяє перевірити точність обраного базисного набору та методів квантово-хімічних розрахунків.

%% --------------------------------------------------------
\subsection{Особливості одноелектронної системи}
%% --------------------------------------------------------

Атом Гідрогену є одноелектронною системою, тому метод Гартрі–Фока (HF) не містить жодних апроксимацій, окрім обмежень базисного набору.
У звичайних багатoелектронних атомах HF наближає взаємодію електронів середнім потенціалом, але для H відсутнє електрон–електронне відштовхування, тож метод дає \emph{точну} хвильову функцію для обраного базису.

\inputcode{code_1.py}

Отримане значення енергії наближається до \(-0.5\) Ha, але точність залежить від базису. STO-3G --- це мінімальний базис, де кожна орбіталь апроксимується трьома гаусовими функціями, тому результат має невелику похибку (близько \(10^{-3}\) Ha).

%% --------------------------------------------------------
\subsection{Залежність від базисного набору}
%% --------------------------------------------------------

Базисний набір визначає якість апроксимації хвильової функції.
Чим більший набір, тим ближче розрахована енергія до аналітичного результату.
Для атома Гідрогену ця збіжність особливо показова, бо ми можемо порівняти з точним розв’язком.

У коді нижче порівнюються кілька популярних базисів, від найменшого STO-3G до розширених кореляційно-узгоджених наборів cc-pV5Z. Для кожного базису обчислюється енергія та похибка відносно \(-0.5\) Ha.

\inputcode{code_2.py}

З графіка видно, що зі збільшенням кількості базисних функцій енергія швидко наближається до точного значення \(-0.5\) Ha.
Набори типу \texttt{cc-pVQZ} або \texttt{cc-pV5Z} практично дають збіжність до повного базисного ліміту (CBS limit).

%% --------------------------------------------------------
\subsection{Аналіз орбіталей}
%% --------------------------------------------------------

Хоча в атома Гідрогену існує лише одна заповнена орбіталь (1s), PySCF дозволяє вивести енергетичні рівні для всіх функцій базису, а також дослідити матрицю густини.
Цей підхід зручний для демонстрації структури HF-розрахунку.

\inputcode{code_3.py}

Матриця густини (\texttt{dm}) відображає заповнення орбіталей. Її слід дорівнює кількості електронів (тут --- 1).
Такі розрахунки є основою для подальшого аналізу електронної густини, побудови орбіталей і візуалізації електронної хмари.

%% --------------------------------------------------------
\subsection{Висновки}
%% --------------------------------------------------------

\begin{itemize}
    \item Для атома Гідрогену метод Гартрі–Фока є \textbf{точним}, бо немає взаємодії між електронами.
    \item Основна похибка виникає через обмеження базисного набору.
    \item Зі збільшенням кількості базисних функцій енергія швидко збігається до точного значення \(-0.5\) Ha.
    \item PySCF дозволяє досліджувати вплив базису, аналізувати орбіталі, матриці густини та підготовлює ґрунт для подальшого розгляду багатoелектронних систем.
\end{itemize}