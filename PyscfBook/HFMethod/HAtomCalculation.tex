% !TeX program = lualatex
% !TeX encoding = utf8
% !TeX spellcheck = uk_UA
% !TeX root =../PyscfBook.tex


%% --------------------------------------------------------
\section{Розрахунок атома Гідрогену}
%% --------------------------------------------------------

Атом Гідрогену є найпростішим квантовим об'єктом, який можна описати в рамках не лише хімії, а й фундаментальної квантової механіки. Його Гамільтоніан має вигляд:
\[
\hat{H} = -\frac{1}{2}\nabla^2 - \frac{1}{r},
\]
де \(-\frac{1}{2}\nabla^2\) --- оператор кінетичної енергії електрона, а \(-\frac{1}{r}\) --- потенціал кулонівської взаємодії між електроном і ядром.
Ця система має аналітичне рішення, і енергія основного стану дорівнює
\[
E_1 = -\frac{1}{2}\ \text{Ha}.
\]
Для атома Гідрогену це рішення можна отримати навіть чисельно в PySCF, що дозволяє перевірити точність обраного базисного набору та методів квантово-хімічних розрахунків.

%% --------------------------------------------------------
\subsection{Особливості одноелектронної системи}
%% --------------------------------------------------------

Атом Гідрогену є одноелектронною системою, тому метод Гартрі–Фока (HF) не містить жодних апроксимацій, окрім обмежень базисного набору.
У звичайних багатoелектронних атомах HF наближає взаємодію електронів середнім потенціалом, але для H відсутнє електрон–електронне відштовхування, тож метод дає \emph{точну} хвильову функцію для обраного базису.

\inputcode{code_1.py}

Отримане значення енергії наближається до \(-0.5\) Ha, але точність залежить від базису. STO-3G --- це мінімальний базис, де кожна орбіталь апроксимується трьома гаусовими функціями, тому результат має невелику похибку (близько \(10^{-3}\) Ha).

%% --------------------------------------------------------
\subsection{Залежність від базисного набору}
%% --------------------------------------------------------

Базисний набір визначає якість апроксимації хвильової функції.
Чим більший набір, тим ближче розрахована енергія до аналітичного результату.
Для атома Гідрогену ця збіжність особливо показова, бо ми можемо порівняти з точним розв’язком.

У коді нижче порівнюються кілька популярних базисів, від найменшого STO-3G до розширених кореляційно-узгоджених наборів cc-pV5Z. Для кожного базису обчислюється енергія та похибка відносно \(-0.5\) Ha.

\inputcode{code_2.py}

%---------------------------------------------------------
\begin{figure}[h!]\centering
\includegraphics[width=\linewidth]{\currfiledir/h_atom_basis_convergence.pdf}
\caption{Залежність енергії атома \ce{H} від вибору базису.}
\label{pic:h_atom_basis_convergence}
\end{figure}
%---------------------------------------------------------


З графіка (рис.~\ref{pic:h_atom_basis_convergence}) видно, що зі збільшенням кількості базисних функцій енергія швидко наближається до точного значення \(-0.5\) Ha.
Набори типу \texttt{cc-pVQZ} або \texttt{cc-pV5Z} практично дають збіжність до повного базисного ліміту (CBS limit).

%% --------------------------------------------------------
\subsection{Аналіз орбіталей}
%% --------------------------------------------------------

Хоча в атома Гідрогену існує лише одна заповнена орбіталь (1s), PySCF дозволяє вивести енергетичні рівні для всіх функцій базису, а також дослідити матрицю густини.
Цей підхід зручний для демонстрації структури HF-розрахунку.

\inputcode{code_3.py}

Виведення програми:
\begin{minted}{text}
Орбітальні енергії (альфа-спін):
--------------------------------------------------
 1. 0 H 1s              :  -0.499810 Ha (occ)
 2. 0 H 2s              :   0.025806 Ha (virt)
 3. 0 H 3s              :   0.298457 Ha (virt)
 4. 0 H 2px             :   0.298457 Ha (virt)
 5. 0 H 2py             :   0.298457 Ha (virt)
 6. 0 H 2pz             :   1.886323 Ha (virt)
 7. 0 H 3px             :   2.824504 Ha (virt)
 8. 0 H 3py             :   2.824504 Ha (virt)
 9. 0 H 3pz             :   2.824504 Ha (virt)
10. 0 H 3dxy            :   2.824504 Ha (virt)
11. 0 H 3dyz            :   2.824504 Ha (virt)
12. 0 H 3dz^2           :   3.199249 Ha (virt)
13. 0 H 3dxz            :   3.199249 Ha (virt)
14. 0 H 3dx2-y2         :   3.199249 Ha (virt)

Енергія 1s орбіталі: -0.49980981 Ha
Теоретична енергія:  -0.50000000 Ha

Матриця густини (альфа): (14, 14)
Слід dm: 0.5 (має дорівнювати 1)
\end{minted}

Матриця густини (\texttt{dm}) відображає заповнення орбіталей. Її слід має дорівнювати кількості електронів.
\begin{commentbox}
Однак у даному випадку ми бачимо $\mathrm{Tr}(\mathrm{dm}) = 0.5$.
Це не помилка: PySCF формує окремі матриці густини для альфа- та бета-спінів.
Для одноконфігураційного стану з одним електроном (1s, альфа-спін) слід матриці густини $\mathrm{dm}_\alpha = 0.5$, оскільки PySCF нормує хвильову функцію на повну густину ($\alpha$ + $\beta$), а не на окремий спін.

Отже,
\[
\mathrm{Tr}(\mathrm{dm}_\alpha) + \mathrm{Tr}(\mathrm{dm}_\beta) = N_{\text{електронів}} = 1,
\]
а тому
\[
\mathrm{Tr}(\mathrm{dm}_\alpha) = 0.5, \quad \mathrm{Tr}(\mathrm{dm}_\beta) = 0.
\]

Якщо обчислити повну густину через \inlinecode{mf.make\_rdm1(mo, mo\_occ, spin=None)}, то отримаємо правильне значення $1$.

\end{commentbox}
Проведені розрахунки є основою для подальшого аналізу електронної густини, побудови орбіталей і візуалізації електронної хмари.


%% --------------------------------------------------------
\subsection{Висновки}
%% --------------------------------------------------------

\begin{itemize}
    \item Для атома Гідрогену метод Гартрі–Фока є \textbf{точним}, бо немає взаємодії між електронами.
    \item Основна похибка виникає через обмеження базисного набору.
    \item Зі збільшенням кількості базисних функцій енергія швидко збігається до точного значення \(-0.5\) Ha.
    \item PySCF дозволяє досліджувати вплив базису, аналізувати орбіталі, матриці густини та підготовлює ґрунт для подальшого розгляду багатoелектронних систем.
\end{itemize}