% !TeX program = lualatex
% !TeX encoding = utf8
% !TeX spellcheck = uk_UA
% !TeX root =../PyscfBook.tex

%=========================================================
\Opensolutionfile{answer}[\currfilebase/\currfilebase-Answers]
\chapter{Метод Хартрі-Фока для атомів}\label{\currfilebase}
%=========================================================

%% --------------------------------------------------------
\section{Теоретичні основи методу Хартрі-Фока}
%% --------------------------------------------------------

%% --------------------------------------------------------
\subsection{Рівняння Хартрі-Фока}
%% --------------------------------------------------------

Метод Хартрі-Фока (HF) є наближеним методом розв'язання рівняння Шредінгера для багатоелектронних систем. Основна ідея полягає у представленні багатоелектронної хвильової функції у вигляді єдиного детермінанта Слейтера:

\begin{equation}
\Psi(\mathbf{r}_1, \mathbf{r}_2, \ldots, \mathbf{r}_N) = \frac{1}{\sqrt{N!}}
\begin{vmatrix}
\psi_1(\mathbf{r}_1) & \psi_2(\mathbf{r}_1) & \cdots & \psi_N(\mathbf{r}_1) \\
\psi_1(\mathbf{r}_2) & \psi_2(\mathbf{r}_2) & \cdots & \psi_N(\mathbf{r}_2) \\
\vdots & \vdots & \ddots & \vdots \\
\psi_1(\mathbf{r}_N) & \psi_2(\mathbf{r}_N) & \cdots & \psi_N(\mathbf{r}_N)
\end{vmatrix}
\end{equation}

де $\psi_i(\mathbf{r})$ --- спін-орбіталі, які є добутком просторової частини та спінової функції.

Канонічні рівняння Хартрі-Фока мають вигляд:

\begin{equation}
\hat{f} \psi_i = \varepsilon_i \psi_i
\end{equation}
%
де $\hat{f}$ --- оператор Фока:

\begin{equation}
\hat{f} = \hat{h} + \sum_{j=1}^{N} (\hat{J}_j - \hat{K}_j)
\end{equation}

Тут $\hat{h}$ --- одноелектронний оператор (кінетична енергія + притягання до ядра), $\hat{J}_j$ --- кулонівський оператор, $\hat{K}_j$ --- обмінний оператор.

%% --------------------------------------------------------
\subsection{Енергія Хартрі-Фока}
%% --------------------------------------------------------

Повна енергія системи у методі HF:

\begin{equation}
E_{HF} = \sum_{i=1}^{N} h_{ii} + \frac{1}{2} \sum_{i,j}^{N} (J_{ij} - K_{ij}) + V_{NN}
\end{equation}

де:
\begin{itemize}
    \item $h_{ii}$ --- одноелектронні інтеграли
    \item $J_{ij}$ --- кулонівські інтеграли
    \item $K_{ij}$ --- обмінні інтеграли
    \item $V_{NN}$ --- енергія міжядерного відштовхування (= 0 для атомів)
\end{itemize}

\paragraph{Віральне співвідношення.}
Для будь-якої стаціонарної хвильової функції (зокрема, у наближенні HF) повинно виконуватись \textbf{віральне співвідношення}:
\[
2 \langle T \rangle + \langle V \rangle = 0,
\]
або еквівалентно:
\[
\frac{\langle V \rangle}{\langle T \rangle} = -2.
\]
Цей критерій є важливим тестом коректності збіжного SCF-розв’язку.

%% --------------------------------------------------------
\subsection{Варіанти методу Хартрі-Фока}
%% --------------------------------------------------------

%% --------------------------------------------------------
\subsubsection{Restricted Hartree-Fock (RHF)}
%% --------------------------------------------------------

RHF використовується для систем з замкненими оболонками, де всі електрони спарені:

\begin{equation}
\psi_i^\alpha(\mathbf{r}) = \psi_i^\beta(\mathbf{r}) = \phi_i(\mathbf{r})
\end{equation}

Підходить для: He, Be, Ne, Mg, Ar, Ca, Zn (у синглетних станах).

%% --------------------------------------------------------
\subsubsection{Unrestricted Hartree-Fock (UHF)}
%% --------------------------------------------------------

UHF дозволяє різні просторові орбіталі для альфа та бета спінів:

\begin{equation}
\psi_i^\alpha(\mathbf{r}) \neq \psi_i^\beta(\mathbf{r})
\end{equation}

Підходить для: всіх відкритих систем (H, Li, B, C, N, O, F та їх іони).


\subsubsection{Restricted Open-shell Hartree-Fock (ROHF)}

ROHF --- компроміс між RHF та UHF. Спарені електрони описуються однаковими орбіталями, неспарені --- різними:

\begin{equation}
\begin{cases}
\psi_i^\alpha = \psi_i^\beta & \text{для спарених} \\
\psi_i^\alpha \neq \psi_i^\beta & \text{для неспарених}
\end{cases}
\end{equation}


%% --------------------------------------------------------
\section{Розрахунок атома Гідрогену}
%% --------------------------------------------------------

Атом Гідрогену є найпростішим квантовим об'єктом, який можна описати в рамках не лише хімії, а й фундаментальної квантової механіки. Його Гамільтоніан має вигляд:
\[
\hat{H} = -\frac{1}{2}\nabla^2 - \frac{1}{r},
\]
де \(-\frac{1}{2}\nabla^2\) --- оператор кінетичної енергії електрона, а \(-\frac{1}{r}\) --- потенціал кулонівської взаємодії між електроном і ядром.
Ця система має аналітичне рішення, і енергія основного стану дорівнює
\[
E_1 = -\frac{1}{2}\ \text{Ha}.
\]
Для атома Гідрогену це рішення можна отримати навіть чисельно в PySCF, що дозволяє перевірити точність обраного базисного набору та методів квантово-хімічних розрахунків.

%% --------------------------------------------------------
\subsection{Особливості одноелектронної системи}
%% --------------------------------------------------------

Атом Гідрогену є одноелектронною системою, тому метод Гартрі–Фока (HF) не містить жодних апроксимацій, окрім обмежень базисного набору.
У звичайних багатoелектронних атомах HF наближає взаємодію електронів середнім потенціалом, але для H відсутнє електрон–електронне відштовхування, тож метод дає \emph{точну} хвильову функцію для обраного базису.

\inputcode{code_1.py}

Отримане значення енергії наближається до \(-0.5\) Ha, але точність залежить від базису. STO-3G --- це мінімальний базис, де кожна орбіталь апроксимується трьома гаусовими функціями, тому результат має невелику похибку (близько \(10^{-3}\) Ha).

%% --------------------------------------------------------
\subsection{Залежність від базисного набору}
%% --------------------------------------------------------

Базисний набір визначає якість апроксимації хвильової функції.
Чим більший набір, тим ближче розрахована енергія до аналітичного результату.
Для атома Гідрогену ця збіжність особливо показова, бо ми можемо порівняти з точним розв’язком.

У коді нижче порівнюються кілька популярних базисів, від найменшого STO-3G до розширених кореляційно-узгоджених наборів cc-pV5Z. Для кожного базису обчислюється енергія та похибка відносно \(-0.5\) Ha.

\inputcode{code_2.py}

З графіка видно, що зі збільшенням кількості базисних функцій енергія швидко наближається до точного значення \(-0.5\) Ha.
Набори типу \texttt{cc-pVQZ} або \texttt{cc-pV5Z} практично дають збіжність до повного базисного ліміту (CBS limit).

%% --------------------------------------------------------
\subsection{Аналіз орбіталей}
%% --------------------------------------------------------

Хоча в атома Гідрогену існує лише одна заповнена орбіталь (1s), PySCF дозволяє вивести енергетичні рівні для всіх функцій базису, а також дослідити матрицю густини.
Цей підхід зручний для демонстрації структури HF-розрахунку.

\inputcode{code_3.py}

Матриця густини (\texttt{dm}) відображає заповнення орбіталей. Її слід дорівнює кількості електронів (тут --- 1).
Такі розрахунки є основою для подальшого аналізу електронної густини, побудови орбіталей і візуалізації електронної хмари.

%% --------------------------------------------------------
\subsection{Висновки}
%% --------------------------------------------------------

\begin{itemize}
    \item Для атома Гідрогену метод Гартрі–Фока є \textbf{точним}, бо немає взаємодії між електронами.
    \item Основна похибка виникає через обмеження базисного набору.
    \item Зі збільшенням кількості базисних функцій енергія швидко збігається до точного значення \(-0.5\) Ha.
    \item PySCF дозволяє досліджувати вплив базису, аналізувати орбіталі, матриці густини та підготовлює ґрунт для подальшого розгляду багатoелектронних систем.
\end{itemize}




%% ========================================================
\section{Розрахунок атома Гелію}
%% ========================================================

%% --------------------------------------------------------
\subsection{Двоелектронна система}
%% --------------------------------------------------------

Атом гелію є фундаментальним прикладом для демонстрації методів \textbf{Хартрі–Фока (HF)}. Два електрони гелію мають протилежні спіни й заповнюють одну й ту саму орбіталь $1s$, тому це \emph{замкнена оболонка}
зі спіном $S = 0$ (синглет, $M = 1$).

%Це перша система, де проявляється \textbf{електрон–електронна кореляція}, відсутня в атомі водню.

Після завершення SCF-розрахунку \texttt{PySCF} надає доступ до одноелектронних інтегралів
(\inlinecode{int1e\_kin}, \inlinecode{int1e\_nuc}) та до двоелектронної частини (\inlinecode{get\_veff}).
Це дозволяє безпосередньо обчислити всі компоненти енергії та перевірити віральне співвідношення.

\inputcode{HeHF.py}

\textbf{Коментар.}
Якщо розрахунок збіжний і фізично коректний, то відношення
\(\langle V \rangle / \langle T \rangle \approx -2.00 \pm 0.01\),
що підтверджує виконання віральної теореми.
Будь-яке суттєве відхилення (наприклад, $-1.7$ чи $-2.3$) свідчить про помилку збіжності або неадекватність базисного набору.

%
%
%
%\inputcode{code_4.py}
%
%\textbf{Коментар.}
%Метод RHF нехтує кореляцією між електронами, тобто вважає, що хвильова функція є добутком одноелектронних орбіталей.
%Тому енергія HF завжди \emph{вище} (менш від’ємна), ніж точна. Для гелію похибка складає близько \(\sim 1.7\%\).
%Ця різниця --- це \textbf{кореляційна енергія}:
%\[
%E_{\text{corr}} = E_{\text{exact}} - E_{\text{HF}}
%\]


%% --------------------------------------------------------
\subsection{Порівняння RHF та UHF}
%% --------------------------------------------------------

Оскільки гелій має замкнену оболонку ($N_\alpha = N_\beta$), обидва методи --- \textbf{RHF} (restricted HF)
і \textbf{UHF} (unrestricted HF) --- дають ідентичні результати.
UHF дозволяє $\alpha$ та $\beta$ орбіталям відрізнятися, але тут ця свобода не використовується.

\inputcode{code_5.py}

\textbf{Коментар.}
UHF може порушувати спінову симетрію (\textit{spin conta\-mination}), особливо для відкрито-оболонкових систем.
Тут же $S^2 = 0$, тобто спінова симетрія зберігається, і RHF $\equiv$ UHF.


%% --------------------------------------------------------
\subsection{Збуджені стани Гелію}
%% --------------------------------------------------------

У збуджених станах один електрон переходить на більш високий рівень (наприклад, $2s$),
а система може існувати в двох спінових конфігураціях:

\begin{itemize}
    \item \textbf{Синглет:} $S=0$ --- антисиметрична за спіном, симетрична за просторовою координатою.
    \item \textbf{Триплет:} $S=1$ --- симетрична за спіном, антисиметрична за просторовою частиною.
\end{itemize}

У PySCF ці стани можна моделювати за допомогою відповідного значення параметра \texttt{spin}
та методу \texttt{ROHF} для частково заповнених оболонок.

\inputcode{code_6.py}

\textbf{Коментар.}
Метод \texttt{MOM (Maximum Overlap Method)} --- це модифікація самозгодженої процедури HF, у якій вибір зайнятих орбіталей на кожній ітерації здійснюється не за найменшими енергіями, а за критерієм \textbf{максимального перекриття} з орбіталями попередньої ітерації:
\[
O_{ij} = \left| \langle \phi_i^{(n)} \mid \phi_j^{(n-1)} \rangle \right|^2.
\]
Таким чином, MOM «закріплює» бажану орбітальну конфігурацію (наприклад, $1s2s$) і не дозволяє хвильовій функції сповзати назад до основного стану ($1s^2$), що часто трапляється у звичайному HF.

Попри те, що MOM стабілізує збуджені конфігурації, він залишається одно-детермінантним наближенням і не враховує електронної кореляції, тому не належить до post-HF методів.
Отримані стани є лише самозгодженими розв’язками рівнянь HF, а не справжніми власними станами гамільтоніана.
Для точного опису енергій і спектрів збудження слід застосовувати багатоконфігураційні або збурювальні підходи (\texttt{CASSCF}, \texttt{CI}, \texttt{TD-DFT}).



\textbf{Підсумок:}
\begin{itemize}
    \item Гелій --- базовий тест для валідації HF-методу.
    \item Різниця між RHF і експериментом демонструє значення кореляційної енергії.
    \item Збуджені стани потребують спеціальних методів (MOM, CASSCF, TD-DFT).
\end{itemize}


%% --------------------------------------------------------
\section{Атоми другого періоду (Li--Ne)}
%% --------------------------------------------------------

%% --------------------------------------------------------
\subsection{Літій (Li, Z=3)}
%% --------------------------------------------------------

Атом Літію --- перший багатoелектронний атом, у якому проявляється \textbf{неспарений електрон} та \textbf{спінова поляризація}.
Електронна конфігурація:
\[
\text{Li: } 1s^2 2s^1.
\]
Два перші електрони утворюють замкнену оболонку (1s), а третій --- одинокий у підоболонці \(2s\), що зумовлює мультиплетність \(2S\) (спін \(S = \frac{1}{2}\)).

Через наявність неспареного електрона метод \textbf{UHF} (Unrestricted Hartree–Fock) є природним вибором.
UHF дозволяє альфа- та бета-орбіталям відрізнятися, тобто хвильова функція не змушена мати однакову просторову форму для різних спінів.

\inputcode{code_7.py}

\paragraph{Коментарі.}
\begin{itemize}
  \item Значення \(\langle S^2 \rangle \approx 0.75\) свідчить про правильний спіновий стан подвійності (мультиплетність \(2S+1=2\)).
  \item Наявність різниці між енергіями альфа- та бета-орбіталей демонструє спінову асиметрію.
  \item Метод UHF може частково порушувати симетрію (так звана \textit{spin contamination}), але для Li це незначно.
\end{itemize}

Енергія, отримана методом HF, для Li становить близько \(-7.432\) Ha у базисі cc-pVDZ, тоді як експериментальна енергія основного стану (повна) --- близько \(-7.478\) Ha.
Таким чином, кореляційна похибка становить близько \(0.046\) Ha (приблизно 1.25 eV).

Цей приклад є першим, де проявляється \emph{кореляція електронів}, яку HF не враховує. В подальших розділах буде показано, як методи MP2, CI та CC враховують ці ефекти.

%% --------------------------------------------------------
\subsection{Берилій (Be, Z=4)}
%% --------------------------------------------------------

Атом Берилію має електронну конфігурацію
\[
\text{Be: } 1s^2 2s^2.
\]
Тут усі орбіталі парні, тому спінова поляризація відсутня, і зручно використовувати \texttt{RHF} (Restricted Hart\-ree–Fock).
Обидва електрони у підоболонці \(2s\) мають протилежні спіни, тож система має мультиплетність \(1S\) (синглетний стан).

\inputcode{code_8.py}

\paragraph{Обговорення.}
\begin{itemize}
  \item Для Be HF-енергія виходить приблизно \(-14.573\) Ha (у базисі cc-pVTZ), тоді як експериментальна --- \(-14.667\) Ha.
  \item Різниця близько \(0.094\) Ha (\(\approx 2.6\) eV) --- це \textbf{кореляційна енергія}, тобто внесок взаємодії електронів, не врахований у HF.
  \item У Берилія ефекти електронної кореляції вже досить значні, адже електрони в орбіталі \(2s\) взаємодіють один з одним.
  \item Цей випадок є гарною ілюстрацією межі застосування методу Гартрі–Фока.
\end{itemize}

\paragraph{Висновки для атомів Li і Be.}
\begin{itemize}
  \item Li демонструє появу неспареного електрона і спінової поляризації (UHF необхідний).
  \item Be --- перший приклад, де \textbf{електронна кореляція} дає помітну похибку енергії.
  \item PySCF дозволяє наочно дослідити вплив базису, спіну, симетрії та методів на точність енергії.
\end{itemize}


%% --------------------------------------------------------
\subsection{Бор--Неон: систематичне дослідження}
%% --------------------------------------------------------

Після розгляду окремих атомів Літію та Берилію доцільно перейти до систематичного аналізу всіх атомів другого періоду (від Бору до Неону).
У цих атомах поступово заповнюється \(2p\)-підрівень, і змінюється як спінова мультиплетність, так і форма електронної густини.
Метод Гартрі–Фока дозволяє побачити, як змінюється енергія системи при збільшенні числа електронів, а також як проявляється спінова поляризація у відкритих оболонках.

\paragraph{Електронні конфігурації.}
\[
\begin{array}{llcl}
\text{Li:} & [\text{He}]\,2s^1 & \quad & {}^2S \\
\text{Be:} & [\text{He}]\,2s^2 & & {}^1S \\
\text{B:}  & [\text{He}]\,2s^2 2p^1 & & {}^2P \\
\text{C:}  & [\text{He}]\,2s^2 2p^2 & & {}^3P \\
\text{N:}  & [\text{He}]\,2s^2 2p^3 & & {}^4S \\
\text{O:}  & [\text{He}]\,2s^2 2p^4 & & {}^3P \\
\text{F:}  & [\text{He}]\,2s^2 2p^5 & & {}^2P \\
\text{Ne:} & [\text{He}]\,2s^2 2p^6 & & {}^1S
\end{array}
\]
Як видно, кількість неспарених електронів збільшується від Бору до Нітрогену, а потім зменшується до Неону, що зумовлює зміну спінового стану та мультиплетності.

\paragraph{Мета дослідження.}
Провести розрахунок енергій Гартрі–Фока для атомів другого періоду в однаковому базисі \texttt{cc-pVDZ}, оцінити правильність спінового стану (через \(\langle S^2 \rangle\)) та проаналізувати систематичні тенденції.

\inputcode{code_9.py}

\paragraph{Коментарі до коду.}
\begin{itemize}
  \item Для кожного атома автоматично обирається тип SCF: \texttt{RHF} (для замкнених оболонок, $S=0$) або \texttt{UHF} (для відкритих).
  \item Параметр \inlinecode{spin} у PySCF означає різницю між кількістю альфа- та бета-електронів:
        \(\text{spin} = N_\alpha - N_\beta = 2S.\)
  \item Розрахунок \(\langle S^2 \rangle\) дозволяє перевірити правильність спінового стану. Для ідеального випадку значення має збігатися з теоретичним \(S(S+1)\).
  \item У файлі \inlinecode{second\_period\_hf.npz} зберігаються всі енергії для подальшого аналізу або побудови графіків.
\end{itemize}

\paragraph{Очікувані тенденції.}
\begin{enumerate}
  \item Повна енергія атома зменшується (стає більш негативною) із зростанням атомного номера $Z$.
  \item Енергія спостерігає стрибки на межі заповнення оболонок: при переходах Be→B, N→O, F→Ne.
  \item Спінова мультиплетність відображає заповнення $2p$-орбіталей згідно з \textbf{правилом Гунда} --- максимальний спін у середині підоболонки (для N).
  \item Значення \(\langle S^2 \rangle\) для UHF повинні бути близькі до теоретичних, але можуть мати невеликі відхилення через \textit{spin contamination}.
\end{enumerate}

\paragraph{Фізичне узагальнення.}
Цей розрахунок демонструє фундаментальну властивість методу Гартрі–Фока:
він добре описує загальну структуру енергетичних рівнів і тенденції в періодичній таблиці, але не враховує електронну кореляцію, через що абсолютні значення енергії мають систематичну похибку.

%\paragraph{Подальші кроки.}
%Наступним логічним етапом є додавання пост-Гартрі–Фок методів (MP2, CI, CCSD), щоб показати, як кореляція електронів уточнює енергії і дозволяє досягати експериментальної точності.


%% --------------------------------------------------------
\subsection{Енергії іонізації}
%% --------------------------------------------------------

Іонізаційна енергія є однією з фундаментальних характеристик атома, що описує енерговитрати на видалення одного електрона.
У квантово-хімічних розрахунках її визначають як різницю повних енергій катіона \(E(A^+)\) і нейтрального атома \(E(A)\):
\[
I = E(A^+) - E(A)
\]
де енергії \(E\) обчислюються в межах методу Гартрі–Фока або його узагальнень.
Для ізольованих атомів другого періоду така процедура дозволяє отримати якісно правильну послідовність енергій іонізації, навіть попри те, що метод HF систематично недооцінює абсолютні значення через відсутність кореляційних ефектів.


\inputcode{code_10.py}

Результатом буде таблиця іонізаційних енергій у електронвольтах.
Порівнюючи їх з експериментальними, можна оцінити якість базисного набору та межі методу Гартрі–Фока.
%Для більш точних результатів використовують після-HF методи, наприклад MP2 або CCSD, які враховують електронну кореляцію.

%% --------------------------------------------------------
\section{Аналіз енергій та орбіталей}
%% --------------------------------------------------------


%% --------------------------------------------------------
\subsection{Енергетична діаграма орбіталей}
%% --------------------------------------------------------


Енергетичні діаграми орбіталей є наочним способом представлення енергетичного спектру молекулярних орбіталей (МО), що виникають у результаті розв’язання рівнянь Гартрі–Фока.


\inputcode{code_11.py}

%% --------------------------------------------------------
\subsection{Розподіл електронної густини}
%% --------------------------------------------------------

Розподіл електронної густини $\rho(\mathbf{r})$ є однією з ключових величин у квантовій хімії.
Саме ця функція визначає, де саме в просторі «перебувають» електрони атома або молекули, і, отже, пояснює більшість хімічних і спектроскопічних властивостей системи.

Згідно з однодетермінантним наближенням Гартрі–Фока, електронна густина визначається як
\[
\rho(\mathbf{r}) = \sum_{i}^{\text{occ}} |\psi_i(\mathbf{r})|^2,
\]
де $\psi_i(\mathbf{r})$ --- орбіталі, а сума береться по всіх зайнятих орбіталях.

У програмному пакеті \texttt{PySCF} густина може бути обчислена як на сітці, так і в окремих точках простору, що дозволяє візуалізувати просторовий розподіл електронів.

%% --------------------------------------------------------
\subsubsection{Обчислення густини вздовж осі $z$}
%% --------------------------------------------------------

У наведеному прикладі реалізовано функцію \inlinecode{plot\_density\_profile}, що обчислює електронну густину вздовж осі $z$ для атома.
Це дозволяє побудувати одноосний «профіль густини» та простежити, як електронна густина спадає при віддаленні від ядра.

Для кожної точки $z$ обчислюється матриця базисних функцій $\phi_i(\mathbf{r})$, і далі густина:
  \[
  \rho(\mathbf{r}) = \sum_{ij} \chi_i(\mathbf{r}) D_{ij} \chi_j(\mathbf{r}),
  \]
 де $D_{ij}$ --- елементи матриці щільності.


Результат зображається графічно: густина $\rho(0,0,z)$ як функція відстані від ядра.
Для нейтральних атомів крива має різкий максимум біля ядра, який швидко спадає експоненційно.

\inputcode{code_12.py}

\paragraph{Інтерпретація результатів.}
На побудованому графіку $\rho(0,0,z)$ спостерігаються чіткі області локалізації електронів:
внутрішні електрони формують центральний пік, тоді як зовнішні оболонки проявляються у вигляді плавних плечей.
Для важчих атомів густина стає більш «зосередженою» біля ядра через сильніше притягання кулонівським потенціалом.

\paragraph{Практична порада.}
При виборі базисного набору (\texttt{cc-pvdz}, \texttt{6-31G**}, тощо) слід мати на увазі, що форма профілю густини істотно залежить від якості базису:
мінімальні базиси дають лише грубу картину, тоді як розширені базиси (\texttt{cc-pVTZ}) відтворюють експоненційний спад більш точно.

%% --------------------------------------------------------
\subsubsection{Сферично усереднена густина}
%% --------------------------------------------------------

Для атомів, що мають сферичну симетрію, зручно розглядати усереднену по всіх напрямах густину:
\[
\rho(r) = \frac{1}{4\pi} \int \rho(\mathbf{r}) \, d\Omega,
\]
де $r = |\mathbf{r}|$ та $d\Omega$ --- елемент тілесного кута.

Ця функція показує, як змінюється густина лише від радіуса $r$, незалежно від напрямку, і є основою для побудови радіальної функції розподілу
\[
P(r) = 4\pi r^2 \rho(r),
\]
що описує, яка частка електронів перебуває у сферичному шарі товщини $dr$ на відстані $r$ від ядра.

\inputcode{code_13.py}


\paragraph{Методика обчислення.}
Для кожного радіуса $r$ вибираються випадкові напрями (кутові координати $\theta$, $\phi$) за методом Монте-Карло.
У цих напрямках обчислюється густина $\rho(x,y,z)$, після чого береться середнє значення:
\[
\rho(r) \approx \frac{1}{N} \sum_{k=1}^{N} \rho(r,\theta_k,\phi_k).
\]
Цей підхід забезпечує добру точність навіть при невеликій кількості напрямів $N \sim 100$.

\paragraph{Радіальна функція розподілу.}
На другому графіку зображується $4\pi r^2\rho(r)$ --- це функція, інтеграл від якої по $r$ дає повну кількість електронів:
\[
\int_0^\infty 4\pi r^2\rho(r)\,dr = N_e.
\]
Таким чином, можна перевірити нормування хвильової функції та коректність чисельного інтегрування.

\paragraph{Приклад інтерпретації.}
Для атома гелію максимум $4\pi r^2\rho(r)$ спостерігається приблизно при $r \approx 0.3$~Bohr,
що відповідає найбільш ймовірній відстані електрона від ядра у $1s$-стані.
Для вуглецю або неону виникають додаткові піки, пов’язані з електронами на $2s$ та $2p$ оболонках.

%\paragraph{Перевірка нормування.}
%Наприкінці виконання функції виводиться значення
%\[
%\int 4\pi r^2 \rho(r) dr,
%\]
%яке має збігатися з кількістю електронів у системі, що є корисним тестом правильності побудованої густини.


%% --------------------------------------------------------
\subsection{Порівняння електронних густин різних атомів}
%% --------------------------------------------------------

Ми розглянули, як отримати електронну густину $\rho(\mathbf{r})$ та сферично усереднену густину $\rho(r)$ для окремого атома.  Однак для повного розуміння періодичних закономірностей важливо порівняти, як змінюється форма та протяжність електронної хмари при переході від легких до важчих елементів.

\subsubsection{Фізична мотивація}

Порівняння електронних густин дозволяє:
\begin{itemize}
    \item побачити, як із зростанням атомного номера $Z$ ядро сильніше притягує електрони;
    \item проаналізувати зміну розмірів атома та ефективного радіуса;
    \item виявити закономірності заповнення електронних оболонок;
    \item візуально оцінити зв’язок між структурою густини та положенням елемента в періодичній системі.
\end{itemize}

%У програмі нижче реалізовано функцію порівняння радіальних профілів густини для кількох атомів одночасно.
%Вона виконує незалежні розрахунки SCF для кожного атома, будує $\rho(r)$ та радіальну функцію $4\pi r^2 \rho(r)$, і виводить результати на одному графіку для порівняння.
%
%\subsubsection{Коментар до реалізації}
%
%Ключові кроки програми:
%\begin{enumerate}
%    \item Створення об'єкта \texttt{Mole} для кожного атома з базисом \texttt{cc-pvdz} або \texttt{cc-pvtz}.
%    \item Проведення SCF-розрахунку типу RHF або UHF залежно від спіну.
%    \item Побудова масиву радіальних точок уздовж осі $z$:
%    $$ r \in [0, 6] \ \text{Bohr} $$
%    \item Обчислення електронної густини як:
%    \[
%    \rho(r) = \sum_{i,j} \chi_i(r) D_{ij} \chi_j(r),
%    \]
%    де $\chi_i$ --- атомні орбіталі, $D_{ij}$ --- елементи матриці густини.
%    \item Для кожного атома будується:
%    \begin{itemize}
%        \item крива $\rho(r)$ --- логарифмічний масштаб показує різке спадання густини;
%        \item радіальний розподіл $4\pi r^2\rho(r)$ --- для аналізу оболонкової структури.
%    \end{itemize}
%\end{enumerate}
%
%\inputcode{code_14.py}
%
%\subsubsection{Аналіз результатів}
%
%На графіку для \textbf{благородних газів} (\ce{He}, \ce{Ne}, \ce{Ar}) чітко видно:
%\begin{itemize}
%    \item з ростом $Z$ максимум густини зміщується ближче до ядра --- електрони сильніше притягуються кулонівським полем;
%    \item протяжність хвоста $\rho(r)$ зменшується, атоми стають компактнішими;
%    \item радіальні криві $4\pi r^2\rho(r)$ демонструють послідовне збільшення кількості максимумів, що відповідає заповненню нових оболонок.
%\end{itemize}
%
%Для \textbf{атомів другого періоду} (\ce{Li} -- \ce{F}) спостерігається:
%\begin{itemize}
%    \item різке зростання центральної густини у міру збільшення $Z$;
%    \item поява другорядного максимуму в $4\pi r^2\rho(r)$, що відповідає електронам $2s$-оболонки;
%    \item звуження зовнішньої частини електронної хмари.
%\end{itemize}

\subsubsection{Зв’язок із електронними оболонками}

Максимуми радіальної функції розподілу $4\pi r^2\rho(r)$ відповідають областям найбільшої ймовірності перебування електронів певних орбіталей:
\begin{align*}
\text{1s} \rightarrow \text{перший максимум}, \quad
\text{2s, 2p} \rightarrow \text{другий максимум}, \\
\text{3s, 3p, 3d} \rightarrow \text{третій тощо.}
\end{align*}

Таким чином, форма $4\pi r^2\rho(r)$ дає прямий зв’язок між результатами квантово-хімічних розрахунків і традиційною атомною структурою, знайомою студентам з курсу атомної фізики.
Візуальний аналіз таких графіків дозволяє простежити закономірності побудови періодичної системи Менделєєва з точки зору електронної густини.

\subsubsection{Практичні завдання}

\begin{enumerate}
    \item Побудуйте на одному графіку $\rho(r)$ для \ce{Li}, \ce{Na}, \ce{K}. Як змінюється радіус атома?
    \item Знайдіть положення максимумів $4\pi r^2\rho(r)$ для атомів \ce{He}, \ce{Ne}, \ce{Ar}. До яких оболонок вони належать?
    \item Порівняйте радіальні функції для атомів \ce{C} і \ce{O}. Як змінюється густина зовнішньої оболонки?
\end{enumerate}


%% --------------------------------------------------------
\subsubsection{Методичні рекомендації}
%% --------------------------------------------------------

\begin{enumerate}
%  \item Для легких атомів (\texttt{H}, \texttt{He}, \texttt{Li}) можна порівняти результати з аналітичними розв’язками рівняння Шредінгера.
  \item Змінюючи базис (наприклад, \texttt{STO-3G}, \texttt{6-31G}, \texttt{cc-pVTZ}), можна дослідити, як збільшується точність відтворення радіального профілю.
  \item Для відкритих оболонок (атомів з неспареними електронами) потрібно враховувати спінову поляризацію --- використовується \texttt{UHF}.
\end{enumerate}

Таким чином, наведені приклади демонструють практичний зв’язок між теоретичними поняттями електронної густини і їх чисельною реалізацією в пакеті \texttt{PySCF}.


%% --------------------------------------------------------
\section{Порівняння методів: RHF vs UHF vs ROHF}
%% --------------------------------------------------------

У квантово-хімічних розрахунках вибір типу наближення Гартрі–Фока залежить від спінового стану системи.
Методи \texttt{RHF}, \texttt{UHF} та \texttt{ROHF} відрізняються тим, як вони поводяться із спіновими функціями електронів — тобто, чи допускають різні орбіталі для $\alpha$- і $\beta$-електронів.

\begin{itemize}
    \item \texttt{RHF (Restricted Hartree–Fock)} --- усі електрони спарені, кожна просторова орбіталь заповнена двома електронами з протилежними спінами.
    Підходить лише для синглетних закритих оболонок.

    \item \texttt{UHF (Unrestricted Hartree–Fock)} --- орбіталі для $\alpha$- і $\beta$-електронів дозволено різні.
    Цей метод придатний для відкритих оболонок (наприклад, атомів з неспареними електронами), однак може страждати від \emph{спінового забруднення (spin contamination)}.

    \item \texttt{ROHF (Restricted Open-Shell Hartree–Fock)} --- компроміс: спільні орбіталі для парних електронів і різні --- лише для неспарених.
    Це забезпечує правильний спіновий симетрійний стан без забруднення.
\end{itemize}

\subsubsection{Енергетичне порівняння}

\texttt{UHF} зазвичай дає трохи нижчу енергію, ніж ROHF, оскільки має більше варіаційної свободи.
Різниця $\Delta E = E_{\mathrm{UHF}} - E_{\mathrm{ROHF}}$ зазвичай незначна (менше кількох міліГартрі), однак у системах з сильним спіновим змішуванням може бути суттєвою.

\subsubsection{Орбітальні енергії}

\texttt{UHF} формує два набори орбіталей --- для $\alpha$- і $\beta$-електронів.
Тому орбітальні енергії можуть відрізнятись:
\[
\varepsilon_i^{(\alpha)} \neq \varepsilon_i^{(\beta)}.
\]
У \texttt{ROHF} усі парні орбіталі спільні, тож орбітальні енергії чітко впорядковані відповідно до симетрії та спіну.

%% --------------------------------------------------------
\subsection{Спінове забруднення}
%% --------------------------------------------------------

У системах з відкритими оболонками метод \texttt{UHF} часто використовується як простіше наближення, що дозволяє займати незалежні орбіталі електронам зі спінами $\alpha$ та $\beta$.
Однак відсутність суворого обмеження на спінову симетрію призводить до важливого недоліку --- \textbf{спінового забруднення}.


\subsubsection*{Математична суть явища}

\texttt{UHF}-хвильова функція не є власним станом оператора $\hat{S}^2$, хоч і залишається власним станом $\hat{S}_z$:
\[
\hat{S}_z \Psi_{\text{UHF}} = M_S \Psi_{\text{UHF}}, \qquad
\hat{S}^2 \Psi_{\text{UHF}} \neq S(S+1)\Psi_{\text{UHF}}.
\]
Отже, очікуване значення
\[
\langle S^2 \rangle = \langle \Psi_{\text{UHF}} | \hat{S}^2 | \Psi_{\text{UHF}} \rangle
\]
зазвичай перевищує істинне $S(S+1)$, тобто:
\[
\Delta S^2 = \langle S^2 \rangle - S(S+1) > 0.
\]

\subsubsection*{Фізичне тлумачення}

UHF-хвильова функція може бути подана як суперпозиція чистих спінових станів:
\[
\Psi_{\text{UHF}} = c_0 \Psi_{S} + c_1 \Psi_{S+1} + c_2 \Psi_{S+2} + \cdots,
\]
де присутність коефіцієнтів $c_i \ne 0$ для $i>0$ означає змішування різних мультиплетів.
Наприклад, для триплетного стану ($S=1$) теоретичне значення $\langle S^2 \rangle = 2.0$, але якщо розрахунок UHF дає $2.25$, це означає, що хвильова функція містить домішку п’ятичленного ($S=2$) стану.

\subsubsection*{Вплив на результати}

\begin{itemize}
    \item \textbf{Енергія.} \texttt{UHF} може давати занадто низьку енергію через змішування спінів і штучне зменшення кулонівського відштовхування.
    \item \textbf{Електронна густина.} Спінова густина стає несиметричною, особливо поблизу атомів з неспареними електронами.
    \item \textbf{Магнітні властивості.} Похибки у спіновій густині ведуть до помилкових магнітних моментів і спотворених мультиплетних розщеплень.
\end{itemize}

%% --------------------------------------------------------
\subsubsection{Як обчислюється спінове забруднення}
%% --------------------------------------------------------

Програми квантово-хімічних розрахунків (зокрема \texttt{PySCF}, \texttt{Gaussian}, \texttt{ORCA}) обчислюють спінове забруднення не безпосередньо через оператор $\hat{S}^2$, а через перекриття між орбіталями спінів $\alpha$ і $\beta$.

Основна формула для середнього значення $\langle S^2 \rangle$ у наближенні \texttt{UHF} має вигляд:
\[
\boxed{
\langle S^2 \rangle =
\frac{(N_\alpha - N_\beta)^2}{4}
+ \frac{N_\alpha + N_\beta}{2}
- \sum_{i=1}^{N_\alpha} \sum_{j=1}^{N_\beta}
|\langle \phi_i^\alpha | \phi_j^\beta \rangle|^2
}
\]
де:
\begin{itemize}
    \item $N_\alpha$, $N_\beta$ — кількість електронів зі спінами $\alpha$ та $\beta$ відповідно;
    \item $\phi_i^\alpha$, $\phi_j^\beta$ --- молекулярні(атомні) орбіталі для спінів $\alpha$ і $\beta$;
    \item $\langle \phi_i^\alpha | \phi_j^\beta \rangle$ --- перекриття між орбіталями різних спінів.
\end{itemize}

Якщо орбіталі $\alpha$ і $\beta$ однакові, тобто $\phi_i^\alpha = \phi_i^\beta$, то:
\[
\langle S^2 \rangle = \frac{(N_\alpha - N_\beta)^2}{4},
\]
і для синглету ($N_\alpha = N_\beta$) це дорівнює нулю --- спінове забруднення відсутнє.

\subsubsection*{Інтерпретація}

Якщо орбіталі $\alpha$ і $\beta$ подібні, перекриття $S_{\alpha\beta}$ велике, тому $\Sigma$ наближається до $N_\beta$, і $\langle S^2 \rangle$ близьке до теоретичного $S(S+1)$ — тобто забруднення мінімальне.
Якщо ж орбіталі сильно відрізняються (наприклад, при розриві зв’язку або в радикалах), $\Sigma$ зменшується, і $\langle S^2 \rangle$ зростає — з’являється суттєве спінове забруднення.


\subsubsection*{Діагностика та критерії}

Значення $\langle S^2 \rangle$ зазвичай друкується у вихідних даних програми.
Практичні межі:
\[
\Delta S^2 = \langle S^2 \rangle - S(S+1),
\]
\[
\begin{cases}
\Delta S^2 < 0.05 & \text{— незначне забруднення, прийнятно;}\\
0.05 < \Delta S^2 < 0.10 & \text{— помірне, бажано перевірити;}\\
\Delta S^2 > 0.10 & \text{— суттєве, слід перейти до ROHF або проєкційних методів.}
\end{cases}
\]


\subsubsection*{Приклад чисельного розрахунку}

Для системи з $N_\alpha = 5$, $N_\beta = 4$ (очікувано $S=\frac{1}{2}$, $\langle S^2 \rangle = 0.75$):

\begin{itemize}
    \item якщо $\Sigma = 3.9$, тоді
    \[
    \langle S^2 \rangle = \frac{1}{4} + \frac{9}{2} - 3.9 = 0.85,
    \]
    \[
    \Delta S^2 = 0.85 - 0.75 = 0.10,
    \]
    тобто забруднення перебуває на межі суттєвого;

    \item якщо $\Sigma = 3.4$, тоді
    \[
    \langle S^2 \rangle = \frac{1}{4} + \frac{9}{2} - 3.4 = 1.35,
    \]
    \[
    \Delta S^2 = 1.35 - 0.75 = 0.60,
    \]
    що свідчить про сильне спінове забруднення.
\end{itemize}

\subsubsection*{Як зменшити спінове забруднення}

\begin{enumerate}
    \item \textbf{ROHF:} обмежена відкрита схема Гартрі–Фока, що зберігає спінову симетрію.
    \item \textbf{Спін-проекційні методи:} після варіації застосовується оператор проєкції
    \[
    \Psi_{\text{proj}} = \hat{P}_S \Psi_{\text{UHF}},
    \]
    який видаляє небажані компоненти.
    \item \textbf{Методи після HF:} MP2, CISD, CCSD частково компенсують спінове забруднення через багатоконфігураційність.
\end{enumerate}


%% --------------------------------------------------------
\subsection{Тестовий випадок: атом Вуглецю}
%% --------------------------------------------------------

Вуглець має конфігурацію $1s^2 2s^2 2p^2$ і основний стан $^3P$ (триплет).
Отже, система має два неспарені електрони ($S=1$), а отже $\langle S^2 \rangle = S(S+1) = 2.0$.

Далі наведено приклад прямого порівняння результатів методів UHF і ROHF для атома C у триплетному стані з використанням базису \texttt{cc-pVTZ}.

\inputcode{code_15.py}

Щоб узагальнити різницю між методами UHF та ROHF, виконаємо серію розрахунків для
кількох відкритооболонкових атомів перших двох періодів. Це дозволить оцінити
енергетичну різницю $\Delta E$ і побачити,
чи зберігає ROHF спінову симетрію без втрати точності.



\inputcode{code_16.py}


\noindent
\textbf{Інтерпретація результатів.}
Як правило, для легких атомів (H, Li, B) енергії UHF і ROHF майже збігаються.
Для атомів із більшою кількістю незапарених електронів (C, N, O, F)
різниця у кілька mHa відображає більшу гнучкість UHF,
що дозволяє частково знизити енергію ціною забруднення спіном
($\langle S^2 \rangle > S(S+1)$).

Таким чином, \texttt{ROHF} --- це більш строгий метод із правильною спіновою симетрією,
а \texttt{UHF} --- енергетично вигідніший, але менш «фізично чистий». У практичних
обчисленнях ROHF часто слугує базовою точкою для подальших кореляційних
методів (MP2, CCSD, CASSCF).

\subsubsection{Практичне значення}

Порівняння методів RHF, UHF і ROHF дозволяє студентам зрозуміти:
\begin{itemize}
    \item коли можна використовувати RHF (синглети закритих оболонок);
    \item коли необхідно застосовувати UHF (відкриті оболонки, магнітні системи);
    \item у яких випадках варто віддати перевагу ROHF (спінова чистота, мінімізація забруднення).
\end{itemize}

\subsubsection{Контрольне завдання}

\begin{enumerate}
    \item Повторіть розрахунок для атомів \ce{O} (спін = 2) і \ce{N} (спін = 3).
    Порівняйте $\langle S^2 \rangle$ для UHF і ROHF.
    \item Для кожного атома побудуйте різницю енергій UHF–ROHF у mHa.
    \item Проаналізуйте, як спінове забруднення зростає зі збільшенням кількості неспарених електронів.
\end{enumerate}



%% --------------------------------------------------------
\section{Складні випадки та збіжність}
%% --------------------------------------------------------

Розрахунки у квантовій хімії не завжди проходять гладко. Навіть для простих систем ітераційна процедура самозгодженого поля (SCF) може не досягати стабільного рішення.
Найчастіше проблеми збіжності виникають для перехідних металів, відкритих оболонок та систем із виродженими орбіталями.
У цьому розділі розглянемо основні джерела труднощів та методи їх подолання.

\subsection{Причини поганої збіжності}

Типові джерела нестабільності SCF:
\begin{itemize}
  \item сильна електронна кореляція у незаповнених $d$- і $f$-оболонках;
  \item мала різниця енергій між орбіталями (виродження);
  \item невдалий початковий вектор коефіцієнтів (погане стартове наближення);
  \item занадто жорсткі або занадто м’які критерії збіжності.
\end{itemize}

Через ці фактори енергія може осцилювати, DIIS — розходитись, а результат — бути нефізичним.

\subsection{Перехідні метали}

Перехідні елементи (Fe, Co, Ni, Cr, Mn тощо) — класичні приклади систем із поганою збіжністю SCF.
Причини:
\begin{itemize}
  \item близькість енергетичних рівнів $3d$ і $4s$-орбіталей;
  \item можливість декількох спінових станів, близьких за енергією;
  \item наявність кількох локальних мінімумів функціоналу енергії.
\end{itemize}

У PySCF це проявляється як:
\begin{itemize}
  \item осциляції енергії між ітераціями;
  \item розбігання DIIS;
  \item стрибки значення $\langle S^2 \rangle$ між кроками.
\end{itemize}

Нижче наведено приклад універсальної функції для стабільного розрахунку атомів перехідних металів із урахуванням практичних прийомів:

\begin{itemize}
  \item використовується UHF (для врахування спіну);
  \item застосовується зсув рівнів (\inlinecode{level\_shift});
  \item розширюється DIIS-простір (\inlinecode{diis\_space});
  \item у разі потреби вмикається уточнення Ньютона–Рафсона;
  \item контролюється спін-забруднення через $\langle S^2 \rangle$.
\end{itemize}

\inputcode{code_18.py}

Такий підхід забезпечує стабільність навіть для важких атомів, де стандартний SCF часто не сходиться.

\subsection{Вироджені орбіталі та дробові заповнення}

Якщо в системі наявні вироджені або майже вироджені орбіталі, SCF може ``коливатись'', не вибираючи між ними.
Для таких випадків ефективним є застосування \textit{дробових заповнень орбіталей} (\textit{fractional occupations}), що згладжують різницю між рівнями енергії.

Ідея полягає у введенні ефективного теплового розмазування Фермі–Дірака:
\[
f_i = \frac{1}{1 + e^{(\varepsilon_i - \mu)/kT}},
\]
де $f_i$ — часткове заповнення орбіталі $i$, $\varepsilon_i$ — її енергія, $\mu$ — хімічний потенціал, а $kT$ — параметр розмазування.

Цей підхід дозволяє SCF уникнути нестійких перестрибувань між виродженими рівнями.

У PySCF для цього достатньо викликати \inlinecode{scf.addons.frac\_occ(mf)}.
Методика добре працює для атомів (V, Cr), радикалів та малих кластерів перехідних металів.

\inputcode{code_19.py}



\subsection{Стратегія досягнення збіжності SCF}

Для надійного отримання фізично коректного рішення рекомендується послідовна стратегія стабілізації SCF — від простих прийомів до складніших:

\begin{enumerate}
  \item стандартний UHF;
  \item зсув рівнів (\inlinecode{level\_shift});
  \item збільшення DIIS-простору (\inlinecode{diis\_space});
  \item атомне початкове наближення (\inlinecode{init\_guess='atom'});
  \item уточнення за методом Ньютона–Рафсона;
  \item дробові заповнення (\inlinecode{frac\_occ}) для вироджених станів.
\end{enumerate}

\inputcode{code_20.py}

\paragraph{Фізичний зміст прийомів.}
\begin{itemize}
  \item \textbf{Level shift} — підвищує енергії віртуальних орбіталей, запобігаючи коливанням.
  \item \textbf{DIIS-простір} — збільшення пам’яті ітерацій покращує апроксимацію поля.
  \item \textbf{Atom guess} — старт з атомних орбіталей ближчий до реального розв’язку.
  \item \textbf{Newton–Raphson} — забезпечує квадратичну збіжність поблизу мінімуму.
  \item \textbf{Fractional occupations} — стабілізують вироджені стани, забезпечуючи плавний перехід.
\end{itemize}

\paragraph{Практичні рекомендації.}
\begin{itemize}
  \item Для великих систем на перших етапах можна зменшити точність: \inlinecode{conv\_tol=1e-6}.
  \item Якщо енергія осцилює — увімкніть \inlinecode{level\_shift=0.5}.
  \item Якщо \texttt{UHF} не сходиться — використайте \inlinecode{init\_guess='atom'} або вимкніть симетрію (\inlinecode{symmetry=False}).
  \item Завжди перевіряйте фізичність результату через $\langle S^2 \rangle$.
\end{itemize}

\medskip

\noindent Таким чином, навіть якщо стандартна процедура SCF не збігається,
послідовне застосування описаних прийомів практично гарантує стабільне та фізично обґрунтоване рішення.



%% --------------------------------------------------------
\section{Практичні завдання}
%% --------------------------------------------------------

%% --------------------------------------------------------
\subsection{Завдання 1: Систематичне дослідження}
%% --------------------------------------------------------

\inputcode{code_21.py}

%% --------------------------------------------------------
\subsection{Завдання 2: Енергії іонізації}
%% --------------------------------------------------------

\inputcode{code_22.py}


%% --------------------------------------------------------
\subsection{Завдання 3: Спектроскопічні константи}
%% --------------------------------------------------------

\inputcode{code_23.py}

%% --------------------------------------------------------
\subsection{Завдання 4: Залежність від базису}
%% --------------------------------------------------------

\inputcode{code_24.py}

%% --------------------------------------------------------
\section{Резюме}
%% --------------------------------------------------------

У цьому розділі ми детально вивчили метод Хартрі-Фока для атомних систем:

\begin{itemize}
    \item \textbf{Теоретичні основи} --- рівняння HF, детермінант Слейтера, оператор Фока
    \item \textbf{Варіанти методу} --- RHF для замкнених оболонок, UHF для відкритих, ROHF як компроміс
    \item \textbf{Одноелектронні системи} --- H атом як тестовий випадок
    \item \textbf{Багатоелектронні атоми} --- He та атоми другого періоду
    \item \textbf{Аналіз результатів} --- орбітальні енергії, заселеності, спінова густина
    \item \textbf{Складні випадки} --- перехідні метали, стратегії конвергенції
\end{itemize}

\subsection{Ключові висновки}

\begin{enumerate}
    \item Метод HF дає добре якісне описання атомних систем, але не враховує кореляцію електронів
    \item Вибір між RHF/UHF/ROHF залежить від спінової структури системи
    \item UHF страждає від забруднення спіном, але зазвичай дає нижчу енергію
    \item Для важких атомів (перехідні метали) потрібні спеціальні техніки конвергенції
    \item Якість результатів сильно залежить від вибору базисного набору
\end{enumerate}

У наступному розділі ми розглянемо теорію функціоналу густини (DFT), яка часто дає кращі результати для багатоелектронних систем.