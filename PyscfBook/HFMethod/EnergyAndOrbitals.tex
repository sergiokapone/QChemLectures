%% --------------------------------------------------------
\section{Аналіз енергій та орбіталей}
%% --------------------------------------------------------


%% --------------------------------------------------------
\subsection{Енергетична діаграма орбіталей}
%% --------------------------------------------------------


Енергетичні діаграми орбіталей є наочним способом представлення енергетичного спектру молекулярних орбіталей (МО), що виникають у результаті розв’язання рівнянь Гартрі–Фока.


\inputcode{code_11.py}

%% --------------------------------------------------------
\subsection{Енергії іонізації}
%% --------------------------------------------------------

Іонізаційна енергія \(I\) є однією з фундаментальних характеристик атома, що визначає мінімальні енерговитрати на видалення одного електрона із нейтрального атома в основному стані:
\[
I = E(A^+) - E(A),
\]
де \(E(A)\) — повна енергія нейтрального атома, а \(E(A^+)\) — енергія його однозарядного катіона.
Ця величина безпосередньо вимірюється експериментально (у~електронвольтах) і є важливим тестом якості будь-якого квантово-хімічного методу.

У межах методу Гартрі–Фока іонізаційна енергія може бути визначена двома способами.

\paragraph{Метод \(\Delta\mathrm{SCF}\).}
Найбільш пряме обчислення полягає у незалежному знаходженні повних енергій нейтрального атома і катіона:
\[
I_{\Delta\mathrm{SCF}} = E(A^+) - E(A).
\]
Такий підхід враховує релаксацію орбіталей після видалення електрона, тому дає точніші результати, ніж прості наближення.

\paragraph{Теорема Купманса.}
У межах наближення «заморожених орбіталей» (\textit{frozen orbital approximation})
повна енергія системи вважається незмінною при видаленні одного електрона.
У цьому випадку зміна енергії дорівнює орбітальній енергії найвищої зайнятої молекулярної орбіталі:
\[
I_\mathrm{Koopmans} \approx -\varepsilon_\mathrm{HOMO}.
\]
Це твердження відоме як \textbf{теорема Купманса}.
Воно дозволяє оцінити енергії іонізації без повторного самозгодження, хоча і не враховує релаксаційні та кореляційні ефекти.

\paragraph{Порівняння підходів.}
Для легких атомів (Li, Be, B, C, N, O, F, Ne) метод Купманса правильно відтворює тенденцію зростання енергії іонізації уздовж періоду, але систематично недооцінює абсолютні значення.
Метод \(\Delta\mathrm{SCF}\) наближається до експерименту краще, оскільки включає релаксацію орбіталей.

\paragraph{Приклад: атом Літію.}
Для Li (конфігурація \(1s^2 2s^1\)):
\[
I_{\text{Koopmans}} = -\varepsilon_{2s},
\quad
I_{\Delta\mathrm{SCF}} = E(\mathrm{Li}^+) - E(\mathrm{Li}),
\quad
I_\text{exp} = 5.39~\text{eV}.
\]
Енергія Купманса дещо завищена, а результат \(\Delta\mathrm{SCF}\) ближчий до експериментального, що демонструє вплив орбітальної релаксації.

\inputcode{KoopmansTheoremCheck.py}

\paragraph{Коментар.}
Розрахунок виводить таблицю порівняння енергій іонізації (\(-\varepsilon_\mathrm{HOMO}\), \(\Delta\mathrm{SCF}\), експеримент) для атомів другого періоду
та відносну похибку у~відсотках:
\[
\delta = 100 \times \frac{I_\text{calc} - I_\text{exp}}{I_\text{exp}}.
\]
Аналіз цих відхилень дозволяє оцінити якість базисного набору і межі застосовності теореми Купманса.


%Для більш точних результатів використовують після-HF методи, наприклад MP2 або CCSD, які враховують електронну кореляцію.

%% --------------------------------------------------------
\subsection{Електронна густина та її інтерпретація}
%% --------------------------------------------------------

%% --------------------------------------------------------
\subsubsection{Розподіл електронної густини}
%% --------------------------------------------------------

Розподіл електронної густини $\rho(\mathbf{r})$ є однією з ключових величин у квантовій хімії.
Саме ця функція визначає, де саме в просторі «перебувають» електрони атома або молекули, і, отже, пояснює більшість хімічних і спектроскопічних властивостей системи.

Згідно з однодетермінантним наближенням Гартрі–Фока, електронна густина визначається як
\[
\rho(\mathbf{r}) = \sum_{i}^{\text{occ}} |\psi_i(\mathbf{r})|^2,
\]
де $\psi_i(\mathbf{r})$ --- орбіталі, а сума береться по всіх зайнятих орбіталях.

У програмному пакеті \texttt{PySCF} густина може бути обчислена як на сітці, так і в окремих точках простору, що дозволяє візуалізувати просторовий розподіл електронів.

%% --------------------------------------------------------
\subsubsection{Обчислення густини вздовж осі $z$}
%% --------------------------------------------------------

У наведеному прикладі реалізовано функцію \inlinecode{plot\_density\_profile}, що обчислює електронну густину вздовж осі $z$ для атома.
Це дозволяє побудувати одноосний «профіль густини» та простежити, як електронна густина спадає при віддаленні від ядра.

Для кожної точки $z$ обчислюється матриця базисних функцій $\phi_i(\mathbf{r})$, і далі густина:
  \[
  \rho(\mathbf{r}) = \sum_{ij} \chi_i(\mathbf{r}) D_{ij} \chi_j(\mathbf{r}),
  \]
 де $D_{ij}$ --- елементи матриці щільності.


Результат зображається графічно: густина $\rho(0,0,z)$ як функція відстані від ядра.
Для нейтральних атомів крива має різкий максимум біля ядра, який швидко спадає експоненційно.

\inputcode{code_12.py}

\paragraph{Інтерпретація результатів.}
На побудованому графіку $\rho(0,0,z)$ спостерігаються чіткі області локалізації електронів:
внутрішні електрони формують центральний пік, тоді як зовнішні оболонки проявляються у вигляді плавних плечей.
Для важчих атомів густина стає більш «зосередженою» біля ядра через сильніше притягання кулонівським потенціалом.

\paragraph{Практична порада.}
При виборі базисного набору (\texttt{cc-pvdz}, \texttt{6-31G**}, тощо) слід мати на увазі, що форма профілю густини істотно залежить від якості базису:
мінімальні базиси дають лише грубу картину, тоді як розширені базиси (\texttt{cc-pVTZ}) відтворюють експоненційний спад більш точно.

%% --------------------------------------------------------
\subsubsection{Сферично усереднена густина}
%% --------------------------------------------------------

Для атомів, що мають сферичну симетрію, зручно розглядати усереднену по всіх напрямах густину:
\[
\rho(r) = \frac{1}{4\pi} \int \rho(\mathbf{r}) \, d\Omega,
\]
де $r = |\mathbf{r}|$ та $d\Omega$ --- елемент тілесного кута.

Ця функція показує, як змінюється густина лише від радіуса $r$, незалежно від напрямку, і є основою для побудови радіальної функції розподілу
\[
P(r) = 4\pi r^2 \rho(r),
\]
що описує, яка частка електронів перебуває у сферичному шарі товщини $dr$ на відстані $r$ від ядра.

\inputcode{code_13.py}


\paragraph{Методика обчислення.}
Для кожного радіуса $r$ вибираються випадкові напрями (кутові координати $\theta$, $\phi$) за методом Монте-Карло.
У цих напрямках обчислюється густина $\rho(x,y,z)$, після чого береться середнє значення:
\[
\rho(r) \approx \frac{1}{N} \sum_{k=1}^{N} \rho(r,\theta_k,\phi_k).
\]
Цей підхід забезпечує добру точність навіть при невеликій кількості напрямів $N \sim 100$.

\paragraph{Радіальна функція розподілу.}
На другому графіку зображується $4\pi r^2\rho(r)$ --- це функція, інтеграл від якої по $r$ дає повну кількість електронів:
\[
\int_0^\infty 4\pi r^2\rho(r)\,dr = N_e.
\]
Таким чином, можна перевірити нормування хвильової функції та коректність чисельного інтегрування.

\paragraph{Приклад інтерпретації.}
Для атома гелію максимум $4\pi r^2\rho(r)$ спостерігається приблизно при $r \approx 0.3$~Bohr,
що відповідає найбільш ймовірній відстані електрона від ядра у $1s$-стані.
Для вуглецю або неону виникають додаткові піки, пов’язані з електронами на $2s$ та $2p$ оболонках.

%\paragraph{Перевірка нормування.}
%Наприкінці виконання функції виводиться значення
%\[
%\int 4\pi r^2 \rho(r) dr,
%\]
%яке має збігатися з кількістю електронів у системі, що є корисним тестом правильності побудованої густини.


%% --------------------------------------------------------
\subsection{Порівняння електронних густин різних атомів}
%% --------------------------------------------------------

Ми розглянули, як отримати електронну густину $\rho(\mathbf{r})$ та сферично усереднену густину $\rho(r)$ для окремого атома.  Однак для повного розуміння періодичних закономірностей важливо порівняти, як змінюється форма та протяжність електронної хмари при переході від легких до важчих елементів.

\subsubsection{Фізична мотивація}

Порівняння електронних густин дозволяє:
\begin{itemize}
    \item побачити, як із зростанням атомного номера $Z$ ядро сильніше притягує електрони;
    \item проаналізувати зміну розмірів атома та ефективного радіуса;
    \item виявити закономірності заповнення електронних оболонок;
    \item візуально оцінити зв’язок між структурою густини та положенням елемента в періодичній системі.
\end{itemize}

%У програмі нижче реалізовано функцію порівняння радіальних профілів густини для кількох атомів одночасно.
%Вона виконує незалежні розрахунки SCF для кожного атома, будує $\rho(r)$ та радіальну функцію $4\pi r^2 \rho(r)$, і виводить результати на одному графіку для порівняння.
%
%\subsubsection{Коментар до реалізації}
%
%Ключові кроки програми:
%\begin{enumerate}
%    \item Створення об'єкта \texttt{Mole} для кожного атома з базисом \texttt{cc-pvdz} або \texttt{cc-pvtz}.
%    \item Проведення SCF-розрахунку типу RHF або UHF залежно від спіну.
%    \item Побудова масиву радіальних точок уздовж осі $z$:
%    $$ r \in [0, 6] \ \text{Bohr} $$
%    \item Обчислення електронної густини як:
%    \[
%    \rho(r) = \sum_{i,j} \chi_i(r) D_{ij} \chi_j(r),
%    \]
%    де $\chi_i$ --- атомні орбіталі, $D_{ij}$ --- елементи матриці густини.
%    \item Для кожного атома будується:
%    \begin{itemize}
%        \item крива $\rho(r)$ --- логарифмічний масштаб показує різке спадання густини;
%        \item радіальний розподіл $4\pi r^2\rho(r)$ --- для аналізу оболонкової структури.
%    \end{itemize}
%\end{enumerate}
%
%\inputcode{code_14.py}
%
%\subsubsection{Аналіз результатів}
%
%На графіку для \textbf{благородних газів} (\ce{He}, \ce{Ne}, \ce{Ar}) чітко видно:
%\begin{itemize}
%    \item з ростом $Z$ максимум густини зміщується ближче до ядра --- електрони сильніше притягуються кулонівським полем;
%    \item протяжність хвоста $\rho(r)$ зменшується, атоми стають компактнішими;
%    \item радіальні криві $4\pi r^2\rho(r)$ демонструють послідовне збільшення кількості максимумів, що відповідає заповненню нових оболонок.
%\end{itemize}
%
%Для \textbf{атомів другого періоду} (\ce{Li} -- \ce{F}) спостерігається:
%\begin{itemize}
%    \item різке зростання центральної густини у міру збільшення $Z$;
%    \item поява другорядного максимуму в $4\pi r^2\rho(r)$, що відповідає електронам $2s$-оболонки;
%    \item звуження зовнішньої частини електронної хмари.
%\end{itemize}

\subsection{Зв’язок із електронними оболонками}

Максимуми радіальної функції розподілу $4\pi r^2\rho(r)$ відповідають областям найбільшої ймовірності перебування електронів певних орбіталей:
\begin{align*}
\text{1s} \rightarrow \text{перший максимум}, \quad
\text{2s, 2p} \rightarrow \text{другий максимум}, \\
\text{3s, 3p, 3d} \rightarrow \text{третій тощо.}
\end{align*}

Таким чином, форма $4\pi r^2\rho(r)$ дає прямий зв’язок між результатами квантово-хімічних розрахунків і традиційною атомною структурою, знайомою студентам з курсу атомної фізики.
Візуальний аналіз таких графіків дозволяє простежити закономірності побудови періодичної системи Менделєєва з точки зору електронної густини.

\subsection{Практичні завдання}

\begin{enumerate}
    \item Побудуйте на одному графіку $\rho(r)$ для \ce{Li}, \ce{Na}, \ce{K}. Як змінюється радіус атома?
    \item Знайдіть положення максимумів $4\pi r^2\rho(r)$ для атомів \ce{He}, \ce{Ne}, \ce{Ar}. До яких оболонок вони належать?
    \item Порівняйте радіальні функції для атомів \ce{C} і \ce{O}. Як змінюється густина зовнішньої оболонки?
\end{enumerate}


%% --------------------------------------------------------
\subsection{Методичні рекомендації}
%% --------------------------------------------------------

\begin{enumerate}
%  \item Для легких атомів (\texttt{H}, \texttt{He}, \texttt{Li}) можна порівняти результати з аналітичними розв’язками рівняння Шредінгера.
  \item Змінюючи базис (наприклад, \texttt{STO-3G}, \texttt{6-31G}, \texttt{cc-pVTZ}), можна дослідити, як збільшується точність відтворення радіального профілю.
  \item Для відкритих оболонок (атомів з неспареними електронами) потрібно враховувати спінову поляризацію --- використовується \texttt{UHF}.
\end{enumerate}

Таким чином, наведені приклади демонструють практичний зв’язок між теоретичними поняттями електронної густини і їх чисельною реалізацією в пакеті \texttt{PySCF}.
