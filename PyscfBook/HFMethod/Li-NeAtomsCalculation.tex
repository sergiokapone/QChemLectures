%% --------------------------------------------------------
\section{Атоми другого періоду (Li--Ne)}
%% --------------------------------------------------------

%% --------------------------------------------------------
\subsection{Літій (Li, Z=3)}
%% --------------------------------------------------------

Атом Літію --- перший багатоелектронний атом, у якому проявляється неспарений електрон та спінова поляризація ($N_\alpha \neq N_\beta$, тобто густини $\rho_\alpha$ і $\rho_\beta$ відрізняються).

Електронна конфігурація:
\[
\text{Li: } 1s^2 2s^1.
\]
Два перші електрони утворюють замкнену оболонку (1s), а третій --- одинокий у підоболонці \(2s\), що зумовлює мультиплетність \(2S\) (спін \(S = \frac{1}{2}\)).

Через наявність неспареного електрона метод \texttt{UHF} є природним вибором.
\texttt{UHF} дозволяє альфа- та бета-орбіталям відрізнятися, тобто хвильова функція не змушена мати однакову просторову форму для різних спінів.

Альтернативою є метод \texttt{ROHF (Restricted Open-Shell Hartree–Fock)}, у якому
\begin{itemize}
    \item усі замкнені орбіталі мають однакові просторові функції для $\alpha$ та $\beta$ спінів;
    \item відкриті орбіталі (неспарені) можуть мати різну зайнятість, але однакову форму.
\end{itemize}
ROHF забезпечує правильне значення $\langle S^2 \rangle$ та зберігає спінову симетрію, проте формулювання рівнянь Фока є складнішим.
UHF, навпаки, простіший у реалізації, але може призводити до \textit{spin contamination}.
У практиці обидва методи дають близькі енергії для атома Li, проте ROHF вважається формально більш коректним.

\inputcode{code_7.py}

\paragraph{Коментарі.}
\begin{itemize}
  \item Значення \(\langle S^2 \rangle \approx 0.75\) свідчить про правильний спіновий стан подвійності (мультиплетність \(2S+1=2\)).
  \item Наявність різниці між енергіями альфа- та бета-орбіталей демонструє спінову асиметрію.
  \item Метод UHF може частково порушувати симетрію (так звана \textit{spin contamination}), але для Li це незначно.
\end{itemize}

Енергія, отримана методом HF, для Li становить близько \(-7.432\) Ha у базисі cc-pVDZ, тоді як експериментальна енергія основного стану (повна) --- близько \(-7.478\) Ha.
Таким чином, кореляційна похибка становить близько \(0.046\) Ha (приблизно 1.25 eV).

Цей приклад є першим, де проявляється \emph{кореляція електронів}, яку HF не враховує. В подальших розділах буде показано, як методи MP2, CI та CC враховують ці ефекти.

%% --------------------------------------------------------
\subsection{Берилій (Be, Z=4)}
%% --------------------------------------------------------

Атом Берилію має електронну конфігурацію
\[
\text{Be: } 1s^2 2s^2.
\]
Тут усі орбіталі парні, тому спінова поляризація відсутня, і зручно використовувати \texttt{RHF} (Restricted Hart\-ree–Fock).
Обидва електрони у підоболонці \(2s\) мають протилежні спіни, тож система має мультиплетність \(1S\) (синглетний стан).

\inputcode{code_8.py}

\paragraph{Обговорення.}
\begin{itemize}
  \item Для Be HF-енергія виходить приблизно \(-14.573\) Ha (у базисі cc-pVTZ), тоді як експериментальна --- \(-14.667\) Ha.
  \item Різниця близько \(0.094\) Ha (\(\approx 2.6\) eV) --- це \textbf{кореляційна енергія}, тобто внесок взаємодії електронів, не врахований у HF.
  \item У Берилія ефекти електронної кореляції вже досить значні, адже електрони в орбіталі \(2s\) взаємодіють один з одним.
  \item Цей випадок є гарною ілюстрацією межі застосування методу Гартрі–Фока.
\end{itemize}

\paragraph{Висновки для атомів Li і Be.}
\begin{itemize}
  \item Li демонструє появу неспареного електрона і спінової поляризації (UHF необхідний).
  \item Be --- перший приклад, де \textbf{електронна кореляція} дає помітну похибку енергії.
  \item PySCF дозволяє наочно дослідити вплив базису, спіну, симетрії та методів на точність енергії.
\end{itemize}


%% --------------------------------------------------------
\subsection{Бор--Неон: систематичне дослідження}
%% --------------------------------------------------------

Після розгляду окремих атомів Літію та Берилію доцільно перейти до систематичного аналізу всіх атомів другого періоду (від Бору до Неону).
У цих атомах поступово заповнюється \(2p\)-підрівень, і змінюється як спінова мультиплетність, так і форма електронної густини.
Метод Гартрі–Фока дозволяє побачити, як змінюється енергія системи при збільшенні числа електронів, а також як проявляється спінова поляризація у відкритих оболонках.

\paragraph{Електронні конфігурації.}
\[
\begin{array}{llcl}
\text{Li:} & [\text{He}]\,2s^1 & \quad & {}^2S \\
\text{Be:} & [\text{He}]\,2s^2 & & {}^1S \\
\text{B:}  & [\text{He}]\,2s^2 2p^1 & & {}^2P \\
\text{C:}  & [\text{He}]\,2s^2 2p^2 & & {}^3P \\
\text{N:}  & [\text{He}]\,2s^2 2p^3 & & {}^4S \\
\text{O:}  & [\text{He}]\,2s^2 2p^4 & & {}^3P \\
\text{F:}  & [\text{He}]\,2s^2 2p^5 & & {}^2P \\
\text{Ne:} & [\text{He}]\,2s^2 2p^6 & & {}^1S
\end{array}
\]
Як видно, кількість неспарених електронів збільшується від Бору до Нітрогену, а потім зменшується до Неону, що зумовлює зміну спінового стану та мультиплетності.

\paragraph{Мета дослідження.}
Провести розрахунок енергій Гартрі–Фока для атомів другого періоду в однаковому базисі \texttt{cc-pVDZ}, оцінити правильність спінового стану (через \(\langle S^2 \rangle\)) та проаналізувати систематичні тенденції.

\inputcode{code_9.py}

\paragraph{Коментарі до коду.}
\begin{itemize}
  \item Для кожного атома автоматично обирається тип SCF: \texttt{RHF} (для замкнених оболонок, $S=0$) або \texttt{UHF} (для відкритих).
  \item Параметр \inlinecode{spin} у PySCF означає різницю між кількістю альфа- та бета-електронів:
        \(\text{spin} = N_\alpha - N_\beta = 2S.\)
  \item Розрахунок \(\langle S^2 \rangle\) дозволяє перевірити правильність спінового стану. Для ідеального випадку значення має збігатися з теоретичним \(S(S+1)\).
  \item У файлі \inlinecode{second\_period\_hf.npz} зберігаються всі енергії для подальшого аналізу або побудови графіків.
\end{itemize}

\paragraph{Очікувані тенденції.}
\begin{enumerate}
  \item Повна енергія атома зменшується (стає більш негативною) із зростанням атомного номера $Z$.
  \item Енергія спостерігає стрибки на межі заповнення оболонок: при переходах Be→B, N→O, F→Ne.
  \item Спінова мультиплетність відображає заповнення $2p$-орбіталей згідно з \textbf{правилом Гунда} --- максимальний спін у середині підоболонки (для N).
  \item Значення \(\langle S^2 \rangle\) для UHF повинні бути близькі до теоретичних, але можуть мати невеликі відхилення через \textit{spin contamination}.
\end{enumerate}

\paragraph{Фізичне узагальнення.}
Цей розрахунок демонструє фундаментальну властивість методу Гартрі–Фока:
він добре описує загальну структуру енергетичних рівнів і тенденції в періодичній таблиці, але не враховує електронну кореляцію, через що абсолютні значення енергії мають систематичну похибку.

%\paragraph{Подальші кроки.}
%Наступним логічним етапом є додавання пост-Гартрі–Фок методів (MP2, CI, CCSD), щоб показати, як кореляція електронів уточнює енергії і дозволяє досягати експериментальної точності.
