%% --------------------------------------------------------
\section{Порівняння методів: RHF vs UHF vs ROHF}
%% --------------------------------------------------------

У квантово-хімічних розрахунках вибір типу наближення Гартрі–Фока залежить від спінового стану системи.
Методи \texttt{RHF}, \texttt{UHF} та \texttt{ROHF} відрізняються тим, як вони поводяться із спіновими функціями електронів — тобто, чи допускають різні орбіталі для $\alpha$- і $\beta$-електронів.

\begin{itemize}
    \item \texttt{RHF (Restricted Hartree–Fock)} --- усі електрони спарені, кожна просторова орбіталь заповнена двома електронами з протилежними спінами.
    Підходить лише для синглетних закритих оболонок.

    \item \texttt{UHF (Unrestricted Hartree–Fock)} --- орбіталі для $\alpha$- і $\beta$-електронів дозволено різні.
    Цей метод придатний для відкритих оболонок (наприклад, атомів з неспареними електронами), однак може страждати від \emph{спінового забруднення (spin contamination)}.

    \item \texttt{ROHF (Restricted Open-Shell Hartree–Fock)} --- компроміс: спільні орбіталі для парних електронів і різні --- лише для неспарених.
    Це забезпечує правильний спіновий симетрійний стан без забруднення.
\end{itemize}

\subsubsection{Енергетичне порівняння}

\texttt{UHF} зазвичай дає трохи нижчу енергію, ніж ROHF, оскільки має більше варіаційної свободи.
Різниця $\Delta E = E_{\mathrm{UHF}} - E_{\mathrm{ROHF}}$ зазвичай незначна (менше кількох міліГартрі), однак у системах з сильним спіновим змішуванням може бути суттєвою.

\subsubsection{Орбітальні енергії}

\texttt{UHF} формує два набори орбіталей --- для $\alpha$- і $\beta$-електронів.
Тому орбітальні енергії можуть відрізнятись:
\[
\varepsilon_i^{(\alpha)} \neq \varepsilon_i^{(\beta)}.
\]
У \texttt{ROHF} усі парні орбіталі спільні, тож орбітальні енергії чітко впорядковані відповідно до симетрії та спіну.

%% --------------------------------------------------------
\subsection{Спінове забруднення}
%% --------------------------------------------------------

У системах з відкритими оболонками метод \texttt{UHF} часто використовується як простіше наближення, що дозволяє займати незалежні орбіталі електронам зі спінами $\alpha$ та $\beta$.
Однак відсутність суворого обмеження на спінову симетрію призводить до важливого недоліку --- \textbf{спінового забруднення}.


\subsubsection*{Математична суть явища}

\texttt{UHF}-хвильова функція не є власним станом оператора $\hat{S}^2$, хоч і залишається власним станом $\hat{S}_z$:
\[
\hat{S}_z \Psi_{\text{UHF}} = M_S \Psi_{\text{UHF}}, \qquad
\hat{S}^2 \Psi_{\text{UHF}} \neq S(S+1)\Psi_{\text{UHF}}.
\]
Отже, очікуване значення
\[
\langle S^2 \rangle = \langle \Psi_{\text{UHF}} | \hat{S}^2 | \Psi_{\text{UHF}} \rangle
\]
зазвичай перевищує істинне $S(S+1)$, тобто:
\[
\Delta S^2 = \langle S^2 \rangle - S(S+1) > 0.
\]

\subsubsection*{Фізичне тлумачення}

UHF-хвильова функція може бути подана як суперпозиція чистих спінових станів:
\[
\Psi_{\text{UHF}} = c_0 \Psi_{S} + c_1 \Psi_{S+1} + c_2 \Psi_{S+2} + \cdots,
\]
де присутність коефіцієнтів $c_i \ne 0$ для $i>0$ означає змішування різних мультиплетів.
Наприклад, для триплетного стану ($S=1$) теоретичне значення $\langle S^2 \rangle = 2.0$, але якщо розрахунок UHF дає $2.25$, це означає, що хвильова функція містить домішку п’ятичленного ($S=2$) стану.

\subsubsection*{Вплив на результати}

\begin{itemize}
    \item \textbf{Енергія.} \texttt{UHF} може давати занадто низьку енергію через змішування спінів і штучне зменшення кулонівського відштовхування.
    \item \textbf{Електронна густина.} Спінова густина стає несиметричною, особливо поблизу атомів з неспареними електронами.
    \item \textbf{Магнітні властивості.} Похибки у спіновій густині ведуть до помилкових магнітних моментів і спотворених мультиплетних розщеплень.
\end{itemize}

%% --------------------------------------------------------
\subsubsection{Як обчислюється спінове забруднення}
%% --------------------------------------------------------

Програми квантово-хімічних розрахунків (зокрема \texttt{PySCF}, \texttt{Gaussian}, \texttt{ORCA}) обчислюють спінове забруднення не безпосередньо через оператор $\hat{S}^2$, а через перекриття між орбіталями спінів $\alpha$ і $\beta$.

Основна формула для середнього значення $\langle S^2 \rangle$ у наближенні \texttt{UHF} має вигляд:
\[
\boxed{
\langle S^2 \rangle =
\frac{(N_\alpha - N_\beta)^2}{4}
+ \frac{N_\alpha + N_\beta}{2}
- \sum_{i=1}^{N_\alpha} \sum_{j=1}^{N_\beta}
|\langle \phi_i^\alpha | \phi_j^\beta \rangle|^2
}
\]
де:
\begin{itemize}
    \item $N_\alpha$, $N_\beta$ — кількість електронів зі спінами $\alpha$ та $\beta$ відповідно;
    \item $\phi_i^\alpha$, $\phi_j^\beta$ --- молекулярні(атомні) орбіталі для спінів $\alpha$ і $\beta$;
    \item $\langle \phi_i^\alpha | \phi_j^\beta \rangle$ --- перекриття між орбіталями різних спінів.
\end{itemize}

Якщо орбіталі $\alpha$ і $\beta$ однакові, тобто $\phi_i^\alpha = \phi_i^\beta$, то:
\[
\langle S^2 \rangle = \frac{(N_\alpha - N_\beta)^2}{4},
\]
і для синглету ($N_\alpha = N_\beta$) це дорівнює нулю --- спінове забруднення відсутнє.

\subsubsection*{Інтерпретація}

Якщо орбіталі $\alpha$ і $\beta$ подібні, перекриття $S_{\alpha\beta}$ велике, тому $\Sigma$ наближається до $N_\beta$, і $\langle S^2 \rangle$ близьке до теоретичного $S(S+1)$ — тобто забруднення мінімальне.
Якщо ж орбіталі сильно відрізняються (наприклад, при розриві зв’язку або в радикалах), $\Sigma$ зменшується, і $\langle S^2 \rangle$ зростає — з’являється суттєве спінове забруднення.


\subsubsection*{Діагностика та критерії}

Значення $\langle S^2 \rangle$ зазвичай друкується у вихідних даних програми.
Практичні межі:
\[
\Delta S^2 = \langle S^2 \rangle - S(S+1),
\]
\[
\begin{cases}
\Delta S^2 < 0.05 & \text{— незначне забруднення, прийнятно;}\\
0.05 < \Delta S^2 < 0.10 & \text{— помірне, бажано перевірити;}\\
\Delta S^2 > 0.10 & \text{— суттєве, слід перейти до ROHF або проєкційних методів.}
\end{cases}
\]


\subsubsection*{Приклад чисельного розрахунку}

Для системи з $N_\alpha = 5$, $N_\beta = 4$ (очікувано $S=\frac{1}{2}$, $\langle S^2 \rangle = 0.75$):

\begin{itemize}
    \item якщо $\Sigma = 3.9$, тоді
    \[
    \langle S^2 \rangle = \frac{1}{4} + \frac{9}{2} - 3.9 = 0.85,
    \]
    \[
    \Delta S^2 = 0.85 - 0.75 = 0.10,
    \]
    тобто забруднення перебуває на межі суттєвого;

    \item якщо $\Sigma = 3.4$, тоді
    \[
    \langle S^2 \rangle = \frac{1}{4} + \frac{9}{2} - 3.4 = 1.35,
    \]
    \[
    \Delta S^2 = 1.35 - 0.75 = 0.60,
    \]
    що свідчить про сильне спінове забруднення.
\end{itemize}

\subsubsection*{Як зменшити спінове забруднення}

\begin{enumerate}
    \item \textbf{ROHF:} обмежена відкрита схема Гартрі–Фока, що зберігає спінову симетрію.
    \item \textbf{Спін-проекційні методи:} після варіації застосовується оператор проєкції
    \[
    \Psi_{\text{proj}} = \hat{P}_S \Psi_{\text{UHF}},
    \]
    який видаляє небажані компоненти.
    \item \textbf{Методи після HF:} MP2, CISD, CCSD частково компенсують спінове забруднення через багатоконфігураційність.
\end{enumerate}


%% --------------------------------------------------------
\subsection{Тестовий випадок: атом Вуглецю}
%% --------------------------------------------------------

Вуглець має конфігурацію $1s^2 2s^2 2p^2$ і основний стан $^3P$ (триплет).
Отже, система має два неспарені електрони ($S=1$), а отже $\langle S^2 \rangle = S(S+1) = 2.0$.

Далі наведено приклад прямого порівняння результатів методів UHF і ROHF для атома C у триплетному стані з використанням базису \texttt{cc-pVTZ}.

\inputcode{code_15.py}

Щоб узагальнити різницю між методами UHF та ROHF, виконаємо серію розрахунків для
кількох відкритооболонкових атомів перших двох періодів. Це дозволить оцінити
енергетичну різницю $\Delta E$ і побачити,
чи зберігає ROHF спінову симетрію без втрати точності.



\inputcode{code_16.py}


\noindent
\textbf{Інтерпретація результатів.}
Як правило, для легких атомів (H, Li, B) енергії UHF і ROHF майже збігаються.
Для атомів із більшою кількістю незапарених електронів (C, N, O, F)
різниця у кілька mHa відображає більшу гнучкість UHF,
що дозволяє частково знизити енергію ціною забруднення спіном
($\langle S^2 \rangle > S(S+1)$).

Таким чином, \texttt{ROHF} --- це більш строгий метод із правильною спіновою симетрією,
а \texttt{UHF} --- енергетично вигідніший, але менш «фізично чистий». У практичних
обчисленнях ROHF часто слугує базовою точкою для подальших кореляційних
методів (MP2, CCSD, CASSCF).

\subsubsection{Практичне значення}

Порівняння методів RHF, UHF і ROHF дозволяє студентам зрозуміти:
\begin{itemize}
    \item коли можна використовувати RHF (синглети закритих оболонок);
    \item коли необхідно застосовувати UHF (відкриті оболонки, магнітні системи);
    \item у яких випадках варто віддати перевагу ROHF (спінова чистота, мінімізація забруднення).
\end{itemize}

\subsubsection{Контрольне завдання}

\begin{enumerate}
    \item Повторіть розрахунок для атомів \ce{O} (спін = 2) і \ce{N} (спін = 3).
    Порівняйте $\langle S^2 \rangle$ для UHF і ROHF.
    \item Для кожного атома побудуйте різницю енергій UHF–ROHF у mHa.
    \item Проаналізуйте, як спінове забруднення зростає зі збільшенням кількості неспарених електронів.
\end{enumerate}