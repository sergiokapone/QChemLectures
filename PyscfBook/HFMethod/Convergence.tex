%% --------------------------------------------------------
\section{Складні випадки та збіжність}
%% --------------------------------------------------------

Розрахунки у квантовій хімії не завжди проходять гладко. Навіть для простих систем ітераційна процедура самозгодженого поля (SCF) може не досягати стабільного рішення.
Найчастіше проблеми збіжності виникають для перехідних металів, відкритих оболонок та систем із виродженими орбіталями.
У цьому розділі розглянемо основні джерела труднощів та методи їх подолання.

\subsection{Причини поганої збіжності}

Типові джерела нестабільності SCF:
\begin{itemize}
  \item сильна електронна кореляція у незаповнених $d$- і $f$-оболонках;
  \item мала різниця енергій між орбіталями (виродження);
  \item невдалий початковий вектор коефіцієнтів (погане стартове наближення);
  \item занадто жорсткі або занадто м’які критерії збіжності.
\end{itemize}

Через ці фактори енергія може осцилювати, DIIS — розходитись, а результат — бути нефізичним.

\subsection{Перехідні метали}

Перехідні елементи (Fe, Co, Ni, Cr, Mn тощо) — класичні приклади систем із поганою збіжністю SCF.
Причини:
\begin{itemize}
  \item близькість енергетичних рівнів $3d$ і $4s$-орбіталей;
  \item можливість декількох спінових станів, близьких за енергією;
  \item наявність кількох локальних мінімумів функціоналу енергії.
\end{itemize}

У PySCF це проявляється як:
\begin{itemize}
  \item осциляції енергії між ітераціями;
  \item розбігання DIIS;
  \item стрибки значення $\langle S^2 \rangle$ між кроками.
\end{itemize}

Нижче наведено приклад універсальної функції для стабільного розрахунку атомів перехідних металів із урахуванням практичних прийомів:

\begin{itemize}
  \item використовується UHF (для врахування спіну);
  \item застосовується зсув рівнів (\inlinecode{level\_shift});
  \item розширюється DIIS-простір (\inlinecode{diis\_space});
  \item у разі потреби вмикається уточнення Ньютона–Рафсона;
  \item контролюється спін-забруднення через $\langle S^2 \rangle$.
\end{itemize}

\inputcode{code_18.py}

Такий підхід забезпечує стабільність навіть для важких атомів, де стандартний SCF часто не сходиться.

\subsection{Вироджені орбіталі та дробові заповнення}

Якщо в системі наявні вироджені або майже вироджені орбіталі, SCF може ``коливатись'', не вибираючи між ними.
Для таких випадків ефективним є застосування \textit{дробових заповнень орбіталей} (\textit{fractional occupations}), що згладжують різницю між рівнями енергії.

Ідея полягає у введенні ефективного теплового розмазування Фермі–Дірака:
\[
f_i = \frac{1}{1 + e^{(\varepsilon_i - \mu)/kT}},
\]
де $f_i$ — часткове заповнення орбіталі $i$, $\varepsilon_i$ — її енергія, $\mu$ — хімічний потенціал, а $kT$ — параметр розмазування.

Цей підхід дозволяє SCF уникнути нестійких перестрибувань між виродженими рівнями.

У PySCF для цього достатньо викликати \inlinecode{scf.addons.frac\_occ(mf)}.
Методика добре працює для атомів (V, Cr), радикалів та малих кластерів перехідних металів.

\inputcode{code_19.py}



\subsection{Стратегія досягнення збіжності SCF}

Для надійного отримання фізично коректного рішення рекомендується послідовна стратегія стабілізації SCF — від простих прийомів до складніших:

\begin{enumerate}
  \item стандартний UHF;
  \item зсув рівнів (\inlinecode{level\_shift});
  \item збільшення DIIS-простору (\inlinecode{diis\_space});
  \item атомне початкове наближення (\inlinecode{init\_guess='atom'});
  \item уточнення за методом Ньютона–Рафсона;
  \item дробові заповнення (\inlinecode{frac\_occ}) для вироджених станів.
\end{enumerate}

\inputcode{code_20.py}

\paragraph{Фізичний зміст прийомів.}
\begin{itemize}
  \item \textbf{Level shift} — підвищує енергії віртуальних орбіталей, запобігаючи коливанням.
  \item \textbf{DIIS-простір} — збільшення пам’яті ітерацій покращує апроксимацію поля.
  \item \textbf{Atom guess} — старт з атомних орбіталей ближчий до реального розв’язку.
  \item \textbf{Newton–Raphson} — забезпечує квадратичну збіжність поблизу мінімуму.
  \item \textbf{Fractional occupations} — стабілізують вироджені стани, забезпечуючи плавний перехід.
\end{itemize}

\paragraph{Практичні рекомендації.}
\begin{itemize}
  \item Для великих систем на перших етапах можна зменшити точність: \inlinecode{conv\_tol=1e-6}.
  \item Якщо енергія осцилює — увімкніть \inlinecode{level\_shift=0.5}.
  \item Якщо \texttt{UHF} не сходиться — використайте \inlinecode{init\_guess='atom'} або вимкніть симетрію (\inlinecode{symmetry=False}).
  \item Завжди перевіряйте фізичність результату через $\langle S^2 \rangle$.
\end{itemize}

\medskip

\noindent Таким чином, навіть якщо стандартна процедура SCF не збігається,
послідовне застосування описаних прийомів практично гарантує стабільне та фізично обґрунтоване рішення.
