%% --------------------------------------------------------
\section{Функціонали обміну-кореляції}
%% --------------------------------------------------------


%% --------------------------------------------------------
\subsection{Класифікація функціоналів: драбина Якова}
%% --------------------------------------------------------

Функціонали DFT класифікуються за "драбиною Якова" (Jacob's ladder), запропонованою Пердью:

\begin{enumerate}
    \item \textbf{LDA/LSDA} (Local Density Approximation) --- залежать тільки від $\rho(\mathbf{r})$
    \item \textbf{GGA} (Generalized Gradient Approximation) --- залежать від $\rho(\mathbf{r})$ та $\nabla\rho(\mathbf{r})$
    \item \textbf{Meta-GGA} --- додатково залежать від $\nabla^2\rho$ або кінетичної густини $\tau$
    \item \textbf{Hybrid} --- включають частку точного обміну HF
    \item \textbf{Double-hybrid} --- включають і HF обмін, і MP2 кореляцію
\end{enumerate}


%% --------------------------------------------------------
\subsection{LDA та LSDA функціонали}
%% --------------------------------------------------------

%% --------------------------------------------------------
\subsubsection{Локальне наближення густини}
%% --------------------------------------------------------

LDA базується на моделі однорідного електронного газу:
\begin{equation}
E_{xc}^{\text{LDA}}[\rho] = \int \rho(\mathbf{r}) \varepsilon_{xc}(\rho(\mathbf{r})) d\mathbf{r}
\end{equation}

де $\varepsilon_{xc}(\rho)$ --- обмінно-кореляційна енергія на електрон в однорідному газі густини $\rho$.

\paragraph{Обмінна частина (Slater/Dirac):}
\begin{equation}
\varepsilon_x^{\text{LDA}}(\rho) = -\frac{3}{4}\left(\frac{3}{\pi}\right)^{1/3} \rho^{1/3}
\end{equation}

\paragraph{Кореляційна частина:} Використовуються параметризації (VWN, PW92).

\subsubsection{LSDA для спін-поляризованих систем}

Для систем з різними $\rho_\alpha$ та $\rho_\beta$:
\begin{equation}
E_{xc}^{\text{LSDA}}[\rho_\alpha, \rho_\beta] = \int \rho \varepsilon_{xc}(\rho_\alpha, \rho_\beta) d\mathbf{r}
\end{equation}

\inputcode{code_1.py}

%% --------------------------------------------------------
\subsection{GGA функціонали}
%% --------------------------------------------------------

GGA функціонали враховують не тільки локальну густину, але й її градієнт:
\begin{equation}
E_{xc}^{\text{GGA}}[\rho] = \int f(\rho(\mathbf{r}), \nabla\rho(\mathbf{r})) d\mathbf{r}
\end{equation}

%% --------------------------------------------------------
\subsubsection{Популярні GGA функціонали}
%% --------------------------------------------------------

\paragraph{PBE (Perdew-Burke-Ernzerhof, 1996)}
Найпопулярніший неемпіричний GGA функціонал:

\inputcode{code_2.py}

\paragraph{BLYP (Becke88 + Lee-Yang-Parr)}
Комбінація обміну Becke88 та кореляції LYP:

\inputcode{code_3.py}

\paragraph{BP86 (Becke88 + Perdew86)}

\inputcode{code_4.py}

\subsubsection{Порівняння GGA функціоналів}

\inputcode{code_5.py}

%% --------------------------------------------------------
\subsection{Meta-GGA функціонали}
%% --------------------------------------------------------

Meta-GGA функціонали включають додаткову інформацію про систему: кінетичну густину $\tau$ або лапласіан густини $\nabla^2\rho$:

\begin{equation}
E_{xc}^{\text{meta-GGA}}[\rho] = \int f(\rho, \nabla\rho, \tau) d\mathbf{r}
\end{equation}

де кінетична густина орбіталей Кона-Шема:
\begin{equation}
\tau(\mathbf{r}) = \frac{1}{2} \sum_{i=1}^{N} |\nabla\psi_i(\mathbf{r})|^2
\end{equation}

\subsubsection{TPSS (Tao-Perdew-Staroverov-Scuseria)}

TPSS --- один з перших успішних meta-GGA функціоналів (2003):

\inputcode{code_6.py}

\subsubsection{M06-L (Minnesota 06 Local)}

Високопараметризований meta-GGA функціонал для широкого спектру задач:

\inputcode{code_7.py}

\subsubsection{SCAN (Strongly Constrained and Appropriately Normed)}

Сучасний meta-GGA, що задовольняє всі відомі точні умови:

\inputcode{code_8.py}

%% --------------------------------------------------------
\subsection{Гібридні функціонали}
%% --------------------------------------------------------

Гібридні функціонали комбінують DFT обмін з точним (HF) обміном:

\begin{equation}
E_{xc}^{\text{hybrid}} = a \cdot E_x^{\text{HF}} + (1-a) \cdot E_x^{\text{DFT}} + E_c^{\text{DFT}}
\end{equation}

де $a$ --- частка HF обміну (зазвичай 0.2--0.3).

\subsubsection{B3LYP (Becke 3-parameter Lee-Yang-Parr)}

Найпопулярніший гібридний функціонал у хімії:

\begin{equation}
E_{xc}^{\text{B3LYP}} = E_x^{\text{LDA}} + 0.2(E_x^{\text{HF}} - E_x^{\text{LDA}}) + 0.72 \cdot E_x^{\text{B88}} + 0.81 \cdot E_c^{\text{LYP}} + 0.19 \cdot E_c^{\text{VWN}}
\end{equation}

\inputcode{code_9.py}

\subsubsection{PBE0 (PBE hybrid)}

Неемпіричний гібрид з 25% HF обміну:

\begin{equation}
E_{xc}^{\text{PBE0}} = 0.25 \cdot E_x^{\text{HF}} + 0.75 \cdot E_x^{\text{PBE}} + E_c^{\text{PBE}}
\end{equation}

\inputcode{code_10.py}


\subsubsection{CAM-B3LYP (Coulomb-Attenuated Method)}

Функціонал з дальньодіючою корекцією (range-separated):

\begin{equation}
\frac{1}{r_{12}} = \frac{\alpha + \beta \cdot \text{erf}(\mu r_{12})}{r_{12}} + \frac{1-[\alpha+\beta \cdot \text{erf}(\mu r_{12})]}{r_{12}}
\end{equation}

\inputcode{code_11.py}

%% --------------------------------------------------------
\subsubsection{M06-2X та інші Minnesota функціонали}
%% --------------------------------------------------------

Родина M06 з різною часткою HF обміну:

\begin{itemize}
    \item M06-L: 0\% HF (meta-GGA)
    \item M06: 27\% HF
    \item M06-2X: 54\% HF (для кінетики, слабких взаємодій)
    \item M06-HF: 100\% HF
\end{itemize}

\inputcode{code_12.py}

%% --------------------------------------------------------
\subsection{Порівняння різних рівнів теорії}
%% --------------------------------------------------------

\inputcode{code_13.py}

\subsection{Вибір функціоналу: рекомендації}

\begin{table}[h]
\centering
\caption{Рекомендовані функціонали для різних задач}
\label{tab:functional_recommendations}
\small
\begin{tabular}{lll}
\hline
\textbf{Задача} & \textbf{Функціонал} & \textbf{Коментар} \\
\hline
Швидкі розрахунки & PBE & Добра точність/швидкість \\
Енергії атомізації & B3LYP, PBE0 & Стандарт у хімії \\
Перехідні метали & TPSSh, M06 & Краще для d-електронів \\
Слабкі взаємодії & M06-2X, ωB97X-D & З дисперсією \\
Високоспінові стани & SCAN, TPSSh & Кращий баланс \\
Загальні розрахунки & PBE0 & Універсальний вибір \\
Максимальна точність & CCSD(T) & Але дуже повільно \\
\hline
\end{tabular}
\end{table}

%% --------------------------------------------------------
\subsection{Практичний приклад: вплив функціоналу на властивості}
%% --------------------------------------------------------

\inputcode{code_14.py}