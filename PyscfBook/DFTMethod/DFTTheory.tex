%% --------------------------------------------------------
\section{Основи теорії функціоналу густини}
%% --------------------------------------------------------


%% --------------------------------------------------------
\subsection{Теореми Хоенберга--Кона}
%% --------------------------------------------------------

Теорія функціоналу густини базується на двох фундаментальних теоремах, доведених Хоенбергом та Коном у 1964 році:

\paragraph{Теорема 1 (Теорема існування).}
Зовнішній потенціал $v_{\text{ext}}(\mathbf{r})$ (а отже, і повна енергія системи) однозначно визначається електронною густиною основного стану $\rho(\mathbf{r})$ з точністю до адитивної константи.

Це означає, що електронна густина містить всю інформацію про систему:
\begin{equation}
E[\rho] = T[\rho] + V_{ee}[\rho] + \int v_{\text{ext}}(\mathbf{r}) \rho(\mathbf{r}) d\mathbf{r}
\end{equation}

\paragraph{Теорема 2 (Варіаційний принцип).}
Існує універсальний функціонал енергії $F[\rho]$, який для будь-якої пробної густини $\tilde{\rho}(\mathbf{r})$ задовольняє:
\begin{equation}
E_0 \leq E[\tilde{\rho}] = F[\tilde{\rho}] + \int v_{\text{ext}}(\mathbf{r}) \tilde{\rho}(\mathbf{r}) d\mathbf{r}
\end{equation}

де $E_0$ --- точна енергія основного стану.

\subsection{Рівняння Кона--Шема}

Кон та Шем (1965) запропонували практичний підхід до DFT, замінивши взаємодіючу систему еквівалентною невзаємодіючою системою з такою ж густиною:

\begin{equation}
\left[-\frac{1}{2}\nabla^2 + v_{\text{eff}}(\mathbf{r})\right] \psi_i(\mathbf{r}) = \varepsilon_i \psi_i(\mathbf{r})
\end{equation}

Ефективний потенціал визначається як:
\begin{equation}
v_{\text{eff}}(\mathbf{r}) = v_{\text{ext}}(\mathbf{r}) + v_H(\mathbf{r}) + v_{xc}(\mathbf{r})
\end{equation}

де:
\begin{itemize}
    \item $v_{\text{ext}}(\mathbf{r})$ --- зовнішній потенціал (від ядер)
    \item $v_H(\mathbf{r}) = \int \frac{\rho(\mathbf{r}')}{\|\mathbf{r}-\mathbf{r}'\|} d\mathbf{r}'$ --- потенціал Хартрі
    \item $v_{xc}(\mathbf{r}) = \frac{\delta E_{xc}[\rho]}{\delta \rho(\mathbf{r})}$ --- обмінно-кореляційний потенціал
\end{itemize}

Електронна густина обчислюється через орбіталі Кона--Шема:
\begin{equation}
\rho(\mathbf{r}) = \sum_{i=1}^{N} |\psi_i(\mathbf{r})|^2
\end{equation}


%% --------------------------------------------------------
\subsection{Обмінно-кореляційна енергія}
%% --------------------------------------------------------

Повна енергія в DFT:
\begin{equation}
E[\rho] = T_s[\rho] + V_{ext}[\rho] + J[\rho] + E_{xc}[\rho]
\end{equation}

де $E_{xc}[\rho]$ --- обмінно-кореляційна енергія, яка містить:
\begin{itemize}
    \item Різницю між точною кінетичною енергією та кінетичною енергією невзаємодіючої системи
    \item Некласичну частину електрон-електронного відштовхування (обмін + кореляція)
\end{itemize}

Точний вигляд $E_{xc}[\rho]$ невідомий, тому використовуються наближення (функціонали).

%% --------------------------------------------------------
\subsection{Порівняння HF та DFT}
%% --------------------------------------------------------

\begin{center}
\captionof{figure}{Порівняння методів Хартрі-Фока та DFT}
\label{tab:hf_vs_dft}
\begin{tabular}{lll}
\hline
\textbf{Аспект} & \textbf{Hartree-Fock} & \textbf{DFT} \\
\hline
Базова змінна & Хвильова функція & Електронна густина \\
Обмін & Точний (нелокальний) & Наближений (локальний) \\
Кореляція & Відсутня & Включена наближено \\
Масштабування & $\mathcal{O}(N^4)$ & $\mathcal{O}(N^3)$ \\
Точність для атомів & Добра якісно & Часто краща кількісно \\
Збуджені стани & Можливі & Складно (TD-DFT) \\
\hline
\end{tabular}
\end{center}


