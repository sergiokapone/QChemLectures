% !TeX program = lualatex
% !TeX encoding = utf8
% !TeX spellcheck = uk_UA
% !TeX root =../PyscfBook.tex

%=========================================================
\Opensolutionfile{answer}[\currfilebase/\currfilebase-Answers]
\chapter{Теорія функціоналу густини (DFT)}\label{\currfilebase}
%=========================================================


%% --------------------------------------------------------
\section{Основи теорії функціоналу густини}
%% --------------------------------------------------------


%% --------------------------------------------------------
\subsection{Теореми Хоенберга--Кона}
%% --------------------------------------------------------

Теорія функціоналу густини базується на двох фундаментальних теоремах, доведених Хоенбергом та Коном у 1964 році:

\paragraph{Теорема 1 (Теорема існування).}
Зовнішній потенціал $v_{\text{ext}}(\mathbf{r})$ (а отже, і повна енергія системи) однозначно визначається електронною густиною основного стану $\rho(\mathbf{r})$ з точністю до адитивної константи.

Це означає, що електронна густина містить всю інформацію про систему:
\begin{equation}
E[\rho] = T[\rho] + V_{ee}[\rho] + \int v_{\text{ext}}(\mathbf{r}) \rho(\mathbf{r}) d\mathbf{r}
\end{equation}

\paragraph{Теорема 2 (Варіаційний принцип).}
Існує універсальний функціонал енергії $F[\rho]$, який для будь-якої пробної густини $\tilde{\rho}(\mathbf{r})$ задовольняє:
\begin{equation}
E_0 \leq E[\tilde{\rho}] = F[\tilde{\rho}] + \int v_{\text{ext}}(\mathbf{r}) \tilde{\rho}(\mathbf{r}) d\mathbf{r}
\end{equation}

де $E_0$ --- точна енергія основного стану.

\subsection{Рівняння Кона--Шема}

Кон та Шем (1965) запропонували практичний підхід до DFT, замінивши взаємодіючу систему еквівалентною невзаємодіючою системою з такою ж густиною:

\begin{equation}
\left[-\frac{1}{2}\nabla^2 + v_{\text{eff}}(\mathbf{r})\right] \psi_i(\mathbf{r}) = \varepsilon_i \psi_i(\mathbf{r})
\end{equation}

Ефективний потенціал визначається як:
\begin{equation}
v_{\text{eff}}(\mathbf{r}) = v_{\text{ext}}(\mathbf{r}) + v_H(\mathbf{r}) + v_{xc}(\mathbf{r})
\end{equation}

де:
\begin{itemize}
    \item $v_{\text{ext}}(\mathbf{r})$ --- зовнішній потенціал (від ядер)
    \item $v_H(\mathbf{r}) = \int \frac{\rho(\mathbf{r}')}{\|\mathbf{r}-\mathbf{r}'\|} d\mathbf{r}'$ --- потенціал Хартрі
    \item $v_{xc}(\mathbf{r}) = \frac{\delta E_{xc}[\rho]}{\delta \rho(\mathbf{r})}$ --- обмінно-кореляційний потенціал
\end{itemize}

Електронна густина обчислюється через орбіталі Кона--Шема:
\begin{equation}
\rho(\mathbf{r}) = \sum_{i=1}^{N} |\psi_i(\mathbf{r})|^2
\end{equation}


%% --------------------------------------------------------
\subsection{Обмінно-кореляційна енергія}
%% --------------------------------------------------------

Повна енергія в DFT:
\begin{equation}
E[\rho] = T_s[\rho] + V_{ext}[\rho] + J[\rho] + E_{xc}[\rho]
\end{equation}

де $E_{xc}[\rho]$ --- обмінно-кореляційна енергія, яка містить:
\begin{itemize}
    \item Різницю між точною кінетичною енергією та кінетичною енергією невзаємодіючої системи
    \item Некласичну частину електрон-електронного відштовхування (обмін + кореляція)
\end{itemize}

Точний вигляд $E_{xc}[\rho]$ невідомий, тому використовуються наближення (функціонали).

%% --------------------------------------------------------
\subsection{Порівняння HF та DFT}
%% --------------------------------------------------------

\begin{center}
\captionof{figure}{Порівняння методів Хартрі-Фока та DFT}
\label{tab:hf_vs_dft}
\begin{tabular}{lll}
\hline
\textbf{Аспект} & \textbf{Hartree-Fock} & \textbf{DFT} \\
\hline
Базова змінна & Хвильова функція & Електронна густина \\
Обмін & Точний (нелокальний) & Наближений (локальний) \\
Кореляція & Відсутня & Включена наближено \\
Масштабування & $\mathcal{O}(N^4)$ & $\mathcal{O}(N^3)$ \\
Точність для атомів & Добра якісно & Часто краща кількісно \\
Збуджені стани & Можливі & Складно (TD-DFT) \\
\hline
\end{tabular}
\end{center}


%% --------------------------------------------------------
\section{Функціонали обміну-кореляції}
%% --------------------------------------------------------


%% --------------------------------------------------------
\subsection{Класифікація функціоналів: драбина Якова}
%% --------------------------------------------------------

Функціонали DFT класифікуються за "драбиною Якова" (Jacob's ladder), запропонованою Пердью:

\begin{enumerate}
    \item \textbf{LDA/LSDA} (Local Density Approximation) --- залежать тільки від $\rho(\mathbf{r})$
    \item \textbf{GGA} (Generalized Gradient Approximation) --- залежать від $\rho(\mathbf{r})$ та $\nabla\rho(\mathbf{r})$
    \item \textbf{Meta-GGA} --- додатково залежать від $\nabla^2\rho$ або кінетичної густини $\tau$
    \item \textbf{Hybrid} --- включають частку точного обміну HF
    \item \textbf{Double-hybrid} --- включають і HF обмін, і MP2 кореляцію
\end{enumerate}


%% --------------------------------------------------------
\subsection{LDA та LSDA функціонали}
%% --------------------------------------------------------

%% --------------------------------------------------------
\subsubsection{Локальне наближення густини}
%% --------------------------------------------------------

LDA базується на моделі однорідного електронного газу:
\begin{equation}
E_{xc}^{\text{LDA}}[\rho] = \int \rho(\mathbf{r}) \varepsilon_{xc}(\rho(\mathbf{r})) d\mathbf{r}
\end{equation}

де $\varepsilon_{xc}(\rho)$ --- обмінно-кореляційна енергія на електрон в однорідному газі густини $\rho$.

\paragraph{Обмінна частина (Slater/Dirac):}
\begin{equation}
\varepsilon_x^{\text{LDA}}(\rho) = -\frac{3}{4}\left(\frac{3}{\pi}\right)^{1/3} \rho^{1/3}
\end{equation}

\paragraph{Кореляційна частина:} Використовуються параметризації (VWN, PW92).

\subsubsection{LSDA для спін-поляризованих систем}

Для систем з різними $\rho_\alpha$ та $\rho_\beta$:
\begin{equation}
E_{xc}^{\text{LSDA}}[\rho_\alpha, \rho_\beta] = \int \rho \varepsilon_{xc}(\rho_\alpha, \rho_\beta) d\mathbf{r}
\end{equation}

\inputcode{code_1.py}

%% --------------------------------------------------------
\subsection{GGA функціонали}
%% --------------------------------------------------------

GGA функціонали враховують не тільки локальну густину, але й її градієнт:
\begin{equation}
E_{xc}^{\text{GGA}}[\rho] = \int f(\rho(\mathbf{r}), \nabla\rho(\mathbf{r})) d\mathbf{r}
\end{equation}

%% --------------------------------------------------------
\subsubsection{Популярні GGA функціонали}
%% --------------------------------------------------------

\paragraph{PBE (Perdew-Burke-Ernzerhof, 1996)}
Найпопулярніший неемпіричний GGA функціонал:

\inputcode{code_2.py}

\paragraph{BLYP (Becke88 + Lee-Yang-Parr)}
Комбінація обміну Becke88 та кореляції LYP:

\inputcode{code_3.py}

\paragraph{BP86 (Becke88 + Perdew86)}

\inputcode{code_4.py}

\subsubsection{Порівняння GGA функціоналів}

\inputcode{code_5.py}

%% --------------------------------------------------------
\subsection{Meta-GGA функціонали}
%% --------------------------------------------------------

Meta-GGA функціонали включають додаткову інформацію про систему: кінетичну густину $\tau$ або лапласіан густини $\nabla^2\rho$:

\begin{equation}
E_{xc}^{\text{meta-GGA}}[\rho] = \int f(\rho, \nabla\rho, \tau) d\mathbf{r}
\end{equation}

де кінетична густина орбіталей Кона-Шема:
\begin{equation}
\tau(\mathbf{r}) = \frac{1}{2} \sum_{i=1}^{N} |\nabla\psi_i(\mathbf{r})|^2
\end{equation}

\subsubsection{TPSS (Tao-Perdew-Staroverov-Scuseria)}

TPSS --- один з перших успішних meta-GGA функціоналів (2003):

\inputcode{code_6.py}

\subsubsection{M06-L (Minnesota 06 Local)}

Високопараметризований meta-GGA функціонал для широкого спектру задач:

\inputcode{code_7.py}

\subsubsection{SCAN (Strongly Constrained and Appropriately Normed)}

Сучасний meta-GGA, що задовольняє всі відомі точні умови:

\inputcode{code_8.py}

%% --------------------------------------------------------
\subsection{Гібридні функціонали}
%% --------------------------------------------------------

Гібридні функціонали комбінують DFT обмін з точним (HF) обміном:

\begin{equation}
E_{xc}^{\text{hybrid}} = a \cdot E_x^{\text{HF}} + (1-a) \cdot E_x^{\text{DFT}} + E_c^{\text{DFT}}
\end{equation}

де $a$ --- частка HF обміну (зазвичай 0.2--0.3).

\subsubsection{B3LYP (Becke 3-parameter Lee-Yang-Parr)}

Найпопулярніший гібридний функціонал у хімії:

\begin{equation}
E_{xc}^{\text{B3LYP}} = E_x^{\text{LDA}} + 0.2(E_x^{\text{HF}} - E_x^{\text{LDA}}) + 0.72 \cdot E_x^{\text{B88}} + 0.81 \cdot E_c^{\text{LYP}} + 0.19 \cdot E_c^{\text{VWN}}
\end{equation}

\inputcode{code_9.py}

\subsubsection{PBE0 (PBE hybrid)}

Неемпіричний гібрид з 25% HF обміну:

\begin{equation}
E_{xc}^{\text{PBE0}} = 0.25 \cdot E_x^{\text{HF}} + 0.75 \cdot E_x^{\text{PBE}} + E_c^{\text{PBE}}
\end{equation}

\inputcode{code_10.py}


\subsubsection{CAM-B3LYP (Coulomb-Attenuated Method)}

Функціонал з дальньодіючою корекцією (range-separated):

\begin{equation}
\frac{1}{r_{12}} = \frac{\alpha + \beta \cdot \text{erf}(\mu r_{12})}{r_{12}} + \frac{1-[\alpha+\beta \cdot \text{erf}(\mu r_{12})]}{r_{12}}
\end{equation}

\inputcode{code_11.py}

%% --------------------------------------------------------
\subsubsection{M06-2X та інші Minnesota функціонали}
%% --------------------------------------------------------

Родина M06 з різною часткою HF обміну:

\begin{itemize}
    \item M06-L: 0\% HF (meta-GGA)
    \item M06: 27\% HF
    \item M06-2X: 54\% HF (для кінетики, слабких взаємодій)
    \item M06-HF: 100\% HF
\end{itemize}

\inputcode{code_12.py}

%% --------------------------------------------------------
\subsection{Порівняння різних рівнів теорії}
%% --------------------------------------------------------

\inputcode{code_13.py}

\subsection{Вибір функціоналу: рекомендації}

\begin{table}[h]
\centering
\caption{Рекомендовані функціонали для різних задач}
\label{tab:functional_recommendations}
\small
\begin{tabular}{lll}
\hline
\textbf{Задача} & \textbf{Функціонал} & \textbf{Коментар} \\
\hline
Швидкі розрахунки & PBE & Добра точність/швидкість \\
Енергії атомізації & B3LYP, PBE0 & Стандарт у хімії \\
Перехідні метали & TPSSh, M06 & Краще для d-електронів \\
Слабкі взаємодії & M06-2X, ωB97X-D & З дисперсією \\
Високоспінові стани & SCAN, TPSSh & Кращий баланс \\
Загальні розрахунки & PBE0 & Універсальний вибір \\
Максимальна точність & CCSD(T) & Але дуже повільно \\
\hline
\end{tabular}
\end{table}

%% --------------------------------------------------------
\subsection{Практичний приклад: вплив функціоналу на властивості}
%% --------------------------------------------------------

\inputcode{code_14.py}


%% --------------------------------------------------------
\section{DFT розрахунки атомів}
%% --------------------------------------------------------


%% --------------------------------------------------------
\subsection{Базова структура DFT розрахунку}
%% --------------------------------------------------------

DFT розрахунки в PySCF використовують класи \texttt{RKS} (Restricted Kohn-Sham) та \texttt{UKS} (Unrestricted Kohn-Sham), аналогічні до RHF/UHF:

\inputcode{code_15.py}


%% --------------------------------------------------------
\subsection{Систематичний розрахунок атомів другого періоду}
%% --------------------------------------------------------

\inputcode{code_16.py}


%% --------------------------------------------------------
\subsection{Розрахунок атомів перехідних металів}
%% --------------------------------------------------------

Атоми перехідних металів --- складна задача через багато близьких за енергією станів:

\inputcode{code_17.py}

%% --------------------------------------------------------
\subsection{Порівняння спінових станів}
%% --------------------------------------------------------

Для деяких атомів важливо порівняти різні спінові стани:

\inputcode{code_18.py}


%% --------------------------------------------------------
\subsection{Аналіз d-орбіталей перехідних металів}
%% --------------------------------------------------------

\inputcode{code_19.py}


%% --------------------------------------------------------
\subsection{Розрахунок важких атомів}
%% --------------------------------------------------------

Для важких атомів (4d, 5d, 4f, 5f) важливі релятивістські ефекти:

\inputcode{code_20.py}


%% --------------------------------------------------------
\subsection{Числові сітки в DFT}
%% --------------------------------------------------------

Точність DFT розрахунків залежить від якості числової сітки для інтегрування:

\inputcode{code_21.py}

%% --------------------------------------------------------
\subsection{Паралелізація DFT розрахунків}
%% --------------------------------------------------------


\inputcode{code_22.py}

%% --------------------------------------------------------
\section{Порівняння HF та DFT результатів}
%% --------------------------------------------------------


%% --------------------------------------------------------
\subsection{Систематичне порівняння енергій}
%% --------------------------------------------------------

\inputcode{code_23.py}


%% --------------------------------------------------------
\subsection{Порівняння енергій іонізації}
%% --------------------------------------------------------

\inputcode{code_24.py}

%% --------------------------------------------------------
\subsection{Електронна спорідненість}
%% --------------------------------------------------------

\inputcode{code_25.py}


%% --------------------------------------------------------
\subsection{Порівняння орбітальних енергій}
%% --------------------------------------------------------

\inputcode{code_26.py}

%% --------------------------------------------------------
\subsection{Забруднення спіном: HF vs DFT}
%% --------------------------------------------------------

\inputcode{code_27.py}

%% --------------------------------------------------------
\subsection{Час обчислень: HF vs DFT}
%% --------------------------------------------------------

\inputcode{code_28.py}

%% --------------------------------------------------------
\subsection{Загальні висновки HF vs DFT}
%% --------------------------------------------------------

\begin{table}[h!]
\centering
\caption{Підсумкове порівняння HF та DFT методів}
\label{tab:hf_dft_summary}
\small
\begin{tblr}{
colspec={X[l, m]X[l, m]X[l, m]},
row{1} = {c,m,font=\bfseries}
}
\hline
{Критерій} & {Hartree-Fock} & {DFT} \\
\hline
Точність енергій & Добра якісно & Краща кількісно \\
Енергії іонізації & Систематично завищені & Ближче до експерименту \\
Електронна спорідненість & Погана для аніонів & Значно краща \\
Орбітальні енергії & HOMO $\approx$ -IE (теорема Купманса) & Відхилення від теореми \\
Забруднення спіном & Присутнє (UHF) & Відсутнє (чисті DFT) \\
Швидкість & Середня & Швидше (чисті DFT) \\
& & Повільніше (гібриди) \\
Перехідні метали & Складно конвергує & Зазвичай краще \\
Дисперсійні взаємодії & Відсутні & Потрібні корекції \\
Систематичність & Добра & Залежить від функціоналу \\
\hline
\end{tblr}
\end{table}

\textbf{Рекомендації:}
\begin{itemize}
    \item \textbf{Використовуйте HF} для швидких якісних оцінок та як початок для post-HF методів
    \item \textbf{Використовуйте чисті DFT} (PBE) для великих систем, перехідних металів
    \item \textbf{Використовуйте гібридні DFT} (B3LYP, PBE0) для найкращого балансу точності та швидкості
    \item \textbf{Для аніонів} обов'язково використовуйте дифузні функції (aug-базиси)
    \item \textbf{Для важких атомів} враховуйте релятивістські ефекти
\end{itemize}


%% --------------------------------------------------------
\section{Вибір функціоналу для атомних систем}
%% --------------------------------------------------------

%% --------------------------------------------------------
\subsection{Тестовий набір даних}
%% --------------------------------------------------------

Для оцінки якості функціоналів використаємо стандартні атомні властивості:

\inputcode{code_29.py}

%% --------------------------------------------------------
\subsection{Рекомендації для різних елементів}
%% --------------------------------------------------------

\begin{table}[h]
\centering
\caption{Рекомендовані функціонали для різних груп елементів}
\label{tab:functional_by_element}
\small
\begin{tabular}{lll}
\hline
\textbf{Група елементів} & \textbf{Рекомендований} & \textbf{Альтернатива} \\
\hline
H, He & HF, PBE0 & Будь-який \\
Li--Ne (2 період) & PBE0, B3LYP & PBE, ωB97X-D \\
Na--Ar (3 період) & PBE0, B3LYP & TPSSh \\
3d метали (Sc--Zn) & TPSSh, M06 & PBE, B3LYP \\
4d метали (Y--Cd) & TPSSh, PBE & M06 \\
5d метали (La--Hg) & PBE, TPSSh & + релятивістські \\
Лантаноїди & PBE, SCAN & + SOC \\
Актиноїди & PBE & + SOC + DFT+U \\
\hline
\end{tabular}
\end{table}

%% --------------------------------------------------------
\section{Практичні завдання}
%% --------------------------------------------------------


%% --------------------------------------------------------
\subsection{Завдання 1: Систематичне дослідження}
%% --------------------------------------------------------

\inputcode{code_30.py}

%% --------------------------------------------------------
\subsection{Завдання 2: Функціональна залежність}
%% --------------------------------------------------------

\inputcode{code_31.py}


%% --------------------------------------------------------
\subsection{Завдання 3: Конвергенція до базисної межі}
%% --------------------------------------------------------

\inputcode{code_32.py}

%% --------------------------------------------------------
\subsection{Завдання 4: Перехідні метали}
%% --------------------------------------------------------

\inputcode{code_33.py}

%% --------------------------------------------------------
\section{Резюме}
%% --------------------------------------------------------

У цьому розділі ми детально вивчили теорію функціоналу густини та її застосування до атомних систем:

\begin{itemize}
    \item \textbf{Теоретичні основи} --- теореми Хоенберга--Кона, рівняння Кона--Шема
    \item \textbf{Функціонали} --- від простих LDA до складних гібридних і meta-GGA
    \item \textbf{Практичні розрахунки} --- атоми різних періодів, перехідні метали
    \item \textbf{Порівняння з HF} --- енергії, орбіталі, спін, швидкість
    \item \textbf{Вибір методу} --- рекомендації для різних задач
\end{itemize}

%% --------------------------------------------------------
\subsection{Ключові висновки}
%% --------------------------------------------------------

\begin{enumerate}
    \item DFT зазвичай дає кращі результати для атомних енергій порівняно з HF
    \item Гібридні функціонали (B3LYP, PBE0) --- золота середина між точністю та швидкістю
    \item Для перехідних металів краще використовувати meta-GGA (TPSS) або спеціалізовані функціонали (M06)
    \item Вибір базису критичний: для аніонів потрібні дифузні функції
    \item Чисті DFT не мають забруднення спіном на відміну від UHF
    \item Якість числової сітки важлива для точних розрахунків
    \item Для важких атомів необхідні релятивістські корекції
\end{enumerate}

%% --------------------------------------------------------
\subsection{Типові помилки}
%% --------------------------------------------------------

\begin{enumerate}
    \item \textbf{Неправильний спін} --- завжди перевіряйте основний стан атома
    \item \textbf{Недостатній базис} --- для точних енергій використовуйте triple-zeta або більше
    \item \textbf{Забування дифузних функцій} --- критично для аніонів та збуджених станів
    \item \textbf{Ігнорування симетрії} --- може уповільнити розрахунок
    \item \textbf{Погана конвергенція} --- використовуйте level shift, змініть початкове наближення
    \item \textbf{Неправильна сітка} --- для точних результатів використовуйте grids.level $\geqslant3$.
\end{enumerate}

%% --------------------------------------------------------
\subsection{Корисні посилання}
%% --------------------------------------------------------

\begin{itemize}
    \item \textbf{Libxc} --- бібліотека DFT функціоналів: \url{https://www.tddft.org/programs/libxc/}
    \item \textbf{NIST} --- експериментальні дані атомів: \url{https://physics.nist.gov/PhysRefData/}
    \item \textbf{Basis Set Exchange} --- база даних базисних наборів: \url{https://www.basissetexchange.org/}
\end{itemize}

У наступному розділі ми розглянемо Post-Hartree-Fock методи (MP2, CCSD, CASSCF), які дозволяють досягти ще вищої точності для атомних систем, враховуючи електронну кореляцію явно.