%% --------------------------------------------------------
\section{DFT для диатомних молекул}
%% --------------------------------------------------------

Після вивчення атомних систем переходимо до молекул. Диатомні молекули --- природний наступний крок, що дозволяє досліджувати хімічні зв'язки, потенційні криві та спектроскопічні властивості.

%% --------------------------------------------------------
\subsection{Базові DFT розрахунки молекул}
%% --------------------------------------------------------

Основна структура DFT розрахунку молекули аналогічна до атомної:

\inputcode{code_34.py}

%% --------------------------------------------------------
\subsection{Оптимізація геометрії}
%% --------------------------------------------------------

Для знаходження рівноважної структури молекули використовується оптимізація геометрії:

\inputcode{code_35.py}

%% --------------------------------------------------------
\subsection{Криві потенційної енергії}
%% --------------------------------------------------------

Побудова кривої потенційної енергії (PES) --- фундаментальна характеристика хімічного зв'язку:

\inputcode{code_36.py}

%% --------------------------------------------------------
\subsection{Порівняння функціоналів для зв'язків}
%% --------------------------------------------------------

Різні функціонали по-різному описують хімічні зв'язки:

\inputcode{code_37.py}

%% --------------------------------------------------------
\subsection{Спектроскопічні константи}
%% --------------------------------------------------------

З кривої потенційної енергії можна отримати спектроскопічні константи:

\begin{itemize}
    \item $r_e$ --- рівноважна довжина зв'язку
    \item $D_e$ --- енергія дисоціації
    \item $\omega_e$ --- частота коливань
    \item $\omega_e x_e$ --- ангармонічність
\end{itemize}

\inputcode{code_38.py}

%% --------------------------------------------------------
\subsection{Гетероядерні молекули}
%% --------------------------------------------------------

Гетероядерні молекули мають полярний зв'язок та дипольний момент:

\inputcode{code_39.py}

%% --------------------------------------------------------
\subsection{Молекули з кратними зв'язками}
%% --------------------------------------------------------

Дослідження молекул з подвійними та потрійними зв'язками:

\inputcode{code_40.py}

%% --------------------------------------------------------
\subsection{Порівняння HF та DFT для молекул}
%% --------------------------------------------------------

\inputcode{code_41.py}

%% --------------------------------------------------------
\subsection{Молекули перехідних металів}
%% --------------------------------------------------------

Диатомні молекули перехідних металів --- складні багатоконфігураційні системи:

\inputcode{code_42.py}

%% --------------------------------------------------------
\subsection{Відкритошкаралупні молекули}
%% --------------------------------------------------------

Молекули з непарними електронами потребують UKS підходу:

\inputcode{code_43.py}

%% --------------------------------------------------------
\subsection{Дисперсійні корекції для слабких зв'язків}
%% --------------------------------------------------------

Стандартні DFT функціонали погано описують дисперсійні взаємодії. Потрібні корекції:

\inputcode{code_44.py}

%% --------------------------------------------------------
\subsection{Аналіз хвильових функцій}
%% --------------------------------------------------------

Детальний аналіз електронної структури молекул:

\inputcode{code_45.py}

%% --------------------------------------------------------
\subsection{Вплив базисного набору}
%% --------------------------------------------------------

Систематичне дослідження конвергенції до базисної межі:

\inputcode{code_46.py}

%% --------------------------------------------------------
\subsection{Дисоціація зв'язків}
%% --------------------------------------------------------

\subsubsection{Проблема симетричного розриву зв'язку}

Restricted методи (RKS) неправильно описують дисоціацію гомолітичного зв'язку:

\inputcode{code_47.py}

%% --------------------------------------------------------
\subsection{Збуджені стани: TD-DFT}
%% --------------------------------------------------------

Time-Dependent DFT для вертикальних збуджень:

\inputcode{code_48.py}

%% --------------------------------------------------------
\subsection{Порівняльна таблиця результатів}
%% --------------------------------------------------------

\begin{table}[h]
\centering
\caption{Порівняння методів для типових диатомних молекул}
\label{tab:diatomic_comparison}
\small
\begin{tblr}{
colspec={X[l]X[c]X[c]X[c]X[c]X[c]},
row{1} = {c,m,font=\bfseries}
}
\hline
Молекула & Експеримент & HF & PBE & B3LYP & PBE0 \\
\hline
H$_2$ $r_e$ (\AA) & 0.741 & 0.735 & 0.748 & 0.744 & 0.742 \\
H$_2$ $D_e$ (eV) & 4.75 & 3.64 & 4.52 & 4.71 & 4.68 \\
N$_2$ $r_e$ (\AA) & 1.098 & 1.065 & 1.104 & 1.098 & 1.091 \\
N$_2$ $D_e$ (eV) & 9.91 & 4.89 & 9.23 & 9.85 & 9.67 \\
O$_2$ $r_e$ (\AA) & 1.208 & 1.162 & 1.229 & 1.216 & 1.208 \\
CO $r_e$ (\AA) & 1.128 & 1.097 & 1.137 & 1.130 & 1.124 \\
\hline
\end{tblr}
\end{table}

%% --------------------------------------------------------
\subsection{Практичні рекомендації}
%% --------------------------------------------------------

\begin{enumerate}
    \item \textbf{Для простих молекул (H$_2$, N$_2$, O$_2$)}:
    \begin{itemize}
        \item Базовий розрахунок: PBE/cc-pVTZ
        \item Висока точність: PBE0/aug-cc-pVQZ
        \item Найкраща точність: CCSD(T)/aug-cc-pV5Z
    \end{itemize}

    \item \textbf{Для полярних молекул (CO, NO, HF)}:
    \begin{itemize}
        \item Гібридні функціонали обов'язкові
        \item Дифузні функції для дипольних моментів
    \end{itemize}

    \item \textbf{Для молекул перехідних металів}:
    \begin{itemize}
        \item TPSSh або M06 функціонали
        \item Велика кількість спінових станів
        \item Можливо, потрібен DFT+U
    \end{itemize}

    \item \textbf{Для слабких зв'язків}:
    \begin{itemize}
        \item Обов'язкові дисперсійні корекції (D3, D4)
        \item Або ωB97X-D, M06-2X функціонали
    \end{itemize}

    \item \textbf{Для збуджених станів}:
    \begin{itemize}
        \item TD-DFT з range-separated функціоналами
        \item Або багатоконфігураційні методи (CASSCF)
    \end{itemize}
\end{enumerate}

%% --------------------------------------------------------
\subsection{Типові помилки при розрахунках молекул}
%% --------------------------------------------------------

\begin{enumerate}
    \item \textbf{Неправильна симетрія}:
    \begin{itemize}
        \item Завжди використовуйте симетрію молекули
        \item Перевіряйте point group командою \texttt{mol.symmetry = True}
    \end{itemize}

    \item \textbf{Хибна мультиплетність}:
    \begin{itemize}
        \item O$_2$ --- триплет, не синглет!
        \item Перевіряйте спін по \texttt{mol.spin}
    \end{itemize}

    \item \textbf{Недостатня сітка інтегрування}:
    \begin{itemize}
        \item Для точних градієнтів: \texttt{grids.level = 4}
        \item Для оптимізації завжди перевіряйте конвергенцію по сітці
    \end{itemize}

    \item \textbf{Погана початкова геометрія}:
    \begin{itemize}
        \item Починайте з розумних відстаней
        \item Використовуйте експериментальні дані або попередні розрахунки
    \end{itemize}

    \item \textbf{Ігнорування BSSE} (Basis Set Superposition Error):
    \begin{itemize}
        \item Для енергій зв'язування використовуйте counterpoise корекцію
        \item Особливо важливо для малих базисів
    \end{itemize}
\end{enumerate}

%% --------------------------------------------------------
\subsection{Задачі для самостійної роботи}
%% --------------------------------------------------------

\begin{enumerate}
    \item \textbf{Задача 1}: Побудуйте криві потенційної енергії молекули \ce{F2} різними функціоналами (LDA, PBE, B3LYP, PBE0) та порівняйте з експериментом ($r_e = 1.412$ \AA, $D_e = 1.66$ eV).

    \item \textbf{Задача 2}: Дослідіть вплив базисного набору на властивості \ce{CO}: розрахуйте $r_e$, $D_e$, $\omega_e$ з базисами cc-pVDZ, cc-pVTZ, cc-pVQZ та екстраполюйте до межі.

    \item \textbf{Задача 3}: Порівняйте RKS та UKS підходи для дисоціації \ce{H2}. Поясніть різницю в поведінці енергії на великих відстанях.

    \item \textbf{Задача 4}: Розрахуйте перші три збуджені стани \ce{N2} методом TD-DFT з різними функціоналами. Порівняйте з експериментальними значеннями.

    \item \textbf{Задача 5}: Дослідіть молекулу CuH методами PBE, TPSSh та B3LYP. Який функціонал краще описує зв'язок Cu--H?

    \item \textbf{Задача 6}: Виконайте систематичне дослідження галогенів (\ce{F2}, \ce{Cl2}, \ce{Br2}, \ce{I2}) з різними функціоналами. Який функціонал найкраще відтворює експериментальні енергії дисоціації?

    \item \textbf{Задача 7}: Побудуйте benchmark для молекул \ce{H2}, \ce{N2}, \ce{CO} з базисами від cc-pVDZ до aug-cc-pV5Z. Визначте, при якому базисі досягається конвергенція з точністю 1 mHa.

    \item \textbf{Задача 8}: Розрахуйте тестовий набір G2 (8 молекул) функціоналами PBE, B3LYP, PBE0, M06-2X. Порівняйте MAE, RMSE та визначте найкращий функціонал.

\end{enumerate}

%% --------------------------------------------------------
\subsection{Типові помилки при розрахунках молекул}
%% --------------------------------------------------------

\begin{enumerate}
    \item \textbf{Неправильна симетрія}:
    \begin{itemize}
        \item Завжди використовуйте симетрію молекули
        \item Перевіряйте point group командою \texttt{mol.symmetry = True}
    \end{itemize}

    \item \textbf{Хибна мультиплетність}:
    \begin{itemize}
        \item O$_2$ --- триплет, не синглет!
        \item Перевіряйте спін по \texttt{mol.spin}
    \end{itemize}

    \item \textbf{Недостатня сітка інтегрування}:
    \begin{itemize}
        \item Для точних градієнтів: \texttt{grids.level = 4}
        \item Для оптимізації завжди перевіряйте конвергенцію по сітці
    \end{itemize}

    \item \textbf{Погана початкова геометрія}:
    \begin{itemize}
        \item Починайте з розумних відстаней
        \item Використовуйте експериментальні дані або попередні розрахунки
    \end{itemize}

    \item \textbf{Ігнорування BSSE} (Basis Set Superposition Error):
    \begin{itemize}
        \item Для енергій зв'язування використовуйте counterpoise корекцію
        \item Особливо важливо для малих базисів
    \end{itemize}
\end{enumerate}

%% --------------------------------------------------------
\subsection{Систематичне дослідження галогенів}
%% --------------------------------------------------------

Галогени (F$_2$, Cl$_2$, Br$_2$, I$_2$) --- цікава серія для тестування функціоналів:

\inputcode{code_49.py}

\textbf{Особливості галогенів:}
\begin{itemize}
    \item F$_2$ має найслабкіший зв'язок через відштовхування lone pairs
    \item Для Br та I необхідні релятивістські псевдопотенціали
    \item Дисперсійні корекції важливі для важких галогенів
    \item LDA/GGA систематично переоцінюють енергії зв'язку
\end{itemize}

%% --------------------------------------------------------
\subsection{Benchmark: час обчислень}
%% --------------------------------------------------------

Практичне порівняння швидкості різних методів:

\inputcode{code_50.py}

\textbf{Масштабування обчислювальної складності:}
\begin{itemize}
    \item \textbf{LDA/GGA}: $\mathcal{O}(N^3)$ --- найшвидші
    \item \textbf{Гібриди}: $\mathcal{O}(N^4)$ --- через точний обмін HF
    \item \textbf{Meta-GGA}: $\mathcal{O}(N^3)$ --- але з більшою константою
    \item \textbf{Double-hybrid}: $\mathcal{O}(N^5)$ --- через MP2 кореляцію
\end{itemize}

%% --------------------------------------------------------
\subsection{Тестовий набір G2}
%% --------------------------------------------------------

Систематична оцінка якості функціоналів на стандартному наборі:

\inputcode{code_51.py}

\textbf{Метрики якості:}
\begin{itemize}
    \item \textbf{MAE} (Mean Absolute Error) --- середня абсолютна помилка
    \item \textbf{RMSE} (Root Mean Square Error) --- середньоквадратична помилка
    \item \textbf{Max error} --- максимальна помилка
\end{itemize}

Типові результати для енергій дисоціації:
\begin{itemize}
    \item LDA: MAE $\sim$ 0.5--1.0 eV (переоцінює зв'язування)
    \item PBE: MAE $\sim$ 0.3--0.5 eV
    \item B3LYP/PBE0: MAE $\sim$ 0.1--0.3 eV (найкраще)
    \item M06-2X: MAE $\sim$ 0.2--0.4 eV
\end{itemize}

%% --------------------------------------------------------
\subsection{Використання симетрії}
%% --------------------------------------------------------

Симетрія молекули може суттєво прискорити розрахунки:

\inputcode{code_52.py}

\textbf{Переваги використання симетрії:}
\begin{enumerate}
    \item \textbf{Прискорення}: 2--10x залежно від точкової групи
    \item \textbf{Менше пам'яті}: блокова структура матриць
    \item \textbf{Класифікація орбіталей}: по незвідним представленням
    \item \textbf{Відбір правил}: для спектроскопії
\end{enumerate}

\textbf{Точкові групи диатомних молекул:}
\begin{itemize}
    \item \textbf{D$_{\infty h}$}: гомоядерні (H$_2$, N$_2$, O$_2$)
    \item \textbf{C$_{\infty v}$}: гетероядерні (CO, HF, NO)
\end{itemize}


%% --------------------------------------------------------
\subsection{Резюме секції}
%% --------------------------------------------------------

У цій секції ми навчилися:
\begin{itemize}
    \item Проводити DFT розрахунки диатомних молекул
    \item Будувати криві потенційної енергії
    \item Оптимізувати геометрію та обчислювати спектроскопічні константи
    \item Порівнювати різні функціонали для хімічних зв'язків
    \item Враховувати дисперсійні взаємодії
    \item Розраховувати збуджені стани методом TD-DFT
    \item Уникати типових помилок при молекулярних розрахунках
\end{itemize}

\textbf{Ключовий висновок}: DFT є ефективним інструментом для опису хімічних зв'язків у диатомних молекулах, але вибір функціоналу критично важливий і залежить від типу зв'язку та властивостей, що досліджуються.