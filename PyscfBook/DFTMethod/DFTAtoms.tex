% !TeX encoding = utf8
% !TeX spellcheck = uk_UA
% !TeX root =../PyscfBook.tex

%% --------------------------------------------------------
\section{DFT розрахунки атомів}
%% --------------------------------------------------------


%% --------------------------------------------------------
\subsection{Базова структура DFT розрахунку}
%% --------------------------------------------------------

DFT розрахунки в PySCF використовують класи \texttt{RKS} (Restricted Kohn-Sham) та \texttt{UKS} (Unrestricted Kohn-Sham), аналогічні до RHF/UHF:

\inputcode{code_15.py}


%% --------------------------------------------------------
\subsection{Систематичний розрахунок атомів другого періоду}
%% --------------------------------------------------------

\inputcode{code_16.py}


%% --------------------------------------------------------
\subsection{Розрахунок атомів перехідних металів}
%% --------------------------------------------------------

Атоми перехідних металів --- складна задача через багато близьких за енергією станів:

\inputcode{code_17.py}

%% --------------------------------------------------------
\subsection{Порівняння спінових станів}
%% --------------------------------------------------------

Для деяких атомів важливо порівняти різні спінові стани:

\inputcode{code_18.py}


%% --------------------------------------------------------
\subsection{Аналіз d-орбіталей перехідних металів}
%% --------------------------------------------------------

\inputcode{code_19.py}


%% --------------------------------------------------------
\subsection{Розрахунок важких атомів}
%% --------------------------------------------------------

Для важких атомів (4d, 5d, 4f, 5f) важливі релятивістські ефекти:

\inputcode{code_20.py}


%% --------------------------------------------------------
\subsection{Числові сітки в DFT}
%% --------------------------------------------------------

Точність DFT розрахунків залежить від якості числової сітки для інтегрування:

\inputcode{code_21.py}

%% --------------------------------------------------------
\subsection{Паралелізація DFT розрахунків}
%% --------------------------------------------------------


\inputcode{code_22.py}

%% --------------------------------------------------------
\section{Порівняння HF та DFT результатів}
%% --------------------------------------------------------


%% --------------------------------------------------------
\subsection{Систематичне порівняння енергій}
%% --------------------------------------------------------

\inputcode{code_23.py}


%% --------------------------------------------------------
\subsection{Порівняння енергій іонізації}
%% --------------------------------------------------------

\inputcode{code_24.py}

%% --------------------------------------------------------
\subsection{Електронна спорідненість}
%% --------------------------------------------------------

\inputcode{code_25.py}


%% --------------------------------------------------------
\subsection{Порівняння орбітальних енергій}
%% --------------------------------------------------------

\inputcode{code_26.py}

%% --------------------------------------------------------
\subsection{Забруднення спіном: HF vs DFT}
%% --------------------------------------------------------

\inputcode{code_27.py}

%% --------------------------------------------------------
\subsection{Час обчислень: HF vs DFT}
%% --------------------------------------------------------

\inputcode{code_28.py}

%% --------------------------------------------------------
\subsection{Загальні висновки HF vs DFT}
%% --------------------------------------------------------

\begin{table}[h!]
\centering
\caption{Підсумкове порівняння HF та DFT методів}
\label{tab:hf_dft_summary}
\small
\begin{tblr}{
colspec={X[l, m]X[l, m]X[l, m]},
row{1} = {c,m,font=\bfseries}
}
\hline
{Критерій} & {Hartree-Fock} & {DFT} \\
\hline
Точність енергій & Добра якісно & Краща кількісно \\
Енергії іонізації & Систематично завищені & Ближче до експерименту \\
Електронна спорідненість & Погана для аніонів & Значно краща \\
Орбітальні енергії & HOMO $\approx$ -IE (теорема Купманса) & Відхилення від теореми \\
Забруднення спіном & Присутнє (UHF) & Відсутнє (чисті DFT) \\
Швидкість & Середня & Швидше (чисті DFT) \\
& & Повільніше (гібриди) \\
Перехідні метали & Складно конвергує & Зазвичай краще \\
Дисперсійні взаємодії & Відсутні & Потрібні корекції \\
Систематичність & Добра & Залежить від функціоналу \\
\hline
\end{tblr}
\end{table}

\textbf{Рекомендації:}
\begin{itemize}
    \item \textbf{Використовуйте HF} для швидких якісних оцінок та як початок для post-HF методів
    \item \textbf{Використовуйте чисті DFT} (PBE) для великих систем, перехідних металів
    \item \textbf{Використовуйте гібридні DFT} (B3LYP, PBE0) для найкращого балансу точності та швидкості
    \item \textbf{Для аніонів} обов'язково використовуйте дифузні функції (aug-базиси)
    \item \textbf{Для важких атомів} враховуйте релятивістські ефекти
\end{itemize}


%% --------------------------------------------------------
\section{Вибір функціоналу для атомних систем}
%% --------------------------------------------------------

%% --------------------------------------------------------
\subsection{Тестовий набір даних}
%% --------------------------------------------------------

Для оцінки якості функціоналів використаємо стандартні атомні властивості:

\inputcode{code_29.py}

%% --------------------------------------------------------
\subsection{Рекомендації для різних елементів}
%% --------------------------------------------------------

\begin{table}[h]
\centering
\caption{Рекомендовані функціонали для різних груп елементів}
\label{tab:functional_by_element}
\small
\begin{tabular}{lll}
\hline
\textbf{Група елементів} & \textbf{Рекомендований} & \textbf{Альтернатива} \\
\hline
H, He & HF, PBE0 & Будь-який \\
Li--Ne (2 період) & PBE0, B3LYP & PBE, ωB97X-D \\
Na--Ar (3 період) & PBE0, B3LYP & TPSSh \\
3d метали (Sc--Zn) & TPSSh, M06 & PBE, B3LYP \\
4d метали (Y--Cd) & TPSSh, PBE & M06 \\
5d метали (La--Hg) & PBE, TPSSh & + релятивістські \\
Лантаноїди & PBE, SCAN & + SOC \\
Актиноїди & PBE & + SOC + DFT+U \\
\hline
\end{tabular}
\end{table}

%% --------------------------------------------------------
\section{Практичні завдання}
%% --------------------------------------------------------


%% --------------------------------------------------------
\subsection{Завдання 1: Систематичне дослідження}
%% --------------------------------------------------------

\inputcode{code_30.py}

%% --------------------------------------------------------
\subsection{Завдання 2: Функціональна залежність}
%% --------------------------------------------------------

\inputcode{code_31.py}


%% --------------------------------------------------------
\subsection{Завдання 3: Конвергенція до базисної межі}
%% --------------------------------------------------------

\inputcode{code_32.py}

%% --------------------------------------------------------
\subsection{Завдання 4: Перехідні метали}
%% --------------------------------------------------------

\inputcode{code_33.py}

%% --------------------------------------------------------
\section{Резюме}
%% --------------------------------------------------------

У цьому розділі ми детально вивчили теорію функціоналу густини та її застосування до атомних систем:

\begin{itemize}
    \item \textbf{Теоретичні основи} --- теореми Хоенберга--Кона, рівняння Кона--Шема
    \item \textbf{Функціонали} --- від простих LDA до складних гібридних і meta-GGA
    \item \textbf{Практичні розрахунки} --- атоми різних періодів, перехідні метали
    \item \textbf{Порівняння з HF} --- енергії, орбіталі, спін, швидкість
    \item \textbf{Вибір методу} --- рекомендації для різних задач
\end{itemize}

%% --------------------------------------------------------
\subsection{Ключові висновки}
%% --------------------------------------------------------

\begin{enumerate}
    \item DFT зазвичай дає кращі результати для атомних енергій порівняно з HF
    \item Гібридні функціонали (B3LYP, PBE0) --- золота середина між точністю та швидкістю
    \item Для перехідних металів краще використовувати meta-GGA (TPSS) або спеціалізовані функціонали (M06)
    \item Вибір базису критичний: для аніонів потрібні дифузні функції
    \item Чисті DFT не мають забруднення спіном на відміну від UHF
    \item Якість числової сітки важлива для точних розрахунків
    \item Для важких атомів необхідні релятивістські корекції
\end{enumerate}

%% --------------------------------------------------------
\subsection{Типові помилки}
%% --------------------------------------------------------

\begin{enumerate}
    \item \textbf{Неправильний спін} --- завжди перевіряйте основний стан атома
    \item \textbf{Недостатній базис} --- для точних енергій використовуйте triple-zeta або більше
    \item \textbf{Забування дифузних функцій} --- критично для аніонів та збуджених станів
    \item \textbf{Ігнорування симетрії} --- може уповільнити розрахунок
    \item \textbf{Погана конвергенція} --- використовуйте level shift, змініть початкове наближення
    \item \textbf{Неправильна сітка} --- для точних результатів використовуйте grids.level $\geqslant3$.
\end{enumerate}

%% --------------------------------------------------------
\subsection{Корисні посилання}
%% --------------------------------------------------------

\begin{itemize}
    \item \textbf{Libxc} --- бібліотека DFT функціоналів: \url{https://www.tddft.org/programs/libxc/}
    \item \textbf{NIST} --- експериментальні дані атомів: \url{https://physics.nist.gov/PhysRefData/}
    \item \textbf{Basis Set Exchange} --- база даних базисних наборів: \url{https://www.basissetexchange.org/}
\end{itemize}
