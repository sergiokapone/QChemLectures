%% ========================================================
\section{Електричні властивості молекул}
%% ========================================================

%% --------------------------------------------------------
\subsection{Дипольний момент}
%% --------------------------------------------------------

Дипольний момент --- найпростіша електрична властивість молекули,
яка характеризує розподіл електронної густини:
\[
\boldsymbol{\mu} = \sum_i q_i \mathbf{r}_i =
\langle\Psi|\hat{\boldsymbol{\mu}}|\Psi\rangle
\]

Для молекулярної системи:
\[
\boldsymbol{\mu} = \sum_A Z_A \mathbf{R}_A -
\int \rho(\mathbf{r}) \mathbf{r} \, d\mathbf{r}
\]

\textbf{Властивості дипольного моменту:}
\begin{itemize}
    \item Векторна величина (має напрямок)
    \item Вимірюється в Дебаях: 1 D $\approx$ 0.3934 ea₀
    \item Залежить від вибору початку координат для зарядженої системи
    \item Для нейтральної молекули не залежить від початку координат
\end{itemize}

\subsubsection{Розрахунок дипольного моменту}

\inputcode{h2o_dipole.py}

\textbf{Результати для H₂O:}
\begin{itemize}
    \item RHF/6-31G: $\mu \approx 2.30$ D
    \item RHF/aug-cc-pVDZ: $\mu \approx 2.18$ D
    \item B3LYP/aug-cc-pVDZ: $\mu \approx 1.97$ D
    \item Експеримент: $\mu = 1.855$ D
\end{itemize}

\textbf{Аналіз похибок:}
\begin{itemize}
    \item RHF завищує дипольний момент через недооцінку кореляції
    \item Малі базиси завищують $\mu$ (недостатня гнучкість)
    \item Дифузні функції (aug-) критично важливі
    \item DFT зазвичай ближче до експерименту
\end{itemize}

%% --------------------------------------------------------
\subsection{Поляризовність}
%% --------------------------------------------------------

Поляризовність $\boldsymbol{\alpha}$ описує відгук електронної густини
на зовнішнє електричне поле:
\[
\mu_i = \mu_i^0 + \sum_j \alpha_{ij} E_j +
\frac{1}{2}\sum_{jk} \beta_{ijk} E_j E_k + \ldots
\]

Статична поляризовність --- тензор другого рангу:
\[
\alpha_{ij} = -\left.\frac{\partial^2 E}{\partial E_i \partial E_j}\right|_{E=0}
\]

\subsubsection{Обчислення поляризовності}

\inputcode{h2o_polarizability.py}

\textbf{Тензор поляризовності H₂O (au³):}
\begin{center}
\begin{tabular}{lccc}
\toprule
Компонента & RHF/aug-cc-pVDZ & B3LYP/aug-cc-pVDZ & Експеримент \\
\midrule
$\alpha_{xx}$ & 8.85 & 9.12 & 9.63 \\
$\alpha_{yy}$ & 9.32 & 9.58 & 9.93 \\
$\alpha_{zz}$ & 8.46 & 8.71 & 9.12 \\
\midrule
$\bar{\alpha}$ & 8.88 & 9.14 & 9.56 \\
\bottomrule
\end{tabular}
\end{center}

де середня поляризовність: $\bar{\alpha} = (\alpha_{xx} + \alpha_{yy} + \alpha_{zz})/3$

\textbf{Анізотропія поляризовності:}
\[
\Delta\alpha = \frac{1}{\sqrt{2}}\sqrt{(\alpha_{xx}-\alpha_{yy})^2 +
(\alpha_{yy}-\alpha_{zz})^2 + (\alpha_{zz}-\alpha_{xx})^2}
\]

Анізотропія важлива для Раман-розсіювання.

%% --------------------------------------------------------
\subsection{Динамічна поляризовність}
%% --------------------------------------------------------

При частотно-залежному полі $\mathbf{E}(t) = \mathbf{E}_0 e^{-i\omega t}$
поляризовність стає функцією частоти:
\[
\alpha_{ij}(\omega) = \alpha_{ij}(-\omega; \omega)
\]

\textbf{Фізичний зміст:}
\begin{itemize}
    \item При $\omega = 0$: статична поляризовність
    \item При $\omega \to \omega_{\text{res}}$: резонансне підсилення
    \item Уявна частина: поглинання енергії
    \item Пов'язана з показником заломлення: $n^2 - 1 \propto \alpha(\omega)$
\end{itemize}

\inputcode{h2o_frequency_dependent_polarizability.py}

%---------------------------------------------------------
\begin{figure}[h!]\centering
\includegraphics[width=0.8\linewidth]{\currfiledir/h2o_dynamic_polarizability.pdf}
\caption{Частотна залежність поляризовності H₂O.}
\label{pic:h2o_dynamic_pol}
\end{figure}
%---------------------------------------------------------

\textbf{Спостереження:}
\begin{itemize}
    \item Монотонне зростання $\alpha(\omega)$ до резонансу
    \item Різка зміна поблизу електронних збуджень
    \item Анізотропія залежить від частоти
    \item Важливо для нелінійної оптики
\end{itemize}

%% --------------------------------------------------------
\subsection{Гіперполяризовності}
%% --------------------------------------------------------

Для сильних полів або нелінійної оптики важливі гіперполяризовності:
\[
\mu_i = \mu_i^0 + \alpha_{ij} E_j + \frac{1}{2}\beta_{ijk} E_j E_k +
\frac{1}{6}\gamma_{ijkl} E_j E_k E_l + \ldots
\]

\textbf{Фізичні процеси:}
\begin{itemize}
    \item $\beta$ (перша гіперполяризовність): генерація другої гармоніки,
          електрооптичний ефект Поккельса
    \item $\gamma$ (друга гіперполяризовність): генерація третьої гармоніки,
          ефект Керра
\end{itemize}

\subsubsection{Розрахунок першої гіперполяризовності}

\inputcode{para_nitroaniline_beta.py}

\textbf{Приклад: пара-нітроанілін (pNA)}

pNA --- класична молекула для нелінійної оптики:
\begin{itemize}
    \item Донорно-акцепторна система (NH₂ --- донор, NO₂ --- акцептор)
    \item Великий переніс заряду вздовж молекули
    \item Велика $\beta$ вздовж довгої осі
\end{itemize}

\textbf{Типові значення для pNA:}
\begin{center}
\begin{tabular}{lcc}
\toprule
Метод & $\beta_{zzz}$ (au) & $\beta_{\text{vec}}$ (au) \\
\midrule
RHF/6-31G          & 520  & 480 \\
B3LYP/6-31+G*      & 890  & 820 \\
CAM-B3LYP/aug-cc-pVDZ & 750 & 690 \\
\midrule
Експеримент (газ)  & ---  & $\sim$800 \\
\bottomrule
\end{tabular}
\end{center}

Векторна гіперполяризовність:
\[
\beta_{\text{vec}} = \sqrt{\beta_x^2 + \beta_y^2 + \beta_z^2}
\]
де $\beta_i = \sum_{jk} \beta_{ijk} \hat{\mu}_j \hat{\mu}_k$ проектується
на напрямок дипольного моменту.

%% --------------------------------------------------------
\subsection{Електричне поле та градієнт}
%% --------------------------------------------------------

Електричне поле в точці $\mathbf{r}$ від молекулярних зарядів:
\[
\mathbf{E}(\mathbf{r}) = -\nabla V(\mathbf{r})
\]

Градієнт електричного поля (тензор):
\[
\nabla_i E_j = \frac{\partial E_j}{\partial r_i}
\]

\textbf{Застосування:}
\begin{itemize}
    \item Взаємодія з квадрупольним моментом ядра (ЯКР спектроскопія)
    \item Розрахунок констант спін-спінової взаємодії в ЯМР
    \item Моделювання сольватації (електростатичний внесок)
\end{itemize}

\inputcode{h2o_electric_field_gradient.py}

\textbf{Електричний градієнт на ядрі O в H₂O:}
\begin{itemize}
    \item Тензор EFG має слід нуль (рівняння Лапласа)
    \item Головні компоненти: $V_{xx}, V_{yy}, V_{zz}$
    \item Параметр асиметрії: $\eta = (V_{xx} - V_{yy})/V_{zz}$
    \item Використовується для інтерпретації ЯКР спектрів
\end{itemize}

%% --------------------------------------------------------
\subsection{Електростатичний потенціал (ESP)}
%% --------------------------------------------------------

ESP показує взаємодію молекули з точковим позитивним зарядом:
\[
V(\mathbf{r}) = \sum_A \frac{Z_A}{|\mathbf{r} - \mathbf{R}_A|} -
\int \frac{\rho(\mathbf{r}')}{|\mathbf{r} - \mathbf{r}'|} d\mathbf{r}'
\]

\textbf{Застосування ESP:}
\begin{itemize}
    \item Передбачення реакційної здатності (нуклеофільні/електрофільні ділянки)
    \item Побудова силових полів (RESP, CHELPG заряди)
    \item Аналіз водневих зв'язків
    \item Вивчення нековалентних взаємодій
\end{itemize}

\inputcode{h2o_esp_analysis.py}

%---------------------------------------------------------
\begin{figure}[h!]\centering
\includegraphics[width=0.7\linewidth]{\currfiledir/h2o_esp_map.pdf}
\caption{Карта електростатичного потенціалу H₂O на поверхні
van der Waals (ізоденситна поверхня 0.001 au).}
\label{pic:h2o_esp}
\end{figure}
%---------------------------------------------------------

\textbf{Інтерпретація карти ESP для H₂O:}
\begin{itemize}
    \item \textcolor{red}{Червона область} (негативний ESP): біля атома O,
          область неподілених пар --- нуклеофільна ділянка
    \item \textcolor{blue}{Синя область} (позитивний ESP): біля атомів H,
          здатні утворювати водневі зв'язки як донори
    \item Величина потенціалу корелює з кислотністю/основністю
\end{itemize}

%% --------------------------------------------------------
\subsection{Парціальні атомні заряди}
%% --------------------------------------------------------

Атомні заряди --- неспостережувані величини, але корисні для інтерпретації.
Існує багато схем розподілу електронної густини по атомах.

\subsubsection{Метод Малікена (Mulliken)}

Найпростіший метод, заснований на розкладанні матриці густини:
\[
q_A^{\text{Mulliken}} = Z_A - \sum_{\mu \in A} (PS)_{\mu\mu}
\]

\textbf{Недоліки:}
\begin{itemize}
    \item Сильно залежить від базису
    \item Нестабільний для великих базисів
    \item Неадекватний для дифузних функцій
\end{itemize}

\subsubsection{Метод Льовдіна (Löwdin)}

Використовує симетричну ортогоналізацію:
\[
q_A^{\text{Löwdin}} = Z_A - \sum_{\mu \in A} (PS^{1/2})_{\mu\mu}
\]

Менш чутливий до базису, ніж Малікен.

\subsubsection{Методи на основі ESP}

\textbf{CHELPG, RESP:} Заряди підбираються для відтворення ESP
на сітці точок навколо молекули:
\[
\min_{\{q_A\}} \sum_i \left[V_i^{\text{exact}} -
\sum_A \frac{q_A}{|\mathbf{r}_i - \mathbf{R}_A|}\right]^2
\]

\textbf{Переваги:}
\begin{itemize}
    \item Фізично обґрунтовані (відтворюють спостережувану величину)
    \item Слабко залежать від базису
    \item Добре працюють для силових полів
\end{itemize}

\inputcode{h2o_atomic_charges_comparison.py}

\textbf{Порівняння зарядів для H₂O:}
\begin{center}
\begin{tabular}{lccc}
\toprule
Метод & $q(\text{O})$ & $q(\text{H})$ & Сума \\
\midrule
Mulliken      & $-0.82$ & $+0.41$ & 0.00 \\
Löwdin        & $-0.68$ & $+0.34$ & 0.00 \\
CHELPG        & $-0.95$ & $+0.48$ & 0.00 \\
RESP          & $-0.93$ & $+0.47$ & 0.00 \\
\midrule
Bader (AIM)   & $-1.12$ & $+0.56$ & 0.00 \\
\bottomrule
\end{tabular}
\end{center}

\textbf{Висновки:}
\begin{itemize}
    \item ESP-методи дають більші заряди (краще для електростатики)
    \item Mulliken сильно залежить від базису
    \item Для силових полів: RESP або CHELPG
    \item Для аналізу хімічного зв'язку: Bader (AIM) або NBO
\end{itemize}

%% --------------------------------------------------------
\subsection{Приклад: вплив розчинника на електричні властивості}
%% --------------------------------------------------------

Розчинник суттєво змінює дипольний момент та поляризовність.
Використовуємо континуальні моделі сольватації (PCM, COSMO).

\inputcode{acetone_solvent_effect.py}

\textbf{Результати для ацетону (CH₃)₂CO:}
\begin{center}
\begin{tabular}{lccc}
\toprule
Розчинник & $\varepsilon$ & $\mu$ (D) & $\bar{\alpha}$ (au³) \\
\midrule
Газ              & 1.0   & 2.88 & 42.5 \\
Гексан           & 1.88  & 2.95 & 43.1 \\
Хлороформ        & 4.71  & 3.08 & 44.3 \\
Етанол           & 24.85 & 3.42 & 46.8 \\
Вода             & 78.36 & 3.56 & 48.2 \\
\midrule
Експ. (газ)      & ---   & 2.91 & --- \\
Експ. (вода)     & ---   & 3.50 & --- \\
\bottomrule
\end{tabular}
\end{center}

\textbf{Спостереження:}
\begin{itemize}
    \item $\mu$ зростає з діелектричною проникністю $\varepsilon$
    \item Поляризовність також збільшується (поляризація розчинником)
    \item Ефект найсильніший для полярних розчинників
    \item Важливо для моделювання реакцій у розчині
\end{itemize}