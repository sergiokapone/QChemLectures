%% ========================================================
\section{Магнітні властивості молекул}
%% ========================================================

%% --------------------------------------------------------
\subsection{Магнітна сприйнятливість}
%% --------------------------------------------------------

Магнітна сприйнятливість $\boldsymbol{\chi}$ описує відгук молекули
на зовнішнє магнітне поле:
\[
\mathbf{M} = \boldsymbol{\chi} \mathbf{B}
\]

де $\mathbf{M}$ --- індукована намагніченість, $\mathbf{B}$ --- магнітне поле.

\textbf{Компоненти магнітної сприйнятливості:}
\begin{itemize}
    \item \textbf{Діамагнітний внесок:} завжди негативний, пов'язаний
          з індукованими струмами (закон Ленца)
    \item \textbf{Парамагнітний внесок:} позитивний, виникає для систем
          з неспареними електронами
    \item Ізотропна сприйнятливість: $\chi_{\text{iso}} = (\chi_{xx} + \chi_{yy} + \chi_{zz})/3$
    \item Анізотропія: $\Delta\chi = \chi_{zz} - (\chi_{xx} + \chi_{yy})/2$
\end{itemize}

Для замкнених оболонок (синглетних систем):
\[
\chi_{ij} = -\frac{e^2}{4m_e c^2} \langle\Psi_0|
\sum_k (r_k^2 \delta_{ij} - r_{ki} r_{kj})|\Psi_0\rangle
\]

\subsubsection{Розрахунок магнітної сприйнятливості}

\inputcode{benzene_susceptibility.py}

\textbf{Результати для бензену (\ce{C6H6}):}

Бензен --- класичний приклад ароматичної молекули з сильною
магнітною анізотропією через кільцеві струми $\pi$-електронів.

\begin{center}
\begin{tabular}{lccc}
\toprule
Метод & $\chi_{\perp}$ (ppm cgs) & $\chi_{\parallel}$ (ppm cgs) & $\chi_{\text{iso}}$ (ppm cgs) \\
\midrule
RHF/6-31G*         & $-48.2$ & $-88.5$ & $-61.6$ \\
B3LYP/6-311+G(2d,p) & $-52.8$ & $-96.3$ & $-67.3$ \\
CCSD(T)/aug-cc-pVDZ & $-54.1$ & $-98.7$ & $-69.0$ \\
\midrule
Експеримент        & $-54.8$ & $-103.0$ & $-70.9$ \\
\bottomrule
\end{tabular}
\end{center}

де $\chi_{\perp}$ --- компонента в площині кільця,
$\chi_{\parallel}$ --- перпендикулярна компонента.

\textbf{Анізотропія:}
\[
\Delta\chi = \chi_{\parallel} - \chi_{\perp} \approx -48 \text{ ppm cgs}
\]

Велика негативна анізотропія --- характерна ознака ароматичності
(діамагнітне екранування кільцевими струмами).

%% --------------------------------------------------------
\subsection{Константи магнітного екранування ЯМР}
%% --------------------------------------------------------

Константа магнітного екранування ядра описує зміщення резонансної
частоти в ЯМР через електронне оточення:
\[
\mathbf{B}_{\text{eff}} = (1 - \boldsymbol{\sigma}) \mathbf{B}_0
\]

Тензор екранування $\boldsymbol{\sigma}$ обчислюється через похідну
від енергії:
\[
\sigma_{ij} = -\left.\frac{\partial^2 E}{\partial B_i \partial \mu_j^N}
\right|_{B=0}
\]

\textbf{Фізичні внески:}
\begin{itemize}
    \item \textbf{Діамагнітний:} локальні електрони навколо ядра
    \item \textbf{Парамагнітний:} змішування з збудженими станами
    \item Анізотропія екранування важлива для твердотільного ЯМР
\end{itemize}

\subsubsection{ЯМР спектр молекули води}

\inputcode{h2o_nmr_shielding.py}

\textbf{Результати для \ce{H2O}:}
\begin{center}
\begin{tabular}{lcccc}
\toprule
Ядро & RHF/6-311+G(2d,p) & B3LYP/aug-cc-pVTZ & CCSD/aug-cc-pVTZ & Експ. \\
\midrule
$^{17}$O & 328.4 & 320.5 & 318.2 & 315.0 \\
$^{1}$H  & 30.8  & 30.2  & 30.6  & 30.1  \\
\bottomrule
\end{tabular}
\end{center}

\textit{Константи екранування в ppm відносно голого ядра.}

\textbf{Хімічний зсув:} зазвичай вимірюють відносно стандарту (TMS для $^1$H):
\[
\delta = \frac{\sigma_{\text{ref}} - \sigma_{\text{sample}}}{\sigma_{\text{ref}}} \times 10^6
\]

%% --------------------------------------------------------
\subsection{Константи спін-спінової взаємодії (J-константи)}
%% --------------------------------------------------------

J-константи описують непряму взаємодію ядерних спінів через електрони:
\[
H_J = \sum_{A<B} \mathbf{I}_A \cdot \mathbf{J}_{AB} \cdot \mathbf{I}_B
\]

\textbf{Механізми взаємодії:}
\begin{itemize}
    \item \textbf{Контактний внесок Фермі (FC):} домінує для $^1$J
    \item \textbf{Спін-дипольний (SD):} важливий для $\pi$-систем
    \item \textbf{Парамагнітний спін-орбітальний (PSO)}
    \item \textbf{Діамагнітний спін-орбітальний (DSO)}
\end{itemize}

\subsubsection{J-константи в молекулі етану}

\inputcode{ethane_j_coupling.py}

\textbf{Результати для етану (\ce{C2H6}):}
\begin{center}
\begin{tabular}{lccc}
\toprule
Зв'язок & B3LYP/pcJ-2 & CCSD/pcJ-2 & Експеримент \\
\midrule
$^1J$(C--C)   & 32.8 & 34.5 & 34.6 \\
$^1J$(C--H)   & 126.4 & 125.2 & 125.3 \\
$^2J$(H--C--H) & $-2.8$ & $-2.5$ & $-2.4$ \\
$^3J$(H--C--C--H) & 7.9 & 7.8 & 8.0 \\
\bottomrule
\end{tabular}
\end{center}

\textit{Константи в Гц.}

\textbf{Залежність $^3J$ від двогранного кута (правило Карплуса):}
\[
^3J_{HH}(\phi) = A \cos^2\phi + B \cos\phi + C
\]

Типово: $A \approx 7$, $B \approx -1$, $C \approx 5$ Гц для $^3J$(H--C--C--H).

%% --------------------------------------------------------
\subsection{g-тензор для радикалів}
%% --------------------------------------------------------

Для систем з неспареними електронами (радикалів) g-тензор описує
взаємодію електронного спіну з магнітним полем:
\[
H = \mu_B \mathbf{B}^T \cdot \mathbf{g} \cdot \mathbf{S}
\]

g-тензор відхиляється від $g_e = 2.0023$ через спін-орбітальну взаємодію:
\[
\Delta g_{ij} = g_{ij} - g_e \delta_{ij}
\]

\subsubsection{Розрахунок g-тензора для радикала OH}

\inputcode{oh_radical_g_tensor.py}

\textbf{Результати для \ce{OH} ($^2\Pi$):}
\begin{center}
\begin{tabular}{lccc}
\toprule
Компонента & B3LYP/EPR-III & CCSD/aug-cc-pVTZ & Експеримент \\
\midrule
$g_{xx}$ & 2.0091 & 2.0095 & 2.0099 \\
$g_{yy}$ & 2.0091 & 2.0095 & 2.0099 \\
$g_{zz}$ & 2.0023 & 2.0024 & 2.0026 \\
\midrule
$g_{\text{iso}}$ & 2.0068 & 2.0071 & 2.0075 \\
\bottomrule
\end{tabular}
\end{center}

\textbf{Інтерпретація:}
\begin{itemize}
    \item $g_{zz} \approx g_e$ вздовж осі O--H (мінімальна \ce{СО}-взаємодія)
    \item $g_{xx}, g_{yy} > g_e$ перпендикулярно (\ce{СО}-змішування з $\pi^*$)
    \item Анізотропія $\Delta g \approx 0.007$ типова для легких атомів
\end{itemize}

%% --------------------------------------------------------
\subsection{Константи надтонкої взаємодії (HFC)}
%% --------------------------------------------------------

HFC описує взаємодію неспареного електрона з ядерними спінами:
\[
H_{\text{HFC}} = \sum_A \mathbf{S} \cdot \mathbf{A}_A \cdot \mathbf{I}_A
\]

\textbf{Ізотропна константа (a-тензор Фермі):}
\[
a_{\text{iso}} = \frac{8\pi}{3} g_e \mu_B g_N \mu_N \rho_{\text{spin}}(\mathbf{R}_A)
\]

де $\rho_{\text{spin}}(\mathbf{R}_A)$ --- спінова густина на ядрі $A$.

\textbf{Анізотропна частина:} дипольна взаємодія електрон-ядро.

\subsubsection{HFC константи для радикала метилу}

\inputcode{methyl_radical_hfc.py}

\textbf{Результати для \ce{CH3} ($^2A_2''$):}
\begin{center}
\begin{tabular}{lccc}
\toprule
Ядро & B3LYP/EPR-III & CCSD(T)/EPR-III & Експеримент \\
\midrule
$^{13}$C ($a_{\text{iso}}$) & $-28.2$ & $-27.8$ & $-27.0$ \\
$^{1}$H ($a_{\text{iso}}$)  & $-23.1$ & $-22.8$ & $-23.0$ \\
\bottomrule
\end{tabular}
\end{center}

\textit{Константи в Гаусах (G). Негативні значення --- поляризаційний механізм.}

\textbf{Механізм:}
\begin{itemize}
    \item Неспарений електрон у $p_z$-орбіталі вуглецю
    \item Прямої спінової густини на H немає ($\rho_\alpha(H) \approx 0$)
    \item Спінова поляризація $\sigma$(C--H) зв'язків створює негативну густину
    \item $a_{\text{iso}}(H) < 0$ --- характерна ознака $\pi$-радикалів
\end{itemize}

%% --------------------------------------------------------
\subsection{Спінове забруднення та нестабільність UHF}
%% --------------------------------------------------------

Для систем з відкритими оболонками UHF часто дає спінове забруднення:
\[
\langle S^2 \rangle = S(S+1) + \Delta
\]

де $\Delta > 0$ --- міра забруднення вищими спіновими станами.

\textbf{Проблеми спінового забруднення:}
\begin{itemize}
    \item Завищення енергії дисоціації
    \item Неправильна спінова густина
    \item Некоректні магнітні властивості
\end{itemize}

\textbf{Рішення:}
\begin{itemize}
    \item ROHF (обмежений відкрита оболонка)
    \item Проекційні методи (PMP2)
    \item DFT з обміннокореляційними функціоналами
\end{itemize}

\subsubsection{Приклад: молекула \ce{O2}}

\inputcode{o2_spin_contamination.py}

\textbf{Основний стан \ce{O2}: $^3\Sigma_g^-$ (триплет)}

\begin{center}
\begin{tabular}{lcccc}
\toprule
Метод & $\langle S^2 \rangle$ & $\Delta$ & $E$ (Hartree) & $r_e$ (\AA) \\
\midrule
UHF/6-311G*         & 2.17 & 0.17 & $-149.618$ & 1.165 \\
ROHF/6-311G*        & 2.00 & 0.00 & $-149.614$ & 1.168 \\
UB3LYP/6-311G*      & 2.03 & 0.03 & $-150.327$ & 1.208 \\
UCCSD(T)/cc-pVTZ    & 2.00 & 0.00 & $-150.168$ & 1.207 \\
\midrule
Експеримент         & 2.00 & --- & --- & 1.208 \\
\bottomrule
\end{tabular}
\end{center}

\textbf{Спостереження:}
\begin{itemize}
    \item UHF має значне спінове забруднення
    \item ROHF вільний від забруднення, але менш точний для енергії
    \item UB3LYP дає малe забруднення та гарну геометрію
    \item CCSD(T) --- "золотий стандарт" для багатоелектронних систем
\end{itemize}

%% --------------------------------------------------------
\subsection{Магнітооптична активність}
%% --------------------------------------------------------

\subsubsection{Обертання площини поляризації (ORD)}

Кут обертання на одиниці довжини при частоті $\omega$:
\[
[\alpha]_\omega = \frac{N}{c} \text{Im}\{\text{Tr}(\mathbf{G}'(\omega))\}
\]

де $\mathbf{G}'$ --- тензор оптичного обертання.

\subsubsection{Круговий дихроїзм (CD)}

Різниця поглинання лівої та правої циркулярно поляризованого світла:
\[
\Delta\varepsilon(\omega) = \varepsilon_L(\omega) - \varepsilon_R(\omega)
\]

Ротаційна сила переходу $i \to f$:
\[
R_{if} = \text{Im}\{\langle i|\hat{\boldsymbol{\mu}}|f\rangle \cdot
\langle f|\hat{\mathbf{m}}|i\rangle\}
\]

де $\hat{\mathbf{m}}$ --- оператор магнітного дипольного моменту.

\inputcode{methyloxirane_ord_cd.py}

\textbf{Приклад: (R)-метилоксиран (пропіленоксид)}

Класична хіральна молекула для демонстрації ORD та CD.

%---------------------------------------------------------
\begin{figure}[h!]\centering
\includegraphics[width=0.75\linewidth]{\currfiledir/methyloxirane_cd_spectrum.pdf}
\caption{Спектр кругового дихроїзму (R)-метилоксирану.
Розрахунок: TD-B3LYP/aug-cc-pVDZ.}
\label{pic:methyloxirane_cd}
\end{figure}
%---------------------------------------------------------

\textbf{Характеристики спектру CD:}
\begin{itemize}
    \item Перший перехід ($n \to \sigma^*$): $\lambda \sim 190$ нм, негативний CD
    \item Другий перехід ($\sigma \to \sigma^*$): $\lambda \sim 155$ нм, позитивний CD
    \item Знак CD визначається абсолютною конфігурацією
    \item Інтенсивність пропорційна ротаційній силі $R_{if}$
\end{itemize}

%% --------------------------------------------------------
\subsection{Магнітний циркулярний дихроїзм (MCD)}
%% --------------------------------------------------------

MCD --- поглинання в магнітному полі залежить від циркулярної поляризації:
\[
\Delta A = A_L - A_R = A_1 B + \frac{B}{kT}(A_2 + \frac{A_3}{kT})
\]

\textbf{Терми MCD:}
\begin{itemize}
    \item \textbf{A-терм:} виникає для вироджених станів (зняття виродження)
    \item \textbf{B-терм:} змішування станів магнітним полем
    \item \textbf{C-терм:} температурно-залежний, для парамагнітних систем
\end{itemize}

\textbf{Застосування:}
\begin{itemize}
    \item Визначення симетрії електронних станів
    \item Вивчення металопротеїнів та координаційних комплексів
    \item Розрізнення близьких за енергією переходів
\end{itemize}

%% --------------------------------------------------------
\subsection{Приклад: порівняння магнітних властивостей NO та O₂}
%% --------------------------------------------------------

\inputcode{no_o2_magnetic_comparison.py}

\textbf{NO ($^2\Pi$) та O₂ ($^3\Sigma_g^-$):}

\begin{center}
\begin{tabular}{lcccc}
\toprule
Молекула & Стан & $\langle S^2 \rangle$ & $\mu_B$ & $g_{\text{iso}}$ \\
\midrule
NO       & $^2\Pi_{1/2}$ & 0.75 & 1.0 & 2.008 \\
O₂       & $^3\Sigma_g^-$ & 2.00 & 2.0 & 2.004 \\
\bottomrule
\end{tabular}
\end{center}

\textbf{Експериментальні властивості:}
\begin{itemize}
    \item \textbf{NO:} один неспарений електрон, слабкий парамагнетик
    \item \textbf{O₂:} два неспарених електрони, сильний парамагнетик
          (рідкий O₂ притягується магнітом!)
    \item Обидві молекули мають орбітальний кутовий момент
    \item g-тензор анізотропний через спін-орбітальну взаємодію
\end{itemize}

\textbf{Висновки:}
\begin{itemize}
    \item Магнітні властивості визначаються електронною конфігурацією
    \item Радикали та триплети вимагають спеціальних методів (UHF, ROHF)
    \item Точний розрахунок потребує врахування кореляції та СО-взаємодії
    \item Експериментальні методи: ЕПР, ЯМР, MCD, магнетометрія
\end{itemize}