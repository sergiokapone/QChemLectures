% !TeX program = lualatex
% !TeX encoding = utf8
% !TeX spellcheck = uk_UA
% !TeX root =../PyscfBook.tex

%=========================================================
\Opensolutionfile{answer}[\currfilebase/\currfilebase-Answers]
\chapter{Властивості молекул}\label{\currfilebase}
%=========================================================


Після обчислень електронної структури молекули природним кроком є
обчислення її спостережуваних властивостей: коливальних спектрів,
оптичних переходів, електричних та магнітних характеристик.
PySCF надає потужні інструменти для розрахунку цих властивостей
на рівні теорії Хартрі-Фока та пост-ХФ методів.


У попередніх розділах ви навчилися виконувати основні квантово-хімічні
розрахунки: оптимізувати геометрію, обчислювати енергії, аналізувати
хвильові функції. Тепер настав час перейти до \textbf{обчислення
молекулярних властивостей} --- фізичних величин, які можна безпосередньо
порівняти з експериментом.

%% --------------------------------------------------------
\subsection*{Що таке молекулярні властивості?}
%% --------------------------------------------------------

\textbf{Молекулярні властивості} --- це спостережувані величини, які
характеризують поведінку молекули у зовнішніх полях або при взаємодії
з випромінюванням. На відміну від енергії чи геометрії, властивості
безпосередньо вимірюються експериментально:

\begin{itemize}
    \item \textbf{Коливальні спектри (ІЧ, Раман):} частоти, інтенсивності.
    \item \textbf{Електронні спектри (УФ-видимі):} довжини хвиль, сили осциляторів.
    \item \textbf{Електричні властивості:} дипольний момент, поляризовність.
    \item \textbf{Магнітні властивості:} ЯМР зсуви, g-тензор, константи спін-спінової взаємодії.
    \item \textbf{Оптична активність:} круговий дихроїзм, оптичне обертання.
\end{itemize}

\textbf{Головна перевага обчислення властивостей:} можна безпосередньо
порівняти теорію з експериментом та перевірити якість обчислень!

%% --------------------------------------------------------
\subsection*{Встановлення додаткових модулів PySCF}
%% --------------------------------------------------------

Базова інсталяція PySCF містить функціонал для розрахунку енергій,
градієнтів та оптимізації геометрії. Для обчислення спектроскопічних
та магнітних властивостей потрібні \textbf{додаткові модулі}.

\subsubsection*{Основні модулі для властивостей}

\begin{enumerate}
    \item \texttt{pyscf-properties} --- головний модуль для властивостей:
    \begin{minted}{bash}
pip install pyscf-properties
    \end{minted}

    Цей модуль включає:
    \begin{itemize}
        \item \texttt{pyscf.prop.nmr} --- ЯМР константи екранування
        \item \texttt{pyscf.prop.esr} --- ЕПР g-тензор, константи надтонкої взаємодії
        \item \texttt{pyscf.prop.polarizability} --- поляризовність, гіперполяризовності
        \item \texttt{pyscf.prop.magnetizability} --- магнітна сприйнятливість
    \end{itemize}

    \item \textbf{pyscf-geomopt} --- для оптимізації геометрії:
    \begin{minted}{bash}
pip install pyscf-geomopt
    \end{minted}

    \item \textbf{geometric} --- покращена оптимізація (рекомендується):
    \begin{minted}{bash}
pip install geometric
    \end{minted}
\end{enumerate}

\subsubsection*{Перевірка встановлення}

Після інсталяції перевірте, чи працюють модулі:

\begin{minted}{python}
import pyscf
from pyscf import gto, scf
from pyscf.prop import nmr, esr  # для магнітних властивостей
from pyscf.hessian import thermo  # для коливальних властивостей
from pyscf import tddft  # для електронних збуджень

print("PySCF версія:", pyscf.__version__)
print("Модулі успішно імпортовані!")
\end{minted}


Якщо імпорт проходить без помилок --- все готово для роботи!

\subsubsection*{Альтернатива: встановлення з conda}

Для користувачів Anaconda/Miniconda:

\begin{minted}{python}
conda install -c pyscf pyscf
conda install -c conda-forge geometric
pip install pyscf-properties  # properties поки немає в conda
\end{minted}

%% --------------------------------------------------------
\subsection*{Концептуальна основа: теорія відгуку}
%% --------------------------------------------------------

Теорія відгуку (response theory) є потужним інструментом для обчислення молекулярних властивостей у квантовій механіці. Її суть полягає в аналізі реакції системи на зовнішні збурення: ми застосовуємо слабке зовнішнє поле (або збурення) і спостерігаємо, як змінюється енергія, хвильова функція чи інші observable величини. Цей підхід дозволяє систематично обчислювати похідні енергії по параметрах збурення, які безпосередньо відповідають фізичним властивостям, таким як поляризовність, магнітні моменти чи спектроскопічні переходи.

Загалом, енергія системи в присутності збурення $\lambda$ (де $\lambda$ може бути будь-яким параметром, наприклад, електричним або магнітним полем чи геометрією молекули тощо) розкладається в ряд Тейлора:

\[
E(\lambda) = E_0 + \left. \frac{\partial E}{\partial \lambda} \right|_{\lambda=0} \lambda + \frac{1}{2!} \left. \frac{\partial^2 E}{\partial \lambda^2} \right|_{\lambda=0} \lambda^2 + \frac{1}{3!} \left. \frac{\partial^3 E}{\partial \lambda^3} \right|_{\lambda=0} \lambda^3 + \ldots
\]

Коефіцієнти при степенях $\lambda$ є похідними енергії по збуренню в точці $\lambda=0$ і визначають лінійний, квадратичний тощо відгук системи. Ці похідні інтерпретуються як фундаментальні властивості:

\begin{itemize}
    \item Перша похідна $\left. \frac{\partial E}{\partial \lambda} \right|_{\lambda=0}$ --- лінійний відгук (наприклад, дипольний момент).
    \item Друга похідна $\left. \frac{\partial^2 E}{\partial \lambda^2} \right|_{\lambda=0}$ --- квадратичний відгук (наприклад, поляризовність).
    \item Вищі похідні --- нелінійні ефекти (гіперполяризовність тощо).
\end{itemize}

\textbf{Загальний принцип:} будь-яка молекулярна властисть, пов'язана з відгуком на збурення, є похідною енергії (або хвильової функції) по відповідному параметру.

\subsubsection*{Приклад: розклад енергії в електричному полі}

Для ілюстрації розглянемо конкретний випадок статичного однорідного електричного поля $\mathbf{E}$. Енергія системи розкладається як:

\[
E(\mathbf{E}) = E_0 - \boldsymbol{\mu} \cdot \mathbf{E}
- \frac{1}{2}\sum_{ij} \alpha_{ij} E_i E_j
- \frac{1}{6}\sum_{ijk} \beta_{ijk} E_i E_j E_k + \ldots
\]

де:
\begin{itemize}
    \item $E_0$ --- енергія системи без поля.
    \item $\boldsymbol{\mu}$ --- статичний дипольний момент (перша похідна $\left. -\frac{\partial E}{\partial \mathbf{E}} \right|_{\mathbf{E}=0}$).
    \item $\alpha_{ij}$ --- тензор поляризовності (друга похідна $\left. -\frac{\partial^2 E}{\partial E_i \partial E_j} \right|_{\mathbf{E}=0}$).
    \item $\beta_{ijk}$ --- тензор першої гіперполяризовності (третя похідна $\left. -\frac{\partial^3 E}{\partial E_i \partial E_j \partial E_k} \right|_{\mathbf{E}=0}$).
\end{itemize}

Цей розклад є частковим випадком загальної теорії відгуку, де $\lambda \equiv \mathbf{E}$. Аналогічно, теорія застосовується до магнітних полів (для магнітної сприйнятливості), геометричних змін (для сил і гессіанів) чи часозалежних збурень (для спектроскопії).

\textbf{Загальний принцип:}
\begin{equation*}
     \text{властивість} = \text{похідна енергії по зовнішньому полю}.
\end{equation*}

\subsubsection*{Методи обчислення похідних}

\begin{enumerate}
    \item \textbf{Числові похідні (finite differences):}
    \[
    \alpha_{xx} = -\frac{\partial^2 E}{\partial E_x^2} \approx
    \frac{E(E_x) + E(-E_x) - 2E(0)}{\Delta E_x^2}
    \]

    \textit{Плюси:} просто, універсально\\
    \textit{Мінуси:} неточно, повільно, багато розрахунків

    \item \textbf{Аналітичні похідні:}
    \[
    \alpha_{ij} = -\left.\frac{\partial^2 E}{\partial E_i \partial E_j}\right|_{E=0}
    = \text{складна формула через хвильову функцію}
    \]

    \textit{Плюси:} точно, швидко, один розрахунок\\
    \textit{Мінуси:} складна реалізація

    \item \textbf{Coupled-Perturbed методи (CP-HF, CP-KS):}

    Розв'язують рівняння для змін орбіталей під дією поля.
    Використовується в PySCF для більшості властивостей.
\end{enumerate}

\textbf{Рекомендація:} завжди використовуйте аналітичні похідні, коли вони доступні!

%% --------------------------------------------------------
\subsection*{Gauge-invariance для магнітних властивостей}
%% --------------------------------------------------------

Магнітні властивості (ЯМР, магнітна сприйнятливість) мають специфічну проблему:
результат залежить від вибору \textbf{gauge} (калібрування) векторного потенціалу.

\subsubsection*{Проблема gauge origin}

Магнітне поле $\mathbf{B}$ можна записати через векторний потенціал:
\[
\mathbf{B} = \nabla \times \mathbf{A}
\]

Але $\mathbf{A}$ не унікальний! Калібрувальне перетворення
$\mathbf{A} \to \mathbf{A} + \nabla\chi$ не змінює $\mathbf{B}$.

Для скінченного базису результат \textbf{залежить від вибору початку координат}
для $\mathbf{A}$ --- це нефізично!

\subsubsection*{Рішення: GIAO (Gauge-Independent Atomic Orbitals)}

У PySCF використовується метод GIAO (також називається IGLO, CSGT):

\begin{itemize}
    \item Кожна атомна орбіталь має своє локальну калібрування.
    \item Результати не залежать від вибору початку координат.
    \item Швидка збіжність з розміром базису.
    \item \textbf{Стандарт} для розрахунків ЯМР властивостей.
\end{itemize}

\textbf{В PySCF це працює автоматично} --- не потрібно нічого додатково налаштовувати!

\begin{minted}{python}
from pyscf.prop import nmr

# GIAO використовується автоматично
nmr_calc = nmr.RHF(mf)
shielding = nmr_calc.kernel()  # gauge-independent результат!
\end{minted}

%% --------------------------------------------------------
\subsection*{Базиси для розрахунку властивостей}
%% --------------------------------------------------------

\textbf{Важливо:} базиси для енергій $\neq$ базиси для властивостей!

\subsubsection*{Загальні вимоги}

\begin{enumerate}
    \item \textbf{Дифузні функції (aug-, d-aug-):}

    Критично важливі для:
    \begin{itemize}
        \item Поляризовностей (опис деформації електронної густини)
        \item Магнітних властивостей (струми далеко від ядер)
        \item Анізотропних властивостейs
    \end{itemize}

    Базиси: \texttt{aug-cc-pVDZ}, \texttt{aug-cc-pVTZ}, \texttt{6-311++G**}

    \item \textbf{Високий angular momentum (поляризаційні функції):}

    Для точних властивостей потрібні $d$, $f$, іноді $g$ функції.

    \item \textbf{Tight functions для ЯМР:}

    ЯМР екранування чутливе до густини \textit{на ядрі} $\rho(\mathbf{R}_A)$.
    Спеціальні базиси: \texttt{pcS-n}, \texttt{pcJ-n}
\end{enumerate}

\subsubsection*{Спеціалізовані базиси}

\begin{center}
\begin{tblr}{lll}
\toprule
Властивість & Рекомендований базис & Коментар \\
\midrule
ІЧ частоти & 6-31G*, 6-311G** & Невеликі базиси OK \\
УФ-видимі спектри & aug-cc-pVDZ/TZ & Дифузні важливі \\
Поляризовність & aug-cc-pVDZ & Обов'язково aug- \\
ЯМР екранування & pcS-2, aug-cc-pVTZ & Tight + diffuse \\
J-константи & pcJ-2, pcJ-3 & Спеціальний базис \\
ЕПР (g-тензор, HFC) & EPR-II, EPR-III & Tight s-функції \\
\bottomrule
\end{tblr}
\end{center}

\subsubsection*{Практичні поради}

\begin{itemize}
    \item Для початку: \texttt{aug-cc-pVDZ} --- універсальний вибір
    \item Для точних результатів: \texttt{aug-cc-pVTZ}
    \item Для великих систем: \texttt{6-311+G(2d,p)} --- компроміс
    \item Для ЯМР: спеціалізовані базиси \texttt{pcS-n} кращі за aug-cc-pVnZ
\end{itemize}

%% --------------------------------------------------------
\subsection*{Структура розділу}
%% --------------------------------------------------------

Розділ організовано за типами властивостей:

\begin{enumerate}
    \item \textbf{Коливальні властивості (Розділ X.1):}
    \begin{itemize}
        \item Гессіан та нормальні моди
        \item ІЧ спектри (інтенсивності, відбір правила)
        \item Раман спектри
        \item Термохімія та ZPE
    \end{itemize}

    \item \textbf{Електронні властивості (Розділ X.2):}
    \begin{itemize}
        \item Електронні збудження (TD-DFT)
        \item УФ-видимі спектри
        \item Флуоресценція та фосфоресценція
    \end{itemize}

    \item \textbf{Електричні властивості (Розділ X.3):}
    \begin{itemize}
        \item Дипольний момент
        \item Поляризовність (статична та динамічна)
        \item Гіперполяризовності
        \item Електростатичний потенціал
    \end{itemize}

    \item \textbf{Магнітні властивості (Розділ X.4):}
    \begin{itemize}
        \item Магнітна сприйнятливість
        \item ЯМР константи екранування та J-константи
        \item ЕПР властивості (g-тензор, HFC)
        \item Оптична активність (ORD, CD)
    \end{itemize}
\end{enumerate}

%% --------------------------------------------------------
\subsection*{Філософія прикладів}
%% --------------------------------------------------------

Кожна підсекція містить:

\begin{itemize}
    \item \textbf{Теоретичне введення:} що ми обчислюємо і чому це важливо
    \item \textbf{Повний робочий код:} готовий до запуску приклад
    \item \textbf{Інтерпретацію результатів:} що означають числа
    \item \textbf{Порівняння з експериментом:} наскільки точна теорія
    \item \textbf{Методологічні поради:} як вибрати метод/базис
\end{itemize}

%\textbf{Важливий принцип:} ми використовуємо \textbf{фізично цікаві} молекули,
%а не просто ``іграшкові'' приклади:
%\begin{itemize}
%    \item \ce{H2O} --- прототип для H-зв'язків
%    \item Бензен --- ароматичність
%    \item \ce{O2}, \ce{NO} --- парамагнетики
%    \item Хіральні молекули --- оптична активність
%\end{itemize}

%% --------------------------------------------------------
\subsection*{Практичні рекомендації перед початком}
%% --------------------------------------------------------

\subsubsection*{Типовий workflow розрахунку властивостей}

\begin{enumerate}
    \item \textbf{Оптимізуйте геометрію:}
    \begin{minted}{python}
from pyscf import gto, scf
from pyscf.geomopt.geometric_solver import optimize

mol = gto.M(atom='...', basis='6-31g*')
mf = scf.RHF(mol).run()
mol_eq = optimize(mf)  # оптимізована геометрія
    \end{minted}

    \item \textbf{Перерахуйте з кращим базисом:}
    \begin{minted}{python}
mol_prop = gto.M(atom=mol_eq.atom, basis='aug-cc-pvdz')
mf_prop = scf.RHF(mol_prop).run()
    \end{minted}

    \item \textbf{Обчисліть властивість:}
    \begin{minted}{python}
from pyscf.prop import nmr
nmr_calc = nmr.RHF(mf_prop)
shielding = nmr_calc.kernel()
    \end{minted}

    \item \textbf{Проаналізуйте результат!}
\end{enumerate}

\subsubsection*{Типові помилки початківців}

\begin{itemize}
    \item Розрахунок властивостей без оптимізації геометрії
    \item Використання малих базисів (6-31G) для поляризовностей
    \item Забування про дифузні функції для анізотропних властивостей
    \item Розрахунок ЯМР без gauge-independent методу
    \item Порівняння абсолютних значень замість зсувів (для ЯМР)
\end{itemize}

\subsubsection*{Контрольний чек-лист}

Перед розрахунком властивості перевірте:

\begin{itemize}
    \item Геометрія оптимізована? (немає уявних частот)
    \item Базис підходить для цієї властивості?
    \item Метод підходить? (HF завищує поляризовність, потрібен DFT/MP2)
    \item Для магнітних: gauge-independent метод?
    \item SCF збігся? (check \inlinecode{mf.converged})
    \item Для радикалів: перевірили $\langle S^2 \rangle$?
\end{itemize}

%% --------------------------------------------------------
\subsection*{Корисні ресурси}
%% --------------------------------------------------------

\begin{itemize}
    \item \textbf{Документація PySCF:} \url{https://pyscf.org/user.html}
    \item \textbf{PySCF properties:} \url{https://github.com/pyscf/properties}
    \item \textbf{Basis Set Exchange:} \url{https://www.basissetexchange.org/}
    \item \textbf{NIST WebBook:} експериментальні дані для порівняння
\end{itemize}


% !TeX program = lualatex
% !TeX encoding = utf8
% !TeX spellcheck = uk_UA
% !TeX root =../PyscfBook.tex

%% ========================================================
\section{Коливальні властивості молекул}
%% ========================================================

Коливальна спектроскопія --- один із найпотужніших методів ідентифікації
молекул та вивчення їх структури. Інфрачервоні (ІЧ) та Раман спектри
надають інформацію про коливальні моди, силові константи зв'язків,
та симетрію молекули.

%% --------------------------------------------------------
\subsection{Гессіан та нормальні моди}
%% --------------------------------------------------------

Коливання атомів у молекулі відбуваються поблизу рівноважної геометрії $\mathbf{R}_0$.
Для малих відхилень потенційна енергія системи можна апроксимувати розкладом у ряд Тейлора до другого порядку:
\[
    E(\mathbf{R}) \approx E(\mathbf{R}_0) + \frac{1}{2}\sum_{ij} H_{ij} \, \Delta R_i \, \Delta R_j,
\]
де
\[
    H_{ij} = \left. \frac{\partial^2 E}{\partial R_i \partial R_j} \right|_{\mathbf{R}_0}
\]
--- елементи \emph{гесіану} (матриці других похідних енергії за декартовими координатами ядер).

Гесіан має розмірність $3N \times 3N$ для молекули з $N$ атомів
і містить повну інформацію про гармонічні коливальні властивості системи.

%% --------------------------------------------------------
\subsubsection{Матриця силових констант}
%% --------------------------------------------------------

Гесіан також називають \textbf{матрицею силових констант}, оскільки його елементи відображають реакцію потенціалу на малі зміщення ядер:
\[
    H_{ij} = \left.\frac{\partial^2 E}{\partial R_i \partial R_j}\right|_{\mathbf{R}_0}
\]

\textbf{Фізичний зміст елементів гесіану:}
\begin{itemize}
    \item \textit{Діагональні елементи} --- характеризують жорсткість потенціалу вздовж окремих координат;
    \item \textit{Недіагональні елементи} --- описують зв’язок між зміщеннями різних атомів;
    \item \textit{Власні значення} $\,\omega^2\,$ --- квадрати коливальних частот;
    \item \textit{Власні вектори} --- напрямки нормальних мод.
\end{itemize}

Таким чином, діагоналізація гесіану дозволяє отримати нормальні координати та спектр гармонічних частот, що повністю характеризують коливальну поведінку молекули поблизу рівноваги.


%% --------------------------------------------------------
\subsection{Обчислення гесіану}
%% --------------------------------------------------------

PySCF може обчислювати гесіан аналітично або чисельно:

\inputcode{h2o_hessian.py}

\textbf{Важливі моменти:}
\begin{itemize}
    \item Аналітичний гесіан доступний для RHF, UHF, RKS, UKS
    \item Чисельний гесіан використовує скінченні різниці:
          $H_{ij} \approx [E(R_i+h) - 2E(R_i) + E(R_i-h)]/h^2$
    \item Аналітичний метод точніший та швидший
    \item Гесіан обчислюється в декартових координатах
\end{itemize}


%% --------------------------------------------------------
\subsubsection{Нормальні коливання}
%% --------------------------------------------------------


Розв'язуючи секулярне рівняння:
\[
    (\mathbf{H} - \omega^2 \mathbf{M}) \mathbf{L} = 0
\]

отримуємо:
\begin{itemize}
    \item $3N - 6$ коливальних мод (нелінійна молекула)
    \item $3N - 5$ коливальних мод (лінійна молекула)
    \item 3 трансляційні моди ($\omega = 0$)
    \item 3 обертальні моди ($\omega = 0$, 2 для лінійної)
\end{itemize}

\textbf{Частоти в різних одиницях:}
\begin{center}
    \begin{tblr}{
        colspec = {X[l] X[r] X[l]},
        row{1} = {font=\bfseries},
        hline{1,2,Z} = {solid},
            }
        Одиниця   & Множник                & Використання             \\
        см$^{-1}$ & 1                      & ІЧ спектроскопія         \\
        Гц        & $2.998 \times 10^{10}$ & СІ одиниці               \\
        еВ        & $1.240 \times 10^{-4}$ & Електронна спектроскопія \\
        Хартрі    & $4.556 \times 10^{-6}$ & Атомні одиниці           \\
    \end{tblr}
\end{center}

%% --------------------------------------------------------
\subsubsection{Розрахунок Гессіану для \ce{H2O}}
%% --------------------------------------------------------


\inputcode{h2o_frequencies.py}

\textbf{Результати для \ce{H2O}:}

\textbf{\footnotesize Частоти в см$^{-1}$. Базис: 6-311++G(3df,3pd)}
\begin{center}
    \begin{tblr}{
        colspec = {X[c] X[r] X[r] X[r] X[l]},
        row{1} = {font=\bfseries},
        hline{1,2,Z} = {solid},
            }
        Мода    & B3LYP & MP2  & Експ. & Опис                    \\
        $\nu_1$ & 3825  & 3832 & 3657  & Симетричне валентне     \\
        $\nu_2$ & 1653  & 1649 & 1595  & Ножичне деформаційне    \\
        $\nu_3$ & 3936  & 3943 & 3756  & Антисиметричне валентне \\
    \end{tblr}
\end{center}


\textbf{Структура гесіану для \ce{H2O}:}
\begin{itemize}
    \item Розмірність: $9 \times 9$ (3 атоми × 3 координати)
    \item Симетричний: $H_{ij} = H_{ji}$
    \item Позитивно визначений у мінімумі енергії
    \item Має 6 нульових власних значень (трансляції та обертання)
\end{itemize}



\textbf{Спостереження:}
\begin{itemize}
    \item Гармонічні частоти завищені на 3--5\% через ангармонізм
    \item Валентні коливання (\ce{O-H}): 3600--4000 см$^{-1}$
    \item Деформаційне коливання: $\sim$1600 см$^{-1}$
    \item Симетрія: $\nu_1$ (A$_1$), $\nu_2$ (A$_1$), $\nu_3$ (B$_2$)
\end{itemize}

%% --------------------------------------------------------
\subsubsection{Масштабування частот}
%% --------------------------------------------------------


Гармонічні частоти систематично завищені. Емпіричне виправлення:
\[
    \nu_{\text{корект}} = \lambda \cdot \nu_{\text{гарм}}
\]

\textbf{Типові масштабувальні фактори:}

\begin{center}
    \begin{tblr}{
        colspec = {X[l,2] X[r] X[l,2]},
        row{1} = {font=\bfseries},
        hline{1,2,Z} = {solid},
            }
        Метод/базис         & $\lambda$ & Коментар               \\
        HF/6-31G*           & 0.8929    & Сильно завищує         \\
        B3LYP/6-31G*        & 0.9613    & Універсальний          \\
        B3LYP/6-311+G(2d,p) & 0.9679    & Кращий базис           \\
        MP2/6-31G*          & 0.9434    & Для точних розрахунків \\
    \end{tblr}
\end{center}

\textit{Джерело: NIST Computational Chemistry Comparison and Benchmark Database}

%% --------------------------------------------------------
\subsection{Інфрачервоні (ІЧ) спектри}
%% --------------------------------------------------------

\subsubsection{Інтенсивності ІЧ поглинання}

ІЧ інтенсивність пропорційна квадрату зміни дипольного моменту:
\[
    I_i \propto \left|\frac{d\boldsymbol{\mu}}{dQ_i}\right|^2
\]

де $Q_i$ --- нормальна координата $i$-ї моди,, $\boldsymbol{\mu}$ --- дипольний момент.

\textbf{Правило відбору:}
\begin{itemize}
    \item ІЧ активна мода: $\frac{d\boldsymbol{\mu}}{dQ_i} \neq 0$.
    \item Симетрична молекула може мати ІЧ-неактивні моди.
    \item Для \ce{H2O} (група $C_{2v}$): всі три моди ІЧ-активні
\end{itemize}


%% --------------------------------------------------------
\subsubsection{ІЧ спектр \ce{H2O}}
%% --------------------------------------------------------


\inputcode{h2o_ir_spectrum.py}

\textbf{Типові інтенсивності для \ce{H2O}:}
\begin{center}
    \begin{tabular}{lcc}
        \toprule
        Мода                       & $\nu$ (см$^{-1}$) & $I$ (км/моль) \\
        \midrule
        $\nu_1$ (симетр. валент.)  & 3657              & 5             \\
        $\nu_2$ (деформаційна)     & 1595              & 73            \\
        $\nu_3$ (антисим. валент.) & 3756              & 58            \\
        \bottomrule
    \end{tabular}
\end{center}

\textbf{Фізична інтерпретація:}
\begin{itemize}
    \item Деформаційна мода найінтенсивніша (велика зміна $\boldsymbol{\mu}$)
    \item Симетричне валентне --- найслабше (компенсація)
    \item Інтенсивність залежить від полярності зв'язків
\end{itemize}

%% --------------------------------------------------------
\subsubsection{ІЧ спектр \ce{CO2}}
%% --------------------------------------------------------

\inputcode{co2_ir_spectrum.py}

\textbf{Коливальні моди \ce{CO2} (лінійна, D$_{\infty h}$):}

\begin{center}
    \begin{tblr}{
        colspec = {X[c] X[r] X[r] X[c] X[l,2]},
        row{1} = {font=\bfseries},
        hline{1,2,Z} = {solid},
            }
        Мода                   & Частота     & Інтенс.   & ІЧ        & Опис                     \\
                               & (см$^{-1}$) & (км/моль) &                                      \\
        $\nu_1$ ($\Sigma_g^+$) & 1333        & 0         & Неактивна & Симетричне валентне      \\
        $\nu_2$ ($\Pi_u$)      & 667         & 85        & Активна   & Деформаційне (вироджене) \\
        $\nu_3$ ($\Sigma_u^+$) & 2349        & 1580      & Активна   & Антисиметричне валентне  \\
    \end{tblr}
\end{center}

\textit{Розрахунок: B3LYP/aug-cc-pVTZ}

%---------------------------------------------------------
%\begin{figure}[h!]\centering
%%    \includegraphics[width=0.85\linewidth]{\currfiledir/co2_ir_spectrum.pdf}
%    \caption{ІЧ спектр \ce{CO2}. Симетрична мода $\nu_1$ не активна в ІЧ через відсутність зміни дипольного моменту.}
%    \label{pic:co2_ir}
%\end{figure}
%---------------------------------------------------------

\textbf{Фізична інтерпретація:}
\begin{itemize}
    \item $\nu_1$: симетричне розтягування, $\mu$ не змінюється (ІЧ-неактивна)
    \item $\nu_2$: згинання молекули, $\mu$ виникає (ІЧ-активна)
    \item $\nu_3$: асиметричне розтягування, велика зміна $\mu$ (дуже інтенсивна)
\end{itemize}

\subsubsection{Характеристичні частоти функціональних груп}

\begin{center}
    \begin{tblr}{
        colspec = {X[l,2] X[r] X[l,2]},
        row{1} = {font=\bfseries},
        hline{1,2,Z} = {solid},
            }
        Група                & Частота (см$^{-1}$) & Коментар              \\
        \ce{O-H} (спирти)    & 3600--3650          & Гострий пік (вільний) \\
        \ce{O-H} (H-зв'язок) & 3200--3550          & Широкий (асоціація)   \\
        \ce{N-H}             & 3300--3500          & Середня інтенсивність \\
        \ce{C-H} (алкани)    & 2850--2960          & Валентні коливання    \\
        \ce{C-H} (алкени)    & 3010--3095          & Вища частота          \\
        \ce{C=O} (кетони)    & 1705--1725          & Дуже інтенсивна       \\
        \ce{C=O} (аміди)     & 1630--1690          & Зміщення резонансом   \\
        \ce{C=C}             & 1620--1680          & Слабка в симетричних  \\
        \ce{C\bond{3}N}      & 2210--2260          & Гострий пік           \\
        \ce{C-O}             & 1050--1150          & Сильна, асиметричне   \\
    \end{tblr}
\end{center}

%% --------------------------------------------------------
\subsection{Раманівський спектр}
%% --------------------------------------------------------

На відміну від ІЧ, активність у Раман-спектрі визначається
поляризовністю $\boldsymbol{\alpha}$, тобто інтенсивність Раман розсіювання пропорційна зміні поляризовності:
\[
    I_i^{\text{Раман}} \propto \left|\frac{d\alpha_{ij}}{dQ_k}\right|^2
\]

%% --------------------------------------------------------
\subsubsection{Інтенсивності Раман розсіювання}
%% --------------------------------------------------------


\textbf{Правило відбору Раман:}
\begin{itemize}
    \item Раман-активна: $\frac{d\alpha}{dQ_i} \neq 0$.
    \item Часто доповнює ІЧ (правило взаємного виключення для центросиметричних).
    \item Для \ce{H2O}: всі три моди також Раман-активні
\end{itemize}

\inputcode{h2o_raman_activity.py}

\textbf{Типові активності для \ce{H2O}:}
\begin{center}
    \begin{tabular}{lcc}
        \toprule
        Мода    & $\nu$ (см$^{-1}$) & Раман-активність (Å$^4$/amu) \\
        \midrule
        $\nu_1$ & 3657              & 1.8                          \\
        $\nu_2$ & 1595              & 0.4                          \\
        $\nu_3$ & 3756              & 3.2                          \\
        \bottomrule
    \end{tabular}
\end{center}


\textbf{Принцип взаємного виключення:}

Для молекул з центром інверсії (наприклад, \ce{CO2}):
\begin{itemize}
    \item Парні моди ($g$): Раман активні, ІЧ неактивні
    \item Непарні моди ($u$): ІЧ активні, Раман неактивні
\end{itemize}

\subsubsection{Раман спектр бензену}

\inputcode{benzene_raman_spectrum.py}

\textbf{Вибрані моди бензену \ce{C6H6} (D$_{6h}$):}

\begin{center}
    \begin{tblr}{
        colspec = {X[c] X[c] X[r] X[c] X[c]},
        row{1} = {font=\bfseries},
        hline{1,2,Z} = {solid},
            }
        Мода       & Симетрія & Частота     & ІЧ        & Раман            \\
                   &          & (см$^{-1}$) &                              \\
        $\nu_1$    & A$_{1g}$ & 993         & Неактивна & Активна          \\
        $\nu_2$    & A$_{1g}$ & 3062        & Неактивна & Активна (сильна) \\
        $\nu_6$    & E$_{2g}$ & 608         & Неактивна & Активна          \\
        $\nu_{18}$ & E$_{1u}$ & 1038        & Активна   & Неактивна        \\
        $\nu_{19}$ & E$_{1u}$ & 3080        & Активна   & Неактивна        \\
    \end{tblr}
\end{center}

\textit{Розрахунок: B3LYP/6-311+G(2d,p)}

%---------------------------------------------------------
\begin{figure}[h!]\centering
%    \includegraphics[width=0.85\linewidth]{\currfiledir/benzene_raman_spectrum.pdf}
    \caption{Раман спектр бензену демонструє принцип взаємного виключення.
        Симетричні моди ($g$) активні в Раман, але неактивні в ІЧ.}
    \label{pic:benzene_raman}
\end{figure}
%---------------------------------------------------------

\textbf{Характерні Раман смуги:}
\begin{itemize}
    \item 993 см$^{-1}$: дихальна мода кільця (ring breathing)
    \item 3062 см$^{-1}$: симетричне валентне \ce{C-H}
    \item 608 см$^{-1}$: деформація кільця
\end{itemize}

\subsubsection{Порівняння ІЧ та Раман}

\begin{center}
    \begin{tblr}{
        colspec = {X[l,2] X[l,3] X[l,3]},
        row{1} = {font=\bfseries},
        hline{1,2,Z} = {solid},
            }
        Властивість               & ІЧ спектроскопія         & Раман спектроскопія         \\
        Правило відбору           & $\frac{d\mu}{dQ} \neq 0$ & $\frac{d\alpha}{dQ} \neq 0$ \\
        Джерело                   & ІЧ лампа                 & Лазер (видиме/УФ)           \\
        Зразок                    & Розчин, плівка, KBr      & Розчин, кристал             \\
        Вода                      & Сильно поглинає          & Слабкий сигнал              \\
        Скло                      & Непрозоре                & Прозоре                     \\
        Полярні групи             & Сильний сигнал           & Слабкий                     \\
        Неполярні                 & Слабкий                  & Сильний сигнал              \\
        \ce{C=C}, \ce{C\bond{3}C} & Слабкі/неактивні         & Сильні                      \\
        \ce{C=O}, \ce{O-H}        & Дуже сильні              & Слабкі                      \\
    \end{tblr}
\end{center}

%% --------------------------------------------------------
\subsection{Термохімія та нульова енергія}
%% --------------------------------------------------------

Використовуючи частоти коливань, можна обчислити термодинамічні
функції в наближенні гармонічного осцилятора та жорсткого ротатора.

%% --------------------------------------------------------
\subsubsection{Нульова коливальна енергія (ZPE)}
%% --------------------------------------------------------


Навіть при $T = 0$ K молекула має енергію коливань:
\[
    E_{\text{ZPE}} = \sum_{i=1}^{3N-6} \frac{1}{2}\hbar\omega_i
\]


\textbf{Важливість ZPE:}
\begin{itemize}
    \item Для точних термохімічних розрахунків
    \item Ізотопні ефекти (HD vs \ce{H2})
    \item Тунелювання через бар'єри
    \item Типово 5--15 ккал/моль для органічних молекул
\end{itemize}

Для \ce{H2O}:
\[
    E_{\text{ZPE}} \approx \frac{1}{2}(3657 + 1595 + 3756) \text{ см}^{-1}
    \times 1.439 \text{ ккал/(моль·см}^{-1}) \approx 13.3 \text{ ккал/моль}
\]

%% --------------------------------------------------------
\subsubsection{Термохімічні поправки}
%% --------------------------------------------------------

При температурі $T$ коливальний внесок:
\[
    E_{\text{vib}}(T) = \sum_k \frac{\hbar\omega_k}{\exp(\hbar\omega_k/k_BT) - 1}.
\]

Повна енергія при температурі $T$:
\[
    E_{\text{total}}(T) = E_{\text{elec}} + E_{\text{ZPE}} + E_{\text{vib}}(T) +
    E_{\text{rot}}(T) + E_{\text{trans}}(T)
\]

\textbf{Ентальпія:}
\[
    H(T) = E_{\text{total}}(T) + RT
\]

\textbf{Ентропія:}
\[
    S(T) = S_{\text{trans}} + S_{\text{rot}} + S_{\text{vib}} + S_{\text{elec}}
\]

\textbf{Вільна енергія Гіббса:}
\[
    G(T) = H(T) - TS(T)
\]

%% --------------------------------------------------------
\subsubsection{Термохімічний розрахунок для \ce{H2O}}
%% --------------------------------------------------------


\inputcode{h2o_thermochemistry.py}

\textbf{Типові поправки для \ce{H2O} при 298.15 K:}
\begin{itemize}
    \item Електронна енергія: $E_{\text{elec}} = -76.067$ Ha
    \item ZPE: +0.021 Ha (13.3 ккал/моль)
    \item Термічна поправка: +0.003 Ha (1.9 ккал/моль)
    \item Ентропія: $S = 45.1$ кал/(моль·K)
\end{itemize}


%% --------------------------------------------------------
\subsubsection{Термохімічний розрахунок для \ce{CH4}}
%% --------------------------------------------------------


\inputcode{ch4_thermochemistry.py}

\textbf{Результати для \ce{CH4} при $298.15$~K:}

\begin{center}
    \begin{tblr}{
        colspec = {X[l,3] X[r,2] X[l,2]},
        row{1} = {font=\bfseries},
        hline{1,2,Z} = {solid},
            }
        Величина              & Значення & Одиниці      \\
        ZPE                   & 28.03    & ккал/моль    \\
        $E_{\text{vib}}(T)$   & 0.10     & ккал/моль    \\
        $E_{\text{rot}}(T)$   & 0.89     & ккал/моль    \\
        $E_{\text{trans}}(T)$ & 0.89     & ккал/моль    \\
        $H(T) - H(0)$         & 2.48     & ккал/моль    \\
        $S(T)$                & 44.5     & кал/(моль·K) \\
        $C_V$                 & 6.0      & кал/(моль·K) \\
    \end{tblr}
\end{center}

\textit{Розрахунок: B3LYP/6-311+G(2d,p)}

\textbf{Порівняння з експериментом:}
\begin{itemize}
    \item ZPE (експ.): 27.8 ккал/моль --- відмінна згода
    \item $S_{298}$ (експ.): 44.5 кал/(моль·K) --- ідеальна згода
    \item Термохімічні поправки надійні для DFT
\end{itemize}

%% --------------------------------------------------------
\subsection{Ізотопні ефекти}
%% --------------------------------------------------------

Заміна ізотопу змінює частоти через зміну маси:
\[
    \frac{\omega_{\text{D}}}{\omega_{\text{H}}} \approx \sqrt{\frac{m_{\text{H}}}{m_{\text{D}}}} = \sqrt{\frac{1}{2}} \approx 0.707
\]

\subsubsection{Ізотопний ефект у \ce{H2O}/\ce{D2O}}

%\inputcode{h2o_d2o_isotope_effect.py}

\textbf{Порівняння \ce{H2O} та \ce{D2O}:}

\begin{center}
    \begin{tblr}{
        colspec = {X[l,3] X[r,2] X[r,2] X[r,2]},
        row{1} = {font=\bfseries},
        hline{1,2,Z} = {solid},
            }
        Мода                & \ce{H2O}    & \ce{D2O}    & Співвідношення \\
                            & (см$^{-1}$) & (см$^{-1}$) &                \\
        Симетричне валентне & 3657        & 2671        & 0.730          \\
        Деформаційне        & 1595        & 1178        & 0.738          \\
        Антисим. валентне   & 3756        & 2788        & 0.742          \\
    \end{tblr}
\end{center}

\textit{Експериментальні дані}

\textbf{ZPE ефект:}
\begin{itemize}
    \item ZPE(\ce{H2O}): 13.26 ккал/моль
    \item ZPE(\ce{D2O}): 9.66 ккал/моль
    \item $\Delta$ZPE: 3.60 ккал/моль
\end{itemize}

\textbf{Наслідки:}
\begin{itemize}
    \item Дейтеровані сполуки стабільніші (нижча ZPE)
    \item Кінетичний ізотопний ефект: $k_{\text{H}}/k_{\text{D}} \approx 7$ для розриву \ce{C-H}
    \item Використання в механістичних дослідженнях
\end{itemize}

%% --------------------------------------------------------
\subsection{Перехідні стани та уявні частоти}
%% --------------------------------------------------------

\subsubsection{Характеристика стаціонарних точок}

Гессіан дозволяє класифікувати стаціонарні точки:

\begin{center}
    \begin{tblr}{
        colspec = {X[l,2] X[c,2] X[l,3]},
        row{1} = {font=\bfseries},
        hline{1,2,Z} = {solid},
            }
        Тип                & Уявних частот & Характеристика             \\
        Мінімум            & 0             & Локальний мінімум          \\
        Перехідний стан    & 1             & Сідлова точка 1-го порядку \\
        Сідло 2-го порядку & 2             & Нестабільна структура      \\
    \end{tblr}
\end{center}

\textbf{Уявна частота:} $\omega^2 < 0 \Rightarrow \omega = i|\omega|$

У виводі PySCF позначається як негативна частота.

\subsubsection{Приклад: інверсія аміаку}

%\inputcode{nh3_inversion.py}

\textbf{Профіль енергії інверсії \ce{NH3}:}

\begin{center}
    \begin{tblr}{
        colspec = {X[l,2] X[r,2] X[r,2] X[c,2]},
        row{1} = {font=\bfseries},
        hline{1,2,Z} = {solid},
            }
        Структура                  & $E_{\text{rel}}$ & Уявних & Тип     \\
                                   & (ккал/моль)      & частот &         \\
        Пірамідальна (\ce{C_{3v}}) & 0.0              & 0      & Мінімум \\
        Планарна (\ce{D_{3h}})     & 5.8              & 1      & ПС      \\
        Пірамідальна (інша)        & 0.0              & 0      & Мінімум \\
    \end{tblr}
\end{center}

%---------------------------------------------------------
\begin{figure}[h!]\centering
%    \includegraphics[width=0.75\linewidth]{\currfiledir/nh3_inversion_profile.pdf}
    \caption{Профіль енергії інверсії \ce{NH3}. Планарна структура є
        перехідним станом з однією уявною частотою (зонтичний рух).}
    \label{pic:nh3_inversion}
\end{figure}
%---------------------------------------------------------

\textbf{Уявна мода в ПС:}
\begin{itemize}
    \item Частота: $\omega = i 1044$ см$^{-1}$ (або $-1044i$ в PySCF)
    \item Напрямок: зонтичний рух (umbrella mode)
    \item Зв'язує два еквівалентні мінімуми
    \item Квантове тунелювання → розщеплення ЯМР сигналу
\end{itemize}

%% --------------------------------------------------------
\subsubsection{Інтрінсична координата реакції (IRC)}
%% --------------------------------------------------------


Від перехідного стану можна простежити шлях реакції:
\begin{itemize}
    \item Рух вздовж уявної моди
    \item З'єднує реагенти та продукти
    \item Доводить, що ПС належить до шуканої реакції
\end{itemize}

%% --------------------------------------------------------
\subsection{Ангармонічні поправки}
%% --------------------------------------------------------

Гармонічне наближення добре для малих амплітуд, але:
\begin{itemize}
    \item Завищує частоти на 3--5\%
    \item Не описує обертонів та комбінаційних смуг
    \item Не враховує асиметрію потенціалу
\end{itemize}

\subsubsection{Ангармонічний потенціал}

Розклад потенціалу до кубічних та квартичних термів:
\[
    V = V_0 + \frac{1}{2}\sum_{ij} k_{ij} Q_i Q_j +
    \frac{1}{6}\sum_{ijk} k_{ijk} Q_i Q_j Q_k +
    \frac{1}{24}\sum_{ijkl} k_{ijkl} Q_i Q_j Q_k Q_l
\]

\textbf{Методи розрахунку:}
\begin{itemize}
    \item VPT2 (Vibrational Perturbation Theory 2-го порядку)
    \item VSCF (Vibrational Self-Consistent Field)
    \item VCI (Vibrational Configuration Interaction)
\end{itemize}

\subsubsection{Ангармонічні частоти \ce{H2O}}

\begin{center}
    \begin{tblr}{
        colspec = {X[c] X[r] X[r] X[r] X[r]},
        row{1} = {font=\bfseries},
        hline{1,2,Z} = {solid},
            }
        Мода    & Гарм.       & VPT2        & Масштаб.    & Експ.       \\
                & (см$^{-1}$) & (см$^{-1}$) & (см$^{-1}$) & (см$^{-1}$) \\
        $\nu_1$ & 3825        & 3707        & 3702        & 3657        \\
        $\nu_2$ & 1653        & 1618        & 1600        & 1595        \\
        $\nu_3$ & 3936        & 3814        & 3808        & 3756        \\
    \end{tblr}
\end{center}

\textit{B3LYP/aug-cc-pVTZ, масштабувальний фактор 0.968}

\textbf{Висновки:}
\begin{itemize}
    \item VPT2 суттєво покращує згоду з експериментом
    \item Масштабування простіше, але менш фізичне
    \item Для валентних коливань ангармонізм $\sim100-150$ см$^{-1}$
\end{itemize}

%% --------------------------------------------------------
\subsection{Практичні рекомендації}
%% --------------------------------------------------------

\subsubsection{Вибір методу та базису}

%\begin{center}
%\begin{tblr}{
%    colspec = {X[l,2] X[l,2] X[l,3]},
%    row{1} = {font=\bfseries},
%    hline{1,2,Z} = {solid},
%}
%Завдання & Рекомендація & Коментар \\
%Ідентифікація мінімуму & B3LYP/6-31G* & Швидко, надійно \\
%Точні частоти & B3LYP/6-311+G(2d,p) & Масштабувати 0.968 \\
%Термохімія & B3LYP/6-311++G(3df,3pd) & Для ZPE \\
%Великі системи & B3LYP/6-31+G* & Компроміс \\
%Benchmark & $\omega$B97X-D/aug-cc-pVTZ & Найкраща точність \\
%Ангармонізм & VPT2, малий базис & Дуже дорого \\


%% ========================================================
\section{Електронні переходи та УФ-видимі спектри}
%% ========================================================

%% --------------------------------------------------------
\subsection{Теорія збуджених станів}
%% --------------------------------------------------------

Для розрахунку електронних переходів використовуємо методи:
\begin{itemize}
    \item \textbf{CIS} (Configuration Interaction Singles) --- базовий
    \item \textbf{TDHF/TDDFT} (Time-Dependent HF/DFT) --- точніший
    \item \textbf{EOM-CCSD} (Equation-of-Motion CCSD) --- високоточний
\end{itemize}

Енергія збудження $n$-го стану:
\[
    \omega_n = E_n - E_0
\]

Сила осцилятора (інтенсивність):
\[
    f_n = \frac{2}{3}\omega_n |\langle\Psi_0|\hat{\boldsymbol{\mu}}|\Psi_n\rangle|^2
\]

%% --------------------------------------------------------
\subsection{TDDFT розрахунок для формальдегіду}
%% --------------------------------------------------------

Розглянемо молекулу формальдегіду \ce{H2CO} як приклад:

\inputcode{h2co_tddft.py}

\textbf{Аналіз збуджень для \ce{H2CO}:}
\begin{center}
    \begin{tabular}{lcccc}
        \toprule
        Перехід & Тип           & $\lambda$ (нм) & $f$   & Характер       \\
        \midrule
        $S_1$   & $n\to\pi^*$   & 355            & 0.001 & Заборонений    \\
        $S_2$   & $\pi\to\pi^*$ & 185            & 0.152 & Дозволений     \\
        $S_3$   & $n\to 3s$     & 172            & 0.021 & Рідбергівський \\
        \bottomrule
    \end{tabular}
\end{center}

\textbf{Фізична інтерпретація:}
\begin{itemize}
    \item $n\to\pi^*$: неподілена пара O $\to$ антизв'язуюча МО C=O
    \item Низька сила осцилятора через симетрію
    \item $\pi\to\pi^*$: основна смуга поглинання в УФ
    \item Енергії залежать від функціоналу DFT
\end{itemize}

%% --------------------------------------------------------
\subsection{Вибір функціоналу для TDDFT}
%% --------------------------------------------------------

Різні функціонали дають різну точність для збуджень:

\inputcode{formaldehyde_functional_comparison.py}

\textbf{Порівняння функціоналів (перехід $n\to\pi^*$):}
\begin{center}
    \begin{tabular}{lcc}
        \toprule
        Функціонал     & $\lambda$ (нм) & Похибка (нм) \\
        \midrule
        B3LYP          & 355            & +25          \\
        PBE0           & 342            & +12          \\
        CAM-B3LYP      & 335            & +5           \\
        $\omega$B97X-D & 332            & +2           \\
        \midrule
        Експеримент    & 330            & ---          \\
        \bottomrule
    \end{tabular}
\end{center}

\textbf{Рекомендації:}
\begin{itemize}
    \item Гібридні функціонали краще за чисті GGA
    \item Range-separated (CAM-B3LYP, $\omega$B97X-D) найточніші
    \item Для переносу заряду обов'язково range-separated
    \item B3LYP часто завищує довжини хвиль
\end{itemize}

%% --------------------------------------------------------
\subsection{Візуалізація спектру}
%% --------------------------------------------------------

Для побудови спектру використовуємо гаусові або лоренцеві контури:
\[
    \varepsilon(\lambda) = \sum_n f_n \cdot \frac{1}{\sigma\sqrt{2\pi}}
    \exp\left[-\frac{(\lambda - \lambda_n)^2}{2\sigma^2}\right]
\]

\inputcode{plot_uv_spectrum.py}

%---------------------------------------------------------
\begin{figure}[h!]\centering
    %\includegraphics[width=\linewidth]{\currfiledir/h2co_uv_spectrum.pdf}
    \caption{УФ-спектр формальдегіду, розрахований методом TDDFT/CAM-B3LYP/aug-cc-pVDZ.}
    \label{pic:h2co_uv_spectrum}
\end{figure}
%---------------------------------------------------------

\textbf{Параметри уширення:}
\begin{itemize}
    \item Типове $\sigma = 0.3$--$0.5$ eV для конденсованої фази
    \item $\sigma = 0.1$--$0.2$ eV для газової фази
    \item Експериментальне уширення враховує розподіл за $T$
\end{itemize}

%% --------------------------------------------------------
\subsection{Аналіз характеру переходів}
%% --------------------------------------------------------

TDDFT надає інформацію про орбіталі, задіяні у переході:

\inputcode{analyze_transitions.py}

\textbf{Приклад виводу для $S_1$ стану \ce{H2CO}:}
\begin{minted}{text}
Excited State 1: 3.492 eV (355 nm)  f=0.0012
    HOMO-1 -> LUMO     0.11 (1.2%)
    HOMO   -> LUMO     0.69 (47.6%)
    HOMO   -> LUMO+1   0.08 (0.6%)
\end{minted}

\textbf{Інтерпретація:}
\begin{itemize}
    \item Основний внесок: HOMO $\to$ LUMO (48\%)
    \item HOMO --- неподілена пара n(O)
    \item LUMO --- антизв'язуюча $\pi^*$(C=O)
    \item Тип переходу: $n\to\pi^*$
\end{itemize}
%%% ========================================================
\section{Електричні властивості молекул}
%% ========================================================

%% --------------------------------------------------------
\subsection{Дипольний момент}
%% --------------------------------------------------------

Дипольний момент --- найпростіша електрична властивість молекули,
яка характеризує розподіл електронної густини:
\[
\boldsymbol{\mu} = \sum_i q_i \mathbf{r}_i =
\langle\Psi|\hat{\boldsymbol{\mu}}|\Psi\rangle
\]

Для молекулярної системи:
\[
\boldsymbol{\mu} = \sum_A Z_A \mathbf{R}_A -
\int \rho(\mathbf{r}) \mathbf{r} \, d\mathbf{r}
\]

\textbf{Властивості дипольного моменту:}
\begin{itemize}
    \item Векторна величина (має напрямок)
    \item Вимірюється в Дебаях: 1 D $\approx$ 0.3934 ea₀
    \item Залежить від вибору початку координат для зарядженої системи
    \item Для нейтральної молекули не залежить від початку координат
\end{itemize}

\subsubsection{Розрахунок дипольного моменту}

\inputcode{h2o_dipole.py}

\textbf{Результати для H₂O:}
\begin{itemize}
    \item RHF/6-31G: $\mu \approx 2.30$ D
    \item RHF/aug-cc-pVDZ: $\mu \approx 2.18$ D
    \item B3LYP/aug-cc-pVDZ: $\mu \approx 1.97$ D
    \item Експеримент: $\mu = 1.855$ D
\end{itemize}

\textbf{Аналіз похибок:}
\begin{itemize}
    \item RHF завищує дипольний момент через недооцінку кореляції
    \item Малі базиси завищують $\mu$ (недостатня гнучкість)
    \item Дифузні функції (aug-) критично важливі
    \item DFT зазвичай ближче до експерименту
\end{itemize}

%% --------------------------------------------------------
\subsection{Поляризовність}
%% --------------------------------------------------------

Поляризовність $\boldsymbol{\alpha}$ описує відгук електронної густини
на зовнішнє електричне поле:
\[
\mu_i = \mu_i^0 + \sum_j \alpha_{ij} E_j +
\frac{1}{2}\sum_{jk} \beta_{ijk} E_j E_k + \ldots
\]

Статична поляризовність --- тензор другого рангу:
\[
\alpha_{ij} = -\left.\frac{\partial^2 E}{\partial E_i \partial E_j}\right|_{E=0}
\]

\subsubsection{Обчислення поляризовності}

\inputcode{h2o_polarizability.py}

\textbf{Тензор поляризовності H₂O (au³):}
\begin{center}
\begin{tabular}{lccc}
\toprule
Компонента & RHF/aug-cc-pVDZ & B3LYP/aug-cc-pVDZ & Експеримент \\
\midrule
$\alpha_{xx}$ & 8.85 & 9.12 & 9.63 \\
$\alpha_{yy}$ & 9.32 & 9.58 & 9.93 \\
$\alpha_{zz}$ & 8.46 & 8.71 & 9.12 \\
\midrule
$\bar{\alpha}$ & 8.88 & 9.14 & 9.56 \\
\bottomrule
\end{tabular}
\end{center}

де середня поляризовність: $\bar{\alpha} = (\alpha_{xx} + \alpha_{yy} + \alpha_{zz})/3$

\textbf{Анізотропія поляризовності:}
\[
\Delta\alpha = \frac{1}{\sqrt{2}}\sqrt{(\alpha_{xx}-\alpha_{yy})^2 +
(\alpha_{yy}-\alpha_{zz})^2 + (\alpha_{zz}-\alpha_{xx})^2}
\]

Анізотропія важлива для Раман-розсіювання.

%% --------------------------------------------------------
\subsection{Динамічна поляризовність}
%% --------------------------------------------------------

При частотно-залежному полі $\mathbf{E}(t) = \mathbf{E}_0 e^{-i\omega t}$
поляризовність стає функцією частоти:
\[
\alpha_{ij}(\omega) = \alpha_{ij}(-\omega; \omega)
\]

\textbf{Фізичний зміст:}
\begin{itemize}
    \item При $\omega = 0$: статична поляризовність
    \item При $\omega \to \omega_{\text{res}}$: резонансне підсилення
    \item Уявна частина: поглинання енергії
    \item Пов'язана з показником заломлення: $n^2 - 1 \propto \alpha(\omega)$
\end{itemize}

\inputcode{h2o_frequency_dependent_polarizability.py}

%---------------------------------------------------------
\begin{figure}[h!]\centering
\includegraphics[width=0.8\linewidth]{\currfiledir/h2o_dynamic_polarizability.pdf}
\caption{Частотна залежність поляризовності H₂O.}
\label{pic:h2o_dynamic_pol}
\end{figure}
%---------------------------------------------------------

\textbf{Спостереження:}
\begin{itemize}
    \item Монотонне зростання $\alpha(\omega)$ до резонансу
    \item Різка зміна поблизу електронних збуджень
    \item Анізотропія залежить від частоти
    \item Важливо для нелінійної оптики
\end{itemize}

%% --------------------------------------------------------
\subsection{Гіперполяризовності}
%% --------------------------------------------------------

Для сильних полів або нелінійної оптики важливі гіперполяризовності:
\[
\mu_i = \mu_i^0 + \alpha_{ij} E_j + \frac{1}{2}\beta_{ijk} E_j E_k +
\frac{1}{6}\gamma_{ijkl} E_j E_k E_l + \ldots
\]

\textbf{Фізичні процеси:}
\begin{itemize}
    \item $\beta$ (перша гіперполяризовність): генерація другої гармоніки,
          електрооптичний ефект Поккельса
    \item $\gamma$ (друга гіперполяризовність): генерація третьої гармоніки,
          ефект Керра
\end{itemize}

\subsubsection{Розрахунок першої гіперполяризовності}

\inputcode{para_nitroaniline_beta.py}

\textbf{Приклад: пара-нітроанілін (pNA)}

pNA --- класична молекула для нелінійної оптики:
\begin{itemize}
    \item Донорно-акцепторна система (NH₂ --- донор, NO₂ --- акцептор)
    \item Великий переніс заряду вздовж молекули
    \item Велика $\beta$ вздовж довгої осі
\end{itemize}

\textbf{Типові значення для pNA:}
\begin{center}
\begin{tabular}{lcc}
\toprule
Метод & $\beta_{zzz}$ (au) & $\beta_{\text{vec}}$ (au) \\
\midrule
RHF/6-31G          & 520  & 480 \\
B3LYP/6-31+G*      & 890  & 820 \\
CAM-B3LYP/aug-cc-pVDZ & 750 & 690 \\
\midrule
Експеримент (газ)  & ---  & $\sim$800 \\
\bottomrule
\end{tabular}
\end{center}

Векторна гіперполяризовність:
\[
\beta_{\text{vec}} = \sqrt{\beta_x^2 + \beta_y^2 + \beta_z^2}
\]
де $\beta_i = \sum_{jk} \beta_{ijk} \hat{\mu}_j \hat{\mu}_k$ проектується
на напрямок дипольного моменту.

%% --------------------------------------------------------
\subsection{Електричне поле та градієнт}
%% --------------------------------------------------------

Електричне поле в точці $\mathbf{r}$ від молекулярних зарядів:
\[
\mathbf{E}(\mathbf{r}) = -\nabla V(\mathbf{r})
\]

Градієнт електричного поля (тензор):
\[
\nabla_i E_j = \frac{\partial E_j}{\partial r_i}
\]

\textbf{Застосування:}
\begin{itemize}
    \item Взаємодія з квадрупольним моментом ядра (ЯКР спектроскопія)
    \item Розрахунок констант спін-спінової взаємодії в ЯМР
    \item Моделювання сольватації (електростатичний внесок)
\end{itemize}

\inputcode{h2o_electric_field_gradient.py}

\textbf{Електричний градієнт на ядрі O в H₂O:}
\begin{itemize}
    \item Тензор EFG має слід нуль (рівняння Лапласа)
    \item Головні компоненти: $V_{xx}, V_{yy}, V_{zz}$
    \item Параметр асиметрії: $\eta = (V_{xx} - V_{yy})/V_{zz}$
    \item Використовується для інтерпретації ЯКР спектрів
\end{itemize}

%% --------------------------------------------------------
\subsection{Електростатичний потенціал (ESP)}
%% --------------------------------------------------------

ESP показує взаємодію молекули з точковим позитивним зарядом:
\[
V(\mathbf{r}) = \sum_A \frac{Z_A}{|\mathbf{r} - \mathbf{R}_A|} -
\int \frac{\rho(\mathbf{r}')}{|\mathbf{r} - \mathbf{r}'|} d\mathbf{r}'
\]

\textbf{Застосування ESP:}
\begin{itemize}
    \item Передбачення реакційної здатності (нуклеофільні/електрофільні ділянки)
    \item Побудова силових полів (RESP, CHELPG заряди)
    \item Аналіз водневих зв'язків
    \item Вивчення нековалентних взаємодій
\end{itemize}

\inputcode{h2o_esp_analysis.py}

%---------------------------------------------------------
\begin{figure}[h!]\centering
\includegraphics[width=0.7\linewidth]{\currfiledir/h2o_esp_map.pdf}
\caption{Карта електростатичного потенціалу H₂O на поверхні
van der Waals (ізоденситна поверхня 0.001 au).}
\label{pic:h2o_esp}
\end{figure}
%---------------------------------------------------------

\textbf{Інтерпретація карти ESP для H₂O:}
\begin{itemize}
    \item \textcolor{red}{Червона область} (негативний ESP): біля атома O,
          область неподілених пар --- нуклеофільна ділянка
    \item \textcolor{blue}{Синя область} (позитивний ESP): біля атомів H,
          здатні утворювати водневі зв'язки як донори
    \item Величина потенціалу корелює з кислотністю/основністю
\end{itemize}

%% --------------------------------------------------------
\subsection{Парціальні атомні заряди}
%% --------------------------------------------------------

Атомні заряди --- неспостережувані величини, але корисні для інтерпретації.
Існує багато схем розподілу електронної густини по атомах.

\subsubsection{Метод Малікена (Mulliken)}

Найпростіший метод, заснований на розкладанні матриці густини:
\[
q_A^{\text{Mulliken}} = Z_A - \sum_{\mu \in A} (PS)_{\mu\mu}
\]

\textbf{Недоліки:}
\begin{itemize}
    \item Сильно залежить від базису
    \item Нестабільний для великих базисів
    \item Неадекватний для дифузних функцій
\end{itemize}

\subsubsection{Метод Льовдіна (Löwdin)}

Використовує симетричну ортогоналізацію:
\[
q_A^{\text{Löwdin}} = Z_A - \sum_{\mu \in A} (PS^{1/2})_{\mu\mu}
\]

Менш чутливий до базису, ніж Малікен.

\subsubsection{Методи на основі ESP}

\textbf{CHELPG, RESP:} Заряди підбираються для відтворення ESP
на сітці точок навколо молекули:
\[
\min_{\{q_A\}} \sum_i \left[V_i^{\text{exact}} -
\sum_A \frac{q_A}{|\mathbf{r}_i - \mathbf{R}_A|}\right]^2
\]

\textbf{Переваги:}
\begin{itemize}
    \item Фізично обґрунтовані (відтворюють спостережувану величину)
    \item Слабко залежать від базису
    \item Добре працюють для силових полів
\end{itemize}

\inputcode{h2o_atomic_charges_comparison.py}

\textbf{Порівняння зарядів для H₂O:}
\begin{center}
\begin{tabular}{lccc}
\toprule
Метод & $q(\text{O})$ & $q(\text{H})$ & Сума \\
\midrule
Mulliken      & $-0.82$ & $+0.41$ & 0.00 \\
Löwdin        & $-0.68$ & $+0.34$ & 0.00 \\
CHELPG        & $-0.95$ & $+0.48$ & 0.00 \\
RESP          & $-0.93$ & $+0.47$ & 0.00 \\
\midrule
Bader (AIM)   & $-1.12$ & $+0.56$ & 0.00 \\
\bottomrule
\end{tabular}
\end{center}

\textbf{Висновки:}
\begin{itemize}
    \item ESP-методи дають більші заряди (краще для електростатики)
    \item Mulliken сильно залежить від базису
    \item Для силових полів: RESP або CHELPG
    \item Для аналізу хімічного зв'язку: Bader (AIM) або NBO
\end{itemize}

%% --------------------------------------------------------
\subsection{Приклад: вплив розчинника на електричні властивості}
%% --------------------------------------------------------

Розчинник суттєво змінює дипольний момент та поляризовність.
Використовуємо континуальні моделі сольватації (PCM, COSMO).

\inputcode{acetone_solvent_effect.py}

\textbf{Результати для ацетону (CH₃)₂CO:}
\begin{center}
\begin{tabular}{lccc}
\toprule
Розчинник & $\varepsilon$ & $\mu$ (D) & $\bar{\alpha}$ (au³) \\
\midrule
Газ              & 1.0   & 2.88 & 42.5 \\
Гексан           & 1.88  & 2.95 & 43.1 \\
Хлороформ        & 4.71  & 3.08 & 44.3 \\
Етанол           & 24.85 & 3.42 & 46.8 \\
Вода             & 78.36 & 3.56 & 48.2 \\
\midrule
Експ. (газ)      & ---   & 2.91 & --- \\
Експ. (вода)     & ---   & 3.50 & --- \\
\bottomrule
\end{tabular}
\end{center}

\textbf{Спостереження:}
\begin{itemize}
    \item $\mu$ зростає з діелектричною проникністю $\varepsilon$
    \item Поляризовність також збільшується (поляризація розчинником)
    \item Ефект найсильніший для полярних розчинників
    \item Важливо для моделювання реакцій у розчині
\end{itemize}
%%% ========================================================
\section{Магнітні властивості молекул}
%% ========================================================

%% --------------------------------------------------------
\subsection{Магнітна сприйнятливість}
%% --------------------------------------------------------

Магнітна сприйнятливість $\chi$ описує відгук молекули на магнітне поле:
\[
\mathbf{M} = \chi \mathbf{H}
\]

де $\mathbf{M}$ --- намагніченість, $\mathbf{H}$ --- напруженість магнітного поля.

\textbf{Типи магнетизму:}
\begin{itemize}
    \item \textbf{Діамагнетизм} ($\chi < 0$): всі електрони спарені,
          молекула відштовхується від магнітного поля
    \item \textbf{Парамагнетизм} ($\chi > 0$): непарені електрони,
          молекула притягується до магнітного поля
\end{itemize}

Для замкненої оболонки (діамагнетизм):
\[
\chi_{ij} = -\frac{1}{2c^2}\left\langle\Psi\left|
\sum_k (r_k^2 \delta_{ij} - r_{k,i} r_{k,j})\right|\Psi\right\rangle
\]

%% --------------------------------------------------------
\subsection{Розрахунок магнітної сприйнятливості}
%% --------------------------------------------------------

\inputcode{h2o_magnetic_susceptibility.py}

\textbf{Результати для \ce{H2O} (діамагнетик):}
\begin{center}
\begin{tabular}{lcc}
\toprule
Компонента & Значення (ppm·cgs) & Значення (10$^{-6}$ cm³/mol) \\
\midrule
$\chi_{xx}$ & $-8.2$ & $-13.1$ \\
$\chi_{yy}$ & $-10.5$ & $-16.7$ \\
$\chi_{zz}$ & $-9.1$ & $-14.5$ \\
\midrule
$\bar{\chi}$ & $-9.3$ & $-14.8$ \\
\bottomrule
\end{tabular}
\end{center}

Експериментальне значення: $\bar{\chi} \approx -13.0 \times 10^{-6}$ cm³/mol

\textbf{Анізотропія магнітної сприйнятливості:}
\[
\Delta\chi = \chi_{\parallel} - \chi_{\perp}
\]
важлива для орієнтації молекул у магнітному полі (ЯМР твердого тіла).

%% --------------------------------------------------------
\subsection{Хімічні зсуви ЯМР}
%% --------------------------------------------------------

Хімічний зсув ядра --- це зміна резонансної частоти через екранування
електронною оболонкою:
\[
\delta = \frac{\nu - \nu_{\text{ref}}}{\nu_{\text{ref}}} \times 10^6 \text{ (ppm)}
\]

Тензор екранування $\boldsymbol{\sigma}$ пов'язаний з індукованим полем:
\[
\mathbf{B}_{\text{ind}} = -\boldsymbol{\sigma} \cdot \mathbf{B}_0
\]

