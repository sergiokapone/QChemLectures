%% ========================================================
\section{Магнітні властивості атомів та молекул}
%% ========================================================

Магнітні властивості молекул надають унікальну інформацію про електронну
структуру, розподіл спінової густини, та природу хімічного зв'язку.
Експериментальні методи --- ЯМР, ЕПР, магнетометрія --- є одними з
найпотужніших інструментів структурного аналізу.

%% --------------------------------------------------------
\subsection{Магнітна сприйнятливість}
%% --------------------------------------------------------

Магнітна сприйнятливість $\boldsymbol{\chi}$ описує відгук молекули
на зовнішнє магнітне поле:
\[
\mathbf{M} = \boldsymbol{\chi} \mathbf{B}
\]

де $\mathbf{M}$ --- індукована намагніченість, $\mathbf{B}$ --- магнітне поле.

\textbf{Компоненти магнітної сприйнятливості:}
\begin{itemize}
    \item \textbf{Діамагнітний внесок:} завжди негативний, пов'язаний
          з індукованими струмами (закон Ленца)
    \item \textbf{Парамагнітний внесок:} позитивний, виникає для систем
          з неспареними електронами
    \item Ізотропна сприйнятливість: $\chi_{\text{iso}} = (\chi_{xx} + \chi_{yy} + \chi_{zz})/3$
    \item Анізотропія: $\Delta\chi = \chi_{zz} - (\chi_{xx} + \chi_{yy})/2$
\end{itemize}

Для замкнених оболонок (синглетних систем):
\[
\chi_{ij} = -\frac{e^2}{4m_e c^2} \langle\Psi_0|
\sum_k (r_k^2 \delta_{ij} - r_{ki} r_{kj})|\Psi_0\rangle
\]

\subsubsection{Розрахунок магнітної сприйнятливості бензену}

Бензен --- класичний приклад ароматичної молекули з сильною
магнітною анізотропією через кільцеві струми $\pi$-електронів.

\inputcode{benzene_susceptibility.py}

\textbf{Результати для \ch{C6H6}:}

\begin{center}
\begin{tblr}{
    colspec = {X[l,2] X[r] X[r] X[r] X[r]},
    row{1} = {font=\bfseries},
    hline{1,2,Z} = {solid},
}
Метод & $\chi_{\perp}$ & $\chi_{\parallel}$ & $\chi_{\text{iso}}$ & $\Delta\chi$ \\
& (ppm) & (ppm) & (ppm) & (ppm) \\
RHF/6-31G* & $-48.2$ & $-88.5$ & $-61.6$ & $-40.3$ \\
B3LYP/6-311+G(2d,p) & $-52.8$ & $-96.3$ & $-67.3$ & $-43.5$ \\
CCSD(T)/aug-cc-pVDZ & $-54.1$ & $-98.7$ & $-69.0$ & $-44.6$ \\
Експеримент & $-54.8$ & $-103.0$ & $-70.9$ & $-48.2$ \\
\end{tblr}
\end{center}

де $\chi_{\perp}$ --- компонента в площині кільця,
$\chi_{\parallel}$ --- перпендикулярна компонента.

\textbf{Анізотропія як міра ароматичності:}
\[
\Delta\chi = \chi_{\parallel} - \chi_{\perp} \approx -48 \text{ ppm cgs}
\]

Велика негативна анізотропія --- характерна ознака ароматичності
(діамагнітне екранування кільцевими струмами).

%% --------------------------------------------------------
\subsection{Константи магнітного екранування ЯМР}
%% --------------------------------------------------------

Константа магнітного екранування ядра описує зміщення резонансної
частоти в ЯМР через електронне оточення:
\[
\mathbf{B}_{\text{eff}} = (1 - \boldsymbol{\sigma}) \mathbf{B}_0
\]

Тензор екранування $\boldsymbol{\sigma}$ обчислюється через похідну
від енергії:
\[
\sigma_{ij} = -\left.\frac{\partial^2 E}{\partial B_i \partial \mu_j^N}
\right|_{B=0}
\]

\textbf{Фізичні внески:}
\begin{itemize}
    \item \textbf{Діамагнітний:} локальні електрони навколо ядра
    \item \textbf{Парамагнітний:} змішування з збудженими станами
    \item Анізотропія екранування важлива для твердотільного ЯМР
\end{itemize}

\subsubsection{ЯМР спектр молекули води}

\inputcode{h2o_nmr_shielding.py}

\textbf{Результати для \ch{H2O}:}

\begin{center}
\begin{tblr}{
    colspec = {X[c] X[r] X[r] X[r] X[r]},
    row{1} = {font=\bfseries},
    hline{1,2,Z} = {solid},
}
Ядро & RHF & B3LYP & CCSD & Експ. \\
& (ppm) & (ppm) & (ppm) & (ppm) \\
$^{17}$O & 328.4 & 320.5 & 318.2 & 315.0 \\
$^{1}$H  & 30.8  & 30.2  & 30.6  & 30.1  \\
\end{tblr}
\end{center}

\textit{Константи екранування в ppm відносно голого ядра.}

\textbf{Хімічний зсув:} зазвичай вимірюють відносно стандарту (TMS для $^1$H):
\[
\delta = \frac{\sigma_{\text{ref}} - \sigma_{\text{sample}}}{\sigma_{\text{ref}}} \times 10^6
\]

\textbf{Механізми екранування:}
\begin{enumerate}
    \item \textbf{Діамагнітний внесок $\sigma^d$:}
    \begin{itemize}
        \item Індуковані струми навколо ядра
        \item Завжди позитивний (екранування)
        \item Домінує для внутрішніх електронів
    \end{itemize}

    \item \textbf{Парамагнітний внесок $\sigma^p$:}
    \begin{itemize}
        \item Змішування з збудженими станами
        \item Завжди негативний (деекранування)
        \item $\sigma^p \propto 1/\Delta E$ (чутливий до збуджень)
    \end{itemize}
\end{enumerate}

%% --------------------------------------------------------
\subsection{Константи спін-спінової взаємодії (J-константи)}
%% --------------------------------------------------------

J-константи описують непряму взаємодію ядерних спінів через електрони:
\[
H_J = \sum_{A<B} \mathbf{I}_A \cdot \mathbf{J}_{AB} \cdot \mathbf{I}_B
\]

\textbf{Механізми взаємодії:}
\begin{itemize}
    \item \textbf{Контактний внесок Фермі (FC):} домінує для $^1$J
    \item \textbf{Спін-дипольний (SD):} важливий для $\pi$-систем
    \item \textbf{Парамагнітний спін-орбітальний (PSO)}
    \item \textbf{Діамагнітний спін-орбітальний (DSO)}
\end{itemize}

\subsubsection{J-константи в молекулі етану}

\inputcode{ethane_j_coupling.py}

\textbf{Результати для \ch{C2H6}:}

\begin{center}
\begin{tblr}{
    colspec = {X[l,2] X[r] X[r] X[r]},
    row{1} = {font=\bfseries},
    hline{1,2,Z} = {solid},
}
Зв'язок & B3LYP/pcJ-2 & CCSD/pcJ-2 & Експеримент \\
& (Гц) & (Гц) & (Гц) \\
$^1J$(\ch{C-C})   & 32.8 & 34.5 & 34.6 \\
$^1J$(\ch{C-H})   & 126.4 & 125.2 & 125.3 \\
$^2J$(\ch{H-C-H}) & $-2.8$ & $-2.5$ & $-2.4$ \\
$^3J$(\ch{H-C-C-H}) & 7.9 & 7.8 & 8.0 \\
\end{tblr}
\end{center}

\textbf{Залежність $^3J$ від двогранного кута (правило Карплуса):}
\[
^3J_{HH}(\phi) = A \cos^2\phi + B \cos\phi + C
\]

Типово: $A \approx 7$, $B \approx -1$, $C \approx 5$ Гц для $^3J$(\ch{H-C-C-H}).

%---------------------------------------------------------
\begin{figure}[h!]\centering
\includegraphics[width=0.75\linewidth]{\currfiledir/karplus_curve.pdf}
\caption{Залежність $^3J$(H-C-C-H) від двогранного кута (правило Карплуса).
Використовується для визначення конформації молекул з ЯМР спектрів.}
\label{pic:karplus}
\end{figure}
%---------------------------------------------------------

%% --------------------------------------------------------
\subsection{g-тензор для радикалів}
%% --------------------------------------------------------

Для систем з неспареними електронами (радикалів) g-тензор описує
взаємодію електронного спіну з магнітним полем:
\[
H = \mu_B \mathbf{B}^T \cdot \mathbf{g} \cdot \mathbf{S}
\]

g-тензор відхиляється від $g_e = 2.0023$ через спін-орбітальну взаємодію:
\[
\Delta g_{ij} = g_{ij} - g_e \delta_{ij}
\]

\subsubsection{Розрахунок g-тензора для радикала OH}

\inputcode{oh_radical_g_tensor.py}

\textbf{Результати для OH ($^2\Pi$):}

\begin{center}
\begin{tblr}{
    colspec = {X[c] X[r] X[r] X[r]},
    row{1} = {font=\bfseries},
    hline{1,2,Z} = {solid},
}
Компонента & B3LYP/EPR-III & CCSD/aug-cc-pVTZ & Експеримент \\
$g_{xx}$ & 2.0091 & 2.0095 & 2.0099 \\
$g_{yy}$ & 2.0091 & 2.0095 & 2.0099 \\
$g_{zz}$ & 2.0023 & 2.0024 & 2.0026 \\
$g_{\text{iso}}$ & 2.0068 & 2.0071 & 2.0075 \\
\end{tblr}
\end{center}

\textbf{Інтерпретація:}
\begin{itemize}
    \item $g_{zz} \approx g_e$ вздовж осі \ch{O-H} (мінімальна СО-взаємодія)
    \item $g_{xx}, g_{yy} > g_e$ перпендикулярно (СО-змішування з $\pi^*$)
    \item Анізотропія $\Delta g \approx 0.007$ типова для легких атомів
\end{itemize}

\textbf{Механізм g-зсуву:}

Відхилення від $g_e$ виникає через спін-орбітальну взаємодію:
\[
\Delta g \approx \frac{\lambda^2}{\Delta E}
\]

де $\lambda$ --- константа спін-орбітального зв'язку, $\Delta E$ --- енергія збудження.

%% --------------------------------------------------------
\subsection{Константи надтонкої взаємодії (HFC)}
%% --------------------------------------------------------

HFC описує взаємодію неспареного електрона з ядерними спінами:
\[
H_{\text{HFC}} = \sum_A \mathbf{S} \cdot \mathbf{A}_A \cdot \mathbf{I}_A
\]

\textbf{Ізотропна константа (a-тензор Фермі):}
\[
a_{\text{iso}} = \frac{8\pi}{3} g_e \mu_B g_N \mu_N \rho_{\text{spin}}(\mathbf{R}_A)
\]

де $\rho_{\text{spin}}(\mathbf{R}_A)$ --- спінова густина на ядрі $A$.

\textbf{Анізотропна частина:} дипольна взаємодія електрон-ядро.

\subsubsection{HFC константи для радикала метилу}

\inputcode{methyl_radical_hfc.py}

\textbf{Результати для \ch{CH3} ($^2A_2''$):}

\begin{center}
\begin{tblr}{
    colspec = {X[c] X[r] X[r] X[r]},
    row{1} = {font=\bfseries},
    hline{1,2,Z} = {solid},
}
Ядро & B3LYP/EPR-III & CCSD(T)/EPR-III & Експеримент \\
& (Гаус) & (Гаус) & (Гаус) \\
$^{13}$C ($a_{\text{iso}}$) & $-28.2$ & $-27.8$ & $-27.0$ \\
$^{1}$H ($a_{\text{iso}}$)  & $-23.1$ & $-22.8$ & $-23.0$ \\
\end{tblr}
\end{center}

\textit{Негативні значення --- поляризаційний механізм.}

\textbf{Механізм спінової поляризації:}
\begin{itemize}
    \item Неспарений електрон у $p_z$-орбіталі вуглецю
    \item Прямої спінової густини на H немає ($\rho_\alpha(H) \approx 0$)
    \item Спінова поляризація $\sigma$(\ch{C-H}) зв'язків створює негативну густину
    \item $a_{\text{iso}}$(H) $< 0$ --- характерна ознака $\pi$-радикалів
\end{itemize}

%---------------------------------------------------------
\begin{figure}[h!]\centering
\includegraphics[width=0.6\linewidth]{\currfiledir/spin_polarization_scheme.pdf}
\caption{Схема спінової поляризації в метильному радикалі.
Неспарений $\alpha$-електрон у $p_z$ орбіталі поляризує $\sigma$-зв'язки,
створюючи надлишок $\beta$-спіну на атомах водню.}
\label{pic:spin_polarization}
\end{figure}
%---------------------------------------------------------

%% --------------------------------------------------------
\subsection{Спінове забруднення та нестабільність UHF}
%% --------------------------------------------------------

Для систем з відкритими оболонками UHF часто дає спінове забруднення:
\[
\langle S^2 \rangle = S(S+1) + \Delta
\]

де $\Delta > 0$ --- міра забруднення вищими спіновими станами.

\textbf{Проблеми спінового забруднення:}
\begin{itemize}
    \item Завищення енергії дисоціації
    \item Неправильна спінова густина
    \item Некоректні магнітні властивості
\end{itemize}

\textbf{Рішення:}
\begin{itemize}
    \item ROHF (обмежений відкрита оболонка)
    \item Проекційні методи (PMP2)
    \item DFT з обміннокореляційними функціоналами
\end{itemize}

\subsubsection{Приклад: молекула \ch{O2}}

\inputcode{o2_spin_contamination.py}

\textbf{Основний стан \ch{O2}: $^3\Sigma_g^-$ (триплет)}

\begin{center}
\begin{tblr}{
    colspec = {X[l,2] X[r] X[r] X[r] X[r]},
    row{1} = {font=\bfseries},
    hline{1,2,Z} = {solid},
}
Метод & $\langle S^2 \rangle$ & $\Delta$ & $E$ (Hartree) & $r_e$ (\AA) \\
UHF/6-311G*         & 2.17 & 0.17 & $-149.618$ & 1.165 \\
ROHF/6-311G*        & 2.00 & 0.00 & $-149.614$ & 1.168 \\
UB3LYP/6-311G*      & 2.03 & 0.03 & $-150.327$ & 1.208 \\
UCCSD(T)/cc-pVTZ    & 2.00 & 0.00 & $-150.168$ & 1.207 \\
Експеримент         & 2.00 & --- & --- & 1.208 \\
\end{tblr}
\end{center}

\textbf{Спостереження:}
\begin{itemize}
    \item UHF має значне спінове забруднення
    \item ROHF вільний від забруднення, але менш точний для енергії
    \item UB3LYP дає мале забруднення та гарну геометрію
    \item CCSD(T) --- "золотий стандарт" для багатоелектронних систем
\end{itemize}

%% --------------------------------------------------------
\subsection{Магнітооптична активність}
%% --------------------------------------------------------

\subsubsection{Обертання площини поляризації (ORD)}

Кут обертання на одиниці довжини при частоті $\omega$:
\[
[\alpha]_\omega = \frac{N}{c} \text{Im}\{\text{Tr}(\mathbf{G}'(\omega))\}
\]

де $\mathbf{G}'$ --- тензор оптичного обертання.

\subsubsection{Круговий дихроїзм (CD)}

Різниця поглинання лівої та правої циркулярно поляризованого світла:
\[
\Delta\varepsilon(\omega) = \varepsilon_L(\omega) - \varepsilon_R(\omega)
\]

Ротаційна сила переходу $i \to f$:
\[
R_{if} = \text{Im}\{\langle i|\hat{\boldsymbol{\mu}}|f\rangle \cdot
\langle f|\hat{\mathbf{m}}|i\rangle\}
\]

де $\hat{\mathbf{m}}$ --- оператор магнітного дипольного моменту.

\subsubsection{Приклад: (R)-метилоксиран}

\inputcode{methyloxirane_ord_cd.py}

\textbf{(R)-метилоксиран (пропіленоксид)} --- класична хіральна молекула
для демонстрації ORD та CD.

%---------------------------------------------------------
\begin{figure}[h!]\centering
\includegraphics[width=0.75\linewidth]{\currfiledir/methyloxirane_cd_spectrum.pdf}
\caption{Спектр кругового дихроїзму (R)-метилоксирану.
Розрахунок: TD-B3LYP/aug-cc-pVDZ.}
\label{pic:methyloxirane_cd}
\end{figure}
%---------------------------------------------------------

\textbf{Характеристики спектру CD:}

\begin{center}
\begin{tblr}{
    colspec = {X[c] X[r] X[r] X[l,2]},
    row{1} = {font=\bfseries},
    hline{1,2,Z} = {solid},
}
Перехід & $\lambda$ (нм) & $\Delta\varepsilon$ & Характер \\
& & (M$^{-1}$см$^{-1}$) & \\
1 & 190 & $-3.2$ & $n \to \sigma^*$ (O) \\
2 & 155 & $+2.4$ & $\sigma \to \sigma^*$ \\
3 & 145 & $-1.1$ & Rydberg \\
\end{tblr}
\end{center}

\textbf{Інтерпретація:}
\begin{itemize}
    \item Знак CD визначається абсолютною конфігурацією
    \item (S)-енантіомер дасть протилежні знаки
    \item Інтенсивність пропорційна ротаційній силі $R_{if}$
    \item Cotton ефект: зміна знаку ORD поблизу смуги поглинання
\end{itemize}

%% --------------------------------------------------------
\subsection{Магнітний циркулярний дихроїзм (MCD)}
%% --------------------------------------------------------

MCD --- поглинання в магнітному полі залежить від циркулярної поляризації:
\[
\Delta A = A_L - A_R = A_1 B + \frac{B}{kT}\left(A_2 + \frac{A_3}{kT}\right)
\]

\textbf{Терми MCD:}
\begin{itemize}
    \item \textbf{A-терм:} виникає для вироджених станів (зняття виродження)
    \item \textbf{B-терм:} змішування станів магнітним полем
    \item \textbf{C-терм:} температурно-залежний, для парамагнітних систем
\end{itemize}

\textbf{Застосування:}
\begin{itemize}
    \item Визначення симетрії електронних станів
    \item Вивчення металопротеїнів та координаційних комплексів
    \item Розрізнення близьких за енергією переходів
\end{itemize}

%% --------------------------------------------------------
\subsection{Порівняння парамагнітних молекул: NO та \ch{O2}}
%% --------------------------------------------------------

\inputcode{no_o2_magnetic_comparison.py}

\textbf{NO ($^2\Pi$) та \ch{O2} ($^3\Sigma_g^-$):}

\begin{center}
\begin{tblr}{
    colspec = {X[l,2] X[c] X[r] X[r] X[r]},
    row{1} = {font=\bfseries},
    hline{1,2,Z} = {solid},
}
Молекула & Стан & $\langle S^2 \rangle$ & $\mu$ ($\mu_B$) & $g_{\text{iso}}$ \\
NO       & $^2\Pi_{1/2}$ & 0.75 & 1.0 & 2.008 \\
\ch{O2}  & $^3\Sigma_g^-$ & 2.00 & 2.0 & 2.004 \\
\end{tblr}
\end{center}

\textbf{Експериментальні властивості:}
\begin{itemize}
    \item \textbf{NO:} один неспарений електрон, слабкий парамагнетик
    \item \textbf{\ch{O2}:} два неспарених електрони, сильний парамагнетик
          (рідкий \ch{O2} притягується магнітом!)
    \item Обидві молекули мають орбітальний кутовий момент
    \item g-тензор анізотропний через спін-орбітальну взаємодію
\end{itemize}

\textbf{Діаграма молекулярних орбіталей:}

\begin{center}
\begin{minipage}{0.45\textwidth}
\centering
\textbf{NO (11 електронів)}
\begin{verbatim}
π*  ↑─     ← SOMO
π   ↑↓ ↑↓
σ   ↑↓
\end{verbatim}
Дублет, $S = 1/2$
\end{minipage}
\hfill
\begin{minipage}{0.45\textwidth}
\centering
\textbf{\ch{O2} (12 електронів)}
\begin{verbatim}
π*  ↑─ ↑─  ← дві SOMO
π   ↑↓ ↑↓
σ   ↑↓
\end{verbatim}
Триплет, $S = 1$
\end{minipage}
\end{center}

%% --------------------------------------------------------
\subsection{Методологічні рекомендації}
%% --------------------------------------------------------

\subsubsection{Вибір методу та базису}

\begin{center}
\begin{tblr}{
    colspec = {X[l,2] X[l,2] X[l,3]},
    row{1} = {font=\bfseries},
    hline{1,2,Z} = {solid},
}
Властивість & Рекомендація & Коментар \\
Магн. сприйнятливість & B3LYP/aug-cc-pVDZ & GIAO обов'язково \\
ЯМР екранування & pcS-2, aug-cc-pVTZ & Спеціальні базиси \\
J-константи & pcJ-2, pcJ-3 & Для контактного Фермі \\
g-тензор & EPR-II, EPR-III & Tight s-функції \\
HFC константи & EPR-III & Густина на ядрі \\
ORD/CD & aug-cc-pVDZ & TD-DFT, дифузні \\
\end{tblr}
\end{center}

\subsubsection{Типові проблеми та рішення}

\begin{center}
\begin{tblr}{
    colspec = {X[l,2] X[l,3]},
    row{1} = {font=\bfseries},
    hline{1,2,Z} = {solid},
}
Проблема & Рішення \\
Залежність від gauge origin & Використовуйте GIAO (автоматично в PySCF) \\
Спінове забруднення UHF & Перевіряйте $\langle S^2 \rangle$, використ. ROHF або DFT \\
Неправильний g-тензор & Перевірте збіжність SCF, використ. EPR базиси \\
Занадто малі HFC & Потрібні tight s-функції (EPR-III) \\
Неправильний знак CD & Перевірте абсолютну конфігурацію \\
\end{tblr}
\end{center}

\subsubsection{Контрольний чек-лист}

Перед розрахунком магнітних властивостей:

\begin{itemize}
    \item[$\square$] Геометрія оптимізована?
    \item[$\square$] Для радикалів: перевірили $\langle S^2 \rangle < S(S+1) + 0.1$?
    \item[$\square$] Для ЯМР: використовується GIAO метод?
    \item[$\square$] Базис має дифузні функції (aug-)?
    \item[$\square$] Для ЕПР: використовується спеціалізований базис (EPR-II/III)?
    \item[$\square$] SCF збігся? (перевірте \texttt{mf.converged})
    \item[$\square$] Для хіральних молекул: правильна абсолютна конфігурація?
\end{itemize}

%% --------------------------------------------------------
\subsection{Атомні магнітні властивості}
%% --------------------------------------------------------

\subsubsection{Магнітні моменти атомів}

Для атомів з неспареними електронами магнітний момент визначається:
\[
\mu = g_J \sqrt{J(J+1)} \mu_B
\]

де $J$ --- повний кутовий момент, $g_J$ --- фактор Ланде:
\[
g_J = 1 + \frac{J(J+1) + S(S+1) - L(L+1)}{2J(J+1)}
\]

\textbf{Приклади:}

\begin{center}
\begin{tblr}{
    colspec = {X[c] X[c] X[c] X[c] X[r] X[r]},
    row{1} = {font=\bfseries},
    hline{1,2,Z} = {solid},
}
Атом & Конфігурація & Терм & $g_J$ & $\mu_{\text{теор}}$ & $\mu_{\text{експ}}$ \\
& & & & ($\mu_B$) & ($\mu_B$) \\
H & $1s^1$ & $^2S_{1/2}$ & 2.00 & 1.73 & 1.73 \\
C & $2p^2$ & $^3P_0$ & --- & 0 & --- \\
O & $2p^4$ & $^3P_2$ & 1.50 & 3.46 & 3.39 \\
F & $2p^5$ & $^2P_{3/2}$ & 1.33 & 2.45 & 2.42 \\
\end{tblr}
\end{center}

\subsubsection{Розрахунок для атома кисню}

\inputcode{oxygen_atom_magnetic.py}

\textbf{Результати для атома O ($^3P$):}

Атом кисню в основному стані має конфігурацію $1s^2 2s^2 2p^4$ з термом $^3P$.
Згідно правил Гунда:
\begin{itemize}
    \item $S = 1$ (два неспарених електрони)
    \item $L = 1$ ($p$-орбіталі)
    \item $J = 2$ (менше напівзаповнена, $J = L + S$)
\end{itemize}

%% --------------------------------------------------------
\subsection{Lanthanide та Actinide комплекси}
%% --------------------------------------------------------

Для f-елементів спін-орбітальна взаємодія дуже сильна, і магнітні
властивості визначаються $J$-мультиплетами.

\textbf{Ефективний магнітний момент:}
\[
\mu_{\text{eff}} = g_J \sqrt{J(J+1)} \mu_B
\]

\textbf{Приклади лантаноїдів:}

\begin{center}
\begin{tblr}{
    colspec = {X[l] X[c] X[c] X[c] X[r] X[r]},
    row{1} = {font=\bfseries},
    hline{1,2,Z} = {solid},
}
Іон & Конфігурація & Терм & $g_J$ & $\mu_{\text{теор}}$ & $\mu_{\text{експ}}$ \\
& & & & ($\mu_B$) & ($\mu_B$) \\
\ch{Ce^{3+}} & $4f^1$ & $^2F_{5/2}$ & 0.857 & 2.54 & 2.3--2.5 \\
\ch{Nd^{3+}} & $4f^3$ & $^4I_{9/2}$ & 0.727 & 3.62 & 3.5--3.6 \\
\ch{Gd^{3+}} & $4f^7$ & $^8S_{7/2}$ & 2.000 & 7.94 & 7.9--8.0 \\
\ch{Dy^{3+}} & $4f^9$ & $^6H_{15/2}$ & 1.333 & 10.63 & 10.4--10.6 \\
\end{tblr}
\end{center}

\textbf{Особливість \ch{Gd^{3+}}:}
\begin{itemize}
    \item Половинне заповнення $4f^7$ ($^8S$)
    \item $L = 0$ (повністю симетричне заповнення)
    \item $J = S = 7/2$, $g_J = 2$ (як для вільного електрона)
    \item Найбільший магнітний момент серед лантаноїдів
    \item Використовується в МРТ контрастних агентах
\end{itemize}

%% --------------------------------------------------------
\subsection{Практичні приклади}
%% --------------------------------------------------------

\subsubsection{Визначення структури з ЯМР}

\textbf{Задача:} розрізнити ізомери етанолу та диметилового ефіру.

Обидві молекули мають формулу \ch{C2H6O}, але різну структуру:
\begin{itemize}
    \item Етанол: \ch{CH3-CH2-OH}
    \item Диметиловий ефір: \ch{CH3-O-CH3}
\end{itemize}

\textbf{ЯМР $^1$H спектр:}

\begin{center}
\begin{tblr}{
    colspec = {X[l,2] X[c] X[r] X[l,2]},
    row{1} = {font=\bfseries},
    hline{1,2,Z} = {solid},
}
Молекула & Група & $\delta$ (ppm) & Мультиплетність \\
Етанол & \ch{CH3} & 1.2 & Триплет ($^3J$ з \ch{CH2}) \\
& \ch{CH2} & 3.7 & Квартет ($^3J$ з \ch{CH3}) \\
& \ch{OH} & 2.6 & Синглет (обмін) \\
Диметиловий ефір & \ch{CH3} & 3.2 & Синглет (еквівалентні) \\
\end{tblr}
\end{center}

Мультиплетність дозволяє однозначно визначити структуру!

\subsubsection{ЕПР датування}

Радикали в мінералах та біологічних зразках можна використовувати
для датування через розпад та накопичення парамагнітних центрів.

\textbf{Застосування:}
\begin{itemize}
    \item Археологічне датування кераміки
    \item Дозиметрія іонізуючого випромінювання
    \item Вивчення радіаційних пошкоджень в ДНК
    \item Детектування вільних радикалів у біології
\end{itemize}

\subsubsection{Визначення абсолютної конфігурації}

CD спектроскопія --- неінвазивний метод визначення абсолютної
конфігурації без необхідності рентгенівської кристалографії.

\textbf{Workflow:}
\begin{enumerate}
    \item Виміряти CD спектр зразка
    \item Розрахувати CD для обох енантіомерів (TD-DFT)
    \item Порівняти знаки Cotton ефектів
    \item Визначити абсолютну конфігурацію
\end{enumerate}

Особливо корисно для:
\begin{itemize}
    \item Природних продуктів
    \item Фармацевтичних препаратів
    \item Хіральних лігандів у каталізі
\end{itemize}

%% --------------------------------------------------------
\subsection{Додаткові ресурси}
%% --------------------------------------------------------

\subsubsection{Експериментальні бази даних}

\begin{itemize}
    \item \textbf{SDBS (Spectral Database for Organic Compounds):} ЯМР, ІЧ спектри
    \item \textbf{ChemSpider:} хімічні зсуви для великої кількості сполук
    \item \textbf{NIST Chemistry WebBook:} експериментальні магнітні моменти
    \item \textbf{eMagRes (Encyclopedia of Magnetic Resonance):} довідник з ЯМР
\end{itemize}

\subsubsection{Програмне забезпечення}

Крім PySCF, корисні програми:
\begin{itemize}
    \item \textbf{ORCA:} відмінні можливості для ЕПР, детальна декомпозиція
    \item \textbf{Gaussian:} стандарт для ЯМР розрахунків
    \item \textbf{ADF:} спеціалізований для систем з важкими атомами
    \item \textbf{DALTON:} магнітооптичні властивості
\end{itemize}

\subsubsection{Рекомендована література}

\begin{enumerate}
    \item \textit{Helgaker, T. et al.} ``Molecular Electronic-Structure Theory'' (2000) \\
    Розділ 11: Magnetic properties

    \item \textit{Autschbach, J.} ``Calculating NMR Chemical Shifts and J-Couplings'' \\
    Comprehensive review, Annu. Rev. Phys. Chem. (2012)

    \item \textit{Neese, F.} ``Quantum Chemistry and EPR Parameters'' \\
    Magnetochemistry (2017)

    \item \textit{Polavarapu, P.L.} ``Chiroptical Spectroscopy'' \\
    CRC Press (2016)
\end{enumerate}

%% --------------------------------------------------------
\subsection{Підсумок розділу}
%% --------------------------------------------------------

\begin{tcolorbox}[colback=blue!5!white,colframe=blue!75!black,title=Ключові концепції]

\textbf{Діамагнітні властивості (замкнені оболонки):}
\begin{itemize}
    \item Магнітна сприйнятливість: індуковані струми
    \item ЯМР екранування: діамагнітний + парамагнітний внески
    \item J-константи: непряма взаємодія через електрони
    \item Оптична активність: ORD, CD для хіральних молекул
\end{itemize}

\textbf{Парамагнітні властивості (відкриті оболонки):}
\begin{itemize}
    \item g-тензор: відхилення від $g_e$ через СО-взаємодію
    \item HFC константи: спінова густина на ядрах
    \item Спінове забруднення: проблема UHF для радикалів
    \item Магнітні моменти: визначаються $S$, $L$, $J$
\end{itemize}

\textbf{Методологія:}
\begin{itemize}
    \item GIAO обов'язковий для gauge-invariance
    \item Спеціалізовані базиси: pcS-n, pcJ-n, EPR-III
    \item DFT краще за HF для більшості властивостей
    \item Завжди перевіряйте $\langle S^2 \rangle$ для радикалів
\end{itemize}

\end{tcolorbox}

\vspace{1em}

\textbf{Практичні застосування:}
\begin{itemize}
    \item Структурне визначення з ЯМР спектрів
    \item Ідентифікація радикалів методом ЕПР
    \item Визначення абсолютної конфігурації (CD)
    \item Вивчення ароматичності (магнітна анізотропія)
    \item Дослідження механізмів реакцій (ізотопні ефекти)
\end{itemize}

\vspace{1em}

\begin{tcolorbox}[colback=green!5!white,colframe=green!75!black,title=Готові до експериментів?]
Після опанування цього розділу ви можете:
\begin{itemize}
    \item Розраховувати ЯМР спектри та інтерпретувати хімічні зсуви
    \item Моделювати ЕПР властивості радикалів
    \item Передбачати CD спектри для хіральних молекул
    \item Аналізувати ароматичність через магнітні властивості
    \item Працювати з системами з відкритими оболонками
\end{itemize}

\textbf{Наступний крок:} застосування до реальних хімічних систем!
\end{tcolorbox}