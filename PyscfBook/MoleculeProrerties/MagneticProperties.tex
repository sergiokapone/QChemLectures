%% ========================================================
\section{Магнітні властивості молекул}
%% ========================================================

%% --------------------------------------------------------
\subsection{Магнітна сприйнятливість}
%% --------------------------------------------------------

Магнітна сприйнятливість $\chi$ описує відгук молекули на магнітне поле:
\[
\mathbf{M} = \chi \mathbf{H}
\]

де $\mathbf{M}$ --- намагніченість, $\mathbf{H}$ --- напруженість магнітного поля.

\textbf{Типи магнетизму:}
\begin{itemize}
    \item \textbf{Діамагнетизм} ($\chi < 0$): всі електрони спарені,
          молекула відштовхується від магнітного поля
    \item \textbf{Парамагнетизм} ($\chi > 0$): непарені електрони,
          молекула притягується до магнітного поля
\end{itemize}

Для замкненої оболонки (діамагнетизм):
\[
\chi_{ij} = -\frac{1}{2c^2}\left\langle\Psi\left|
\sum_k (r_k^2 \delta_{ij} - r_{k,i} r_{k,j})\right|\Psi\right\rangle
\]

%% --------------------------------------------------------
\subsection{Розрахунок магнітної сприйнятливості}
%% --------------------------------------------------------

\inputcode{h2o_magnetic_susceptibility.py}

\textbf{Результати для \ce{H2O} (діамагнетик):}
\begin{center}
\begin{tabular}{lcc}
\toprule
Компонента & Значення (ppm·cgs) & Значення (10$^{-6}$ cm³/mol) \\
\midrule
$\chi_{xx}$ & $-8.2$ & $-13.1$ \\
$\chi_{yy}$ & $-10.5$ & $-16.7$ \\
$\chi_{zz}$ & $-9.1$ & $-14.5$ \\
\midrule
$\bar{\chi}$ & $-9.3$ & $-14.8$ \\
\bottomrule
\end{tabular}
\end{center}

Експериментальне значення: $\bar{\chi} \approx -13.0 \times 10^{-6}$ cm³/mol

\textbf{Анізотропія магнітної сприйнятливості:}
\[
\Delta\chi = \chi_{\parallel} - \chi_{\perp}
\]
важлива для орієнтації молекул у магнітному полі (ЯМР твердого тіла).

%% --------------------------------------------------------
\subsection{Хімічні зсуви ЯМР}
%% --------------------------------------------------------

Хімічний зсув ядра --- це зміна резонансної частоти через екранування
електронною оболонкою:
\[
\delta = \frac{\nu - \nu_{\text{ref}}}{\nu_{\text{ref}}} \times 10^6 \text{ (ppm)}
\]

Тензор екранування $\boldsymbol{\sigma}$ пов'язаний з індукованим полем:
\[
\mathbf{B}_{\text{ind}} = -\boldsymbol{\sigma} \cdot \mathbf{B}_0
\]
