% !TeX program = lualatex
% !TeX encoding = utf8
% !TeX spellcheck = uk_UA
% !TeX root =../PyscfBook.tex

%% ========================================================
\section{Коливальні властивості молекул}
%% ========================================================

Коливальна спектроскопія --- один із найпотужніших методів ідентифікації
молекул та вивчення їх структури. Інфрачервоні (ІЧ) та Раман спектри
надають інформацію про коливальні моди, силові константи зв'язків,
та симетрію молекули.

%% --------------------------------------------------------
\subsection{Гессіан та нормальні моди}
%% --------------------------------------------------------

Коливання атомів у молекулі відбуваються поблизу рівноважної геометрії $\mathbf{R}_0$.
Для малих відхилень потенційна енергія системи можна апроксимувати розкладом у ряд Тейлора до другого порядку:
\[
    E(\mathbf{R}) \approx E(\mathbf{R}_0) + \frac{1}{2}\sum_{ij} H_{ij} \, \Delta R_i \, \Delta R_j,
\]
де
\[
    H_{ij} = \left. \frac{\partial^2 E}{\partial R_i \partial R_j} \right|_{\mathbf{R}_0}
\]
--- елементи \emph{гесіану} (матриці других похідних енергії за декартовими координатами ядер).

Гесіан має розмірність $3N \times 3N$ для молекули з $N$ атомів
і містить повну інформацію про гармонічні коливальні властивості системи.

%% --------------------------------------------------------
\subsubsection{Матриця силових констант}
%% --------------------------------------------------------

Гесіан також називають \textbf{матрицею силових констант}, оскільки його елементи відображають реакцію потенціалу на малі зміщення ядер:
\[
    H_{ij} = \left.\frac{\partial^2 E}{\partial R_i \partial R_j}\right|_{\mathbf{R}_0}
\]

\textbf{Фізичний зміст елементів гесіану:}
\begin{itemize}
    \item \textit{Діагональні елементи} --- характеризують жорсткість потенціалу вздовж окремих координат;
    \item \textit{Недіагональні елементи} --- описують зв’язок між зміщеннями різних атомів;
    \item \textit{Власні значення} $\,\omega^2\,$ --- квадрати коливальних частот;
    \item \textit{Власні вектори} --- напрямки нормальних мод.
\end{itemize}

Таким чином, діагоналізація гесіану дозволяє отримати нормальні координати та спектр гармонічних частот, що повністю характеризують коливальну поведінку молекули поблизу рівноваги.


%% --------------------------------------------------------
\subsection{Обчислення гесіану}
%% --------------------------------------------------------

PySCF може обчислювати гесіан аналітично або чисельно:

\inputcode{h2o_hessian.py}

\textbf{Важливі моменти:}
\begin{itemize}
    \item Аналітичний гесіан доступний для RHF, UHF, RKS, UKS
    \item Чисельний гесіан використовує скінченні різниці:
          $H_{ij} \approx [E(R_i+h) - 2E(R_i) + E(R_i-h)]/h^2$
    \item Аналітичний метод точніший та швидший
    \item Гесіан обчислюється в декартових координатах
\end{itemize}


%% --------------------------------------------------------
\subsubsection{Нормальні коливання}
%% --------------------------------------------------------


Розв'язуючи секулярне рівняння:
\[
    (\mathbf{H} - \omega^2 \mathbf{M}) \mathbf{L} = 0
\]

отримуємо:
\begin{itemize}
    \item $3N - 6$ коливальних мод (нелінійна молекула)
    \item $3N - 5$ коливальних мод (лінійна молекула)
    \item 3 трансляційні моди ($\omega = 0$)
    \item 3 обертальні моди ($\omega = 0$, 2 для лінійної)
\end{itemize}

\textbf{Частоти в різних одиницях:}
\begin{center}
    \begin{tblr}{
        colspec = {X[l] X[r] X[l]},
        row{1} = {font=\bfseries},
        hline{1,2,Z} = {solid},
            }
        Одиниця   & Множник                & Використання             \\
        см$^{-1}$ & 1                      & ІЧ спектроскопія         \\
        Гц        & $2.998 \times 10^{10}$ & СІ одиниці               \\
        еВ        & $1.240 \times 10^{-4}$ & Електронна спектроскопія \\
        Хартрі    & $4.556 \times 10^{-6}$ & Атомні одиниці           \\
    \end{tblr}
\end{center}

%% --------------------------------------------------------
\subsubsection{Розрахунок Гессіану для \ce{H2O}}
%% --------------------------------------------------------


\inputcode{h2o_frequencies.py}

\textbf{Результати для \ce{H2O}:}

\textbf{\footnotesize Частоти в см$^{-1}$. Базис: 6-311++G(3df,3pd)}
\begin{center}
    \begin{tblr}{
        colspec = {X[c] X[r] X[r] X[r] X[l]},
        row{1} = {font=\bfseries},
        hline{1,2,Z} = {solid},
            }
        Мода    & B3LYP & MP2  & Експ. & Опис                    \\
        $\nu_1$ & 3825  & 3832 & 3657  & Симетричне валентне     \\
        $\nu_2$ & 1653  & 1649 & 1595  & Ножичне деформаційне    \\
        $\nu_3$ & 3936  & 3943 & 3756  & Антисиметричне валентне \\
    \end{tblr}
\end{center}


\textbf{Структура гесіану для \ce{H2O}:}
\begin{itemize}
    \item Розмірність: $9 \times 9$ (3 атоми × 3 координати)
    \item Симетричний: $H_{ij} = H_{ji}$
    \item Позитивно визначений у мінімумі енергії
    \item Має 6 нульових власних значень (трансляції та обертання)
\end{itemize}



\textbf{Спостереження:}
\begin{itemize}
    \item Гармонічні частоти завищені на 3--5\% через ангармонізм
    \item Валентні коливання (\ce{O-H}): 3600--4000 см$^{-1}$
    \item Деформаційне коливання: $\sim$1600 см$^{-1}$
    \item Симетрія: $\nu_1$ (A$_1$), $\nu_2$ (A$_1$), $\nu_3$ (B$_2$)
\end{itemize}

%% --------------------------------------------------------
\subsubsection{Масштабування частот}
%% --------------------------------------------------------


Гармонічні частоти систематично завищені. Емпіричне виправлення:
\[
    \nu_{\text{корект}} = \lambda \cdot \nu_{\text{гарм}}
\]

\textbf{Типові масштабувальні фактори:}

\begin{center}
    \begin{tblr}{
        colspec = {X[l,2] X[r] X[l,2]},
        row{1} = {font=\bfseries},
        hline{1,2,Z} = {solid},
            }
        Метод/базис         & $\lambda$ & Коментар               \\
        HF/6-31G*           & 0.8929    & Сильно завищує         \\
        B3LYP/6-31G*        & 0.9613    & Універсальний          \\
        B3LYP/6-311+G(2d,p) & 0.9679    & Кращий базис           \\
        MP2/6-31G*          & 0.9434    & Для точних розрахунків \\
    \end{tblr}
\end{center}

\textit{Джерело: NIST Computational Chemistry Comparison and Benchmark Database}

%% --------------------------------------------------------
\subsection{Інфрачервоні (ІЧ) спектри}
%% --------------------------------------------------------

\subsubsection{Інтенсивності ІЧ поглинання}

ІЧ інтенсивність пропорційна квадрату зміни дипольного моменту:
\[
    I_i \propto \left|\frac{d\boldsymbol{\mu}}{dQ_i}\right|^2
\]

де $Q_i$ --- нормальна координата $i$-ї моди,, $\boldsymbol{\mu}$ --- дипольний момент.

\textbf{Правило відбору:}
\begin{itemize}
    \item ІЧ активна мода: $\frac{d\boldsymbol{\mu}}{dQ_i} \neq 0$.
    \item Симетрична молекула може мати ІЧ-неактивні моди.
    \item Для \ce{H2O} (група $C_{2v}$): всі три моди ІЧ-активні
\end{itemize}


%% --------------------------------------------------------
\subsubsection{ІЧ спектр \ce{H2O}}
%% --------------------------------------------------------


\inputcode{h2o_ir_spectrum.py}

\textbf{Типові інтенсивності для \ce{H2O}:}
\begin{center}
    \begin{tabular}{lcc}
        \toprule
        Мода                       & $\nu$ (см$^{-1}$) & $I$ (км/моль) \\
        \midrule
        $\nu_1$ (симетр. валент.)  & 3657              & 5             \\
        $\nu_2$ (деформаційна)     & 1595              & 73            \\
        $\nu_3$ (антисим. валент.) & 3756              & 58            \\
        \bottomrule
    \end{tabular}
\end{center}

\textbf{Фізична інтерпретація:}
\begin{itemize}
    \item Деформаційна мода найінтенсивніша (велика зміна $\boldsymbol{\mu}$)
    \item Симетричне валентне --- найслабше (компенсація)
    \item Інтенсивність залежить від полярності зв'язків
\end{itemize}

%% --------------------------------------------------------
\subsubsection{ІЧ спектр \ce{CO2}}
%% --------------------------------------------------------

\inputcode{co2_ir_spectrum.py}

\textbf{Коливальні моди \ce{CO2} (лінійна, D$_{\infty h}$):}

\begin{center}
    \begin{tblr}{
        colspec = {X[c] X[r] X[r] X[c] X[l,2]},
        row{1} = {font=\bfseries},
        hline{1,2,Z} = {solid},
            }
        Мода                   & Частота     & Інтенс.   & ІЧ        & Опис                     \\
                               & (см$^{-1}$) & (км/моль) &                                      \\
        $\nu_1$ ($\Sigma_g^+$) & 1333        & 0         & Неактивна & Симетричне валентне      \\
        $\nu_2$ ($\Pi_u$)      & 667         & 85        & Активна   & Деформаційне (вироджене) \\
        $\nu_3$ ($\Sigma_u^+$) & 2349        & 1580      & Активна   & Антисиметричне валентне  \\
    \end{tblr}
\end{center}

\textit{Розрахунок: B3LYP/aug-cc-pVTZ}

%---------------------------------------------------------
%\begin{figure}[h!]\centering
%%    \includegraphics[width=0.85\linewidth]{\currfiledir/co2_ir_spectrum.pdf}
%    \caption{ІЧ спектр \ce{CO2}. Симетрична мода $\nu_1$ не активна в ІЧ через відсутність зміни дипольного моменту.}
%    \label{pic:co2_ir}
%\end{figure}
%---------------------------------------------------------

\textbf{Фізична інтерпретація:}
\begin{itemize}
    \item $\nu_1$: симетричне розтягування, $\mu$ не змінюється (ІЧ-неактивна)
    \item $\nu_2$: згинання молекули, $\mu$ виникає (ІЧ-активна)
    \item $\nu_3$: асиметричне розтягування, велика зміна $\mu$ (дуже інтенсивна)
\end{itemize}

\subsubsection{Характеристичні частоти функціональних груп}

\begin{center}
    \begin{tblr}{
        colspec = {X[l,2] X[r] X[l,2]},
        row{1} = {font=\bfseries},
        hline{1,2,Z} = {solid},
            }
        Група                & Частота (см$^{-1}$) & Коментар              \\
        \ce{O-H} (спирти)    & 3600--3650          & Гострий пік (вільний) \\
        \ce{O-H} (H-зв'язок) & 3200--3550          & Широкий (асоціація)   \\
        \ce{N-H}             & 3300--3500          & Середня інтенсивність \\
        \ce{C-H} (алкани)    & 2850--2960          & Валентні коливання    \\
        \ce{C-H} (алкени)    & 3010--3095          & Вища частота          \\
        \ce{C=O} (кетони)    & 1705--1725          & Дуже інтенсивна       \\
        \ce{C=O} (аміди)     & 1630--1690          & Зміщення резонансом   \\
        \ce{C=C}             & 1620--1680          & Слабка в симетричних  \\
        \ce{C\bond{3}N}      & 2210--2260          & Гострий пік           \\
        \ce{C-O}             & 1050--1150          & Сильна, асиметричне   \\
    \end{tblr}
\end{center}

%% --------------------------------------------------------
\subsection{Раманівський спектр}
%% --------------------------------------------------------

На відміну від ІЧ, активність у Раман-спектрі визначається
поляризовністю $\boldsymbol{\alpha}$, тобто інтенсивність Раман розсіювання пропорційна зміні поляризовності:
\[
    I_i^{\text{Раман}} \propto \left|\frac{d\alpha_{ij}}{dQ_k}\right|^2
\]

%% --------------------------------------------------------
\subsubsection{Інтенсивності Раман розсіювання}
%% --------------------------------------------------------


\textbf{Правило відбору Раман:}
\begin{itemize}
    \item Раман-активна: $\frac{d\alpha}{dQ_i} \neq 0$.
    \item Часто доповнює ІЧ (правило взаємного виключення для центросиметричних).
    \item Для \ce{H2O}: всі три моди також Раман-активні
\end{itemize}

\inputcode{h2o_raman_activity.py}

\textbf{Типові активності для \ce{H2O}:}
\begin{center}
    \begin{tabular}{lcc}
        \toprule
        Мода    & $\nu$ (см$^{-1}$) & Раман-активність (Å$^4$/amu) \\
        \midrule
        $\nu_1$ & 3657              & 1.8                          \\
        $\nu_2$ & 1595              & 0.4                          \\
        $\nu_3$ & 3756              & 3.2                          \\
        \bottomrule
    \end{tabular}
\end{center}


\textbf{Принцип взаємного виключення:}

Для молекул з центром інверсії (наприклад, \ce{CO2}):
\begin{itemize}
    \item Парні моди ($g$): Раман активні, ІЧ неактивні
    \item Непарні моди ($u$): ІЧ активні, Раман неактивні
\end{itemize}

\subsubsection{Раман спектр бензену}

\inputcode{benzene_raman_spectrum.py}

\textbf{Вибрані моди бензену \ce{C6H6} (D$_{6h}$):}

\begin{center}
    \begin{tblr}{
        colspec = {X[c] X[c] X[r] X[c] X[c]},
        row{1} = {font=\bfseries},
        hline{1,2,Z} = {solid},
            }
        Мода       & Симетрія & Частота     & ІЧ        & Раман            \\
                   &          & (см$^{-1}$) &                              \\
        $\nu_1$    & A$_{1g}$ & 993         & Неактивна & Активна          \\
        $\nu_2$    & A$_{1g}$ & 3062        & Неактивна & Активна (сильна) \\
        $\nu_6$    & E$_{2g}$ & 608         & Неактивна & Активна          \\
        $\nu_{18}$ & E$_{1u}$ & 1038        & Активна   & Неактивна        \\
        $\nu_{19}$ & E$_{1u}$ & 3080        & Активна   & Неактивна        \\
    \end{tblr}
\end{center}

\textit{Розрахунок: B3LYP/6-311+G(2d,p)}

%---------------------------------------------------------
\begin{figure}[h!]\centering
%    \includegraphics[width=0.85\linewidth]{\currfiledir/benzene_raman_spectrum.pdf}
    \caption{Раман спектр бензену демонструє принцип взаємного виключення.
        Симетричні моди ($g$) активні в Раман, але неактивні в ІЧ.}
    \label{pic:benzene_raman}
\end{figure}
%---------------------------------------------------------

\textbf{Характерні Раман смуги:}
\begin{itemize}
    \item 993 см$^{-1}$: дихальна мода кільця (ring breathing)
    \item 3062 см$^{-1}$: симетричне валентне \ce{C-H}
    \item 608 см$^{-1}$: деформація кільця
\end{itemize}

\subsubsection{Порівняння ІЧ та Раман}

\begin{center}
    \begin{tblr}{
        colspec = {X[l,2] X[l,3] X[l,3]},
        row{1} = {font=\bfseries},
        hline{1,2,Z} = {solid},
            }
        Властивість               & ІЧ спектроскопія         & Раман спектроскопія         \\
        Правило відбору           & $\frac{d\mu}{dQ} \neq 0$ & $\frac{d\alpha}{dQ} \neq 0$ \\
        Джерело                   & ІЧ лампа                 & Лазер (видиме/УФ)           \\
        Зразок                    & Розчин, плівка, KBr      & Розчин, кристал             \\
        Вода                      & Сильно поглинає          & Слабкий сигнал              \\
        Скло                      & Непрозоре                & Прозоре                     \\
        Полярні групи             & Сильний сигнал           & Слабкий                     \\
        Неполярні                 & Слабкий                  & Сильний сигнал              \\
        \ce{C=C}, \ce{C\bond{3}C} & Слабкі/неактивні         & Сильні                      \\
        \ce{C=O}, \ce{O-H}        & Дуже сильні              & Слабкі                      \\
    \end{tblr}
\end{center}

%% --------------------------------------------------------
\subsection{Термохімія та нульова енергія}
%% --------------------------------------------------------

Використовуючи частоти коливань, можна обчислити термодинамічні
функції в наближенні гармонічного осцилятора та жорсткого ротатора.

%% --------------------------------------------------------
\subsubsection{Нульова коливальна енергія (ZPE)}
%% --------------------------------------------------------


Навіть при $T = 0$ K молекула має енергію коливань:
\[
    E_{\text{ZPE}} = \sum_{i=1}^{3N-6} \frac{1}{2}\hbar\omega_i
\]


\textbf{Важливість ZPE:}
\begin{itemize}
    \item Для точних термохімічних розрахунків
    \item Ізотопні ефекти (HD vs \ce{H2})
    \item Тунелювання через бар'єри
    \item Типово 5--15 ккал/моль для органічних молекул
\end{itemize}

Для \ce{H2O}:
\[
    E_{\text{ZPE}} \approx \frac{1}{2}(3657 + 1595 + 3756) \text{ см}^{-1}
    \times 1.439 \text{ ккал/(моль·см}^{-1}) \approx 13.3 \text{ ккал/моль}
\]

%% --------------------------------------------------------
\subsubsection{Термохімічні поправки}
%% --------------------------------------------------------

При температурі $T$ коливальний внесок:
\[
    E_{\text{vib}}(T) = \sum_k \frac{\hbar\omega_k}{\exp(\hbar\omega_k/k_BT) - 1}.
\]

Повна енергія при температурі $T$:
\[
    E_{\text{total}}(T) = E_{\text{elec}} + E_{\text{ZPE}} + E_{\text{vib}}(T) +
    E_{\text{rot}}(T) + E_{\text{trans}}(T)
\]

\textbf{Ентальпія:}
\[
    H(T) = E_{\text{total}}(T) + RT
\]

\textbf{Ентропія:}
\[
    S(T) = S_{\text{trans}} + S_{\text{rot}} + S_{\text{vib}} + S_{\text{elec}}
\]

\textbf{Вільна енергія Гіббса:}
\[
    G(T) = H(T) - TS(T)
\]

%% --------------------------------------------------------
\subsubsection{Термохімічний розрахунок для \ce{H2O}}
%% --------------------------------------------------------


\inputcode{h2o_thermochemistry.py}

\textbf{Типові поправки для \ce{H2O} при 298.15 K:}
\begin{itemize}
    \item Електронна енергія: $E_{\text{elec}} = -76.067$ Ha
    \item ZPE: +0.021 Ha (13.3 ккал/моль)
    \item Термічна поправка: +0.003 Ha (1.9 ккал/моль)
    \item Ентропія: $S = 45.1$ кал/(моль·K)
\end{itemize}


%% --------------------------------------------------------
\subsubsection{Термохімічний розрахунок для \ce{CH4}}
%% --------------------------------------------------------


\inputcode{ch4_thermochemistry.py}

\textbf{Результати для \ce{CH4} при $298.15$~K:}

\begin{center}
    \begin{tblr}{
        colspec = {X[l,3] X[r,2] X[l,2]},
        row{1} = {font=\bfseries},
        hline{1,2,Z} = {solid},
            }
        Величина              & Значення & Одиниці      \\
        ZPE                   & 28.03    & ккал/моль    \\
        $E_{\text{vib}}(T)$   & 0.10     & ккал/моль    \\
        $E_{\text{rot}}(T)$   & 0.89     & ккал/моль    \\
        $E_{\text{trans}}(T)$ & 0.89     & ккал/моль    \\
        $H(T) - H(0)$         & 2.48     & ккал/моль    \\
        $S(T)$                & 44.5     & кал/(моль·K) \\
        $C_V$                 & 6.0      & кал/(моль·K) \\
    \end{tblr}
\end{center}

\textit{Розрахунок: B3LYP/6-311+G(2d,p)}

\textbf{Порівняння з експериментом:}
\begin{itemize}
    \item ZPE (експ.): 27.8 ккал/моль --- відмінна згода
    \item $S_{298}$ (експ.): 44.5 кал/(моль·K) --- ідеальна згода
    \item Термохімічні поправки надійні для DFT
\end{itemize}

%% --------------------------------------------------------
\subsection{Ізотопні ефекти}
%% --------------------------------------------------------

Заміна ізотопу змінює частоти через зміну маси:
\[
    \frac{\omega_{\text{D}}}{\omega_{\text{H}}} \approx \sqrt{\frac{m_{\text{H}}}{m_{\text{D}}}} = \sqrt{\frac{1}{2}} \approx 0.707
\]

\subsubsection{Ізотопний ефект у \ce{H2O}/\ce{D2O}}

%\inputcode{h2o_d2o_isotope_effect.py}

\textbf{Порівняння \ce{H2O} та \ce{D2O}:}

\begin{center}
    \begin{tblr}{
        colspec = {X[l,3] X[r,2] X[r,2] X[r,2]},
        row{1} = {font=\bfseries},
        hline{1,2,Z} = {solid},
            }
        Мода                & \ce{H2O}    & \ce{D2O}    & Співвідношення \\
                            & (см$^{-1}$) & (см$^{-1}$) &                \\
        Симетричне валентне & 3657        & 2671        & 0.730          \\
        Деформаційне        & 1595        & 1178        & 0.738          \\
        Антисим. валентне   & 3756        & 2788        & 0.742          \\
    \end{tblr}
\end{center}

\textit{Експериментальні дані}

\textbf{ZPE ефект:}
\begin{itemize}
    \item ZPE(\ce{H2O}): 13.26 ккал/моль
    \item ZPE(\ce{D2O}): 9.66 ккал/моль
    \item $\Delta$ZPE: 3.60 ккал/моль
\end{itemize}

\textbf{Наслідки:}
\begin{itemize}
    \item Дейтеровані сполуки стабільніші (нижча ZPE)
    \item Кінетичний ізотопний ефект: $k_{\text{H}}/k_{\text{D}} \approx 7$ для розриву \ce{C-H}
    \item Використання в механістичних дослідженнях
\end{itemize}

%% --------------------------------------------------------
\subsection{Перехідні стани та уявні частоти}
%% --------------------------------------------------------

\subsubsection{Характеристика стаціонарних точок}

Гессіан дозволяє класифікувати стаціонарні точки:

\begin{center}
    \begin{tblr}{
        colspec = {X[l,2] X[c,2] X[l,3]},
        row{1} = {font=\bfseries},
        hline{1,2,Z} = {solid},
            }
        Тип                & Уявних частот & Характеристика             \\
        Мінімум            & 0             & Локальний мінімум          \\
        Перехідний стан    & 1             & Сідлова точка 1-го порядку \\
        Сідло 2-го порядку & 2             & Нестабільна структура      \\
    \end{tblr}
\end{center}

\textbf{Уявна частота:} $\omega^2 < 0 \Rightarrow \omega = i|\omega|$

У виводі PySCF позначається як негативна частота.

\subsubsection{Приклад: інверсія аміаку}

%\inputcode{nh3_inversion.py}

\textbf{Профіль енергії інверсії \ce{NH3}:}

\begin{center}
    \begin{tblr}{
        colspec = {X[l,2] X[r,2] X[r,2] X[c,2]},
        row{1} = {font=\bfseries},
        hline{1,2,Z} = {solid},
            }
        Структура                  & $E_{\text{rel}}$ & Уявних & Тип     \\
                                   & (ккал/моль)      & частот &         \\
        Пірамідальна (\ce{C_{3v}}) & 0.0              & 0      & Мінімум \\
        Планарна (\ce{D_{3h}})     & 5.8              & 1      & ПС      \\
        Пірамідальна (інша)        & 0.0              & 0      & Мінімум \\
    \end{tblr}
\end{center}

%---------------------------------------------------------
\begin{figure}[h!]\centering
%    \includegraphics[width=0.75\linewidth]{\currfiledir/nh3_inversion_profile.pdf}
    \caption{Профіль енергії інверсії \ce{NH3}. Планарна структура є
        перехідним станом з однією уявною частотою (зонтичний рух).}
    \label{pic:nh3_inversion}
\end{figure}
%---------------------------------------------------------

\textbf{Уявна мода в ПС:}
\begin{itemize}
    \item Частота: $\omega = i 1044$ см$^{-1}$ (або $-1044i$ в PySCF)
    \item Напрямок: зонтичний рух (umbrella mode)
    \item Зв'язує два еквівалентні мінімуми
    \item Квантове тунелювання → розщеплення ЯМР сигналу
\end{itemize}

%% --------------------------------------------------------
\subsubsection{Інтрінсична координата реакції (IRC)}
%% --------------------------------------------------------


Від перехідного стану можна простежити шлях реакції:
\begin{itemize}
    \item Рух вздовж уявної моди
    \item З'єднує реагенти та продукти
    \item Доводить, що ПС належить до шуканої реакції
\end{itemize}

%% --------------------------------------------------------
\subsection{Ангармонічні поправки}
%% --------------------------------------------------------

Гармонічне наближення добре для малих амплітуд, але:
\begin{itemize}
    \item Завищує частоти на 3--5\%
    \item Не описує обертонів та комбінаційних смуг
    \item Не враховує асиметрію потенціалу
\end{itemize}

\subsubsection{Ангармонічний потенціал}

Розклад потенціалу до кубічних та квартичних термів:
\[
    V = V_0 + \frac{1}{2}\sum_{ij} k_{ij} Q_i Q_j +
    \frac{1}{6}\sum_{ijk} k_{ijk} Q_i Q_j Q_k +
    \frac{1}{24}\sum_{ijkl} k_{ijkl} Q_i Q_j Q_k Q_l
\]

\textbf{Методи розрахунку:}
\begin{itemize}
    \item VPT2 (Vibrational Perturbation Theory 2-го порядку)
    \item VSCF (Vibrational Self-Consistent Field)
    \item VCI (Vibrational Configuration Interaction)
\end{itemize}

\subsubsection{Ангармонічні частоти \ce{H2O}}

\begin{center}
    \begin{tblr}{
        colspec = {X[c] X[r] X[r] X[r] X[r]},
        row{1} = {font=\bfseries},
        hline{1,2,Z} = {solid},
            }
        Мода    & Гарм.       & VPT2        & Масштаб.    & Експ.       \\
                & (см$^{-1}$) & (см$^{-1}$) & (см$^{-1}$) & (см$^{-1}$) \\
        $\nu_1$ & 3825        & 3707        & 3702        & 3657        \\
        $\nu_2$ & 1653        & 1618        & 1600        & 1595        \\
        $\nu_3$ & 3936        & 3814        & 3808        & 3756        \\
    \end{tblr}
\end{center}

\textit{B3LYP/aug-cc-pVTZ, масштабувальний фактор 0.968}

\textbf{Висновки:}
\begin{itemize}
    \item VPT2 суттєво покращує згоду з експериментом
    \item Масштабування простіше, але менш фізичне
    \item Для валентних коливань ангармонізм $\sim100-150$ см$^{-1}$
\end{itemize}

%% --------------------------------------------------------
\subsection{Практичні рекомендації}
%% --------------------------------------------------------

\subsubsection{Вибір методу та базису}

%\begin{center}
%\begin{tblr}{
%    colspec = {X[l,2] X[l,2] X[l,3]},
%    row{1} = {font=\bfseries},
%    hline{1,2,Z} = {solid},
%}
%Завдання & Рекомендація & Коментар \\
%Ідентифікація мінімуму & B3LYP/6-31G* & Швидко, надійно \\
%Точні частоти & B3LYP/6-311+G(2d,p) & Масштабувати 0.968 \\
%Термохімія & B3LYP/6-311++G(3df,3pd) & Для ZPE \\
%Великі системи & B3LYP/6-31+G* & Компроміс \\
%Benchmark & $\omega$B97X-D/aug-cc-pVTZ & Найкраща точність \\
%Ангармонізм & VPT2, малий базис & Дуже дорого \\


%% ========================================================
\section{Електронні переходи та УФ-видимі спектри}
%% ========================================================

%% --------------------------------------------------------
\subsection{Теорія збуджених станів}
%% --------------------------------------------------------

Для розрахунку електронних переходів використовуємо методи:
\begin{itemize}
    \item \textbf{CIS} (Configuration Interaction Singles) --- базовий
    \item \textbf{TDHF/TDDFT} (Time-Dependent HF/DFT) --- точніший
    \item \textbf{EOM-CCSD} (Equation-of-Motion CCSD) --- високоточний
\end{itemize}

Енергія збудження $n$-го стану:
\[
    \omega_n = E_n - E_0
\]

Сила осцилятора (інтенсивність):
\[
    f_n = \frac{2}{3}\omega_n |\langle\Psi_0|\hat{\boldsymbol{\mu}}|\Psi_n\rangle|^2
\]

%% --------------------------------------------------------
\subsection{TDDFT розрахунок для формальдегіду}
%% --------------------------------------------------------

Розглянемо молекулу формальдегіду \ce{H2CO} як приклад:

\inputcode{h2co_tddft.py}

\textbf{Аналіз збуджень для \ce{H2CO}:}
\begin{center}
    \begin{tabular}{lcccc}
        \toprule
        Перехід & Тип           & $\lambda$ (нм) & $f$   & Характер       \\
        \midrule
        $S_1$   & $n\to\pi^*$   & 355            & 0.001 & Заборонений    \\
        $S_2$   & $\pi\to\pi^*$ & 185            & 0.152 & Дозволений     \\
        $S_3$   & $n\to 3s$     & 172            & 0.021 & Рідбергівський \\
        \bottomrule
    \end{tabular}
\end{center}

\textbf{Фізична інтерпретація:}
\begin{itemize}
    \item $n\to\pi^*$: неподілена пара O $\to$ антизв'язуюча МО C=O
    \item Низька сила осцилятора через симетрію
    \item $\pi\to\pi^*$: основна смуга поглинання в УФ
    \item Енергії залежать від функціоналу DFT
\end{itemize}

%% --------------------------------------------------------
\subsection{Вибір функціоналу для TDDFT}
%% --------------------------------------------------------

Різні функціонали дають різну точність для збуджень:

\inputcode{formaldehyde_functional_comparison.py}

\textbf{Порівняння функціоналів (перехід $n\to\pi^*$):}
\begin{center}
    \begin{tabular}{lcc}
        \toprule
        Функціонал     & $\lambda$ (нм) & Похибка (нм) \\
        \midrule
        B3LYP          & 355            & +25          \\
        PBE0           & 342            & +12          \\
        CAM-B3LYP      & 335            & +5           \\
        $\omega$B97X-D & 332            & +2           \\
        \midrule
        Експеримент    & 330            & ---          \\
        \bottomrule
    \end{tabular}
\end{center}

\textbf{Рекомендації:}
\begin{itemize}
    \item Гібридні функціонали краще за чисті GGA
    \item Range-separated (CAM-B3LYP, $\omega$B97X-D) найточніші
    \item Для переносу заряду обов'язково range-separated
    \item B3LYP часто завищує довжини хвиль
\end{itemize}

%% --------------------------------------------------------
\subsection{Візуалізація спектру}
%% --------------------------------------------------------

Для побудови спектру використовуємо гаусові або лоренцеві контури:
\[
    \varepsilon(\lambda) = \sum_n f_n \cdot \frac{1}{\sigma\sqrt{2\pi}}
    \exp\left[-\frac{(\lambda - \lambda_n)^2}{2\sigma^2}\right]
\]

\inputcode{plot_uv_spectrum.py}

%---------------------------------------------------------
\begin{figure}[h!]\centering
    %\includegraphics[width=\linewidth]{\currfiledir/h2co_uv_spectrum.pdf}
    \caption{УФ-спектр формальдегіду, розрахований методом TDDFT/CAM-B3LYP/aug-cc-pVDZ.}
    \label{pic:h2co_uv_spectrum}
\end{figure}
%---------------------------------------------------------

\textbf{Параметри уширення:}
\begin{itemize}
    \item Типове $\sigma = 0.3$--$0.5$ eV для конденсованої фази
    \item $\sigma = 0.1$--$0.2$ eV для газової фази
    \item Експериментальне уширення враховує розподіл за $T$
\end{itemize}

%% --------------------------------------------------------
\subsection{Аналіз характеру переходів}
%% --------------------------------------------------------

TDDFT надає інформацію про орбіталі, задіяні у переході:

\inputcode{analyze_transitions.py}

\textbf{Приклад виводу для $S_1$ стану \ce{H2CO}:}
\begin{minted}{text}
Excited State 1: 3.492 eV (355 nm)  f=0.0012
    HOMO-1 -> LUMO     0.11 (1.2%)
    HOMO   -> LUMO     0.69 (47.6%)
    HOMO   -> LUMO+1   0.08 (0.6%)
\end{minted}

\textbf{Інтерпретація:}
\begin{itemize}
    \item Основний внесок: HOMO $\to$ LUMO (48\%)
    \item HOMO --- неподілена пара n(O)
    \item LUMO --- антизв'язуюча $\pi^*$(C=O)
    \item Тип переходу: $n\to\pi^*$
\end{itemize}