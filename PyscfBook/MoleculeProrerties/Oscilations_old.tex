% !TeX program = lualatex
% !TeX encoding = utf8
% !TeX spellcheck = uk_UA
% !TeX root =../PyscfBook.tex

%% ========================================================
\section{Коливальні властивості молекул}
%% ========================================================

%% --------------------------------------------------------
\subsection{Теоретичні основи}
%% --------------------------------------------------------

Коливання атомів у молекулі відбуваються навколо рівноважної геометрії.
Для малих відхилень потенційна енергія апроксимується квадратичною функцією:
\[
    E(\mathbf{R}) \approx E(\mathbf{R}_0) + \frac{1}{2}\sum_{ij} H_{ij} \Delta R_i \Delta R_j,
\]
де $H_{ij}$ --- елементи матриці других похідних енергії (гесіану):
\[
    H_{ij} = \left.\frac{\partial^2 E}{\partial R_i \partial R_j}\right|_{\mathbf{R}_0}
\]

\textbf{Гесіан} має розмірність $3N \times 3N$ для молекули з $N$ атомів
та містить повну інформацію про коливальні властивості.

%% --------------------------------------------------------
\subsection{Обчислення гесіану}
%% --------------------------------------------------------

PySCF може обчислювати гесіан аналітично або чисельно:

\inputcode{h2o_hessian.py}

\textbf{Важливі моменти:}
\begin{itemize}
    \item Аналітичний гесіан доступний для RHF, UHF, RKS, UKS
    \item Чисельний гесіан використовує скінченні різниці:
          $H_{ij} \approx [E(R_i+h) - 2E(R_i) + E(R_i-h)]/h^2$
    \item Аналітичний метод точніший та швидший
    \item Гесіан обчислюється в декартових координатах
\end{itemize}

\textbf{Структура гесіану для \ce{H2O}:}
\begin{itemize}
    \item Розмірність: $9 \times 9$ (3 атоми × 3 координати)
    \item Симетричний: $H_{ij} = H_{ji}$
    \item Позитивно визначений у мінімумі енергії
    \item Має 6 нульових власних значень (трансляції та обертання)
\end{itemize}

%% --------------------------------------------------------
\subsection{Нормальні моди коливань}
%% --------------------------------------------------------

Для знаходження нормальних мод диагоналізуємо масо-зважений гесіан:
\[
    \mathbf{M}^{-1/2} \mathbf{H} \mathbf{M}^{-1/2} \mathbf{L} = \mathbf{L} \boldsymbol{\Lambda},
\]
де $\mathbf{M}$ --- діагональна матриця мас атомів,
$\mathbf{L}$ --- власні вектори (нормальні моди),
$\boldsymbol{\Lambda}$ --- власні значення.

Частоти коливань:
\[
    \omega_k = \sqrt{\lambda_k}, \quad \nu_k = \frac{\omega_k}{2\pi c}
\]

\inputcode{h2o_frequencies.py}

\textbf{Аналіз результатів для \ce{H2O}:}
\begin{center}
    \begin{tabular}{lccc}
        \toprule
        Мода    & Тип   & $\nu$ (см$^{-1}$) & Опис                    \\
        \midrule
        $\nu_1$ & $A_1$ & 3657              & Симетричне валентне     \\
        $\nu_2$ & $A_1$ & 1595              & Деформаційне (ножиці)   \\
        $\nu_3$ & $B_2$ & 3756              & Антисиметричне валентне \\
        \bottomrule
    \end{tabular}
\end{center}

\textbf{Порівняння з експериментом:}
\begin{itemize}
    \item RHF/6-31G: завищує частоти на 10--15\%
    \item RHF/cc-pVTZ: завищує на 8--12\%
    \item B3LYP/cc-pVTZ: відхилення < 5\% (з масштабуванням)
    \item Типовий масштабуючий фактор для RHF: 0.89
\end{itemize}

%% --------------------------------------------------------
\subsection{ІЧ-спектр: інтенсивності}
%% --------------------------------------------------------

Інтенсивність ІЧ-поглинання визначається зміною дипольного моменту:
\[
    I_k \propto \left|\frac{\partial \boldsymbol{\mu}}{\partial Q_k}\right|^2,
\]
де $Q_k$ --- нормальна координата $k$-ї моди,
$\boldsymbol{\mu}$ --- дипольний момент.

\textbf{Правила відбору:}
\begin{itemize}
    \item ІЧ-активна: $\partial\boldsymbol{\mu}/\partial Q_k \neq 0$
    \item Залежить від симетрії молекули
    \item Для \ce{H2O} (група $C_{2v}$): всі три моди ІЧ-активні
\end{itemize}

\inputcode{h2o_ir_spectrum.py}

\textbf{Типові інтенсивності для \ce{H2O}:}
\begin{center}
    \begin{tabular}{lcc}
        \toprule
        Мода                       & $\nu$ (см$^{-1}$) & $I$ (км/моль) \\
        \midrule
        $\nu_1$ (симетр. валент.)  & 3657              & 5             \\
        $\nu_2$ (деформаційна)     & 1595              & 73            \\
        $\nu_3$ (антисим. валент.) & 3756              & 58            \\
        \bottomrule
    \end{tabular}
\end{center}

\textbf{Фізична інтерпретація:}
\begin{itemize}
    \item Деформаційна мода найінтенсивніша (велика зміна $\boldsymbol{\mu}$)
    \item Симетричне валентне --- найслабше (компенсація)
    \item Інтенсивність залежить від полярності зв'язків
\end{itemize}

%% --------------------------------------------------------
\subsection{Раманівський спектр}
%% --------------------------------------------------------

На відміну від ІЧ, активність у Раман-спектрі визначається
поляризовністю $\boldsymbol{\alpha}$:
\[
    I_k^{\text{Raman}} \propto \left|\frac{\partial\boldsymbol{\alpha}}{\partial Q_k}\right|^2
\]

\textbf{Правила відбору:}
\begin{itemize}
    \item Раман-активна: $\partial\boldsymbol{\alpha}/\partial Q_k \neq 0$
    \item Відрізняється від ІЧ (принцип взаємовиключення для центросиметричних)
    \item Для \ce{H2O}: всі три моди також Раман-активні
\end{itemize}

\inputcode{h2o_raman_activity.py}

\textbf{Типові активності для \ce{H2O}:}
\begin{center}
    \begin{tabular}{lcc}
        \toprule
        Мода    & $\nu$ (см$^{-1}$) & Раман-активність (Å$^4$/amu) \\
        \midrule
        $\nu_1$ & 3657              & 1.8                          \\
        $\nu_2$ & 1595              & 0.4                          \\
        $\nu_3$ & 3756              & 3.2                          \\
        \bottomrule
    \end{tabular}
\end{center}

%% --------------------------------------------------------
\subsection{Термохімічні поправки}
%% --------------------------------------------------------

Використовуючи частоти коливань, можна обчислити термодинамічні
функції в наближенні гармонічного осцилятора та жорсткого ротатора.

\subsubsection{Нульова коливальна енергія (ZPE)}

\[
    E_{\text{ZPE}} = \sum_k \frac{1}{2}\hbar\omega_k = \sum_k \frac{hc\nu_k}{2}
\]

Для \ce{H2O}:
\[
    E_{\text{ZPE}} \approx \frac{1}{2}(3657 + 1595 + 3756) \text{ см}^{-1}
    \times 1.439 \text{ ккал/(моль·см}^{-1}) \approx 13.3 \text{ ккал/моль}
\]

\subsubsection{Термічні поправки}

При температурі $T$ коливальний внесок:
\[
    E_{\text{vib}}(T) = \sum_k \frac{\hbar\omega_k}{\exp(\hbar\omega_k/k_BT) - 1}
\]

Повна енергія при $T$:
\[
    E(T) = E_{\text{elec}} + E_{\text{ZPE}} + E_{\text{vib}}(T) + E_{\text{rot}}(T) + E_{\text{trans}}(T)
\]

\inputcode{h2o_thermochemistry.py}

\textbf{Типові поправки для \ce{H2O} при 298.15 K:}
\begin{itemize}
    \item Електронна енергія: $E_{\text{elec}} = -76.067$ Ha
    \item ZPE: +0.021 Ha (13.3 ккал/моль)
    \item Термічна поправка: +0.003 Ha (1.9 ккал/моль)
    \item Ентропія: $S = 45.1$ кал/(моль·K)
\end{itemize}

%% ========================================================
\section{Електронні переходи та УФ-видимі спектри}
%% ========================================================

%% --------------------------------------------------------
\subsection{Теорія збуджених станів}
%% --------------------------------------------------------

Для розрахунку електронних переходів використовуємо методи:
\begin{itemize}
    \item \textbf{CIS} (Configuration Interaction Singles) --- базовий
    \item \textbf{TDHF/TDDFT} (Time-Dependent HF/DFT) --- точніший
    \item \textbf{EOM-CCSD} (Equation-of-Motion CCSD) --- високоточний
\end{itemize}

Енергія збудження $n$-го стану:
\[
    \omega_n = E_n - E_0
\]

Сила осцилятора (інтенсивність):
\[
    f_n = \frac{2}{3}\omega_n |\langle\Psi_0|\hat{\boldsymbol{\mu}}|\Psi_n\rangle|^2
\]

%% --------------------------------------------------------
\subsection{TDDFT розрахунок для формальдегіду}
%% --------------------------------------------------------

Розглянемо молекулу формальдегіду H₂CO як приклад:

\inputcode{h2co_tddft.py}

\textbf{Аналіз збуджень для \ce{H2CO}:}
\begin{center}
    \begin{tabular}{lcccc}
        \toprule
        Перехід & Тип           & $\lambda$ (нм) & $f$   & Характер       \\
        \midrule
        $S_1$   & $n\to\pi^*$   & 355            & 0.001 & Заборонений    \\
        $S_2$   & $\pi\to\pi^*$ & 185            & 0.152 & Дозволений     \\
        $S_3$   & $n\to 3s$     & 172            & 0.021 & Рідбергівський \\
        \bottomrule
    \end{tabular}
\end{center}

\textbf{Фізична інтерпретація:}
\begin{itemize}
    \item $n\to\pi^*$: неподілена пара O $\to$ антизв'язуюча МО C=O
    \item Низька сила осцилятора через симетрію
    \item $\pi\to\pi^*$: основна смуга поглинання в УФ
    \item Енергії залежать від функціоналу DFT
\end{itemize}

%% --------------------------------------------------------
\subsection{Вибір функціоналу для TDDFT}
%% --------------------------------------------------------

Різні функціонали дають різну точність для збуджень:

\inputcode{formaldehyde_functional_comparison.py}

\textbf{Порівняння функціоналів (перехід $n\to\pi^*$):}
\begin{center}
    \begin{tabular}{lcc}
        \toprule
        Функціонал     & $\lambda$ (нм) & Похибка (нм) \\
        \midrule
        B3LYP          & 355            & +25          \\
        PBE0           & 342            & +12          \\
        CAM-B3LYP      & 335            & +5           \\
        $\omega$B97X-D & 332            & +2           \\
        \midrule
        Експеримент    & 330            & ---          \\
        \bottomrule
    \end{tabular}
\end{center}

\textbf{Рекомендації:}
\begin{itemize}
    \item Гібридні функціонали краще за чисті GGA
    \item Range-separated (CAM-B3LYP, $\omega$B97X-D) найточніші
    \item Для переносу заряду обов'язково range-separated
    \item B3LYP часто завищує довжини хвиль
\end{itemize}

%% --------------------------------------------------------
\subsection{Візуалізація спектру}
%% --------------------------------------------------------

Для побудови спектру використовуємо гаусові або лоренцеві контури:
\[
    \varepsilon(\lambda) = \sum_n f_n \cdot \frac{1}{\sigma\sqrt{2\pi}}
    \exp\left[-\frac{(\lambda - \lambda_n)^2}{2\sigma^2}\right]
\]

\inputcode{plot_uv_spectrum.py}

%---------------------------------------------------------
\begin{figure}[h!]\centering
    %\includegraphics[width=\linewidth]{\currfiledir/h2co_uv_spectrum.pdf}
    \caption{УФ-спектр формальдегіду, розрахований методом TDDFT/CAM-B3LYP/aug-cc-pVDZ.}
    \label{pic:h2co_uv_spectrum}
\end{figure}
%---------------------------------------------------------

\textbf{Параметри уширення:}
\begin{itemize}
    \item Типове $\sigma = 0.3$--$0.5$ eV для конденсованої фази
    \item $\sigma = 0.1$--$0.2$ eV для газової фази
    \item Експериментальне уширення враховує розподіл за $T$
\end{itemize}

%% --------------------------------------------------------
\subsection{Аналіз характеру переходів}
%% --------------------------------------------------------

TDDFT надає інформацію про орбіталі, задіяні у переході:

\inputcode{analyze_transitions.py}

\textbf{Приклад виводу для $S_1$ стану \ce{H2CO}:}
\begin{minted}{text}
Excited State 1: 3.492 eV (355 nm)  f=0.0012
    HOMO-1 -> LUMO     0.11 (1.2%)
    HOMO   -> LUMO     0.69 (47.6%)
    HOMO   -> LUMO+1   0.08 (0.6%)
\end{minted}

\textbf{Інтерпретація:}
\begin{itemize}
    \item Основний внесок: HOMO $\to$ LUMO (48\%)
    \item HOMO --- неподілена пара n(O)
    \item LUMO --- антизв'язуюча $\pi^*$(C=O)
    \item Тип переходу: $n\to\pi^*$
\end{itemize}