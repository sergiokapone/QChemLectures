% !TeX program = lualatex
% !TeX encoding = utf8
% !TeX spellcheck = uk_UA
% !TeX root =../PyscfBook.tex

%=========================================================
\Opensolutionfile{answer}[\currfilebase/\currfilebase-Answers]
\chapter{Пост-Гартрі-Фоківські методи}\label{\currfilebase}
%=========================================================

%% --------------------------------------------------------
\section{Вступ до електронної кореляції}
%% --------------------------------------------------------


%% --------------------------------------------------------
\subsection{Що таке електронна кореляція?}
%% --------------------------------------------------------

Електронна кореляція --- це різниця між точною енергією системи та енергією, отриманою методом Хартрі-Фока:

\begin{equation}
    E_{\text{corr}} = E_{\text{exact}} - E_{\text{HF}}
\end{equation}

Метод Хартрі-Фока не враховує миттєву кореляцію рухів електронів, оскільки кожен електрон рухається в усередненому полі всіх інших електронів. Насправді ж електрони "уникають" один одного через кулонівське відштовхування.

%% --------------------------------------------------------
\subsection{Типи електронної кореляції}
%% --------------------------------------------------------

\paragraph{Динамічна кореляція}
Пов'язана з миттєвими флуктуаціями електронної густини. Проявляється на коротких відстанях між електронами. Може бути враховані методами:
\begin{itemize}
    \item Теорія збурень Møller-Plesset (MP2, MP3, MP4)
    \item Coupled Cluster (CCSD, CCSD(T))
    \item Configuration Interaction (CISD, QCISD)
\end{itemize}

\paragraph{Статична (нединамічна) кореляція}
Виникає, коли кілька конфігурацій близькі за енергією. Важлива для:
\begin{itemize}
    \item Розриву хімічних зв'язків
    \item Збуджених станів
    \item Диелектронних систем
    \item Перехідних металів з близькими d-орбіталями
\end{itemize}

Методи: CASSCF, CASPT2, MRCI.

%% --------------------------------------------------------
\subsection{Кореляційна енергія атомів}
%% --------------------------------------------------------

Для атомів кореляційна енергія становить 1--5\% від повної енергії, але вона критична для точних розрахунків:

\begin{center}
    \captionof{table}{Кореляційна енергія атомів (Ha)}
    \label{tab:correlation_energy}
    \begin{tabular}{lccccc}
        \hline
        \textbf{Атом} & \textbf{$E_{HF}$} & \textbf{$E_{exact}$} & \textbf{$E_{corr}$} & \textbf{\% від $E_{HF}$} \\
        \hline
        He            & $-2.8617 $        & $-2.9037  $          & $-0.0420 $          & $1.5$\%                  \\
        Be            & $-14.573 $        & $-14.667  $          & $-0.094  $          & $0.6$\%                  \\
        Ne            & $-128.547$        & $-128.937 $          & $-0.390  $          & $0.3$\%                  \\
        Ar            & $-526.817$        & $-527.540 $          & $-0.723  $          & $0.1$\%                  \\
        \hline
    \end{tabular}
\end{center}


%% --------------------------------------------------------
\subsection{Концепція одно- та багатоконфігураційних методів}
%% --------------------------------------------------------


\paragraph{Одноконфігураційні методи}

Базуються на одному детермінанті Слейтера (HF) і додають кореляцію через збудження:

У найпростішому випадку хвильова функція є одним детермінантом:
\begin{equation}
    |\Psi_{\text{SR}}\rangle = |\Phi_0\rangle
    = |\phi_1 \phi_2 \dots \phi_N|.
\end{equation}

Далі на нього накладають кореляційні поправки --- не у вигляді нових детермінантів у самій хвильовій функції, а через операторні або збурювальні члени у виразі для енергії.
Це дає <<однореференсні>> методи:

\[
    E = E_{\text{HF}} + E_{\text{corr}},
\]
де \(E_{\text{corr}}\) визначається з теорії збурень (MP2, MP3...), або з експоненційного кластерного розкладу
\[
    |\Psi_{\text{CC}}\rangle = e^{\hat{T}}|\Phi_0\rangle,
    \quad
    \hat{T} = \hat{T}_1 + \hat{T}_2 + \dots,
\]
у методі CCSD тощо.

До однодетермінантних методів належать наступні:
\begin{itemize}
    \item MP2, MP3, MP4 --- теорія збурень
    \item CCSD, CCSD(T) --- coupled cluster
    \item CISD, QCISD --- configuration interaction
\end{itemize}

Добре працюють для основних станів, де один детермінант домінує у хвильовій функції.


\paragraph{Багатоконфігураційні методи}

Коли один детермінант не здатен описати систему (наприклад, при розриві хімічного зв’язку), хвильова функція має вигляд:
\begin{equation}
    |\Psi_{\text{MR}}\rangle = \sum_I c_I |\Phi_I\rangle.
\end{equation}

Кожен детермінант \(|\Phi_I\rangle\) утворюється перестановкою заповнених і вільних орбіталей у певному активному просторі.
Наприклад, у методі CASSCF(4,4) ми розглядаємо всі можливі розподіли 4 електронів по 4 активних орбіталях.
Це породжує набір детермінантів:
\[
    \{|\Phi_I\rangle\} =
    \{ |(2s)^2(2p_x)^2|,\ |(2s)^1(2p_x)^1(2p_y)^2|,\ \dots \},
\]
а хвильова функція є їхньою суперпозицією з коефіцієнтами \(c_I\), які обчислюються варіаційно.


До багатоконфігураційних належать методи:
\begin{itemize}
    \item CASSCF --- вибір активного простору
    \item CASPT2 --- додавання динамічної кореляції до CASSCF
    \item MRCI --- multireference CI
\end{itemize}

Необхідні для: розриву зв'язків, збуджених станів, діелектронних систем.

\inputcode{code_3.py}

%% --------------------------------------------------------
\subsection{Коли потрібні post-HF методи?}
%% --------------------------------------------------------

\begin{enumerate}
    \item \textbf{Спектроскопічна точність} --- похибка < 1 kcal/mol (0.04 eV)
    \item \textbf{Порівняння близьких станів} --- різниці енергій між мультиплетами
    \item \textbf{Електронна спорідненість} --- HF і DFT часто не достатньо
    \item \textbf{Еталонні розрахунки} --- для валідації DFT функціоналів
    \item \textbf{Системи з сильною кореляцією} --- перехідні метали, f-елементи
\end{enumerate}


%% --------------------------------------------------------
\section{Теорія збурень Møller-Plesset (MP2)}
%% --------------------------------------------------------

%% --------------------------------------------------------
\subsection{Теоретичні основи MP2}
%% --------------------------------------------------------

Теорія збурень Møller-Plesset розділяє гамільтоніан на незбурену частину (HF) та збурення:

\begin{equation}
    \hat{H} = \hat{H}_0 + \lambda \hat{V}
\end{equation}

де $\hat{H}_0$ --- оператор Фока, а $\hat{V}$ --- різниця між точним гамільтоніаном та HF.

Енергія розкладається в ряд за степенями $\lambda$:

\begin{equation}
    E = E^{(0)} + E^{(1)} + E^{(2)} + E^{(3)} + \ldots
\end{equation}

де:
\begin{itemize}
    \item $E^{(0)} + E^{(1)} = E_{HF}$ --- енергія Хартрі-Фока
    \item $E^{(2)}$ --- MP2 корекція (перший порядок кореляції)
    \item $E^{(3)}, E^{(4)}$ --- вищі порядки (MP3, MP4)
\end{itemize}


%% --------------------------------------------------------
\subsection{Формула MP2 кореляційної енергії}
%% --------------------------------------------------------

Для замкненої оболонки (RMP2):

\begin{equation}
    E^{(2)} = -\sum_{i<j}^{\text{occ}} \sum_{a<b}^{\text{virt}} \frac{|\langle ij||ab \rangle|^2}{\varepsilon_i + \varepsilon_j - \varepsilon_a - \varepsilon_b}
\end{equation}

де:
\begin{itemize}
    \item $i, j$ --- заповнені орбіталі
    \item $a, b$ --- віртуальні орбіталі
    \item $\langle ij||ab \rangle$ --- антисиметризовані двоелектронні інтеграли
    \item $\varepsilon$ --- орбітальні енергії
\end{itemize}

\subsection{MP2 розрахунок атома Неону}

\inputcode{code_4.py}


%% --------------------------------------------------------
\subsection{MP2 для відкритих оболонок (UMP2)}
%% --------------------------------------------------------

Для систем з неспареними електронами використовується UMP2:

\inputcode{code_5.py}


%% --------------------------------------------------------
\subsection{Систематичне порівняння HF vs MP2}
%% --------------------------------------------------------

\inputcode{code_6.py}


%% --------------------------------------------------------
\subsection{Залежність MP2 від базисного набору}
%% --------------------------------------------------------

MP2 сильно залежить від базису, особливо потрібні функції високих $l$:

\inputcode{code_7.py}

%% --------------------------------------------------------
\subsection{Переваги та недоліки MP2}
%% --------------------------------------------------------

\paragraph{Переваги:}
\begin{itemize}
    \item Відносно швидкий ($\mathcal{O}(N^5)$) порівняно з іншими post-HF
    \item Добре відновлює динамічну кореляцію
    \item Size-consistent (правильне масштабування для великих систем)
    \item Доступний для великих атомів/молекул
    \item Добрий базис для вищих методів (MP3, MP4)
\end{itemize}

\paragraph{Недоліки:}
\begin{itemize}
    \item Не виправляє забруднення спіном UHF
    \item Може розходитись для систем з малим gap HOMO-LUMO
    \item Погано працює для сильно корельованих систем
    \item Потребує великих базисів для точних результатів
    \item Не варіаційний (може давати енергію нижчу за точну)
\end{itemize}

%% --------------------------------------------------------
\subsection{Рекомендації по використанню MP2}
%% --------------------------------------------------------

\begin{table}[h]
    \centering
    \caption{Коли використовувати MP2}
    \label{tab:mp2_recommendations}
    \small
    \begin{tabular}{ll}
        \hline
        \textbf{Добре для:}     & \textbf{Погано для:}          \\
        \hline
        Замкнені оболонки       & Системи з малим HOMO-LUMO gap \\
        Якісні оцінки кореляції & Розрив зв'язків               \\
        Великі системи          & Збуджені стани                \\
        Слабкі взаємодії        & Перехідні стани               \\
        Відносні енергії        & Діелектронні системи          \\
        \hline
    \end{tabular}
\end{table}

%% --------------------------------------------------------
\section{Coupled Cluster методи (CCSD, CCSD(T))}
%% --------------------------------------------------------

%% --------------------------------------------------------
\subsection{Теоретичні основи Coupled Cluster}
%% --------------------------------------------------------

Coupled Cluster (CC) --- один з найточніших методів для врахування електронної кореляції. На відміну від CI, CC є size-extensive (правильно масштабується для великих систем).

Хвильова функція у CC записується через експоненціальний оператор збудження:

\begin{equation}
    |\Psi_{CC}\rangle = e^{\hat{T}} |\Phi_0\rangle
\end{equation}

де $|\Phi_0\rangle$ --- HF детермінант, а $\hat{T}$ --- оператор кластерів:

\begin{equation}
    \hat{T} = \hat{T}_1 + \hat{T}_2 + \hat{T}_3 + \ldots
\end{equation}

\paragraph{CCSD (Coupled Cluster Singles and Doubles)}
Враховує тільки $\hat{T}_1$ (одиночні збудження) та $\hat{T}_2$ (подвійні збудження):

\begin{align}
    \hat{T}_1 & = \sum_{i}^{\text{occ}} \sum_{a}^{\text{virt}} t_i^a \, \hat{a}^\dagger_a \hat{a}_i                                                 \\
    \hat{T}_2 & = \frac{1}{4} \sum_{ij}^{\text{occ}} \sum_{ab}^{\text{virt}} t_{ij}^{ab} \, \hat{a}^\dagger_a \hat{a}^\dagger_b \hat{a}_j \hat{a}_i
\end{align}

\paragraph{CCSD(T) --- <<золотий стандарт>>}
Додає пертурбативну корекцію для потрійних збуджень $\hat{T}_3$:

\begin{equation}
    E_{CCSD(T)} = E_{CCSD} + E_{(T)}
\end{equation}

де $E_{(T)}$ обчислюється за формулами четвертого порядку теорії збурень.


%% --------------------------------------------------------
\subsection{CCSD розрахунок атома Гелію}
%% --------------------------------------------------------

\inputcode{code_8.py}

%% --------------------------------------------------------
\subsection{CCSD для відкритих оболонок (UCCSD)}
%% --------------------------------------------------------

\inputcode{code_9.py}

%% --------------------------------------------------------
\subsection{Порівняння ієрархії методів}
%% --------------------------------------------------------

\inputcode{code_10.py}

%% --------------------------------------------------------
\subsection{T1 діагностика та багатоконфігураційний характер}
%% --------------------------------------------------------

T1 діагностика --- індикатор того, наскільки система є одноконфігураційною:

\begin{equation}
    T_1 = \frac{\|t_1\|}{\sqrt{N_{\text{elec}}}}
\end{equation}

\inputcode{code_11.py}

%% --------------------------------------------------------
\subsection{Обчислювальна складність та масштабування}
%% --------------------------------------------------------

\begin{table}[h]
    \centering
    \caption{Порівняння обчислювальної складності методів}
    \label{tab:cc_complexity}
    \begin{tabular}{lccc}
        \hline
        \textbf{Метод} & \textbf{Масштабування} & \textbf{Пам'ять}   & \textbf{Disk}      \\
        \hline
        HF             & $\mathcal{O}(N^4)$     & $\mathcal{O}(N^2)$ & Мало               \\
        MP2            & $\mathcal{O}(N^5)$     & $\mathcal{O}(N^4)$ & $\mathcal{O}(N^4)$ \\
        CCSD           & $\mathcal{O}(N^6)$     & $\mathcal{O}(N^4)$ & $\mathcal{O}(N^4)$ \\
        CCSD(T)        & $\mathcal{O}(N^7)$     & $\mathcal{O}(N^4)$ & $\mathcal{O}(N^4)$ \\
        CCSDT          & $\mathcal{O}(N^8)$     & $\mathcal{O}(N^6)$ & $\mathcal{O}(N^6)$ \\
        FCI            & $\mathcal{O}(e^N)$     & $\mathcal{O}(e^N)$ & $\mathcal{O}(e^N)$ \\
        \hline
    \end{tabular}
\end{table}

де $N$ --- характерний розмір системи (базисних функцій або електронів).

%% --------------------------------------------------------
\section{Configuration Interaction (CI) та Full CI}
%% --------------------------------------------------------

%% --------------------------------------------------------
\subsection{Основи Configuration Interaction}
%% --------------------------------------------------------

Configuration Interaction (CI) --- варіаційний метод, де хвильова функція представлена як лінійна комбінація детермінантів Слейтера:

\begin{equation}
    |\Psi_{CI}\rangle = c_0 |\Phi_0\rangle + \sum_i^{\text{occ}} \sum_a^{\text{virt}} c_i^a |\Phi_i^a\rangle + \sum_{i<j} \sum_{a<b} c_{ij}^{ab} |\Phi_{ij}^{ab}\rangle + \ldots
\end{equation}

де:
\begin{itemize}
    \item $|\Phi_0\rangle$ --- HF детермінант (основна конфігурація)
    \item $|\Phi_i^a\rangle$ --- одиночно збуджені детермінанти (S)
    \item $|\Phi_{ij}^{ab}\rangle$ --- подвійно збуджені детермінанти (D)
    \item $|\Phi_{ijk}^{abc}\rangle$ --- потрійно збуджені детермінанти (T)
    \item і т.д.
\end{itemize}

\paragraph{Варіанти CI методу:}
\begin{itemize}
    \item \textbf{CIS} (CI Singles) --- тільки одиночні збудження (для збуджених станів)
    \item \textbf{CISD} (CI Singles and Doubles) --- S + D
    \item \textbf{CISDT} --- S + D + T
    \item \textbf{Full CI} --- всі можливі збудження (точний у межах базису)
\end{itemize}


%% --------------------------------------------------------
\subsection{Full CI --- точний розв'язок}
%% --------------------------------------------------------

Full CI включає всі можливі детермінанти і дає точну хвильову функцію та енергію в межах обраного базису:

\begin{equation}
    E_{FCI} = \min_{\{c_i\}} \langle \Psi_{FCI} | \hat{H} | \Psi_{FCI} \rangle
\end{equation}

\textbf{Проблема:} Кількість детермінантів зростає як:
\begin{equation}
    N_{det} = \binom{N_{orb}}{N_{\alpha}} \times \binom{N_{orb}}{N_{\beta}}
\end{equation}

Для 10 електронів у 20 орбіталях: $N_{det} \approx 10^{10}$ детермінантів!

\inputcode{FullCI.py}
%% --------------------------------------------------------
\subsection{CISD розрахунки}
%% --------------------------------------------------------

CISD --- практичний варіант CI, але не є size-extensive:

\inputcode{CISDcalc.py}

%% --------------------------------------------------------
\subsection{Порівняння CI та CC методів}
%% --------------------------------------------------------

\inputcode{CIvsCD.py}

%% --------------------------------------------------------
\subsection{Size-extensivity проблема CI}
%% --------------------------------------------------------

Важлива відмінність між CI та CC:

\begin{equation}
    E_{CI}(2A) \neq 2 \times E_{CI}(A)
\end{equation}

\begin{equation}
    E_{CC}(2A) = 2 \times E_{CC}(A)
\end{equation}

\inputcode{SizeExtensivityExample.py}


%% --------------------------------------------------------
\subsection{Коли використовувати CI vs CC}
%% --------------------------------------------------------
\nopagebreak
\begin{center}
    \captionof{table}{Порівняння CI та CC методів}
    \label{tab:ci_vs_cc}
    \small
    \begin{tblr}{lll}
        \hline
        \textbf{Критерій}       & \textbf{CI}      & \textbf{CC}      \\
        \hline
        Варіаційність           & Так              & Ні               \\
        Size-extensivity        & Ні (окрім FCI)   & Так              \\
        Точність                & CISD < CCSD      & CCSD(T) ≈ FCI    \\
        Збуджені стани          & EOM-CI, CIS      & EOM-CC           \\
        Швидкість               & CISD швидше CCSD & Повільніше       \\
        Стабільність            & Варіаційна межа  & Може розходитись \\
        Багатоконфігураційність & Природно (FCI)   & Потребує MRCC    \\
        \hline
    \end{tblr}
\end{center}

\textbf{Рекомендації:}
\begin{itemize}
    \item \textbf{Використовуйте FCI} для малих систем як еталон
    \item \textbf{Використовуйте CCSD(T)} для точних розрахунків одноконфігураційних систем
    \item \textbf{Використовуйте CISD} для швидких оцінок кореляції
    \item \textbf{Для багатоконфігураційних систем} використовуйте CASSCF/CASPT2
\end{itemize}

%%% --------------------------------------------------------
%\subsection{Ієрархія методів}
%%% --------------------------------------------------------
%
%Post-HF методи утворюють ієрархію за точністю та обчислювальною складністю:
%
%\begin{enumerate}
%    \item \textbf{HF} --- базовий рівень, без кореляції
%    \item \textbf{MP2} --- $\mathcal{O}(N^5)$ --- найпростіша кореляція
%    \item \textbf{MP3, MP4} --- $\mathcal{O}(N^6), \mathcal{O}(N^7)$ --- вищі порядки теорії збурень
%    \item \textbf{CCSD} --- $\mathcal{O}(N^6)$ --- надійна динамічна кореляція
%    \item \textbf{CCSD(T)} --- $\mathcal{O}(N^7)$ --- "золотий стандарт" квантової хімії
%    \item \textbf{Full CI} --- $\mathcal{O}(e^N)$ --- точний розв'язок (у межах базису)
%\end{enumerate}
%
%\inputcode{code_1.py}

%\subsection{Важливість кореляції для різних властивостей}
%
%\begin{center}
%\captionof{table}{Вплив електронної кореляції на різні властивості}
%\label{tab:correlation_importance}
%\small
%\begin{tblr}{lll}
%\hline
%\textbf{Властивість} & \textbf{Важливість кореляції} & \textbf{Рекомендований метод} \\
%\hline
%Абсолютні енергії & Помірна & MP2, DFT \\
%Енергії іонізації & Висока & CCSD(T) \\
%Електронна спорідненість & Дуже висока & CCSD(T) + дифузні \\
%Енергії збудження & Висока & EOM-CCSD, TD-DFT \\
%Відстані між станами & Висока & CASPT2, MRCI \\
%Дисоціація & Критична & CASSCF, MRCI \\
%Слабкі взаємодії & Критична & CCSD(T), MP2-F12 \\
%\hline
%\end{tblr}
%\end{center}
%
%
%%% --------------------------------------------------------
%\subsection{Порівняння методів для \ce{He}}
%%% --------------------------------------------------------
%
%Розглянемо найпростішу багатоелектронну систему --- атом Гелію:
%
%\inputcode{code_2.py}

%% --------------------------------------------------------
\section{CASSCF --- багатоконфігураційний метод}
%% --------------------------------------------------------

%% --------------------------------------------------------
\subsection{Ідея Complete Active Space Self-Consistent Field}
%% --------------------------------------------------------

CASSCF (Complete Active Space SCF) --- багатоконфігураційний метод, який поєднує:
\begin{itemize}
    \item \textbf{Активний простір} --- набір орбіталей, де проводиться Full CI
    \item \textbf{SCF оптимізацію} --- орбіталі та CI коефіцієнти оптимізуються одночасно
\end{itemize}

Орбіталі розділяються на три групи:
\begin{enumerate}
    \item \textbf{Core (остов)} --- завжди подвійно заповнені, не беруть участь у кореляції
    \item \textbf{Active (активні)} --- електрони розподілені по всіх можливих способах (Full CI)
    \item \textbf{Virtual (віртуальні)} --- завжди порожні
\end{enumerate}

\begin{equation}
    |\Psi_{CASSCF}\rangle = \sum_I c_I |\Phi_I^{\text{active}}\rangle \otimes |\text{core}\rangle
\end{equation}

%% --------------------------------------------------------
\subsection{Позначення CASSCF(n,m)}
%% --------------------------------------------------------

CASSCF(n,m) означає:
\begin{itemize}
    \item $n$ --- кількість електронів в активному просторі
    \item $m$ --- кількість орбіталей в активному просторі
\end{itemize}

\textbf{Приклади:}
\begin{itemize}
    \item \textbf{Be}: CASSCF(4,8) --- 4 валентні електрони у 8 орбіталях (2s, 2p, 3s, 3p)
    \item \textbf{C}: CASSCF(4,4) --- 4 валентні електрони у 4 орбіталях (2s, 2p)
    \item \textbf{Cr}: CASSCF(6,5) --- 6 електронів у 5 d-орбіталях
\end{itemize}

%% --------------------------------------------------------
\subsection{CASSCF розрахунок атома Берилію}
%% --------------------------------------------------------

Берилій має близькі за енергією конфігурації 2s² та 2p², що робить його класичним прикладом для CASSCF:

\inputcode{CASSCFBe.py}

%% --------------------------------------------------------
\subsection{CASSCF для перехідних металів}
%% --------------------------------------------------------


Для перехідних металів d-орбіталі часто близькі за енергією, потребуючи багатоконфігураційного підходу:

\inputcode{CASSCFTransitionMetals.py}


%% --------------------------------------------------------
\subsection{Вибір активного простору}
%% --------------------------------------------------------

Правильний вибір активного простору критичний для CASSCF:

\begin{table}[h]
    \centering
    \caption{Рекомендації по вибору активного простору}
    \label{tab:casscf_active_space}
    \small
    \begin{tabular}{lll}
        \hline
        \textbf{Система} & \textbf{Активний простір} & \textbf{Коментар}          \\
        \hline
        Be               & (4,8): 2s,2p,3s,3p        & Враховує 2s²↔2p²           \\
        C                & (4,4): 2s,2p              & Мінімальний валентний      \\
        O                & (6,6): 2s,2p + корел.     & З кореляційними орбіталями \\
        3d метали        & (n,5): 3d                 & Тільки d-орбіталі          \\
        3d метали        & (n+2,6): 3d,4s            & d + s орбіталі             \\
        Лантаноїди       & (n,7): 4f                 & f-орбіталі                 \\
        \hline
    \end{tabular}
\end{table}

\textbf{Принципи вибору:}
\begin{enumerate}
    \item Включити орбіталі, близькі за енергією
    \item Включити орбіталі з значною хімічною активністю
    \item Більший простір → точніше, але експоненційно повільніше
    \item Перевірити заселеності природних орбіталей
    \item Якщо заселеності ≈ 2 або ≈ 0, можна виключити з активного простору
\end{enumerate}


%% --------------------------------------------------------
\subsection{State-averaged CASSCF}
%% --------------------------------------------------------

Для розрахунку декількох станів одночасно (наприклад, різних мультиплетів):

\inputcode{StateAvaragedCASSCF.py}

%% --------------------------------------------------------
\subsection{CASPT2 --- динамічна кореляція поверх CASSCF}
%% --------------------------------------------------------

CASSCF враховує тільки статичну кореляцію. Для повної точності потрібно додати динамічну кореляцію через CASPT2:

\inputcode{CASPT2.py}

%% --------------------------------------------------------
\subsection{Переваги та недоліки CASSCF}
%% --------------------------------------------------------

\paragraph{Переваги:}
\begin{itemize}
    \item Правильно описує статичну кореляцію
    \item Може розрахувати декілька станів одночасно
    \item Варіаційний метод
    \item Підходить для розриву зв'язків, збуджених станів
    \item Дає природні орбіталі та заселеності
\end{itemize}

\paragraph{Недоліки:}
\begin{itemize}
    \item Вибір активного простору не завжди очевидний
    \item Експоненційне масштабування з розміром активного простору
    \item Не враховує динамічну кореляцію (потрібен CASPT2)
    \item Може бути проблеми з конвергенцією
    \item Потребує досвіду для правильного використання
\end{itemize}

\paragraph{Коли використовувати CASSCF:}
\begin{itemize}
    \item T1 діагностика > 0.05 (сильна багатоконфігураційність)
    \item Перехідні метали з близькими d-орбіталями
    \item Збуджені стани атомів
    \item Діелектронні системи
    \item Відкрито-оболонкові синглети
\end{itemize}


%% --------------------------------------------------------
\subsection{Ієрархія методів}
%% --------------------------------------------------------

Post-HF методи утворюють ієрархію за точністю та обчислювальною складністю:

\begin{enumerate}
    \item \textbf{HF} --- базовий рівень, без кореляції
    \item \textbf{MP2} --- $\mathcal{O}(N^5)$ --- найпростіша кореляція
    \item \textbf{MP3, MP4} --- $\mathcal{O}(N^6), \mathcal{O}(N^7)$ --- вищі порядки теорії збурень
    \item \textbf{CCSD} --- $\mathcal{O}(N^6)$ --- надійна динамічна кореляція
    \item \textbf{CCSD(T)} --- $\mathcal{O}(N^7)$ --- "золотий стандарт" квантової хімії
    \item \textbf{Full CI} --- $\mathcal{O}(e^N)$ --- точний розв'язок (у межах базису)
\end{enumerate}


%% --------------------------------------------------------
\subsection{Важливість кореляції для різних властивостей}
%% --------------------------------------------------------

\begin{center}
    \captionof{table}{Вплив електронної кореляції на різні властивості}
    \label{tab:correlation_importance}
    \small
    \begin{tblr}{lll}
        \hline
        \textbf{Властивість}     & \textbf{Важливість кореляції} & \textbf{Рекомендований метод} \\
        \hline
        Абсолютні енергії        & Помірна                       & MP2, DFT                      \\
        Енергії іонізації        & Висока                        & CCSD(T)                       \\
        Електронна спорідненість & Дуже висока                   & CCSD(T) + дифузні             \\
        Енергії збудження        & Висока                        & EOM-CCSD, TD-DFT              \\
        Відстані між станами     & Висока                        & CASPT2, MRCI                  \\
        Дисоціація               & Критична                      & CASSCF, MRCI                  \\
        Слабкі взаємодії         & Критична                      & CCSD(T), MP2-F12              \\
        \hline
    \end{tblr}
\end{center}

\inputcode{code_12.py}


%% --------------------------------------------------------
\subsection{Порівняння методів для He}
%% --------------------------------------------------------

Розглянемо найпростішу багатоелектронну систему --- атом Гелію:

\inputcode{code_13.py}


\section{Практичні завдання}

\subsection{Завдання 1: Порівняння методів для He}

\begin{minted}{python}
"""
ЗАВДАННЯ 1: Для атома Гелію розрахуйте енергії методами
HF, MP2, CCSD, CCSD(T) та FCI з базисами cc-pVDZ, cc-pVTZ,
cc-pVQZ.

1) Побудуйте графік збіжності кожного методу до базисної межі
2) Екстраполюйте до CBS за формулою E(X) = E_CBS + A/X³
3) Порівняйте з експериментальним значенням -2.90372 Ha
4) Яка частка кореляційної енергії відновлюється кожним методом?

Базиси: cc-pVDZ, cc-pVTZ, cc-pVQZ
"""

# Ваш код тут
\end{minted}

\subsection*{Завдання 2: T1 діагностика для другого періоду}

\begin{minted}{python}
"""
ЗАВДАННЯ 2: Розрахуйте T1 діагностику для всіх атомів
другого періоду (Li-Ne) з базисом cc-pVDZ.

1) Для яких атомів T1 > 0.02 (багатоконфігураційні)?
2) Чи є кореляція між T1 та положенням у періоді?
3) Порівняйте T1 для основного та збудженого спінового
   стану Карбону (³P vs ¹D)
4) Побудуйте графік залежності T1 від атомного номера

Базис: cc-pVDZ
"""

# Ваш код тут
\end{minted}

\subsection*{Завдання 3: Size-extensivity тест}

\begin{minted}{python}
"""
ЗАВДАННЯ 3: Дослідіть size-extensivity для атомів Ne.

Розрахуйте методами HF, MP2, CISD, CCSD:
1) Енергію одного атома Ne
2) Енергію двох атомів Ne на відстані 100 Bohr
3) Обчисліть похибку: E(2×Ne) - 2×E(Ne)

Питання:
- Який метод є size-extensive?
- Яка похибка для CISD (у mHa)?
- Чому MP2 є size-extensive, а CISD - ні?

Базис: cc-pVTZ
"""

# Ваш код тут
\end{minted}

\subsection*{Завдання 4: CASSCF для перехідного металу}

\begin{minted}{python}
"""
ЗАВДАННЯ 4: CASSCF аналіз атома Титану (Ti).

Ti: [Ar] 3d² 4s², основний стан ³F

1) Розрахуйте UHF, CCSD та CASSCF(4,5) енергії
   (4 електрони у 5 d-орбіталях)
2) Проаналізуйте заселеності природних орбіталей
3) Обчисліть статичну кореляцію (CASSCF - HF)
4) Спробуйте різні активні простори: (4,5), (4,6), (6,11)
   Який найкращий?

Базис: def2-SVP
"""

# Ваш код тут
\end{minted}

\subsection*{Завдання 5: Енергії збудження}

\begin{minted}{python}
"""
ЗАВДАННЯ 5: Розрахуйте енергію збудження ³P → ¹D для
атома Карбону.

Використайте методи:
1) ΔSCF (окремі розрахунки для кожного стану)
2) State-averaged CASSCF
3) EOM-CCSD (якщо доступно)

Порівняйте з експериментальним значенням 1.26 eV.

Який метод дає найточніший результат?

Базис: aug-cc-pVTZ
"""

# Ваш код тут
\end{minted}

\subsection*{Завдання 6: Кореляційна енергія vs Z}

\begin{minted}{python}
"""
ЗАВДАННЯ 6: Дослідіть залежність кореляційної енергії
від атомного номера.

Розрахуйте MP2 кореляційну енергію для:
- Благородних газів: He, Ne, Ar, Kr
- У абсолютних величинах (mHa)
- У відсотках від повної енергії

Побудуйте графіки:
1) |E_corr| vs Z
2) |E_corr|/|E_HF| vs Z

Яка тенденція спостерігається?

Базис: cc-pVTZ
"""

# Ваш код тут
\end{minted}

%% --------------------------------------------------------
\section{Резюме}
%% --------------------------------------------------------

У цьому розділі ми детально вивчили Post-Hartree-Fock методи для врахування електронної кореляції:

\subsection{Основні методи та їх характеристики}

\begin{table}[h]
\centering
\caption{Порівняння Post-HF методів}
\label{tab:posthf_summary}
\small
\begin{tabular}{llcccc}
\hline
\textbf{Метод} & \textbf{Складність} & \textbf{Size-ext.} & \textbf{Варіац.} & \textbf{Точність} & \textbf{МК?} \\
\hline
MP2 & $O(N^5)$ & Так & Ні & Помірна & Ні \\
MP3/MP4 & $O(N^6/N^7)$ & Так & Ні & Добра & Ні \\
CISD & $O(N^6)$ & Ні & Так & Помірна & Ні \\
CCSD & $O(N^6)$ & Так & Ні & Добра & Ні \\
CCSD(T) & $O(N^7)$ & Так & Ні & Відмінна & Ні \\
CASSCF & $O(e^{n_{act}})$ & Так & Так & Статична & Так \\
CASPT2 & $O(e^{n_{act}})$ & Так & Ні & Відмінна & Так \\
Full CI & $O(e^N)$ & Так & Так & Точна & Так \\
\hline
\end{tabular}
\end{table}

\textit{МК = Багатоконфігураційний}

\subsection{Ключові висновки}

\paragraph{1. Електронна кореляція}
\begin{itemize}
    \item Становить 1-5\% від повної енергії, але критична для точних розрахунків
    \item Поділяється на динамічну (короткодіючу) та статичну (близькі стани)
    \item HF і DFT не враховують кореляцію повністю
\end{itemize}

\paragraph{2. MP2 --- найпростіший post-HF метод}
\begin{itemize}
    \item Швидкий ($O(N^5)$), size-extensive
    \item Добре працює для замкнених оболонок з великим HOMO-LUMO gap
    \item Відновлює 80-95\% кореляційної енергії
    \item Не виправляє забруднення спіном UHF
    \item Потребує великих базисів для збіжності
\end{itemize}

\paragraph{3. CCSD(T) --- "золотий стандарт"}
\begin{itemize}
    \item Найточніший одноконфігураційний метод
    \item Size-extensive, відновлює >99\% кореляції
    \item Повільний ($O(N^7)$), але надійний
    \item Виправляє забруднення спіном
    \item Обмежений розміром системи (до ~20-30 атомів)
\end{itemize}

\paragraph{4. T1 діагностика}
\begin{itemize}
    \item T1 < 0.02: одноконфігураційна система (CCSD надійний)
    \item 0.02 < T1 < 0.05: слабка багатоконфігураційність
    \item T1 > 0.05: потрібен CASSCF/MRCI
\end{itemize}

\paragraph{5. CI методи}
\begin{itemize}
    \item Варіаційні (енергія завжди вище точної)
    \item Full CI --- точний у межах базису, але непрактичний
    \item CISD не є size-extensive (на відміну від CCSD)
    \item Корисні для аналізу хвильової функції
\end{itemize}

\paragraph{6. CASSCF --- для багатоконфігураційних систем}
\begin{itemize}
    \item Необхідний коли T1 > 0.05 або близькі стани
    \item Вибір активного простору критичний
    \item Враховує статичну кореляцію
    \item Потребує CASPT2/NEVPT2 для динамічної кореляції
    \item Ідеальний для перехідних металів, збуджених станів
\end{itemize}

\subsection{Рекомендації по використанню}

\begin{table}[h]
\centering
\caption{Вибір методу залежно від задачі}
\label{tab:method_selection}
\small
\begin{tabular}{lll}
\hline
\textbf{Задача} & \textbf{Рекомендований метод} & \textbf{Альтернатива} \\
\hline
Швидка оцінка кореляції & MP2 & DFT \\
Точні енергії (1 kcal/mol) & CCSD(T) & --- \\
Замкнені оболонки & CCSD(T), MP2 & DFT \\
Відкриті оболонки & UCCSD(T) & ROHF-CCSD \\
Енергії збудження & EOM-CCSD, CASSCF & TD-DFT \\
Перехідні метали & CASSCF, CASPT2 & DFT+U \\
Багатоконфігураційні & CASSCF/CASPT2 & MRCI \\
Еталонні розрахунки & CCSD(T)/CBS & Full CI \\
Великі системи & MP2, DFT & DLPNO-CCSD(T) \\
\hline
\end{tabular}
\end{table}

\subsection{Типові помилки}

\begin{enumerate}
    \item \textbf{Використання MP2 для систем з малим gap} --- може давати нефізичні результати
    \item \textbf{Забування перевірки T1 діагностики} --- CCSD може бути ненадійним
    \item \textbf{Занадто малий активний простір у CASSCF} --- не врахує всю статичну кореляцію
    \item \textbf{Занадто великий активний простір} --- неможливо розрахувати
    \item \textbf{Недостатній базис} --- post-HF методи дуже чутливі до базису
    \item \textbf{Ігнорування size-extensivity} --- CISD не підходить для енергій дисоціації
    \item \textbf{CASSCF без CASPT2} --- відсутня динамічна кореляція
\end{enumerate}

\subsection{Практичні поради}

\begin{enumerate}
    \item Завжди починайте з HF розрахунку та перевірте конвергенцію
    \item Для нових систем спочатку протестуйте на малому базисі
    \item Перевіряйте T1 діагностику після CCSD
    \item Для CASSCF: починайте з малого активного простору, збільшуйте поступово
    \item Аналізуйте природні орбіталі для вибору активного простору
    \item Використовуйте симетрію коли можливо
    \item Для екстраполяції до CBS потрібні принаймні 3 базиси
    \item Зберігайте проміжні результати (checkpoint файли)
\end{enumerate}

\subsection{Обчислювальні ресурси}

\begin{table}[h]
\centering
\caption{Орієнтовний час розрахунку (відносно HF)}
\label{tab:computational_cost}
\small
\begin{tabular}{lcccc}
\hline
\textbf{Метод} & \textbf{He (DZ)} & \textbf{Ne (TZ)} & \textbf{Ar (DZ)} & \textbf{Пам'ять} \\
\hline
HF & 1× & 1× & 1× & $N^2$ \\
MP2 & 2× & 5× & 10× & $N^4$ \\
CCSD & 10× & 50× & 200× & $N^4$ \\
CCSD(T) & 20× & 200× & 1000× & $N^4$ \\
CASSCF(4,8) & 5× & --- & --- & $e^{n_{act}}$ \\
\hline
\end{tabular}
\end{table}

\textit{Примітка: реальний час залежить від якості початкового наближення та конвергенції}

\subsection{Подальше вивчення}

Для поглибленого розуміння post-HF методів рекомендуємо:

\begin{itemize}
    \item \textbf{Книги:}
    \begin{itemize}
        \item T. Helgaker et al., "Molecular Electronic-Structure Theory"
        \item I. Shavitt, R. J. Bartlett, "Many-Body Methods in Chemistry and Physics"
        \item R. J. Bartlett, M. Musiał, "Coupled-cluster theory in quantum chemistry"
    \end{itemize}

    \item \textbf{Огляди:}
    \begin{itemize}
        \item CCSD(T): Reviews in Computational Chemistry, Vol. 14
        \item CASSCF: Chemical Reviews 2012, 112, 108
        \item Post-HF methods: Wiley Interdisciplinary Reviews: Computational Molecular Science
    \end{itemize}

    \item \textbf{Документація PySCF:}
    \begin{itemize}
        \item \url{https://pyscf.org/user/mp.html} --- MP2, MP3
        \item \url{https://pyscf.org/user/cc.html} --- Coupled Cluster
        \item \url{https://pyscf.org/user/mcscf.html} --- CASSCF, CASPT2
        \item \url{https://pyscf.org/user/fci.html} --- Full CI
    \end{itemize}
\end{itemize}

\subsection{Що далі?}

У наступному розділі ми розглянемо аналіз результатів квантово-хімічних розрахунків: орбітальні енергії, густини, заселеності, спектроскопічні властивості та візуалізацію даних.

