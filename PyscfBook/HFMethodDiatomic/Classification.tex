%% --------------------------------------------------------
\section{Орбіталі і електронні стани молекул}
%% --------------------------------------------------------

Електронна структура двоатомних молекул описується двома рівнями:
\begin{enumerate}
    \item класифікацією \textbf{молекулярних орбіталей (МО)};
    \item класифікацією \textbf{електронних станів молекули}, які утворюються внаслідок заповнення цих МО електронами.
\end{enumerate}

%% --------------------------------------------------------
\subsection{Класифікація молекулярних орбіталей}
%% --------------------------------------------------------

Молекулярні орбіталі позначаються символами
$\sigma$, $\pi$, $\delta$, $\varphi$, \ldots,
що відповідають проєкції орбітального моменту $\Lambda$ на між'ядерну вісь:
\[
\Lambda = 0,1,2,3,\ldots \quad \Rightarrow \quad
\sigma,\, \pi,\, \delta,\, \varphi, \ldots
\]

Кожна орбіталь може бути:
\begin{itemize}
    \item \textbf{зв’язувальною} або \textbf{антизв’язувальною} (позначається зірочкою ${}^*$);
    \item \textbf{симетричною (gerade, $g$)} або \textbf{антисиметричною (ungerade, $u$)} відносно інверсії в центрі молекули.
\end{itemize}

Таким чином, типовий запис молекулярної орбіталі має вигляд:
\[
\sigma_g, \quad \pi_u, \quad \sigma_u^*, \quad \pi_g^*, \ldots
\]

%\textbf{Приклади класифікації молекулярних орбіталей:}
%
%\noindent%
%\begin{tblr}{
%  colspec = {X[c] X[l] X[l] X[l] X[c, m]},
%  row{1} = {font=\bfseries, bg=gray!10, c},
%  hlines = {1pt,2pt},
%  vlines = {0.6pt},
%}
%МО & Симетрія & Тип & Характер і походження & Схематичне перекривання \\
%
%$\sigma_g$ & симетрична (gerade) & зв’язувальна & Лінійне перекривання атомних $1s$ або $p_y$ орбіталей уздовж осі &
%\begin{tikzpicture}[scale=0.6]
%  \orbital[pos={(0,0)}]{s}
%  \orbital[pos={(1.2,0)}]{s}
%\end{tikzpicture}
%
%\begin{tikzpicture}[scale=0.6]
%  \orbital[pos={(0,0)}]{s}
%  \orbital[pos={(1.2,0)}]{-py}
%\end{tikzpicture}
%
%\\
%
%$\sigma_u^*$ & антисиметрична (ungerade) & антизв’язувальна & Протифазне перекривання $p_z$-орбіталей уздовж осі &
%\begin{tikzpicture}
%  \orbital[pos={(0,0)}]{pz}
%  \orbital[pos={(0.5,0)}]{-pz}
%\end{tikzpicture} \\
%
%$\pi_u$ & антисиметрична & зв’язувальна & Бічне перекривання $p_x$, $p_y$ орбіталей (один вузол уздовж осі) &
%\begin{tikzpicture}[scale=0.6]
%  \orbital[pos={(0,0)}]{px}
%  \orbital[pos={(1.4,0)}]{px}
%\end{tikzpicture} \\
%
%$\pi_g^*$ & симетрична & антизв’язувальна & Протифазне бічне перекривання $p_x$, $p_y$ орбіталей &
%\begin{tikzpicture}[scale=0.6]
%  \orbital[pos={(0,0)}]{px}
%  \orbital[pos={(1.4,0)}]{-px}
%\end{tikzpicture} \\
%
%$\delta_g$ & симетрична & зв’язувальна & Перекривання $d_{xy}$, $d_{x^2-y^2}$; дві вузлові площини &
%\begin{tikzpicture}[scale=0.6]
%  \orbital[pos={(0,0)}]{dxy}
%  \orbital[pos={(1.4,0)}]{dxy}
%\end{tikzpicture} \\
%\end{tblr}
%
%\medskip

\textbf{Зв’язок із симетрією:}
Парність $g/u$ визначається поведінкою хвильової функції при інверсії координат:
\[
\psi_g(\mathbf{r}) = +\psi_g(-\mathbf{r}), \qquad
\psi_u(\mathbf{r}) = -\psi_u(-\mathbf{r}).
\]

%% --------------------------------------------------------
\subsection{Класифікація електронних станів молекули}
%% --------------------------------------------------------

\textbf{Електронний терм}  --- це позначення квантового стану молекули, який характеризується певними значеннями
повного спіну $S$, проєкції орбітального моменту на між’ядерну вісь $\Lambda$,
та симетріями відносно інверсії ($g/u$) і відбиття ($+/-$).
Він має вигляд:
\[
^{2S+1}\Lambda_{g/u}^{\pm}.
\]
де:
\begin{itemize}
    \item $S$ --- повний спін системи;
    \item $2S+1$ --- \textbf{мультиплетність} (синглет, дублет, триплет тощо);
    \item $\Lambda$ --- \textbf{сумарна проєкція орбітального моменту} на між'ядерну вісь:
    \[
    \Sigma~(\Lambda=0), \quad
    \Pi~(\Lambda=1), \quad
    \Delta~(\Lambda=2), \quad
    \Phi~(\Lambda=3), \ldots
    \]
    \item верхній індекс $+$ або $-$ --- \textbf{симетрія відносно площини, що містить вісь молекули} (лише для станів типу $\Sigma$);
    \item нижній індекс $g/u$ (від \textit{gerade}/\textit{ungerade}) --- \textbf{парність хвильової функції} відносно інверсії в центрі молекули.
\end{itemize}

%% --------------------------------------------------------
\subsection*{Типові терми двоатомних молекул}
%% --------------------------------------------------------

\noindent%
\begin{tblr}{
  colspec = {c c c X[l]},
  hlines,
  vlines,
  row{1} = {c, m, font=\bfseries, bg=gray!10},
}
Терм & Мультиплетність & Симетрія & Фізичний зміст \\
${}^1\Sigma_g^+$ & Синглет & $g$, $+$ & Основний стан молекули H$_2$; усі електрони спарені, повна симетрія \\
${}^3\Sigma_u^+$ & Триплет & $u$, $+$ & Збуджений стан з паралельними спінами (наприклад, у \ce{O2}) \\
${}^1\Pi_u$ & Синглет & $u$ & Орбітальний момент $\Lambda=1$, без інверсійної симетрії \\
${}^3\Pi_g$ & Триплет & $g$ & Орбітальний момент $\Lambda=1$, паралельні спіни \\
${}^1\Delta_g$ & Синглет & $g$ & Орбітальний момент $\Lambda=2$, спарені електрони \\
${}^2\Sigma_u^+$ & Дублет & $u$, $+$ & Один неспарений електрон, як у радикалах типу \ce{NO} \\
\end{tblr}

\bigskip
Нижче наведено покроковий алгоритм визначення терму для заданої електронної конфігурації.

%% --------------------------------------------------------
\subsection*{Алгоритм}
%% --------------------------------------------------------

\begin{enumerate}
  \item \textbf{Записати електронну конфігурацію.}
  Визначити, які орбіталі заповнені, частково заповнені чи вакантні:
  \[
  (\sigma_g)^2(\sigma_u^*)^2(\pi_u)^4(\pi_g^*)^2, \ldots
  \]

  \item \textbf{Визначити тип орбіталей і відповідні проєкції орбітального моменту.}
  Для кожної орбіталі встановити:
  \[
  \sigma \rightarrow \Lambda_i = 0, \quad
  \pi \rightarrow \Lambda_i = 1, \quad
  \delta \rightarrow \Lambda_i = 2, \quad
  \text{тощо.}
  \]

  \item \textbf{Знайти можливі комбінації орбітальних моментів.}
  Для активних (частково заповнених) орбіталей обчислити всі можливі значення:
  \[
  \Lambda = |\Lambda_1 + \Lambda_2|,\, |\Lambda_1 - \Lambda_2|, \ldots
  \]
  Це дає можливі типи станів: $\Sigma, \Pi, \Delta, \ldots$

  \item \textbf{Комбінувати спіни електронів.}
  Для кожного електрона $S_i = \tfrac{1}{2}$.
  Обчислити всі можливі значення повного спіну:
  \[
  S = |S_1 + S_2|,\, |S_1 - S_2|, \ldots
  \]
  Відповідно визначається мультиплетність $2S+1$ (синглет, дублет, триплет тощо).

  \item \textbf{Перевірити принцип Паулі.}
  Загальна хвильова функція має бути антисиметричною при перестановці двох електронів:
  \[
  \Psi_{\text{заг}} = \Psi_{\text{просторова}} \, \Psi_{\text{спінова}} \, \Psi_{\text{симетрії}}.
  \]
  Якщо просторова частина симетрична $\Rightarrow S=0$ (синглет),
  якщо антисиметрична $\Rightarrow S=1$ (триплет).

  \item \textbf{Визначити симетрії $g/u$ і $+/-$.}
  \begin{itemize}
    \item Індекс $g/u$ показує парність хвильової функції при інверсії в центрі молекули.
    \item Для станів $\Sigma$ додається верхній індекс $+/-$, що вказує на симетрію при відбитті у площині, яка містить між’ядерну вісь.
  \end{itemize}

  \item \textbf{Записати всі можливі терми.}
  Комбінуючи знайдені $S$, $\Lambda$, $g/u$ та $+/-$, записуємо:
  \[
  {}^{2S+1}\Lambda_{g/u}^{\pm}.
  \]

  \item \textbf{Визначити основний терм за правилами Гунда.}
  \begin{enumerate}[label*=\arabic*.]
    \item Найнижчу енергію має стан з \textbf{максимальним спіном} $S$.
    \item Для однакового $S$ — стан з \textbf{найбільшим} $\Lambda$.
    \item Для даних $S, \Lambda$ — визначається парність ($g/u$) і знак ($+/-)$ з урахуванням симетрії конфігурації.
  \end{enumerate}
\end{enumerate}

\textbf{Приклад.}
Для молекули водню $\mathrm{H_2}$ основний електронний стан має конфігурацію $\sigma_g^2$.
Оскільки обидва електрони спарені ($S=0$), повний терм записується як:
\[
^{2S+1}\Lambda_g^+ = {}^1\Sigma_g^+.
\]
Це \textbf{синглетний стан} ($S=0$), симетричний відносно інверсії (індекс $g$) і з позитивною симетрією відносно площини, що містить між'ядерну вісь ($+$).

%%% --------------------------------------------------------
%\subsection*{Приклад: молекула \ce{O2}}
%%% --------------------------------------------------------
%
%\begin{tblr}{
%  colspec = {Q[1.2cm] Q[2cm] Q[2cm] X[l]},
%  row{1} = {font=\bfseries, bg=gray!10},
%  hlines = {1pt, 2pt},
%  vlines = {0.6pt},
%}
%Крок & Дія & Результат & Пояснення \\
%1 & Конфігурація & $(\pi_g^*)^2$ & Активні орбіталі — два електрони на вироджених $\pi_g^*$ \\
%2 & Орбітальний момент & $\Lambda_i = 1$ & Для $\pi$-орбіталей \\
%3 & Комбінації $\Lambda$ & $0,1,2$ & Відповідають станам $\Sigma$, $\Pi$, $\Delta$ \\
%4 & Комбінації спіну & $S=0,1$ & Синглет і триплет \\
%5 & Умови Паулі & $S=1$ дозволено для антисиметричної просторової частини & Визначає триплетний стан \\
%6 & Симетрія & $g$, $-$ & За характером $\pi_g^*$ орбіталей \\
%7 & Терм & ${}^3\Sigma_g^-$ & Основний стан \\
%\end{tblr}