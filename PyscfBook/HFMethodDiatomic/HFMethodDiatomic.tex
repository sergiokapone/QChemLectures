% !TeX program = lualatex
% !TeX encoding = utf8
% !TeX spellcheck = uk_UA
% !TeX root =../PyscfBook.tex

%=========================================================
\Opensolutionfile{answer}[\currfilebase/\currfilebase-Answers]
\chapter{Метод Хартрі-Фока для простих молекул}\label{\currfilebase}
%=========================================================


Після вивчення атомних систем природним кроком є перехід до молекул.
Найпростіші молекули --- іон молекули водню \ce{H2+}
та молекула водню \ce{H2} --- є ідеальними об'єктами для демонстрації
можливостей та обмежень методу Хартрі--Фока.

%% --------------------------------------------------------
\section{Орбіталі і електронні стани молекул}
%% --------------------------------------------------------

Електронна структура двоатомних молекул описується двома рівнями:
\begin{enumerate}
    \item класифікацією \textbf{молекулярних орбіталей (МО)};
    \item класифікацією \textbf{електронних станів молекули}, які утворюються внаслідок заповнення цих МО електронами.
\end{enumerate}

%% --------------------------------------------------------
\subsection{Класифікація молекулярних орбіталей}
%% --------------------------------------------------------

Молекулярні орбіталі позначаються символами
$\sigma$, $\pi$, $\delta$, $\varphi$, \ldots,
що відповідають проєкції орбітального моменту $\Lambda$ на між'ядерну вісь:
\[
\Lambda = 0,1,2,3,\ldots \quad \Rightarrow \quad
\sigma,\, \pi,\, \delta,\, \varphi, \ldots
\]

Кожна орбіталь може бути:
\begin{itemize}
    \item \textbf{зв’язувальною} або \textbf{антизв’язувальною} (позначається зірочкою ${}^*$);
    \item \textbf{симетричною (gerade, $g$)} або \textbf{антисиметричною (ungerade, $u$)} відносно інверсії в центрі молекули.
\end{itemize}

Таким чином, типовий запис молекулярної орбіталі має вигляд:
\[
\sigma_g, \quad \pi_u, \quad \sigma_u^*, \quad \pi_g^*, \ldots
\]

%\textbf{Приклади класифікації молекулярних орбіталей:}
%
%\noindent%
%\begin{tblr}{
%  colspec = {X[c] X[l] X[l] X[l] X[c, m]},
%  row{1} = {font=\bfseries, bg=gray!10, c},
%  hlines = {1pt,2pt},
%  vlines = {0.6pt},
%}
%МО & Симетрія & Тип & Характер і походження & Схематичне перекривання \\
%
%$\sigma_g$ & симетрична (gerade) & зв’язувальна & Лінійне перекривання атомних $1s$ або $p_y$ орбіталей уздовж осі &
%\begin{tikzpicture}[scale=0.6]
%  \orbital[pos={(0,0)}]{s}
%  \orbital[pos={(1.2,0)}]{s}
%\end{tikzpicture}
%
%\begin{tikzpicture}[scale=0.6]
%  \orbital[pos={(0,0)}]{s}
%  \orbital[pos={(1.2,0)}]{-py}
%\end{tikzpicture}
%
%\\
%
%$\sigma_u^*$ & антисиметрична (ungerade) & антизв’язувальна & Протифазне перекривання $p_z$-орбіталей уздовж осі &
%\begin{tikzpicture}
%  \orbital[pos={(0,0)}]{pz}
%  \orbital[pos={(0.5,0)}]{-pz}
%\end{tikzpicture} \\
%
%$\pi_u$ & антисиметрична & зв’язувальна & Бічне перекривання $p_x$, $p_y$ орбіталей (один вузол уздовж осі) &
%\begin{tikzpicture}[scale=0.6]
%  \orbital[pos={(0,0)}]{px}
%  \orbital[pos={(1.4,0)}]{px}
%\end{tikzpicture} \\
%
%$\pi_g^*$ & симетрична & антизв’язувальна & Протифазне бічне перекривання $p_x$, $p_y$ орбіталей &
%\begin{tikzpicture}[scale=0.6]
%  \orbital[pos={(0,0)}]{px}
%  \orbital[pos={(1.4,0)}]{-px}
%\end{tikzpicture} \\
%
%$\delta_g$ & симетрична & зв’язувальна & Перекривання $d_{xy}$, $d_{x^2-y^2}$; дві вузлові площини &
%\begin{tikzpicture}[scale=0.6]
%  \orbital[pos={(0,0)}]{dxy}
%  \orbital[pos={(1.4,0)}]{dxy}
%\end{tikzpicture} \\
%\end{tblr}
%
%\medskip

\textbf{Зв’язок із симетрією:}
Парність $g/u$ визначається поведінкою хвильової функції при інверсії координат:
\[
\psi_g(\mathbf{r}) = +\psi_g(-\mathbf{r}), \qquad
\psi_u(\mathbf{r}) = -\psi_u(-\mathbf{r}).
\]

%% --------------------------------------------------------
\subsection{Класифікація електронних станів молекули}
%% --------------------------------------------------------

\textbf{Електронний терм}  --- це позначення квантового стану молекули, який характеризується певними значеннями
повного спіну $S$, проєкції орбітального моменту на між’ядерну вісь $\Lambda$,
та симетріями відносно інверсії ($g/u$) і відбиття ($+/-$).
Він має вигляд:
\[
^{2S+1}\Lambda_{g/u}^{\pm}.
\]
де:
\begin{itemize}
    \item $S$ --- повний спін системи;
    \item $2S+1$ --- \textbf{мультиплетність} (синглет, дублет, триплет тощо);
    \item $\Lambda$ --- \textbf{сумарна проєкція орбітального моменту} на між'ядерну вісь:
    \[
    \Sigma~(\Lambda=0), \quad
    \Pi~(\Lambda=1), \quad
    \Delta~(\Lambda=2), \quad
    \Phi~(\Lambda=3), \ldots
    \]
    \item верхній індекс $+$ або $-$ --- \textbf{симетрія відносно площини, що містить вісь молекули} (лише для станів типу $\Sigma$);
    \item нижній індекс $g/u$ (від \textit{gerade}/\textit{ungerade}) --- \textbf{парність хвильової функції} відносно інверсії в центрі молекули.
\end{itemize}

%% --------------------------------------------------------
\subsection*{Типові терми двоатомних молекул}
%% --------------------------------------------------------

\noindent%
\begin{tblr}{
  colspec = {c c c X[l]},
  hlines,
  vlines,
  row{1} = {c, m, font=\bfseries, bg=gray!10},
}
Терм & Мультиплетність & Симетрія & Фізичний зміст \\
${}^1\Sigma_g^+$ & Синглет & $g$, $+$ & Основний стан молекули H$_2$; усі електрони спарені, повна симетрія \\
${}^3\Sigma_u^+$ & Триплет & $u$, $+$ & Збуджений стан з паралельними спінами (наприклад, у \ce{O2}) \\
${}^1\Pi_u$ & Синглет & $u$ & Орбітальний момент $\Lambda=1$, без інверсійної симетрії \\
${}^3\Pi_g$ & Триплет & $g$ & Орбітальний момент $\Lambda=1$, паралельні спіни \\
${}^1\Delta_g$ & Синглет & $g$ & Орбітальний момент $\Lambda=2$, спарені електрони \\
${}^2\Sigma_u^+$ & Дублет & $u$, $+$ & Один неспарений електрон, як у радикалах типу \ce{NO} \\
\end{tblr}

\bigskip
Нижче наведено покроковий алгоритм визначення терму для заданої електронної конфігурації.

%% --------------------------------------------------------
\subsection*{Алгоритм}
%% --------------------------------------------------------

\begin{enumerate}
  \item \textbf{Записати електронну конфігурацію.}
  Визначити, які орбіталі заповнені, частково заповнені чи вакантні:
  \[
  (\sigma_g)^2(\sigma_u^*)^2(\pi_u)^4(\pi_g^*)^2, \ldots
  \]

  \item \textbf{Визначити тип орбіталей і відповідні проєкції орбітального моменту.}
  Для кожної орбіталі встановити:
  \[
  \sigma \rightarrow \Lambda_i = 0, \quad
  \pi \rightarrow \Lambda_i = 1, \quad
  \delta \rightarrow \Lambda_i = 2, \quad
  \text{тощо.}
  \]

  \item \textbf{Знайти можливі комбінації орбітальних моментів.}
  Для активних (частково заповнених) орбіталей обчислити всі можливі значення:
  \[
  \Lambda = |\Lambda_1 + \Lambda_2|,\, |\Lambda_1 - \Lambda_2|, \ldots
  \]
  Це дає можливі типи станів: $\Sigma, \Pi, \Delta, \ldots$

  \item \textbf{Комбінувати спіни електронів.}
  Для кожного електрона $S_i = \tfrac{1}{2}$.
  Обчислити всі можливі значення повного спіну:
  \[
  S = |S_1 + S_2|,\, |S_1 - S_2|, \ldots
  \]
  Відповідно визначається мультиплетність $2S+1$ (синглет, дублет, триплет тощо).

  \item \textbf{Перевірити принцип Паулі.}
  Загальна хвильова функція має бути антисиметричною при перестановці двох електронів:
  \[
  \Psi_{\text{заг}} = \Psi_{\text{просторова}} \, \Psi_{\text{спінова}} \, \Psi_{\text{симетрії}}.
  \]
  Якщо просторова частина симетрична $\Rightarrow S=0$ (синглет),
  якщо антисиметрична $\Rightarrow S=1$ (триплет).

  \item \textbf{Визначити симетрії $g/u$ і $+/-$.}
  \begin{itemize}
    \item Індекс $g/u$ показує парність хвильової функції при інверсії в центрі молекули.
    \item Для станів $\Sigma$ додається верхній індекс $+/-$, що вказує на симетрію при відбитті у площині, яка містить між’ядерну вісь.
  \end{itemize}

  \item \textbf{Записати всі можливі терми.}
  Комбінуючи знайдені $S$, $\Lambda$, $g/u$ та $+/-$, записуємо:
  \[
  {}^{2S+1}\Lambda_{g/u}^{\pm}.
  \]

  \item \textbf{Визначити основний терм за правилами Гунда.}
  \begin{enumerate}[label*=\arabic*.]
    \item Найнижчу енергію має стан з \textbf{максимальним спіном} $S$.
    \item Для однакового $S$ — стан з \textbf{найбільшим} $\Lambda$.
    \item Для даних $S, \Lambda$ — визначається парність ($g/u$) і знак ($+/-)$ з урахуванням симетрії конфігурації.
  \end{enumerate}
\end{enumerate}

\textbf{Приклад.}
Для молекули водню $\mathrm{H_2}$ основний електронний стан має конфігурацію $\sigma_g^2$.
Оскільки обидва електрони спарені ($S=0$), повний терм записується як:
\[
^{2S+1}\Lambda_g^+ = {}^1\Sigma_g^+.
\]
Це \textbf{синглетний стан} ($S=0$), симетричний відносно інверсії (індекс $g$) і з позитивною симетрією відносно площини, що містить між'ядерну вісь ($+$).

%%% --------------------------------------------------------
%\subsection*{Приклад: молекула \ce{O2}}
%%% --------------------------------------------------------
%
%\begin{tblr}{
%  colspec = {Q[1.2cm] Q[2cm] Q[2cm] X[l]},
%  row{1} = {font=\bfseries, bg=gray!10},
%  hlines = {1pt, 2pt},
%  vlines = {0.6pt},
%}
%Крок & Дія & Результат & Пояснення \\
%1 & Конфігурація & $(\pi_g^*)^2$ & Активні орбіталі — два електрони на вироджених $\pi_g^*$ \\
%2 & Орбітальний момент & $\Lambda_i = 1$ & Для $\pi$-орбіталей \\
%3 & Комбінації $\Lambda$ & $0,1,2$ & Відповідають станам $\Sigma$, $\Pi$, $\Delta$ \\
%4 & Комбінації спіну & $S=0,1$ & Синглет і триплет \\
%5 & Умови Паулі & $S=1$ дозволено для антисиметричної просторової частини & Визначає триплетний стан \\
%6 & Симетрія & $g$, $-$ & За характером $\pi_g^*$ орбіталей \\
%7 & Терм & ${}^3\Sigma_g^-$ & Основний стан \\
%\end{tblr}

% !TeX program = lualatex
% !TeX encoding = utf8
% !TeX spellcheck = uk_UA
% !TeX root =../PyscfBook.tex


Після вивчення атомних систем природним кроком є перехід до молекул.
Найпростіші диатомні молекули --- іон молекули водню \ce{H2+}
та молекула водню \ce{H2} --- є ідеальними об'єктами для демонстрації
можливостей та обмежень методу Хартрі--Фока.

%% ========================================================
\section{Чому саме \ce{H2+} та \ce{H2}?}
%% ========================================================

\begin{itemize}
    \item \textbf{\ce{H2+}} --- найпростіша молекула (один електрон),
          має аналітичний розв'язок у еліптичних координатах
    \item \textbf{\ce{H2}} --- найпростіша двоелектронна молекула,
          демонструє фундаментальну проблему методу ХФ
    \item Обидві системи достатньо малі для детального аналізу
    \item Результати можна порівняти з високоточними експериментальними даними
\end{itemize}

%% ========================================================
\section{Іон молекули водню \ce{H2+}}
%% ========================================================

%% --------------------------------------------------------
\subsection{Теоретичні основи}
%% --------------------------------------------------------

Іон \ce{H2+} складається з двох протонів та одного електрона.
Гамільтоніан системи (в атомних одиницях):
\[
\hat{H} = -\frac{1}{2}\nabla^2 - \frac{1}{r_A} - \frac{1}{r_B} + \frac{1}{R},
\]
де $r_A$, $r_B$ --- відстані електрона до ядер A і B,
$R$ --- міжядерна відстань.

Оскільки система має лише один електрон, метод Хартрі--Фока
\textbf{не містить апроксимацій} (крім базисних обмежень),
тому HF-енергія збігається з точною в межах обраного базису.

%% --------------------------------------------------------
\subsection{Визначення молекули в PySCF}
%% --------------------------------------------------------

Для визначення молекули в PySCF використовується рядкова нотація:

\inputcode{h2plus_single.py}

\textbf{Пояснення коду:}
\begin{itemize}
    \item \inlinecode{atom = 'H 0 0 0; H 0 0 0.74'} ---
          координати атомів (Ангстреми за замовчуванням)
    \item \inlinecode{basis = 'sto-3g'} --- мінімальний базис
    \item \inlinecode{charge = 1} --- заряд системи (+1 для \ce{H2+})
    \item \inlinecode{spin = 1} --- $2S = N_\alpha - N_\beta = 1$
    \item \inlinecode{scf.UHF} --- необмежений ХФ (один непарний електрон)
\end{itemize}

\textbf{Результат:} Енергія $E \approx -0.566$ Ha для $R = 0.74$ Å.

%% --------------------------------------------------------
\subsection{Крива потенційної енергії (PES)}
%% --------------------------------------------------------

Крива потенційної енергії (Potential Energy Surface, PES)
показує залежність повної енергії системи від міжядерної відстані $R$.
Для диатомної молекули це одновимірна функція $E(R)$.

\subsubsection{Сканування по відстаням}

Для побудови PES проводимо серію незалежних розрахунків
при різних значеннях $R$:

\inputcode{h2plus_pes.py}

\textbf{Важливі моменти:}
\begin{itemize}
    \item Відстані задаються в бораx (1 bohr $\approx$ 0.529 Å)
    \item Діапазон $R \in [0.5, 5.0]$ bohr охоплює від стиснення
          до повної дисоціації
    \item Кожна точка --- незалежний SCF-розрахунок
    \item Результат зберігається для подальшого аналізу
\end{itemize}

\subsubsection{Аналіз кривої}

%---------------------------------------------------------
\begin{figure}[h!]\centering
\includegraphics[width=\linewidth]{\currfiledir/h2plus_pes.pdf}
\caption{Графіку PES для \ce{H2+}.}
\label{pic:h2plus_pes}
\end{figure}
%---------------------------------------------------------

На графіку (рис.~\ref{pic:h2plus_pes}) PES спостерігаються характерні області:

\begin{enumerate}
    \item \textbf{Область стиснення} ($R < R_e$):
          Енергія різко зростає через відштовхування ядер

    \item \textbf{Мінімум енергії} ($R = R_e$):
          Рівноважна відстань, де сили притягання
          та відштовхування збалансовані

    \item \textbf{Дисоціаційна межа} ($R \to \infty$):
          Енергія прямує до суми енергій ізольованих фрагментів
          \[
          E(R\to\infty) \to E(\text{H}) + E(\text{H}^+) = -0.5 \text{ Ha}
          \]
\end{enumerate}

\textbf{Порівняння з аналітичним розв'язком:}
Для \ce{H2+} існує точний розв'язок у еліптичних координатах:
\begin{itemize}
    \item $R_e = 2.00$ bohr (експеримент: 2.00 bohr)
    \item $D_e = 0.102$ Ha = 2.79 eV (експеримент: 2.79 eV)
    \item Різниця HF--точне $< 0.001$ Ha для великих базисів
\end{itemize}

%% --------------------------------------------------------
\subsection{Оптимізація геометрії}
%% --------------------------------------------------------

Замість ручного сканування можна автоматично знайти мінімум
енергії (рівноважну геометрію):

\inputcode{h2plus_optimization.py}

\textbf{Алгоритм оптимізації:}
\begin{itemize}
    \item PySCF використовує градієнти енергії $\nabla E$
    \item Метод quasi-Newton (BFGS за замовчуванням)
    \item Критерій збіжності: $|\nabla E| < 10^{-4}$ Ha/bohr
    \item Зазвичай потрібно 5--10 ітерацій
\end{itemize}

\textbf{Спектроскопічні константи:}
Після оптимізації можна обчислити:
\begin{itemize}
    \item \textbf{Рівноважна відстань} $R_e$ --- з оптимізованої геометрії
    \item \textbf{Енергія дисоціації}:
          \[
          D_e = E(\text{H}) + E(\text{H}^+) - E(\text{H}_2^+, R_e)
          \]
    \item \textbf{Частота коливань} $\omega_e$ --- з другої похідної
          (гесіану) енергії
\end{itemize}

%% --------------------------------------------------------
\subsection{Аналіз молекулярних орбіталей}
%% --------------------------------------------------------

Для \ce{H2+} єдина зайнята молекулярна орбіталь (МО)
є зв'язуючою $\sigma_g$ комбінацією атомних орбіталей:
\[
\psi_{\sigma_g} \approx \frac{1}{\sqrt{2}}(\phi_{1s}^A + \phi_{1s}^B)
\]

\inputcode{h2plus_mo_analysis.py}

\textbf{Енергетична діаграма:}
\begin{itemize}
    \item Зайнята МО: $\varepsilon_{\sigma_g} \approx -0.6$ Ha (зв'язуюча)
    \item Віртуальна МО: $\varepsilon_{\sigma_u^*} \approx +0.2$ Ha
          (антизв'язуюча)
    \item HOMO--LUMO gap: $\Delta \approx 0.8$ Ha $\approx$ 22 eV
\end{itemize}

%% --------------------------------------------------------
\subsection{Вплив базисного набору}
%% --------------------------------------------------------

Оскільки \ce{H2+} має аналітичний розв'язок, це ідеальна система
для тестування базисів:

\inputcode{h2plus_basis_convergence.py}

\textbf{Очікувані результати:}
\begin{center}
\begin{tabular}{lccc}
\toprule
Базис & $R_e$ (bohr) & $E_e$ (Ha) & Похибка (mHa) \\
\midrule
STO-3G      & 2.05 & $-0.564$ & 4.5 \\
6-31G       & 2.02 & $-0.566$ & 2.1 \\
cc-pVDZ     & 2.01 & $-0.567$ & 1.2 \\
cc-pVTZ     & 2.00 & $-0.5686$ & 0.3 \\
cc-pVQZ     & 2.00 & $-0.5688$ & 0.1 \\
\midrule
Точне       & 2.00 & $-0.5689$ & --- \\
\bottomrule
\end{tabular}
\end{center}

\textbf{Висновки:}
\begin{itemize}
    \item Навіть STO-3G дає якісно правильну картину
    \item Для кількісної точності потрібен cc-pVTZ або більший
    \item Збіжність монотонна: $E_{\text{basis}} \to E_{\text{CBS}}$
    \item Для \ce{H2+} CBS limit досягається при cc-pV5Z
\end{itemize}

%% ========================================================
\section{Молекула водню \ce{H2}}
%% ========================================================

%% --------------------------------------------------------
\subsection{Двоелектронна система}
%% --------------------------------------------------------

Молекула \ce{H2} --- перша справжня двоелектронна система.
Гамільтоніан:
\[
\hat{H} = \hat{h}_1 + \hat{h}_2 + \frac{1}{r_{12}} + \frac{1}{R},
\]
де $\frac{1}{r_{12}}$ --- електрон-електронне відштовхування,
яке метод ХФ апроксимує середнім полем.

\textbf{Електронна конфігурація:}
Основний стан --- синглет ($^1\Sigma_g^+$) з конфігурацією $\sigma_g^2$:
\[
\Psi = \frac{1}{\sqrt{2}}[\psi_{\sigma_g}(\mathbf{r}_1)\alpha(1)
\psi_{\sigma_g}(\mathbf{r}_2)\beta(2)
- \psi_{\sigma_g}(\mathbf{r}_1)\beta(1)
\psi_{\sigma_g}(\mathbf{r}_2)\alpha(2)]
\]

%% --------------------------------------------------------
\subsection{RHF розрахунок}
%% --------------------------------------------------------

Для замкненої оболонки використовуємо RHF:

\inputcode{h2_rhf_single.py}

\textbf{Результат при $R = 1.4$ bohr:}
\begin{itemize}
    \item Енергія: $E_{\text{RHF}} \approx -1.133$ Ha
    \item Експериментальна: $E_{\text{exp}} \approx -1.174$ Ha
    \item Кореляційна енергія: $E_{\text{corr}} = 0.041$ Ha $\approx$ 1.1 eV
\end{itemize}

%% --------------------------------------------------------
\subsection{Крива потенційної енергії}
%% --------------------------------------------------------

Побудуємо PES для \ce{H2} методами RHF та UHF:

\inputcode{h2_pes_comparison.py}


%---------------------------------------------------------
\begin{figure}[h!]\centering
\includegraphics[width=\linewidth]{\currfiledir/h2_rhf_vs_uhf.pdf}
\caption{}
\label{}
\end{figure}
%---------------------------------------------------------


\textbf{Спостереження на графіку:}
\begin{enumerate}
    \item \textbf{Навколо мінімуму} ($R \approx 1.4$ bohr):
          RHF та UHF дають практично однакові результати

    \item \textbf{При розтягуванні} ($R > 3$ bohr):
          Криві розходяться! RHF завищує енергію

    \item \textbf{Дисоціаційна межа} ($R \to \infty$):
          \begin{itemize}
              \item UHF: $E \to 2 \times E(\text{H}) = -1.0$ Ha ✓
              \item RHF: $E \to E(\text{H}^+) + E(\text{H}^-) \approx -0.7$ Ha ✗
          \end{itemize}
\end{enumerate}

%% ========================================================
\section{Проблема дисоціації в методі Хартрі--Фока}
%% ========================================================

%% --------------------------------------------------------
\subsection{RHF: некоректна дисоціація}
%% --------------------------------------------------------

\subsubsection{Фізична картина}

При розриві зв'язку \ce{H2} $\to$ 2\ce{H} очікується:
\[
\lim_{R\to\infty} E(\text{H}_2) = 2 \times E(\text{H}) = 2 \times (-0.5) = -1.0 \text{ Ha}
\]

Однак RHF дає:
\[
\lim_{R\to\infty} E_{\text{RHF}}(\text{H}_2) \approx -0.7 \text{ Ha}
\]

\textbf{Чому?} RHF змушує електрони мати однакові просторові орбіталі:
\[
\psi_{\text{RHF}} = |\sigma_g\alpha\, \sigma_g\beta\rangle
\]

При великих $R$ це означає:
\[
\psi \sim |(1s_A + 1s_B)\alpha\, (1s_A + 1s_B)\beta\rangle
\]

Розкриваючи детермінант, отримуємо \textbf{іонні внески}:
\[
\psi \sim \text{H--H} + \text{H}^+\text{H}^- + \text{H}^-\text{H}^+
\]

При $R \to \infty$ іонні стани мають завищену енергію,
тому RHF-енергія неправильна!

\subsubsection{Математичне обґрунтування}

RHF-хвильова функція не є власним станом оператора $\hat{S}^2$
при великих $R$. Вона містить домішки триплетних станів,
що призводить до неправильної енергії дисоціації.

%% --------------------------------------------------------
\subsection{UHF: правильна дисоціація, але спінове забруднення}
%% --------------------------------------------------------

UHF дозволяє різні орбіталі для $\alpha$ та $\beta$ спінів:
\[
\psi_{\text{UHF}} = |\phi_\alpha\, \phi_\beta\rangle,
\quad \phi_\alpha \neq \phi_\beta
\]

При $R \to \infty$:
\[
\phi_\alpha \to 1s_A, \quad \phi_\beta \to 1s_B
\]

Це дає правильну енергію:
\[
\lim_{R\to\infty} E_{\text{UHF}} = E(\text{H}) + E(\text{H}) = -1.0 \text{ Ha}
\]

\textbf{Але!} Виникає спінове забруднення:

\inputcode{h2_spin_contamination.py}

\textbf{Результат:}
\begin{itemize}
    \item При $R = 1.4$ bohr: $\langle S^2 \rangle \approx 0.00$
          (чистий синглет)
    \item При $R = 5.0$ bohr: $\langle S^2 \rangle \approx 1.00$
          (забруднення триплетом!)
    \item Теоретично для $S=0$: $\langle S^2 \rangle = 0$
\end{itemize}

\textbf{Інтерпретація:} UHF-хвильова функція при великих $R$ є сумішшю
синглету та триплету, хоча правильний стан --- чистий синглет.

%% --------------------------------------------------------
\subsection{Візуалізація проблеми}
%% --------------------------------------------------------

Порівняємо всі методи на одному графіку:

\inputcode{h2_dissociation_comparison.py}

%---------------------------------------------------------
\begin{figure}[h!]\centering
\includegraphics[width=\linewidth]{\currfiledir/h2_complete_comparison.pdf}
\caption{Повне порівняння методів для молекули \ce{H2}.}
\label{pic:h2_complete_comparison}
\end{figure}
%---------------------------------------------------------

\textbf{Графік (рис.~\ref{pic:h2_complete_comparison}) показує:}
\begin{itemize}
    \item \textbf{Точний розв'язок} (Full CI) --- монотонна крива
    \item \textbf{RHF} --- завищена енергія при $R > 2.5$ bohr
    \item \textbf{UHF} --- правильна асимптотика, але забруднений спін
    \item \textbf{ROHF} --- між RHF та UHF (не показано, складніший)
\end{itemize}

%% --------------------------------------------------------
\subsection{Відсутність кореляції як джерело проблеми}
%% --------------------------------------------------------

Фундаментальна причина --- метод ХФ описує електрон-електронну
взаємодію через \textbf{середнє поле}, ігноруючи миттєву кореляцію
положень електронів.

\textbf{Кореляційна енергія:}
\[
E_{\text{corr}}(R) = E_{\text{exact}}(R) - E_{\text{HF}}(R)
\]

\begin{center}
\begin{tabular}{lcc}
\toprule
$R$ (bohr) & $E_{\text{corr}}$ (mHa) & \% від $D_e$ \\
\midrule
1.4 (рівновага) & 41 & 9\% \\
3.0             & 85 & 18\% \\
5.0             & 150 & 32\% \\
$\infty$        & 300 & --- \\
\bottomrule
\end{tabular}
\end{center}

\textbf{Висновок:} Кореляція стає критичною при розриві зв'язків!

%% --------------------------------------------------------
\section*{Завдання}
%% --------------------------------------------------------
\subsection*{Завдання 1: Порівняння базисів}

Побудуйте PES для \ce{H2} у базисах STO-3G, 6-31G**, cc-pVDZ, cc-pVTZ.
Порівняйте:
\begin{itemize}
    \item Рівноважні відстані $R_e$
    \item Енергії дисоціації $D_e$
    \item Час обчислень
\end{itemize}

\subsection*{Завдання 2: Інші диатомні молекули}

Повторіть аналіз для:
\begin{itemize}
    \item \ce{LiH} --- гетероядерна молекула
    \item \ce{N2} --- потрійний зв'язок
    \item \ce{F2} --- слабкий зв'язок
\end{itemize}

Чи зберігається проблема дисоціації?


%% --------------------------------------------------------
\section{Резюме}
%% --------------------------------------------------------

У цьому розділі ми детально вивчили найпростіші диатомні молекули:

\begin{itemize}
    \item \ce{H2+}: Метод ХФ є точним (один електрон),
          демонструє техніки PES, оптимізації, аналізу МО.

    \item \ce{H2}: Виявляє фундаментальну проблему RHF ---
          некоректну дисоціацію через відсутність кореляції

    \item \textbf{Кореляція}: Стає критичною при розриві зв'язків,
          потребує пост-ХФ методів (FCI, CASSCF, CCSD, CI)
\end{itemize}

\textbf{Ключовий урок:} Метод ХФ добре працює навколо рівноваги,
але неадекватний для опису хімічних реакцій, де відбувається
розрив/утворення зв'язків.


