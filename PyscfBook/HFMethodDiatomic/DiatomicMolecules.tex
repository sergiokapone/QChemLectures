% !TeX program = lualatex
% !TeX encoding = utf8
% !TeX spellcheck = uk_UA
% !TeX root =../PyscfBook.tex


%% ========================================================
\section{Початкові модельні молекули: \ce{H2+} та \ce{H2}}
%% ========================================================

Першим кроком у квантово-хімічному описі молекул є вивчення найпростіших систем, для яких можна чітко простежити природу хімічного зв’язку.
Молекули \ce{H2+} та \ce{H2} є \textbf{модельними} у тому сенсі, що вони:
\begin{itemize}
    \item мають найменшу можливу кількість електронів (один і два відповідно);
    \item дозволяють побудувати хвильові функції, що безпосередньо демонструють
          утворення зв’язувальних і розривних (антизв’язувальних) орбіталей;
    \item відтворюють основні риси електронної структури складніших молекул,
          але без зайвих ускладнень багаточастинкової взаємодії;
    \item зручні для порівняння точних, наближених і експериментальних результатів;
    \item використовуються як тестові системи для перевірки методів квантової хімії.
\end{itemize}

\begin{itemize}
    \item \textbf{\ce{H2+}} — найпростіша двоатомна молекула, що містить один електрон.
          Її можна розв’язати аналітично в еліптичних координатах,
          тому вона є класичним прикладом формування зв’язувальної орбіталі $\sigma_g$;
    \item \textbf{\ce{H2}} — найпростіша двоелектронна молекула,
          у якій вперше проявляється електронна кореляція
          та обмеження методу Хартрі–Фока.
\end{itemize}

Таким чином, системи \ce{H2+} і \ce{H2} є \textbf{відправною точкою}
для розуміння природи хімічного зв’язку, ролі симетрії,
утворення молекулярних орбіталей і спінових станів.


%% ========================================================
\section{Іон молекули водню \ce{H2+}}
%% ========================================================

%% --------------------------------------------------------
\subsection{Теоретичні основи}
%% --------------------------------------------------------

Іон \ce{H2+} --- це найпростіша молекулярна система, яка складається з двох протонів
та одного електрона. Незважаючи на її простоту, саме на прикладі \ce{H2+}
вперше стає очевидною суть хімічного зв’язку як результату
квантового перекривання хвильових функцій.

У нерелятивістському наближенні (та з використанням атомних одиниць)
гамільтоніан системи має вигляд:
\[
\hat{H} = -\frac{1}{2}\nabla^2 - \frac{1}{r_A} - \frac{1}{r_B} + \frac{1}{R},
\]
де $r_A$, $r_B$ --- відстані електрона до ядер $A$ і $B$,
а $R$ --- між'ядерна відстань.

Перші три члени описують кінетичну енергію електрона
та його притягання до двох ядер, а останній доданок --- кулонівське
відштовхування між ядрами.

Оскільки у системі лише один електрон, тут \emph{відсутні електрон-елек\-трон\-ні взаємодії}.
Тому метод Хартрі--Фока не містить ніяких додаткових апроксимацій (крім обмежень базису),
і його результат збігається з точною розв’язною енергією при нескінченному базисі.
Таким чином, \ce{H2+} --- це ідеальний тест для перевірки
коректності реалізації ХФ-методу в програмних пакетах.



%% --------------------------------------------------------
\subsection{Визначення молекули в PySCF}
%% --------------------------------------------------------

Для визначення молекули в PySCF використовується рядкова нотація:

\inputcode{h2plus_single.py}

\textbf{Пояснення коду:}
\begin{itemize}
    \item \inlinecode{atom = 'H 0 0 0; H 0 0 0.74'} ---
          координати атомів (Ангстреми за замовчуванням)
    \item \inlinecode{basis = 'sto-3g'} --- мінімальний базис
    \item \inlinecode{charge = 1} --- заряд системи (+1 для \ce{H2+})
    \item \inlinecode{spin = 1} --- $2S = N_\alpha - N_\beta = 1$
    \item \inlinecode{scf.UHF} --- необмежений ХФ (один непарний електрон)
\end{itemize}


%% --------------------------------------------------------
\subsection{Інтерпретація результатів}
%% --------------------------------------------------------

Вивід програми містить основні характеристики розрахованої системи \ce{H2+} при між’ядерній відстані
\[
R = 0.74~\text{Å} = 1.399~a_0,
\]
у базисному наборі STO-3G.

\begin{itemize}
    \item \textbf{Повна енергія:}
          \[
          E = -0.538205~\text{Ha},
          \]
          що є нижчою за енергію вільного атома водню ($-0.5$~Ha).
          Це свідчить про стабілізацію системи внаслідок утворення молекулярного зв’язку —
          електрон делокалізується між двома ядрами, зменшуючи електростатичне відштовхування між протонами.

    \item \textbf{Енергії орбіталей (для $\alpha$-спіну):}
          \begin{align*}
              \varepsilon_{\mathrm{HOMO}} &= -1.2533~\text{Ha}, \\
              \varepsilon_{\mathrm{LUMO}} &= -0.4751~\text{Ha}, \\
              \Delta_{\mathrm{H-L}} &= 0.7782~\text{Ha} = 21.18~\text{еВ}.
          \end{align*}
          HOMO має симетрію $\sigma_g$ і є зв’язувальною орбіталлю,
          тоді як LUMO відповідає антизв’язувальному стану $\sigma_u^*$.
          Значний енергетичний розрив між ними свідчить про сильний і стабільний зв’язок.

    \item \textbf{Спіновий стан:}
          \[
          \langle S^2 \rangle = 0.75 \quad \Rightarrow \quad S = \frac{1}{2}, \quad 2S+1 = 2.
          \]
          Це відповідає дублетному стану з одним неспареним електроном, що характерно для \ce{H2+}.

\end{itemize}

Таким чином, результати розрахунку підтверджують,
що при відстані $R = 0.74$~Å іон \ce{H2+} перебуває у стабільному зв’язаному стані.


%% --------------------------------------------------------
\subsection{Крива потенційної енергії (PES)}
%% --------------------------------------------------------

\textbf{Крива потенційної енергії} (Potential Energy Surface, PES) ---
це залежність повної електронної енергії системи $E(\mathbf{R})$
від положень ядер $\mathbf{R} = (R_1, R_2, \dots, R_N)$.

Для загальної $N$-атомної молекули PES --- це багатовимірна поверхня
у $3N - 6$ координатах (три координати на кожне ядро мінус
три для поступального руху і три для обертання).
У випадку двоатомної молекули (наприклад, \ce{H2+}) поверхня зводиться до
\emph{одновимірної кривої} $E(R)$, де $R$ --- відстань між ядрами:

\[
E(R) = \text{мінімальна енергія системи при фіксованому } R.
\]

\textbf{Фізичний зміст:}
\begin{itemize}
    \item Мінімум $E(R)$ відповідає \emph{рівноважній відстані} $R_\text{eq}$,
          де сили на ядра взаємно компенсуються ($\partial E / \partial R = 0$);
    \item Глибина ями потенціалу визначає \emph{енергію зв’язку};
    \item Крутизна поблизу мінімуму пов’язана з \emph{жорсткістю зв’язку}
          та \emph{частотою коливань};
    \item Для великих $R$ енергія прямує до суми енергій ізольованих атомів.
\end{itemize}

\textbf{Обчислення PES} у квантовій хімії полягає у серії
незалежних SCF-розрахунків при різних значеннях $R$:
\[
R = 0.5,\, 0.7,\, 1.0,\, 1.5,\, 2.0,\, \dots~\text{Å}.
\]
На кожному кроці визначається повна енергія $E(R_i)$, після чого будується
графік $E(R)$, який ілюструє поведінку потенціальної енергії системи.

\textbf{Типова форма PES для \ce{H2+}:}
\begin{itemize}
    \item На малих $R$ енергія різко зростає через кулонівське відштовхування ядер;
    \item Поблизу $R \approx 0.74$~Å --- глибокий мінімум (стабільний хімічний зв’язок);
    \item При $R \to \infty$ енергія прямує до енергії одного атома Гідрогену $E = -0.5$~Ha.
\end{itemize}


\subsubsection{Сканування по відстанях}

Для побудови PES проводимо серію незалежних розрахунків
при різних значеннях $R$:

\inputcode{h2plus_pes.py}

\textbf{Важливі моменти:}
\begin{itemize}
    \item Відстані задаються в борах (1 bohr $\approx$ 0.529 Å);
    \item Діапазон $R \in [0.5, 5.0]$ bohr охоплює від сильно стисненого стану
          до повної дисоціації;
    \item Кожна точка --- окремий SCF-розрахунок;
    \item Результати $E(R_i)$ зберігаються для подальшого аналізу.
\end{itemize}

%---------------------------------------------------------
\begin{figure}[h!]\centering
\includegraphics[width=\linewidth]{\currfiledir/pictures/h2plus_pes.pdf}
\caption{Крива потенційної енергії (PES) для \ce{H2+}.}
\label{pic:h2plus_pes}
\end{figure}
%---------------------------------------------------------

На графіку (рис.~\ref{pic:h2plus_pes}) можна виділити три характерні області:

\begin{enumerate}
    \item \textbf{Стиснення} ($R < R_e$) — енергія зростає через відштовхування ядер;
    \item \textbf{Мінімум} ($R = R_e$) — рівноважна відстань, де $\partial E / \partial R = 0$;
    \item \textbf{Дисоціація} ($R \to \infty$) — енергія прямує до $E(\text{H}) + E(\text{H}^+) = -0.5$~Ha.
\end{enumerate}

\textbf{Порівняння з аналітичним розв’язком:}
\begin{itemize}
    \item $R_e = 2.00$ bohr (експеримент: 2.00 bohr);
    \item $D_e = 0.102$ Ha = 2.79 eV (експеримент: 2.79 eV);
    \item Різниця між HF та точним розв’язком $< 0.001$~Ha при великих базисах.
\end{itemize}

%% --------------------------------------------------------
\subsection{Оптимізація геометрії}
%% --------------------------------------------------------


Замість <<ручного>> пошуку мінімуму енергії на кривій потенційної енергії (PES)
можна скористатися автоматичною процедурою, яка сама визначає оптимальне
взаємне розташування атомів.
Така задача називається \textbf{оптимізацією геометрії}, оскільки метою є знайти
такі координати ядер, при яких система має найменшу можливу енергію.

Інакше кажучи, \textbf{оптимізація геометрії} --- це процедура пошуку просторової
конфігурації атомів, що відповідає мінімуму потенційної енергії.
Ми шукаємо \emph{рівноважну геометрію} $\mathbf{R}_\text{eq}$, у якій сили,
що діють на ядра, взаємно компенсуються:
\[
\nabla_{\mathbf{R}} E(\mathbf{R}_\text{eq}) = 0.
\]

На практиці алгоритм оптимізації виконує:
\begin{enumerate}
    \item обчислення енергії $E(\mathbf{R})$ та її градієнтів $\nabla E$;
    \item зміну координат ядер у напрямку зменшення енергії;
    \item повторення кроків до досягнення умови $|\nabla E| < 10^{-4}$~Ha/bohr.
\end{enumerate}

У PySCF ця процедура реалізована через зовнішній модуль \texttt{geomeTRIC},
який викликається функцією:
\begin{minted}{python}
from pyscf.geomopt.geometric_solver import optimize
\end{minted}

Щоб цей модуль працював, потрібно мати встановлений geometric:

\begin{minted}{bash}
pip install geometric
\end{minted}

Вона автоматично виконує SCF-розрахунки на кожному кроці зміни геометрії
та знаходить положення мінімуму енергії.

\inputcode{h2plus_optimization.py}

\textbf{Алгоритм оптимізації (PySCF + \href{https://geometric.readthedocs.io}{geomeTRIC}):}
\begin{itemize}
    \item Використовується метод quasi-Newton (BFGS);
    \item Для оцінки напрямку руху застосовуються градієнти енергії $\nabla E$;
    \item Критерій збіжності: $|\nabla E| < 10^{-4}$~Ha/bohr;
    \item Зазвичай потрібно 5--10 ітерацій для простої двоатомної системи.
\end{itemize}

%Для подальшого аналізу стабільності отриманої структури та вивчення коливальних мод
%використовується модуль:
%\begin{minted}{python}
%from pyscf import hessian
%\end{minted}
%який дозволяє обчислити матрицю Гессіана (другі похідні енергії)
%і визначити нормальні коливання молекули поблизу рівноважної геометрії.



%% --------------------------------------------------------
\subsection{Аналіз молекулярних орбіталей}
%% --------------------------------------------------------

У квантовій хімії молекулярні орбіталі (МО) будуються за принципом
\textbf{лінійної комбінації атомних орбіталей (ЛКАО)}:
\[
\psi_i = \sum_\mu c_{\mu i} \, \phi_\mu,
\]
де $\phi_\mu$ --- атомні орбіталі, а $c_{\mu i}$ — коефіцієнти, що визначаються з рівнянь Хартрі–Фока (ХФ).

Метод ХФ дає набір МО $\{\psi_i\}$ з відповідними енергіями $\varepsilon_i$, які розділяються на:
\begin{itemize}
    \item \textbf{зайняті орбіталі (occupied)} — електрони займають найнижчі за енергією МО;
    \item \textbf{віртуальні орбіталі (virtual)} — порожні орбіталі.
\end{itemize}

Кожна МО є власною функцією рівняння Фока:
\[
\hat{f}\,\psi_i = \varepsilon_i\,\psi_i,
\]
і всі $\psi_i$ утворюють ортонормовану систему.

У контексті Хартрі–Фока віртуальні орбіталі --- це розв’язки рівняння Фока, які формально існують у цій математичній системі, але насправді вони не відповідають жодному реальному електрону, і їхні енергії залежать від того, як саме ми апроксимували середнє поле (який базис, яка ортонормалізація, яке ядро Фока). Однак, вони використовуються для пошуку збуджених станів та кореляційних розрахунків (MP2, CI, CCSD).

Для іона \ce{H2+} атомні орбіталі $1s_A$ та $1s_B$ поєднуються у дві молекулярні орбіталі:
\[
\psi_{\sigma_g} = \frac{1}{\sqrt{2(1+S)}} (1s_A + 1s_B), \qquad
\psi_{\sigma_u^*} = \frac{1}{\sqrt{2(1-S)}} (1s_A - 1s_B),
\]
де $S = \langle 1s_A | 1s_B \rangle$ — інтеграл перекривання.

\begin{itemize}
    \item $\psi_{\sigma_g}$ --- \emph{зв’язуюча орбіталь}, електронна густина зосереджена між ядрами;
    \item $\psi_{\sigma_u^*}$ --- \emph{антизв’язуюча орбіталь}, густина між ядрами зменшена.
\end{itemize}

Іон \ce{H2+} має один електрон, який займає нижчу зв’язуючу орбіталь $\sigma_g$, стабілізуючи систему.
Віртуальна орбіталь $\sigma_u^*$ залишається порожньою.

\inputcode{h2plus_mo_analysis.py}  % <- тут приклад коду аналізу МО


%% --------------------------------------------------------
\subsection{Вплив базисного набору}
%% --------------------------------------------------------

Оскільки \ce{H2+} має аналітичний розв'язок, це ідеальна система
для тестування базисів:

\inputcode{h2plus_basis_convergence.py}

\textbf{Очікувані результати:}
\begin{center}
\begin{tabular}{lccc}
\toprule
Базис & $R_e$ (bohr) & $E_e$ (Ha) & Похибка (mHa) \\
\midrule
STO-3G      & 2.05 & $-0.564$ & 4.5 \\
6-31G       & 2.02 & $-0.566$ & 2.1 \\
cc-pVDZ     & 2.01 & $-0.567$ & 1.2 \\
cc-pVTZ     & 2.00 & $-0.5686$ & 0.3 \\
cc-pVQZ     & 2.00 & $-0.5688$ & 0.1 \\
\midrule
Точне       & 2.00 & $-0.5689$ & --- \\
\bottomrule
\end{tabular}
\end{center}

\textbf{Висновки:}
\begin{itemize}
    \item Навіть STO-3G дає якісно правильну картину
    \item Для кількісної точності потрібен cc-pVTZ або більший
    \item Збіжність монотонна: $E_{\text{basis}} \to E_{\text{CBS}}$
    \item Для \ce{H2+} CBS limit досягається при cc-pV5Z
\end{itemize}

%% ========================================================
\section{Молекула водню \ce{H2}}
%% ========================================================

%% --------------------------------------------------------
\subsection{Двоелектронна система}
%% --------------------------------------------------------

Молекула \ce{H2} --- перша справжня двоелектронна система.
Гамільтоніан:
\[
\hat{H} = \hat{h}_1 + \hat{h}_2 + \frac{1}{r_{12}} + \frac{1}{R},
\]
де $\frac{1}{r_{12}}$ --- електрон-електронне відштовхування,
яке метод ХФ апроксимує середнім полем.

\textbf{Електронна конфігурація:}
Основний стан --- синглет ($^1\Sigma_g^+$) з конфігурацією $\sigma_g^2$:
\[
\Psi = \frac{1}{\sqrt{2}}[\psi_{\sigma_g}(\mathbf{r}_1)\alpha(1)
\psi_{\sigma_g}(\mathbf{r}_2)\beta(2)
- \psi_{\sigma_g}(\mathbf{r}_1)\beta(1)
\psi_{\sigma_g}(\mathbf{r}_2)\alpha(2)]
\]

%% --------------------------------------------------------
\subsection{RHF розрахунок}
%% --------------------------------------------------------

Для замкненої оболонки використовуємо RHF:

\inputcode{h2_rhf_single.py}

\textbf{Результат при $R = 1.4$ bohr:}
\begin{itemize}
    \item Енергія: $E_{\text{RHF}} \approx -1.133$ Ha
    \item Експериментальна: $E_{\text{exp}} \approx -1.174$ Ha
    \item Кореляційна енергія: $E_{\text{corr}} = 0.041$ Ha $\approx$ 1.1 eV
\end{itemize}


%% ========================================================
\section{Проблема дисоціації в методі Хартрі--Фока}
%% ========================================================


Нижче наведено приклад розрахунку поверхні потенційної енергії молекули \ce{H2} методом RHF.
Цей розрахунок ілюструє класичну проблему RHF --- при великих міжядерних відстанях:
метод змушує обидва електрони займати одну і ту ж молекулярну орбіталь,
що призводить до неправильного опису дисоціації молекули \ce{H2}.


\inputcode{h2_pes_RHF.py}

%---------------------------------------------------------
\begin{figure}[h!]\centering
\includegraphics[width=\linewidth]{\currfiledir/pictures/h2_rhf.pdf}
\caption{Поверхня потенційна енергії молекули \ce{H2} методом RHF.}
\label{pic:h2_rhf}
\end{figure}
%---------------------------------------------------------


\subsection{Фізична картина}

При розриві зв'язку \ce{H2} $\to$ 2\ce{H} очікується:
\[
\lim_{R\to\infty} E(\text{H}_2) = 2 \times E(\text{H}) = 2 \times (-0.5) = -1.0 \text{ Ha}
\]

Однак RHF дає (рис.~\ref{pic:h2_rhf}):
\[
\lim_{R\to\infty} E_{\text{RHF}}(\text{H}_2) \approx -0.7 \text{ Ha}
\]

\textbf{Чому?} RHF змушує електрони мати однакові просторові орбіталі:
\[
\psi_{\text{RHF}} = |\sigma_g\alpha\, \sigma_g\beta\rangle
\]

При великих $R$ це означає:
\[
\psi \sim |(1s_A + 1s_B)\alpha\, (1s_A + 1s_B)\beta\rangle
\]

Розкриваючи детермінант, отримуємо \textbf{іонні внески}:
\[
\psi \sim \ce{H-H} + \ce{H+} + \ce{H-} + \ce{H+}\ce{H-}
\]

При $R \to \infty$ іонні стани мають завищену енергію,
тому RHF-енергія неправильна!


%% --------------------------------------------------------
\subsection{Візуалізація проблеми}
%% --------------------------------------------------------

Для наочного аналізу порівняємо результати розрахунку потенційної енергії молекули \ce{H2} у двох підходах: \texttt{RHF} та \texttt{FCI}\footnote{FCI (Full Configuration Interaction) --- виступає \emph{еталонним розв’язком},
який максимально точно (в межах вибраного базисного набору) описує електронну структуру системи.}


\inputcode{h2_dissociation_comparison.py}

%---------------------------------------------------------
\begin{figure}[h!]\centering
\includegraphics[width=\linewidth]{\currfiledir/pictures/h2_RHF_vs_FCI.pdf}
\caption{Повне порівняння методів для молекули \ce{H2}.}
\label{pic:h2_complete_comparison}
\end{figure}
%---------------------------------------------------------

\textbf{Графік (рис.~\ref{pic:h2_complete_comparison}) показує:}
\begin{itemize}
    \item \textbf{Точний розв'язок} (Full CI) --- монотонна крива.
    \item \textbf{RHF} --- завищена енергія при $R > 2.5$ bohr.
\end{itemize}

%% --------------------------------------------------------
\subsection{Відсутність кореляції як джерело проблеми}
%% --------------------------------------------------------

Фундаментальна причина --- метод ХФ описує електрон-електронну
взаємодію через \textbf{середнє поле}, ігноруючи миттєву кореляцію
положень електронів.

\textbf{Кореляційна енергія:}
\[
E_{\text{corr}}(R) = E_{\text{exact}}(R) - E_{\text{HF}}(R)
\]

\begin{center}
\begin{tabular}{lcc}
\toprule
$R$ (bohr) & $E_{\text{corr}}$ (mHa) & \% від $D_e$ \\
\midrule
1.4 (рівновага) & 41 & 9\% \\
3.0             & 85 & 18\% \\
5.0             & 150 & 32\% \\
$\infty$        & 300 & --- \\
\bottomrule
\end{tabular}
\end{center}

\textbf{Висновок:} Кореляція стає критичною при розриві зв'язків!