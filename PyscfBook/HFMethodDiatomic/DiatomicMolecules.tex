% !TeX program = lualatex
% !TeX encoding = utf8
% !TeX spellcheck = uk_UA
% !TeX root =../PyscfBook.tex


Після вивчення атомних систем природним кроком є перехід до молекул.
Найпростіші диатомні молекули --- іон молекули водню \ce{H2+}
та молекула водню \ce{H2} --- є ідеальними об'єктами для демонстрації
можливостей та обмежень методу Хартрі--Фока.

%% ========================================================
\section{Чому саме \ce{H2+} та \ce{H2}?}
%% ========================================================

\begin{itemize}
    \item \textbf{\ce{H2+}} --- найпростіша молекула (один електрон),
          має аналітичний розв'язок у еліптичних координатах
    \item \textbf{\ce{H2}} --- найпростіша двоелектронна молекула,
          демонструє фундаментальну проблему методу ХФ
    \item Обидві системи достатньо малі для детального аналізу
    \item Результати можна порівняти з високоточними експериментальними даними
\end{itemize}

%% ========================================================
\section{Іон молекули водню \ce{H2+}}
%% ========================================================

%% --------------------------------------------------------
\subsection{Теоретичні основи}
%% --------------------------------------------------------

Іон \ce{H2+} складається з двох протонів та одного електрона.
Гамільтоніан системи (в атомних одиницях):
\[
\hat{H} = -\frac{1}{2}\nabla^2 - \frac{1}{r_A} - \frac{1}{r_B} + \frac{1}{R},
\]
де $r_A$, $r_B$ --- відстані електрона до ядер A і B,
$R$ --- міжядерна відстань.

Оскільки система має лише один електрон, метод Хартрі--Фока
\textbf{не містить апроксимацій} (крім базисних обмежень),
тому HF-енергія збігається з точною в межах обраного базису.

%% --------------------------------------------------------
\subsection{Визначення молекули в PySCF}
%% --------------------------------------------------------

Для визначення молекули в PySCF використовується рядкова нотація:

\inputcode{h2plus_single.py}

\textbf{Пояснення коду:}
\begin{itemize}
    \item \inlinecode{atom = 'H 0 0 0; H 0 0 0.74'} ---
          координати атомів (Ангстреми за замовчуванням)
    \item \inlinecode{basis = 'sto-3g'} --- мінімальний базис
    \item \inlinecode{charge = 1} --- заряд системи (+1 для \ce{H2+})
    \item \inlinecode{spin = 1} --- $2S = N_\alpha - N_\beta = 1$
    \item \inlinecode{scf.UHF} --- необмежений ХФ (один непарний електрон)
\end{itemize}

\textbf{Результат:} Енергія $E \approx -0.566$ Ha для $R = 0.74$ Å.

%% --------------------------------------------------------
\subsection{Крива потенційної енергії (PES)}
%% --------------------------------------------------------

Крива потенційної енергії (Potential Energy Surface, PES)
показує залежність повної енергії системи від міжядерної відстані $R$.
Для диатомної молекули це одновимірна функція $E(R)$.

\subsubsection{Сканування по відстаням}

Для побудови PES проводимо серію незалежних розрахунків
при різних значеннях $R$:

\inputcode{h2plus_pes.py}

\textbf{Важливі моменти:}
\begin{itemize}
    \item Відстані задаються в бораx (1 bohr $\approx$ 0.529 Å)
    \item Діапазон $R \in [0.5, 5.0]$ bohr охоплює від стиснення
          до повної дисоціації
    \item Кожна точка --- незалежний SCF-розрахунок
    \item Результат зберігається для подальшого аналізу
\end{itemize}

\subsubsection{Аналіз кривої}

%---------------------------------------------------------
\begin{figure}[h!]\centering
\includegraphics[width=\linewidth]{\currfiledir/h2plus_pes.pdf}
\caption{Графіку PES для \ce{H2+}.}
\label{pic:h2plus_pes}
\end{figure}
%---------------------------------------------------------

На графіку (рис.~\ref{pic:h2plus_pes}) PES спостерігаються характерні області:

\begin{enumerate}
    \item \textbf{Область стиснення} ($R < R_e$):
          Енергія різко зростає через відштовхування ядер

    \item \textbf{Мінімум енергії} ($R = R_e$):
          Рівноважна відстань, де сили притягання
          та відштовхування збалансовані

    \item \textbf{Дисоціаційна межа} ($R \to \infty$):
          Енергія прямує до суми енергій ізольованих фрагментів
          \[
          E(R\to\infty) \to E(\text{H}) + E(\text{H}^+) = -0.5 \text{ Ha}
          \]
\end{enumerate}

\textbf{Порівняння з аналітичним розв'язком:}
Для \ce{H2+} існує точний розв'язок у еліптичних координатах:
\begin{itemize}
    \item $R_e = 2.00$ bohr (експеримент: 2.00 bohr)
    \item $D_e = 0.102$ Ha = 2.79 eV (експеримент: 2.79 eV)
    \item Різниця HF--точне $< 0.001$ Ha для великих базисів
\end{itemize}

%% --------------------------------------------------------
\subsection{Оптимізація геометрії}
%% --------------------------------------------------------

Замість ручного сканування можна автоматично знайти мінімум
енергії (рівноважну геометрію):

\inputcode{h2plus_optimization.py}

\textbf{Алгоритм оптимізації:}
\begin{itemize}
    \item PySCF використовує градієнти енергії $\nabla E$
    \item Метод quasi-Newton (BFGS за замовчуванням)
    \item Критерій збіжності: $|\nabla E| < 10^{-4}$ Ha/bohr
    \item Зазвичай потрібно 5--10 ітерацій
\end{itemize}

\textbf{Спектроскопічні константи:}
Після оптимізації можна обчислити:
\begin{itemize}
    \item \textbf{Рівноважна відстань} $R_e$ --- з оптимізованої геометрії
    \item \textbf{Енергія дисоціації}:
          \[
          D_e = E(\text{H}) + E(\text{H}^+) - E(\text{H}_2^+, R_e)
          \]
    \item \textbf{Частота коливань} $\omega_e$ --- з другої похідної
          (гесіану) енергії
\end{itemize}

%% --------------------------------------------------------
\subsection{Аналіз молекулярних орбіталей}
%% --------------------------------------------------------

Для \ce{H2+} єдина зайнята молекулярна орбіталь (МО)
є зв'язуючою $\sigma_g$ комбінацією атомних орбіталей:
\[
\psi_{\sigma_g} \approx \frac{1}{\sqrt{2}}(\phi_{1s}^A + \phi_{1s}^B)
\]

\inputcode{h2plus_mo_analysis.py}

\textbf{Енергетична діаграма:}
\begin{itemize}
    \item Зайнята МО: $\varepsilon_{\sigma_g} \approx -0.6$ Ha (зв'язуюча)
    \item Віртуальна МО: $\varepsilon_{\sigma_u^*} \approx +0.2$ Ha
          (антизв'язуюча)
    \item HOMO--LUMO gap: $\Delta \approx 0.8$ Ha $\approx$ 22 eV
\end{itemize}

%% --------------------------------------------------------
\subsection{Вплив базисного набору}
%% --------------------------------------------------------

Оскільки \ce{H2+} має аналітичний розв'язок, це ідеальна система
для тестування базисів:

\inputcode{h2plus_basis_convergence.py}

\textbf{Очікувані результати:}
\begin{center}
\begin{tabular}{lccc}
\toprule
Базис & $R_e$ (bohr) & $E_e$ (Ha) & Похибка (mHa) \\
\midrule
STO-3G      & 2.05 & $-0.564$ & 4.5 \\
6-31G       & 2.02 & $-0.566$ & 2.1 \\
cc-pVDZ     & 2.01 & $-0.567$ & 1.2 \\
cc-pVTZ     & 2.00 & $-0.5686$ & 0.3 \\
cc-pVQZ     & 2.00 & $-0.5688$ & 0.1 \\
\midrule
Точне       & 2.00 & $-0.5689$ & --- \\
\bottomrule
\end{tabular}
\end{center}

\textbf{Висновки:}
\begin{itemize}
    \item Навіть STO-3G дає якісно правильну картину
    \item Для кількісної точності потрібен cc-pVTZ або більший
    \item Збіжність монотонна: $E_{\text{basis}} \to E_{\text{CBS}}$
    \item Для \ce{H2+} CBS limit досягається при cc-pV5Z
\end{itemize}

%% ========================================================
\section{Молекула водню \ce{H2}}
%% ========================================================

%% --------------------------------------------------------
\subsection{Двоелектронна система}
%% --------------------------------------------------------

Молекула \ce{H2} --- перша справжня двоелектронна система.
Гамільтоніан:
\[
\hat{H} = \hat{h}_1 + \hat{h}_2 + \frac{1}{r_{12}} + \frac{1}{R},
\]
де $\frac{1}{r_{12}}$ --- електрон-електронне відштовхування,
яке метод ХФ апроксимує середнім полем.

\textbf{Електронна конфігурація:}
Основний стан --- синглет ($^1\Sigma_g^+$) з конфігурацією $\sigma_g^2$:
\[
\Psi = \frac{1}{\sqrt{2}}[\psi_{\sigma_g}(\mathbf{r}_1)\alpha(1)
\psi_{\sigma_g}(\mathbf{r}_2)\beta(2)
- \psi_{\sigma_g}(\mathbf{r}_1)\beta(1)
\psi_{\sigma_g}(\mathbf{r}_2)\alpha(2)]
\]

%% --------------------------------------------------------
\subsection{RHF розрахунок}
%% --------------------------------------------------------

Для замкненої оболонки використовуємо RHF:

\inputcode{h2_rhf_single.py}

\textbf{Результат при $R = 1.4$ bohr:}
\begin{itemize}
    \item Енергія: $E_{\text{RHF}} \approx -1.133$ Ha
    \item Експериментальна: $E_{\text{exp}} \approx -1.174$ Ha
    \item Кореляційна енергія: $E_{\text{corr}} = 0.041$ Ha $\approx$ 1.1 eV
\end{itemize}

%% --------------------------------------------------------
\subsection{Крива потенційної енергії}
%% --------------------------------------------------------

Побудуємо PES для \ce{H2} методами RHF та UHF:

\inputcode{h2_pes_comparison.py}


%---------------------------------------------------------
\begin{figure}[h!]\centering
\includegraphics[width=\linewidth]{\currfiledir/h2_rhf_vs_uhf.pdf}
\caption{}
\label{}
\end{figure}
%---------------------------------------------------------


\textbf{Спостереження на графіку:}
\begin{enumerate}
    \item \textbf{Навколо мінімуму} ($R \approx 1.4$ bohr):
          RHF та UHF дають практично однакові результати

    \item \textbf{При розтягуванні} ($R > 3$ bohr):
          Криві розходяться! RHF завищує енергію

    \item \textbf{Дисоціаційна межа} ($R \to \infty$):
          \begin{itemize}
              \item UHF: $E \to 2 \times E(\text{H}) = -1.0$ Ha ✓
              \item RHF: $E \to E(\text{H}^+) + E(\text{H}^-) \approx -0.7$ Ha ✗
          \end{itemize}
\end{enumerate}

%% ========================================================
\section{Проблема дисоціації в методі Хартрі--Фока}
%% ========================================================

%% --------------------------------------------------------
\subsection{RHF: некоректна дисоціація}
%% --------------------------------------------------------

\subsubsection{Фізична картина}

При розриві зв'язку \ce{H2} $\to$ 2\ce{H} очікується:
\[
\lim_{R\to\infty} E(\text{H}_2) = 2 \times E(\text{H}) = 2 \times (-0.5) = -1.0 \text{ Ha}
\]

Однак RHF дає:
\[
\lim_{R\to\infty} E_{\text{RHF}}(\text{H}_2) \approx -0.7 \text{ Ha}
\]

\textbf{Чому?} RHF змушує електрони мати однакові просторові орбіталі:
\[
\psi_{\text{RHF}} = |\sigma_g\alpha\, \sigma_g\beta\rangle
\]

При великих $R$ це означає:
\[
\psi \sim |(1s_A + 1s_B)\alpha\, (1s_A + 1s_B)\beta\rangle
\]

Розкриваючи детермінант, отримуємо \textbf{іонні внески}:
\[
\psi \sim \text{H--H} + \text{H}^+\text{H}^- + \text{H}^-\text{H}^+
\]

При $R \to \infty$ іонні стани мають завищену енергію,
тому RHF-енергія неправильна!

\subsubsection{Математичне обґрунтування}

RHF-хвильова функція не є власним станом оператора $\hat{S}^2$
при великих $R$. Вона містить домішки триплетних станів,
що призводить до неправильної енергії дисоціації.

%% --------------------------------------------------------
\subsection{UHF: правильна дисоціація, але спінове забруднення}
%% --------------------------------------------------------

UHF дозволяє різні орбіталі для $\alpha$ та $\beta$ спінів:
\[
\psi_{\text{UHF}} = |\phi_\alpha\, \phi_\beta\rangle,
\quad \phi_\alpha \neq \phi_\beta
\]

При $R \to \infty$:
\[
\phi_\alpha \to 1s_A, \quad \phi_\beta \to 1s_B
\]

Це дає правильну енергію:
\[
\lim_{R\to\infty} E_{\text{UHF}} = E(\text{H}) + E(\text{H}) = -1.0 \text{ Ha}
\]

\textbf{Але!} Виникає спінове забруднення:

\inputcode{h2_spin_contamination.py}

\textbf{Результат:}
\begin{itemize}
    \item При $R = 1.4$ bohr: $\langle S^2 \rangle \approx 0.00$
          (чистий синглет)
    \item При $R = 5.0$ bohr: $\langle S^2 \rangle \approx 1.00$
          (забруднення триплетом!)
    \item Теоретично для $S=0$: $\langle S^2 \rangle = 0$
\end{itemize}

\textbf{Інтерпретація:} UHF-хвильова функція при великих $R$ є сумішшю
синглету та триплету, хоча правильний стан --- чистий синглет.

%% --------------------------------------------------------
\subsection{Візуалізація проблеми}
%% --------------------------------------------------------

Порівняємо всі методи на одному графіку:

\inputcode{h2_dissociation_comparison.py}

%---------------------------------------------------------
\begin{figure}[h!]\centering
\includegraphics[width=\linewidth]{\currfiledir/h2_complete_comparison.pdf}
\caption{Повне порівняння методів для молекули \ce{H2}.}
\label{pic:h2_complete_comparison}
\end{figure}
%---------------------------------------------------------

\textbf{Графік (рис.~\ref{pic:h2_complete_comparison}) показує:}
\begin{itemize}
    \item \textbf{Точний розв'язок} (Full CI) --- монотонна крива
    \item \textbf{RHF} --- завищена енергія при $R > 2.5$ bohr
    \item \textbf{UHF} --- правильна асимптотика, але забруднений спін
    \item \textbf{ROHF} --- між RHF та UHF (не показано, складніший)
\end{itemize}

%% --------------------------------------------------------
\subsection{Відсутність кореляції як джерело проблеми}
%% --------------------------------------------------------

Фундаментальна причина --- метод ХФ описує електрон-електронну
взаємодію через \textbf{середнє поле}, ігноруючи миттєву кореляцію
положень електронів.

\textbf{Кореляційна енергія:}
\[
E_{\text{corr}}(R) = E_{\text{exact}}(R) - E_{\text{HF}}(R)
\]

\begin{center}
\begin{tabular}{lcc}
\toprule
$R$ (bohr) & $E_{\text{corr}}$ (mHa) & \% від $D_e$ \\
\midrule
1.4 (рівновага) & 41 & 9\% \\
3.0             & 85 & 18\% \\
5.0             & 150 & 32\% \\
$\infty$        & 300 & --- \\
\bottomrule
\end{tabular}
\end{center}

\textbf{Висновок:} Кореляція стає критичною при розриві зв'язків!