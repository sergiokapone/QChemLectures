% !TeX program = lualatex
% !TeX encoding = utf8
% !TeX spellcheck = uk_UA
% !TeX root =../PyscfBook.tex

%=========================================================
\Opensolutionfile{answer}[\currfilebase/\currfilebase-Answers]
\chapter{Аналіз результатів розрахунків}\label{\currfilebase}
%=========================================================

%% --------------------------------------------------------
\section{Енергії атомів та енергії іонізації}
%% --------------------------------------------------------


%% --------------------------------------------------------
\subsection{Абсолютні та відносні енергії}
%% --------------------------------------------------------

Абсолютні енергії атомів, отримані з квантово-хімічних розрахунків, є від'ємними величинами у Hartree (атомних одиницях). Проте для практичних застосувань важливіші відносні енергії:

\begin{itemize}
    \item \textbf{Енергії іонізації (IE)} --- енергія, необхідна для видалення електрона
    \item \textbf{Електронна спорідненість (EA)} --- енергія при приєднанні електрона
    \item \textbf{Енергії збудження} --- різниця між станами
    \item \textbf{Енергії атомізації} --- для молекул
\end{itemize}


%% --------------------------------------------------------
\subsection{Перша енергія іонізації}
%% --------------------------------------------------------

Перша енергія іонізації визначається як:

\begin{equation}
    \text{IE}_1 = E(A^+) - E(A)
\end{equation}

де $E(A)$ --- енергія нейтрального атома, $E(A^+)$ --- енергія катіона.


\subsubsection{Розрахунок енергій іонізації}

\inputcode{FirstIonizationEnergyCalculation.py}


%% --------------------------------------------------------
\subsection{Систематичний розрахунок IE для другого періоду}
%% --------------------------------------------------------

\inputcode{SystematicIECalc.py}


%% --------------------------------------------------------
\subsection{Теорема Купманса}
%% --------------------------------------------------------

Згідно з теоремою Купманса, енергія іонізації наближено дорівнює від'ємній енергії HOMO орбіталі:

\begin{equation}
    \text{IE} \approx -\varepsilon_{\text{HOMO}}
\end{equation}

Це точно виконується для HF у границі нескінченного базису і в наближенні замороженого остова.

\inputcode{KoopmansTheoremCheck.py}


%% --------------------------------------------------------
\subsection{Друга та вища енергії іонізації}
%% --------------------------------------------------------

\inputcode{SequentialIEcalc.py}