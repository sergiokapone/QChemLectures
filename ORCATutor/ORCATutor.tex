% !TeX program = lualatex
% !TeX encoding = utf8
% !TeX spellcheck = uk_UA

\documentclass[10pt]{beamer}
\usetheme{QuantumChemistry}
\usepackage{QuantumChemistry}
\usepackage{makecell, booktabs, pgf}
\usepackage{minted, xcolor}
\graphicspath{{pictures/}}
\title[Лекції з квантової хімії]{\bfseries\huge ORCA \\ Принципи роботи з програмою}
\subtitle{Лекції з квантової хімії}
\author{Пономаренко С. М.}
\date{}
\begin{document}
%==============================================C=========================================================
\usebackgroundtemplate{
	\tikz\node[opacity=0.3]{\includegraphics[width=\paperwidth,height=\paperheight]{background}};%
}
\begin{frame}
	\thispagestyle{empty}
	\titlepage
\end{frame}
%=======================================================================================================
\usebackgroundtemplate{}

%============================================================================
\begin{frame}{Моделювання структури та властивостей молекул}{}
	Моделювання структури та опис властивостей  базується на
	декількох теоретичних методах і підходах:
	\begin{itemize}
		\item \alert{квантова хімія} --- опис взаємодій атомів і молекул, а також
		      хімічних перетворень методами квантової механіки;
		\item \alert{молекулярна механіка} --- опис взаємодій атомів і молекул на
		      основі заданих (емпіричних) класичних потенціалів;
		\item \alert{молекулярна динаміка} --- опис властивостей речовини і хімічних
		      перетворень шляхом розрахунку, відстеження і усереднення траєкторій руху
		      великого ансамблю молекул або атомів. Траєкторії будуються на основі
		      класичних рівнянь руху (законів Ньютона) молекул;
		\item \alert{моделювання методом Монте-Карло} --- опис властивостей речовини на
		      основі генерації і усереднення великої кількості випадково вибраних
		      конфігурацій молекул рідини, розчину, твердого тіла.
	\end{itemize}
\end{frame}
%============================================================================
\begin{frame}{Базисні набори}{}
	Розв'язки рівнянь Хартрі-Фока-Рутана, а також аналогічних рівнянь Кона-Шема в DFT вимагає вибору базисних функцій в розкладанні
	\begin{equation*}\label{}
		\phi_i = \sum\limits_{m = 1}^{N_b} c_{im}\chi_m
	\end{equation*}
	для подання хвильової функції молекули або її електронної густини. Вибір базисних функцій (базисного набору або базису) --- відповідальний етап квантовохімічного моделювання. Від правильного вибору залежить точність розв'язків, час обчислень і навіть можливість домогтися самоузгодження в процедурі SCF.
	В даний час розрізняють три основних типи базисних функцій:
	\begin{enumerate}
		\item слейтерівські,
		\item гаусові,
		\item базиси плоских хвиль.
	\end{enumerate}
\end{frame}
%============================================================================


\begin{frame}{Способи опису структури хімічної системи}{}
	Молекулярне моделювання базується на описі
	властивостей хімічної системи в залежності від координат атомів, я яких вона складається. Опис системи починається з задавання координат ядер, груп атомів або молекул. Залежно від способу задавання
	розрізняють два основних типи систем:
	\begin{itemize}
		\item системи, що не мають періодичність в просторі: атом,
		      молекула, комплекс (кластер) двох або декількох атомів або молекул;
		\item системи, які можуть бути описані набором періодично
		      повторюваних образів у просторі: кристал, аморфне тіло, розчин,
		      рідина, газ, поверхня між двома фазами.
	\end{itemize}
	Для опису атомно-молекулярних систем існує кілька способів задавання координат, кожен з яких має свої переваги і
	недоліки:
	\begin{enumerate}
		\item декартові координати;
		\item внутрішні координати.
		      %\item координати $Z$-матриці;
		      %\item дробні координати.
	\end{enumerate}
\end{frame}
%============================================================================

%============================================================================
\begin{frame}[fragile, t]{Декартові координати}{}
	\begin{columns}
		\begin{column}{0.4\linewidth}\scriptsize
			Декартові координати --- це просторові координати в обраній користувачем декартовій системі координат, доповнені інформацією про тип атома (тип хімічного елемента).
			\\~\\
			В молекулярному моделюванні декартові координати зазвичай записується рядком з четвірки чисел або символів <<$N\ X\ Y\ Z$>>, де $N$ --- хімічний символ атома, а $X$, $Y$, $Z$ --- власне декартові координати атома.
			\\~\\
			Роздільниками між числами або символами є пробіли. Такий спосіб записи будемо називати $NXYZ$-координатами (або $NXYZ$-форматом декартових координат).
		\end{column}
		\begin{column}{0.6\linewidth}\scriptsize
			Одиниці виміру координат $X$, $Y$, $Z$ можуть бути а ангстремах (\AA) або в атомних одиниці довжини (а.о., бори) (a.u., Bohr).
			\\~\\
			Приклад задавання координат атомів \ce{H} в молекулі \ce{H2}
			\begin{minted}[mathescape,
        gobble=8,
        %        breaklines,
        fontsize=\scriptsize,
        ]
        {ruby}
        %coords
        CTyp xyz   # the type of coordinates = xyz or internal
        Charge 0   # the total charge of the molecule
        Mult 1     # the multiplicity = 2S+1 S = 0
        Units Angs # the unit of length = angs or bohrs
        coords
            H    0.00000    0.00000    0.00000
            H    0.74000    0.00000    0.00000
        end
        end
        \end{minted}
		\end{column}
	\end{columns}

\end{frame}

%============================================================================
\begin{frame}[fragile, t]{Внутрішні координати}{}
	\begin{columns}
		\begin{column}{0.4\linewidth}\scriptsize
			Найбільш часто в якості внутрішніх координат вибираються міжатомні відстані $r$, кути між трійками атомів $\alpha$ (валентні кути, хоча ці кути можуть бути задані і між атомами, не пов'язаними хімічним зв'язком). Крім того, це можуть бути диедральні (торсіонні) кути $\theta$ між двома площинами $ABC$ і $BCD$, утвореними четвірками атомів $A$, $B$, $C$, $D$.
		\end{column}
		\begin{column}{0.6\linewidth}\scriptsize
			Приклад задавання внутрішніх атомів \ce{H} в молекулі \ce{H2}
			\begin{minted}[mathescape,
        gobble=8,
        %        breaklines,
        fontsize=\scriptsize,
        ]
        {ruby}
        %coords
        CTyp zmatrix # the type of coordinates = xyz or internal
        Charge 0      # the total charge of the molecule
        Mult 1        # the multiplicity = 2S+1 S = 0
        Units Angs    # the unit of length = angs or bohrs
        coords
           H     0    0    0    0.00000    0.00000    0.00000
           H     1    0    0    0.74000    0.00000    0.00000
        end
        end
        \end{minted}
		\end{column}
	\end{columns}

	Найбільш часто для такої прив'язки використовуються наступні додаткові припущення, які в подальшому ми будемо називати стандартною прив'язкою:
	\begin{itemize}
		\item перший по порядку атом системи розташовується в центрі декартової системи координат (в точці $О$) $\vec{r}_1 = (0, 0, 0)$;
		\item другий атом системи розташовується на осі $OX$: $\vec{r}_2 = (r_{12}, 0, 0)$;
		\item третій атом розташовується в площині $OXY$: $\vec{r}_3 = (x_3, y_3, 0)$. Конкретні значення координат $x_3$, $y_3$ залежать від атомів, з якими атом 3 утворює зв'язок і валентний кут;
	\end{itemize}

\end{frame}
%============================================================================


\begin{frame}[fragile, t]{Оптимізація геометрії}{}
	\begin{columns}[t]
		\begin{column}{0.45\linewidth}\scriptsize
			Оптимізація молекулярної геометрії є основним типом комп'ютерного експерименту в квантової хімії. Вона полягає в мінімізації повної енергії молекули  $E_\text{Total}$ при варіації
			координат атомів. Оскільки залежність $E_\text{Total}$ від координат ядер є
			поверхнею потенційної енергії (ППЕ), оптимізація геометрії є
			пошуком точок локальних мінімумів на ППЕ.
			\\~\\
			Оптимізація геометрії може бути \alert{повною} або \alert{частковою}. В першому
			випадку варіюються всі молекулярні координати, у другому --- деякі з
			них вважаються фіксованими (замороженими). Така «заморозка»
			координат може бути виконана шляхом спеціальних команд, які оголошують постійними деякі з внутрішніх
			координат.

		\end{column}
		\begin{column}{0.55\linewidth}
			\vspace*{-1em}
			\begin{minted}[mathescape,
        gobble=20,
%        breaklines,
        fontsize=\tiny,
]
                    {ruby}
                    ! RHF STO-3G Opt
                    %coords
                        CTyp internal  # the type of coordinates = xyz or internal
                        Charge 0       # the total charge of the molecule
                        Mult 1         # the multiplicity = 2S+1 S = 0
                        Units Angs     # the unit of length = angs or bohrs
                        coords
                           H    0    0    0    0.00000    0.00000    0.00000
                           H    1    0    0    0.30000    0.00000    0.00000
                        end
                    end
                    %geom
                    Calc_Hess true # Request an exact Hessian (here analytical)
                    in the first optimization step
                    end
                \end{minted}
			\begin{center}
				\includegraphics[width=0.9\linewidth]{pes.pdf}
			\end{center}
		\end{column}
	\end{columns}
\end{frame}
%============================================================================

%============================================================================
\begin{frame}[fragile, t]{Розрахунок властивостей при фіксованій геометрії}{}
	\begin{columns}
		\begin{column}{0.45\linewidth}\scriptsize
			Розрахунок при фіксованій геометрії (single-point calculation, SP) використовується якщо структура молекулярної системи фізично обґрунтована, наприклад, знайдена в ході попередньої оптимізації геометрії або обрана на основі експериментальних даних.
			\\~\\
			Метою розрахунку при фіксованій геометрії є оцінка статичних молекулярних властивостей, які не вимагають варіації геометрії. Такими властивостями є, наприклад: орбітальні енергії; потенціали іонізації; атомні заряди; мультипольні мультипольні моменти молекули; поверхні електростатичного потенціалу молекули, електронної густини, молекулярних орбіталей; вертикальні енергії електронно-збуджених станів і сили осциляторів електронних збуджень, що дозволяють відтворити УФ оптичний електронний спектр; поправки до енергії молекули, зумовлені впливом середовища.
		\end{column}
		\begin{column}{0.55\linewidth}
			\vspace*{-1em}
			\begin{minted}[mathescape,
        gobble=20,
%        breaklines,
        fontsize=\tiny,
]
                    {ruby}
                    ! RHF STO-3G SP
                    %coords
                        CTyp internal  # the type of coordinates = xyz or internal
                        Charge 0       # the total charge of the molecule
                        Mult 1         # the multiplicity = 2S+1 S = 0
                        Units Angs     # the unit of length = angs or bohrs
                        coords
                           H    0    0    0    0.00000    0.00000    0.00000
                           H    1    0    0    0.74000    0.00000    0.00000
                        end
                    end
                \end{minted}
		\end{column}
	\end{columns}
\end{frame}
%============================================================================

%============================================================================
%\begin{frame}{Розрахунок коливальних частот і термодинамічних параметрів}{}
%
%\end{frame}
%============================================================================

\end{document}
